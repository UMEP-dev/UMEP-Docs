%% Generated by Sphinx.
\def\sphinxdocclass{report}
\documentclass[letterpaper,10pt,english]{sphinxmanual}
\ifdefined\pdfpxdimen
   \let\sphinxpxdimen\pdfpxdimen\else\newdimen\sphinxpxdimen
\fi \sphinxpxdimen=.75bp\relax

\PassOptionsToPackage{warn}{textcomp}
\usepackage[utf8]{inputenc}
\ifdefined\DeclareUnicodeCharacter
 \ifdefined\DeclareUnicodeCharacterAsOptional
  \DeclareUnicodeCharacter{"00A0}{\nobreakspace}
  \DeclareUnicodeCharacter{"2500}{\sphinxunichar{2500}}
  \DeclareUnicodeCharacter{"2502}{\sphinxunichar{2502}}
  \DeclareUnicodeCharacter{"2514}{\sphinxunichar{2514}}
  \DeclareUnicodeCharacter{"251C}{\sphinxunichar{251C}}
  \DeclareUnicodeCharacter{"2572}{\textbackslash}
 \else
  \DeclareUnicodeCharacter{00A0}{\nobreakspace}
  \DeclareUnicodeCharacter{2500}{\sphinxunichar{2500}}
  \DeclareUnicodeCharacter{2502}{\sphinxunichar{2502}}
  \DeclareUnicodeCharacter{2514}{\sphinxunichar{2514}}
  \DeclareUnicodeCharacter{251C}{\sphinxunichar{251C}}
  \DeclareUnicodeCharacter{2572}{\textbackslash}
 \fi
\fi
\usepackage{cmap}
\usepackage[T1]{fontenc}
\usepackage{amsmath,amssymb,amstext}
\usepackage{babel}
\usepackage{times}
\usepackage[Bjarne]{fncychap}
\usepackage[,numfigreset=1]{sphinx}

\usepackage{geometry}

% Include hyperref last.
\usepackage{hyperref}
% Fix anchor placement for figures with captions.
\usepackage{hypcap}% it must be loaded after hyperref.
% Set up styles of URL: it should be placed after hyperref.
\urlstyle{same}
\addto\captionsenglish{\renewcommand{\contentsname}{UMEP Manual}}

\addto\captionsenglish{\renewcommand{\figurename}{Fig.}}
\addto\captionsenglish{\renewcommand{\tablename}{Table}}
\addto\captionsenglish{\renewcommand{\literalblockname}{Listing}}

\addto\captionsenglish{\renewcommand{\literalblockcontinuedname}{continued from previous page}}
\addto\captionsenglish{\renewcommand{\literalblockcontinuesname}{continues on next page}}

\addto\extrasenglish{\def\pageautorefname{page}}

\setcounter{tocdepth}{1}



\title{UMEP Manual Documentation}
\date{Jun 16, 2018}
\release{}
\author{Fredrik Lindberg, Ting Sun, Sue Grimmond, Yihao Tang}
\newcommand{\sphinxlogo}{\sphinxincludegraphics{UMEP_logo.png}\par}
\renewcommand{\releasename}{}
\makeindex

\begin{document}

\maketitle
\sphinxtableofcontents
\phantomsection\label{\detokenize{index::doc}}


The Urban Multi-scale Environmental Predictor (\sphinxstylestrong{UMEP}) is a climate
service tool, designed for researchers and service providers (e.g.
architects, climatologists, energy, health and urban planners). This
tool can be used for a variety of applications related to outdoor
thermal comfort, urban energy consumption, climate change mitigation
etc. \sphinxstylestrong{UMEP} consists of a coupled modelling system which combines
“state of the art” 1D and 2D models related to the processes essential
for scale independent urban climate estimations.
\begin{wrapfigure}{r}{0pt}
\centering
\noindent\sphinxincludegraphics[scale=1.0]{{Header_umep}.png}
\caption{Source area modelling with UMEP}\label{\detokenize{index:id1}}\end{wrapfigure}

\sphinxstylestrong{UMEP} is a open source model, where users can contribute as well as extend the tool
to improve modelling capabilities. It is freely available for download.
A major feature is the ability for a user to interact with spatial
information to determine model parameters. The spatial data across a
range of scales and sources are accessed through
\sphinxhref{http://qgis.org}{QGIS} - a cross-platform, free, open source
desktop geographic information systems
({\hyperref[\detokenize{Abbreviations::doc}]{\sphinxcrossref{\DUrole{doc,doc,doc}{GIS}}}}) application \textendash{}
that provides data viewing, editing, and analysis capabilities.


\bigskip\hrule\bigskip


\sphinxstylestrong{To downloading instructions and initial settings:}  {\hyperref[\detokenize{Getting_Started:getting-started}]{\sphinxcrossref{\DUrole{std,std-ref}{Getting Started}}}}


\bigskip\hrule\bigskip


\begin{sphinxadmonition}{note}{Note:}
The migration of UMEP into QGIS3 is planned for the Autumn of 2018. In the meantime, use UMEP with the LTR version of QGIS (2.18.x).
\end{sphinxadmonition}

We are keen to get inputs and contributions from others. Plesae contribute in the following ways:
\begin{enumerate}
\item {} 
Submit comments or issues to the \sphinxhref{https://bitbucket.org/fredrik\_ucg/umep/issues}{repository}

\item {} 
Participate in {\hyperref[\detokenize{How_to_Contribute::doc}]{\sphinxcrossref{\DUrole{doc,doc,doc}{coding or adding new features}}}}.

\item {} 
Contribute by writing {\hyperref[\detokenize{Tutorials/Tutorials::doc}]{\sphinxcrossref{\DUrole{doc,doc,doc}{tutorials and guidelines}}}} for other users to give examples of the many application possibilities of UMEP.

\end{enumerate}


\chapter{Introduction}
\label{\detokenize{Introduction:introduction}}\label{\detokenize{Introduction:id1}}\label{\detokenize{Introduction::doc}}
The Urban Multi-scale Environmental Predictor (\sphinxstylestrong{UMEP}) is a climate
services tool, designed for researchers, architects, urban planners,
climatologists, and meteorologists. This tool can be used for a variety
of applications related to outdoor thermal comfort, urban energy
consumption, climate change mitigation etc. \sphinxstylestrong{UMEP} consists of a
coupled modelling system which combines “state of the art” 1D and 2D
models related to the processes essen tial for urban climate
interactions.

\sphinxstylestrong{UMEP} is a {\hyperref[\detokenize{People_Involved___Acknowledgements::doc}]{\sphinxcrossref{\DUrole{doc,doc,doc}{community}}}} open
source model that users can contribute to improve and extend the
modelling capabilities. It is free to download. A major feature is the
ability for a user to interact with spatial information to determine
model parameters. The spatial data across a range of scales and sources
are accessed through \sphinxstylestrong{QGIS} - a cross-platform, free, open source
desktop geographic information systems
({\hyperref[\detokenize{Abbreviations::doc}]{\sphinxcrossref{\DUrole{doc,doc,doc}{GIS}}}}) application \textendash{}
that provides data viewing, editing, and analysis capabilities.

This software is in continuous development. There are two types of
releases:
\begin{enumerate}
\item {} 
\sphinxstylestrong{Long term release} - this may be obtained from the QGIS plugin
manager {\hyperref[\detokenize{Getting_Started::doc}]{\sphinxcrossref{\DUrole{doc,doc,doc}{(see details)}}}}.

\item {} 
\sphinxstylestrong{Current development version} - this can be obtained from the plugin
\sphinxhref{http://www.bitbucket.org/fredrik\_ucg/umep}{repository}. This
version you need to manually install yourself {\hyperref[\detokenize{Getting_Started::doc}]{\sphinxcrossref{\DUrole{doc,doc,doc}{(see details)}}}}.

\end{enumerate}

The UMEP plugin consist of three
parts; a pre-processor, a processor and a post-processor. The
pre-processor prepares spatial and meteorological data as inputs to the
modelling system. The processor includes all the main models for the
main calculations. To provide initial “quick looks” the post-processor
will enable results to be plotted, statistics calculated etc. based on
the model output. For more information on the content and archetecture,
see {\hyperref[\detokenize{Introduction:pluginarchitecture}]{\sphinxcrossref{\DUrole{std,std-ref,std,std-ref}{Plugin Architecture}}}}.

Detailed information on how to install can be found {\hyperref[\detokenize{Getting_Started::doc}]{\sphinxcrossref{\DUrole{doc,doc,doc}{here}}}}.

Information on version history can be found \sphinxhref{https://bitbucket.org/fredrik\_ucg/umep/commits/branch/master}{here}.


\section{UMEP: How to Cite}
\label{\detokenize{Introduction:umep-how-to-cite}}
Please use the reference below when UMEP is used:
\begin{quote}

{\color{red}\bfseries{}*}Lindberg F, Grimmond CSB, Gabey A, Huang B, Kent CW, Sun T, Theeuwes N, Järvi L, Ward H, Capel-
Timms I, Chang YY, Jonsson P, Krave N, Liu D, Meyer D, Olofson F, Tan JG, Wästberg D, Xue L,
Zhang Z (2017) Urban Multi-scale Environmental Predictor (UMEP) - An integrated tool for city-based
climate services. Environmen tal Modelling and Software.99, 70-87 \sphinxurl{https://doi.org/10.1016/j.envsoft.2017.09.020} *
\end{quote}

The manual should be cited as:
\begin{quote}

\sphinxstyleemphasis{Lindberg F, Grimmond CSB, A Gabey, L Jarvi, CW Kent, N Krave, T Sun, N Wallenberg, HC Ward (2018)
Urban Multi-scale Environmental Predictor (UMEP) Manual. https://umep-docs.readthedocs.io/
University of Reading UK, University of Gothenburg Sweden, SIMS China}
\end{quote}


\section{Plugin Architecture}
\label{\detokenize{Introduction:plugin-architecture}}\label{\detokenize{Introduction:pluginarchitecture}}
The UMEP plugin consist of three different parts; a pre-processor, a
processor and a post-processor. The pre-processor is used to prepare
spatial and meteorological data as inputs to the modelling system. The
processor includes all the main models for the main calculations. To
provide initial “quick looks” the post-processor will be enable results
to be plotted, statistics calculated etc. based on the model output.

\begin{sphinxadmonition}{note}{Note:}
It is strongly recommended to transform all geodatasets into the same projected coordinate system (CRS) before any processing starts.
\end{sphinxadmonition}


\subsection{Pre-Processor}
\label{\detokenize{Introduction:pre-processor}}
\sphinxstylestrong{Meteorological Data}


\begin{savenotes}\sphinxattablestart
\centering
\begin{tabular}[t]{|\X{35}{100}|\X{65}{100}|}
\hline

{\hyperref[\detokenize{pre-processor/Meteorological Data MetPreprocessor:metpreprocessor}]{\sphinxcrossref{\DUrole{std,std-ref,std,std-ref}{Prepare Existing Data}}}}
&
Transforms meteorological data into UMEP format
\\
\hline
{\hyperref[\detokenize{pre-processor/Meteorological Data Download data (WATCH):watch}]{\sphinxcrossref{\DUrole{std,std-ref,std,std-ref}{Download data (WATCH)}}}}
&
Prepare meteorological dataset from \sphinxhref{http://www.eu-watch.org/data\_availability}{WATCH}
\\
\hline
\end{tabular}
\par
\sphinxattableend\end{savenotes}

\sphinxstylestrong{Spatial Data}


\begin{savenotes}\sphinxattablestart
\centering
\begin{tabular}[t]{|\X{35}{100}|\X{65}{100}|}
\hline

{\hyperref[\detokenize{pre-processor/Spatial Data Spatial Data Downloader:spatialdatadownloader}]{\sphinxcrossref{\DUrole{std,std-ref,std,std-ref}{Spatial Data Downloader}}}}
&
Plugin for retrieving geodata from online services suitable for various UMEP related tools
\\
\hline
{\hyperref[\detokenize{pre-processor/Spatial Data DSM Generator:dsmgenerator}]{\sphinxcrossref{\DUrole{std,std-ref,std,std-ref}{DSM generator}}}}
&
Creation/manipulation of a DSM based on user-specified building footprint vector data and/or \sphinxhref{http://www.openstreetmap.org}{Open Street Map} data (if available)
\\
\hline
{\hyperref[\detokenize{pre-processor/Spatial Data Tree Generator:treegenerator}]{\sphinxcrossref{\DUrole{std,std-ref,std,std-ref}{Tree generator}}}}
&
Creation/manipulation of vegetation input data
\\
\hline
{\hyperref[\detokenize{pre-processor/Spatial Data LCZ Converter:lczconverter}]{\sphinxcrossref{\DUrole{std,std-ref,std,std-ref}{LCZ Converter}}}}
&
Conversion from Local Climate Zones (LCZs) in the WUDAPT database into SUEWS input data
\\
\hline
\end{tabular}
\par
\sphinxattableend\end{savenotes}

\sphinxstylestrong{Urban geometry}


\begin{savenotes}\sphinxattablestart
\centering
\begin{tabular}[t]{|\X{35}{100}|\X{65}{100}|}
\hline

{\hyperref[\detokenize{pre-processor/Urban Geometry Sky View Factor Calculator:skyviewfactorcalculator}]{\sphinxcrossref{\DUrole{std,std-ref,std,std-ref}{Sky View Factor}}}}
&
Calculation of continuous maps of Sky View Factors (SVF) based on high resolution digital surface models (DSM). \sphinxstyleemphasis{Solar access, urban heat island}
\\
\hline
{\hyperref[\detokenize{pre-processor/Urban Geometry Wall Height and Aspect:wallheightandaspect}]{\sphinxcrossref{\DUrole{std,std-ref,std,std-ref}{Wall Height and Aspect}}}}
&
Calculation of height and aspect of building walls based on a DSM
\\
\hline
\end{tabular}
\par
\sphinxattableend\end{savenotes}

\sphinxstylestrong{Urban land cover}


\begin{savenotes}\sphinxattablestart
\centering
\begin{tabular}[t]{|\X{35}{100}|\X{65}{100}|}
\hline

{\hyperref[\detokenize{pre-processor/Urban Land Cover Land Cover Reclassifier:landcoverreclassifier}]{\sphinxcrossref{\DUrole{std,std-ref,std,std-ref}{Land Cover Reclassifier}}}}
&
Reclassifies a grid into UMEP format land cover grid. \sphinxstyleemphasis{Land surface models}
\\
\hline
{\hyperref[\detokenize{pre-processor/Urban Land Cover Land Cover Fraction (Point):landcoverfraction-point}]{\sphinxcrossref{\DUrole{std,std-ref,std,std-ref}{Land Cover Fraction (Point)}}}}
&
Land cover fractions estimates from a land cover grid based on a specific point in space
\\
\hline
{\hyperref[\detokenize{pre-processor/Urban Land Cover Land Cover Fraction (Grid):landcoverfraction-grid}]{\sphinxcrossref{\DUrole{std,std-ref,std,std-ref}{Land Cover Fraction (Grid)}}}}
&
Land cover fractions estimates from a land cover grid based on a polygon grid
\\
\hline
\end{tabular}
\par
\sphinxattableend\end{savenotes}

\sphinxstylestrong{Urban Morphology}


\begin{savenotes}\sphinxattablestart
\centering
\begin{tabular}[t]{|\X{35}{100}|\X{65}{100}|}
\hline

{\hyperref[\detokenize{pre-processor/Urban Morphology Morphometric Calculator (Point):morphometriccalculator-point}]{\sphinxcrossref{\DUrole{std,std-ref,std,std-ref}{Morphometric Calculator (Point)}}}}
&
Morphometric parameters from a DSM based on a specific point in space
\\
\hline
\sphinxhref{MorphometricCalculator(Grid)}{Morphometric Calculator (Grid)}
&
Morphometric parameters estimated from a DSM based on a polygon grid
\\
\hline
{\hyperref[\detokenize{pre-processor/Urban Morphology Source Area (Point):sourcearea-point}]{\sphinxcrossref{\DUrole{std,std-ref,std,std-ref}{Source Area Model (Point)}}}}
&
Source area calculated from a DSM based on a specific point in space. \sphinxstyleemphasis{Interpretation of observations}
\\
\hline
\end{tabular}
\par
\sphinxattableend\end{savenotes}

\sphinxstylestrong{Other}


\begin{savenotes}\sphinxattablestart
\centering
\begin{tabular}[t]{|\X{35}{100}|\X{65}{100}|}
\hline

{\hyperref[\detokenize{pre-processor/SUEWS Prepare:suewsprepare}]{\sphinxcrossref{\DUrole{std,std-ref,std,std-ref}{SUEWS Prepare}}}}
&
Preprocessing and preparing input data for the SUEWS model
\\
\hline
\end{tabular}
\par
\sphinxattableend\end{savenotes}


\subsection{Processor}
\label{\detokenize{Introduction:processor}}
\sphinxstylestrong{Outdoor Thermal Comfort}


\begin{savenotes}\sphinxattablestart
\centering
\begin{tabular}[t]{|\X{35}{100}|\X{65}{100}|}
\hline

Comfort Index (PET/UTCI) (planned)
&
Spatial variations of thermal comfort indices in complex urban environments
\\
\hline
{\hyperref[\detokenize{processor/Outdoor Thermal Comfort SOLWEIG:solweig}]{\sphinxcrossref{\DUrole{std,std-ref,std,std-ref}{Mean Radiant Temperature (SOLWEIG)}}}}
&
Spatial variations of T$_{\text{mrt}}$ in complex urban environments. \sphinxstyleemphasis{Human Health: Outdoor thermal comfort; Park planning; Heat/Health warning; Daily Operations: visitors to parks}
\\
\hline
Pedestrian Wind Speed (planned)
&
Spatial variations of pedestrian wind speed in complex urban environments
\\
\hline
{\hyperref[\detokenize{processor/Outdoor Thermal Comfort ExtremeFinder:extremefinder}]{\sphinxcrossref{\DUrole{std,std-ref,std,std-ref}{ExtremeFinder}}}}
&
Identify heat waves and cold waves for a certain location. \sphinxstyleemphasis{Human Health: Outdoor thermal comfort; Daily City Operations: Energy use; Gas consumption}
\\
\hline
\end{tabular}
\par
\sphinxattableend\end{savenotes}

\sphinxstylestrong{Urban Energy Balance}


\begin{savenotes}\sphinxattablestart
\centering
\begin{tabular}[t]{|\X{35}{100}|\X{65}{100}|}
\hline

{\hyperref[\detokenize{processor/Urban Energy Balance LQ:lqf}]{\sphinxcrossref{\DUrole{std,std-ref,std,std-ref}{LQF}}}}
&
Spatial variations anthropogenic heat release for urban areas
\\
\hline
{\hyperref[\detokenize{processor/Urban Energy Balance GQ:gqf}]{\sphinxcrossref{\DUrole{std,std-ref,std,std-ref}{GQF}}}}
&
Anthropogenic Heat (Q$_{\text{F}}$). \sphinxstyleemphasis{Daily City Operations: Energy use; Gas consumption; Traffic heat loads}
\\
\hline
{\hyperref[\detokenize{processor/Urban Energy Balance Urban Energy Balance (SUEWS, simple):suewssimple}]{\sphinxcrossref{\DUrole{std,std-ref,std,std-ref}{SUEWS (Simple)}}}}
&
Urban Energy and Water Balance. \sphinxstyleemphasis{Disaster Risk Management: Drought, Heat; Environment evaluation for construction, Water Management, Green infrastructure}
\\
\hline
{\hyperref[\detokenize{processor/Urban Energy Balance Urban Energy Balance (SUEWS.BLUEWS, advanced):suewsadvanced}]{\sphinxcrossref{\DUrole{std,std-ref,std,std-ref}{SUEWS (Advanced)}}}}
&
Urban Energy and Water Balance. \sphinxstyleemphasis{Disaster Risk Management: Drought, Heat; Environment evaluation for construction, Water Management, Green infrastructure}
\\
\hline
\end{tabular}
\par
\sphinxattableend\end{savenotes}

\sphinxstylestrong{Solar Radiation}


\begin{savenotes}\sphinxattablestart
\centering
\begin{tabular}[t]{|\X{35}{100}|\X{65}{100}|}
\hline

{\hyperref[\detokenize{Tutorials/SEBE:sebe}]{\sphinxcrossref{\DUrole{std,std-ref,std,std-ref}{Solar Energy on Building Envelopes (SEBE)}}}}
&
Solar irradiance on building roofs and walls in urban environments. \sphinxstyleemphasis{Economy and planning: Energy production, resource planning}
\\
\hline
{\hyperref[\detokenize{processor/Solar Radiation Daily Shadow Pattern:dailyshadowpattern}]{\sphinxcrossref{\DUrole{std,std-ref,std,std-ref}{Daily Shadow Patterns}}}}
&
Shadow patterns on a DSM and CDSM. \sphinxstyleemphasis{Economy and planning: Resource planning Human Health: Outdoor thermal comfort; Park planning}
\\
\hline
\end{tabular}
\par
\sphinxattableend\end{savenotes}


\subsection{Post-Processor}
\label{\detokenize{Introduction:post-processor}}
\sphinxstylestrong{Solar Radiation}


\begin{savenotes}\sphinxattablestart
\centering
\begin{tabular}[t]{|\X{35}{100}|\X{65}{100}|}
\hline

{\hyperref[\detokenize{post_processor/Solar Radiation SEBE (Visualisation):sebevisualisation}]{\sphinxcrossref{\DUrole{std,std-ref,std,std-ref}{SEBE Visualisation}}}}
&
Plugin to visualse output irradiation from SEBE on building roofs, walls and ground
\\
\hline
\end{tabular}
\par
\sphinxattableend\end{savenotes}

\sphinxstylestrong{Outdoor Thermal Comfort}


\begin{savenotes}\sphinxattablestart
\centering
\begin{tabular}[t]{|\X{35}{100}|\X{65}{100}|}
\hline

{\hyperref[\detokenize{post_processor/Outdoor Thermal Comfort SOLWEIG Analyzer:solweiganalyzer}]{\sphinxcrossref{\DUrole{std,std-ref,std,std-ref}{SOLWEIG analyzer}}}}
&
Plugin for plotting, statistical analysis and post-processing of model results from SOLWEIG
\\
\hline
\end{tabular}
\par
\sphinxattableend\end{savenotes}

\sphinxstylestrong{Urban Energy Balance}


\begin{savenotes}\sphinxattablestart
\centering
\begin{tabular}[t]{|\X{35}{100}|\X{65}{100}|}
\hline

{\hyperref[\detokenize{post_processor/Urban Energy Balance SUEWS Analyser:suewsanalyser}]{\sphinxcrossref{\DUrole{std,std-ref,std,std-ref}{SUEWS analyser}}}}
&
Plugin for plotting and statistical analysis of model results from SUEWS simple and SUEWS advanced
\\
\hline
\end{tabular}
\par
\sphinxattableend\end{savenotes}

\sphinxstylestrong{Benchmark}


\begin{savenotes}\sphinxattablestart
\centering
\begin{tabular}[t]{|\X{35}{100}|\X{65}{100}|}
\hline

{\hyperref[\detokenize{post_processor/Benchmark System:benchmark}]{\sphinxcrossref{\DUrole{std,std-ref,std,std-ref}{Benchmark System}}}}
&
For statistical analysis of model results, such as SUEWS
\\
\hline
\end{tabular}
\par
\sphinxattableend\end{savenotes}


\section{Tool Applications}
\label{\detokenize{Introduction:tool-applications}}\label{\detokenize{Introduction:toolapplications}}
A key element of UMEP is to facilitate the preparation of input data
needed for City-Based Climate Services (CBCS). UMEP provides both
guidance and tools that enable data preparation and manipulation. This
is particularly important as many end-users have familiarity with some,
but not the full spectrum, of the data needed for applications. Below
you can find some examples on applications and workflows for the
modelling procedure in UMEP and what tools that are connected to each
other.

\begin{figure}[htbp]
\centering
\capstart

\noindent\sphinxincludegraphics{{SUEWSworkflow}.png}
\caption{Workflow and geodata used for analysing urban energy balance
using the SUEWS model. Bold outlined boxes are mandatory items.
Yellow, orange and red indicates pre-processor, processor and
post-processor tools, respectively. Grey boxes indicate geodatasets.}\label{\detokenize{Introduction:id4}}\end{figure}

\begin{figure}[htbp]
\centering
\capstart

\noindent\sphinxincludegraphics{{SOLWEIGworkflow}.png}
\caption{Workflow and geodata used for analysing mean radiant
temperature using the SOLWEIG model. Bold outlines are mandatory
items. Yellow, orange and red indicates pre-processor, processor and
post-processor tools, respectively. Grey boxes indicate geodatasets.}\label{\detokenize{Introduction:id5}}\end{figure}

Other application examples can be found
\sphinxhref{http://www.urban-climate.net/umep/Example\_Applications}{here}.


\subsection{Evaluation and application studies}
\label{\detokenize{Introduction:evaluation-and-application-studies}}\begin{itemize}
\item {} \begin{description}
\item[{Mean Radiant Temperature (\sphinxhref{http://urban-climate.net/umep/SOLWEIG}{SOLWEIG})}] \leavevmode\begin{itemize}
\item {} 
References: Evaluation

\end{itemize}


\begin{savenotes}\sphinxattablestart
\centering
\begin{tabular}[t]{|\X{50}{100}|\X{50}{100}|}
\hline
\sphinxstyletheadfamily 
Spatial reference
&\sphinxstyletheadfamily 
Reference
\\
\hline
Gothenburg, Sweden
&
\sphinxhref{http://link.springer.com/article/10.1007/s00484-008-0162-7}{Lindberg et al. (2008)}
\\
\hline
Gothenburg, Sweden
&
\sphinxhref{http://link.springer.com/article/10.1007/s00704-010-0382-8}{Lindberg and Grimmond (2011)}
\\
\hline
Freiburg, Germany
&
\sphinxhref{http://link.springer.com/article/10.1007/s00704-010-0382-8}{Lindberg and Grimmond (2011)}
\\
\hline
Kassel, Germany
&
\sphinxhref{http://link.springer.com/article/10.1007/s00704-010-0382-8}{Lindberg and Grimmond (2011)}
\\
\hline
London, UK
&
\sphinxhref{http://link.springer.com/article/10.1007/s00484-016-1135-x}{Lindberg et al. (2016)}
\\
\hline
Hong Kong, China
&
\sphinxhref{http://www.sciencedirect.com/science/article/pii/S0378778815300645}{Lau et al. (2016)}
\\
\hline
Shanghai, China
&
\sphinxhref{http://www.sciencedirect.com/science/article/pii/S037877881630812X}{Chen et al. (2016)}
\\
\hline
\end{tabular}
\par
\sphinxattableend\end{savenotes}
\begin{itemize}
\item {} 
References: Application

\end{itemize}


\begin{savenotes}\sphinxattablestart
\centering
\begin{tabular}[t]{|\X{50}{100}|\X{50}{100}|}
\hline
\sphinxstyletheadfamily 
Spatial reference
&\sphinxstyletheadfamily 
Reference
\\
\hline
London, UK
&
\sphinxhref{http://link.springer.com/article/10.1007/s11252-011-0184-5}{Lindberg and Grimmond (2011)}
\\
\hline
Gothenburg, Sweden
&
\sphinxhref{http://link.springer.com/article/10.1007/s00484-013-0638-y}{Lindberg et al. (2013)}
\\
\hline
Stockholm, Sweden
&
\sphinxhref{http://link.springer.com/article/10.1007/s00484-013-0638-y}{Lindberg et al. (2013)}
\\
\hline
Luleå, Sweden
&
\sphinxhref{http://link.springer.com/article/10.1007/s00484-013-0638-y}{Lindberg et al. (2013)}
\\
\hline
Adelaide, Australia
&
\sphinxhref{http://www.sciencedirect.com/science/article/pii/S1618866716301297}{Thom et al. (2016)}
\\
\hline
Berlin, Germany
&
\sphinxhref{http://www.sciencedirect.com/science/article/pii/S2212095515300341}{Jänicke et al. (2015)}
\\
\hline
Gothenburg, Sweden
&
\sphinxhref{http://link.springer.com/article/10.1007/s00484-014-0898-1}{Lau et al. (2014)}
\\
\hline
Frankfurt, Germany
&
\sphinxhref{http://link.springer.com/article/10.1007/s00484-014-0898-1}{Lau et al. (2014)}
\\
\hline
Porto, Portugal
&
\sphinxhref{http://link.springer.com/article/10.1007/s00484-014-0898-1}{Lau et al. (2014)}
\\
\hline
Gothenburg, Sweden
&
\sphinxhref{http://www.sciencedirect.com/science/article/pii/S2210670716300579}{Lindberg et al. (2016)}
\\
\hline
Gothenburg, Sweden
&
\sphinxhref{http://onlinelibrary.wiley.com/doi/10.1002/joc.2231/abstract}{Thorsson et al. (2011)}
\\
\hline
Stockholm, Sweden
&
\sphinxhref{http://www.sciencedirect.com/science/article/pii/S2212095514000054}{Thorsson et al. (2014)}
\\
\hline
\end{tabular}
\par
\sphinxattableend\end{savenotes}

\end{description}

\item {} \begin{description}
\item[{Pedestrian Wind Speed}] \leavevmode\begin{itemize}
\item {} 
References: Evaluation

\end{itemize}


\begin{savenotes}\sphinxattablestart
\centering
\begin{tabular}[t]{|\X{50}{100}|\X{50}{100}|}
\hline
\sphinxstyletheadfamily 
Spatial reference
&\sphinxstyletheadfamily 
Reference
\\
\hline
Global
&
\sphinxhref{http://link.springer.com/article/10.1007/s00704-015-1405-2}{Johansson et al. (2015)}
\\
\hline
\end{tabular}
\par
\sphinxattableend\end{savenotes}

\end{description}

\item {} \begin{description}
\item[{Anthropogenic Heat (Qf) (LUCY)}] \leavevmode\begin{itemize}
\item {} 
References: Evaluation

\end{itemize}


\begin{savenotes}\sphinxattablestart
\centering
\begin{tabular}[t]{|\X{50}{100}|\X{50}{100}|}
\hline
\sphinxstyletheadfamily 
Spatial reference
&\sphinxstyletheadfamily 
Reference
\\
\hline
Global
&
\sphinxhref{http://onlinelibrary.wiley.com/doi/10.1002/joc.2210/abstract}{Allen et al. (2011)}
\\
\hline
\end{tabular}
\par
\sphinxattableend\end{savenotes}
\begin{itemize}
\item {} 
References: Application

\end{itemize}


\begin{savenotes}\sphinxattablestart
\centering
\begin{tabular}[t]{|\X{50}{100}|\X{50}{100}|}
\hline
\sphinxstyletheadfamily 
Spatial reference
&\sphinxstyletheadfamily 
Reference
\\
\hline
Europe
&
\sphinxhref{http://www.sciencedirect.com/science/article/pii/S2212095513000059}{Lindberg et al. (2013)}
\\
\hline
\end{tabular}
\par
\sphinxattableend\end{savenotes}

\end{description}

\item {} \begin{description}
\item[{Urban Energy and Water Balance (\sphinxhref{http://urban-climate.net/umep/SUEWS}{SUEWS})}] \leavevmode\begin{itemize}
\item {} 
References: Evaluation

\end{itemize}


\begin{savenotes}\sphinxattablestart
\centering
\begin{tabular}[t]{|\X{50}{100}|\X{50}{100}|}
\hline
\sphinxstyletheadfamily 
Spatial reference
&\sphinxstyletheadfamily 
Reference
\\
\hline
Vancouver, Canada
&
\sphinxhref{http://www.sciencedirect.com/science/article/pii/S0022169411006937}{Järvi et al. (2011)}
\\
\hline
Los Angeles, USA
&
\sphinxhref{http://www.sciencedirect.com/science/article/pii/S0022169411006937}{Järvi et al. (2011)}
\\
\hline
Helsinki, Finland
&
\sphinxhref{http://www.geosci-model-dev.net/7/1691/2014/}{Järvi et al. (2014)}
\\
\hline
Montreal, Canada
&
\sphinxhref{http://www.geosci-model-dev.net/7/1691/2014/}{Järvi et al. (2014)}
\\
\hline
Dublin, Ireland
&
\sphinxhref{http://dx.doi.org/10.1016/j.uclim.2015.05.001}{Alexander et al. (2015)}
\\
\hline
Swindon, UK
&
\sphinxhref{http://www.sciencedirect.com/science/article/pii/S2212095516300256}{Ward et al. (2016)}
\\
\hline
London, UK
&
\sphinxhref{http://www.sciencedirect.com/science/article/pii/S2212095516300256}{Ward et al. (2016)}
\\
\hline
Helsinki, Finlamd
&
\sphinxhref{http://onlinelibrary.wiley.com/doi/10.1002/qj.2659/full}{Karsisto et al. (2016)}
\\
\hline
Shanghai, China
&
(Radiation) \sphinxhref{http://journals.ametsoc.org/doi/abs/10.1175/JAMC-D-16-0082.1}{Ao et al. (2016)}
\\
\hline
Sacramento, US
&
\sphinxhref{http://www.sciencedirect.com/science/article/pii/S2212095514000856}{Onomura et al. (2015)}
\\
\hline
\end{tabular}
\par
\sphinxattableend\end{savenotes}
\begin{itemize}
\item {} 
References: Application

\end{itemize}


\begin{savenotes}\sphinxattablestart
\centering
\begin{tabular}[t]{|\X{50}{100}|\X{50}{100}|}
\hline
\sphinxstyletheadfamily 
Spatial reference
&\sphinxstyletheadfamily 
Reference
\\
\hline
London, UK
&
Ward and Grimmond (2017)
\\
\hline
Helsinki, Finland
&
\sphinxhref{http://www.sciencedirect.com/science/article/pii/S221209551500019X}{Nordbo et al. (2015)}
\\
\hline
Dublin, Ireland
&
\sphinxhref{http://www.sciencedirect.com/science/article/pii/S0169204616000128}{Alexander et al. (2016)}
\\
\hline
Porto, Portugal
&
\sphinxhref{http://www.sciencedirect.com/science/article/pii/S0048969716312086}{Rafael et al. (2016)}
\\
\hline
\end{tabular}
\par
\sphinxattableend\end{savenotes}

\end{description}

\item {} \begin{description}
\item[{Solar Energy on Building Envelopes (SEBE)}] \leavevmode\begin{itemize}
\item {} 
References: Evaluation

\end{itemize}


\begin{savenotes}\sphinxattablestart
\centering
\begin{tabular}[t]{|\X{50}{100}|\X{50}{100}|}
\hline
\sphinxstyletheadfamily 
Spatial reference
&\sphinxstyletheadfamily 
Reference
\\
\hline
Gothenburg, Sweden
&
\sphinxhref{http://www.sciencedirect.com/science/article/pii/S0038092X15001164}{Lindberg et al. (2015)}
\\
\hline
\end{tabular}
\par
\sphinxattableend\end{savenotes}
\begin{itemize}
\item {} 
References: Application

\end{itemize}


\begin{savenotes}\sphinxattablestart
\centering
\begin{tabular}[t]{|\X{50}{100}|\X{50}{100}|}
\hline
\sphinxstyletheadfamily 
Spatial reference
&\sphinxstyletheadfamily 
Reference
\\
\hline
Dar es Salam, Tanzania
&
\sphinxhref{http://www.sciencedirect.com/science/article/pii/S2210670716304267}{Lau et al. (2016)}
\\
\hline
Stockholm, Sweden
&
\sphinxhref{http://www.energiradgivningen.se/sites/all/themes/energi/map/index.html}{Online mapping service (in Swedish)}
\\
\hline
Uppsala, Sweden
&
\sphinxhref{http://ec2-54-77-203-12.eu-west-1.compute.amazonaws.com/uppsala/}{Online mapping service (in Swedish)}
\\
\hline
Gothenburg, Sweden
&
\sphinxhref{http://www.goteborgenergi.se/Privat/Projekt\_och\_etableringar/Fornybar\_energi/Solceller/Solkartan/}{Online mapping service (in Swedish)}
\\
\hline
Eskilstuna, Sweden
&
\sphinxhref{http://karta.eskilstuna.se/eskilstunakartan/x/\#maps/1069}{Online mapping service (in Swedish)}
\\
\hline
\end{tabular}
\par
\sphinxattableend\end{savenotes}

\end{description}

\item {} \begin{description}
\item[{Daily Shadow Patterns}] \leavevmode\begin{itemize}
\item {} 
References: Evaluation

\end{itemize}


\begin{savenotes}\sphinxattablestart
\centering
\begin{tabular}[t]{|\X{50}{100}|\X{50}{100}|}
\hline
\sphinxstyletheadfamily 
Spatial reference
&\sphinxstyletheadfamily 
Reference
\\
\hline
Borås, Sweden
&
\sphinxhref{http://link.springer.com/article/10.1007/s00704-015-1508-9}{Hu et al. (2015)}
\\
\hline
\end{tabular}
\par
\sphinxattableend\end{savenotes}
\begin{itemize}
\item {} 
References: Application

\end{itemize}


\begin{savenotes}\sphinxattablestart
\centering
\begin{tabular}[t]{|\X{50}{100}|\X{50}{100}|}
\hline
\sphinxstyletheadfamily 
Spatial reference
&\sphinxstyletheadfamily 
Reference
\\
\hline
London, UK
&
\sphinxhref{http://www.sciencedirect.com/science/article/pii/S221209551400090X}{Lindberg et al. (2015)}
\\
\hline
Gothenburg, Sweden
&
\sphinxhref{http://www.sciencedirect.com/science/article/pii/S0266352X11000693}{Lindberg et al. (2011)}
\\
\hline
\end{tabular}
\par
\sphinxattableend\end{savenotes}

\end{description}

\end{itemize}


\chapter{Getting Started}
\label{\detokenize{Getting_Started:getting-started}}\label{\detokenize{Getting_Started:id1}}\label{\detokenize{Getting_Started::doc}}\begin{description}
\item[{UMEP is developed as a plugin for \sphinxhref{http://www.qgis.org}{QGIS}. Two different versions are available:}] \leavevmode\begin{itemize}
\item {} 
\sphinxstyleemphasis{Long term release} - This version is recommended for most users.

\item {} 
\sphinxstyleemphasis{Development release} - This version is for testing. Could be unstable.

\end{itemize}

\end{description}

For a more detailed description including how to install QGIS on a Windows PC (see below) or watch this instruction \sphinxhref{https://www.youtube.com/watch?v=ZEw\_DVl772Q}{video}. You can find more introductory videos on how to use UMEP on our \sphinxhref{https://www.youtube.com/channel/UCTPkXncD3ghb5ZTdZe\_u7gA}{YouTube-channel}.
UMEP has been developed using Python 2.7.x and QGIS 2.x in a Windows environment. Since QGIS is a multi-platform software system it works on other platforms as well. UMEP is still under development so there may be missing documentation and instability. Please report any issues to the \sphinxhref{https://bitbucket.org/fredrik\_ucg/umep}{code repository}. Also, have a look in {\hyperref[\detokenize{FAQ::doc}]{\sphinxcrossref{\DUrole{doc,doc,doc}{FAQ}}}} for further installation tips and issues.


\section{Installation}
\label{\detokenize{Getting_Started:installation}}

\subsection{Recommended Installation Steps of QGIS on Windows}
\label{\detokenize{Getting_Started:recommended-installation-steps-of-qgis-on-windows}}
\begin{sphinxadmonition}{note}{Note:}
The migration of UMEP into QGIS3 is planned for the Autumn of 2018. In the meantime, use UMEP with the LTR version of QGIS (2.18).
\end{sphinxadmonition}
\begin{enumerate}
\item {} 
Visit \sphinxhref{http://www.qgis.org}{QGIS} and go to the download page. Preferably, choose the \sphinxhref{http://download.osgeo.org/osgeo4w/osgeo4w-setup-x86\_64.exe}{OSGEO4W Network Installer (64-bit)}, start the installation and choose \sphinxstyleemphasis{installation (64-bit) For Advanced Users}.

\item {} 
\sphinxstylestrong{To install the latest version (3.x)}, start the installation and choose \sphinxstyleemphasis{Express Desktop Install}.

\item {} 
\sphinxstylestrong{To install the} {\hyperref[\detokenize{Abbreviations::doc}]{\sphinxcrossref{\DUrole{doc,doc,doc}{LTR}}}}, start the installation and choose \sphinxstyleemphasis{Advanced Install}. Click through to \sphinxstyleemphasis{Select Packages} and select \sphinxstyleemphasis{qgis-ltr-full}.

\end{enumerate}

Visit \sphinxhref{http://www.qgis.org}{www.qgis.org} for installation on other operating systems.


\subsection{Download and installation of the UMEP-plugin from within QGIS}
\label{\detokenize{Getting_Started:download-and-installation-of-the-umep-plugin-from-within-qgis}}
\sphinxstylestrong{Long-term release (Recommended)}
\begin{enumerate}
\item {} 
Start \sphinxstylestrong{QGIS}

\item {} 
Go to: \sphinxstyleemphasis{Plugins -\textgreater{} Manage and Install Plugins…}

\item {} 
Search for \sphinxstylestrong{UMEP}

\item {} 
Click \sphinxstylestrong{Install Plugin} (or \sphinxstylestrong{Upgrade} if already have an older version installed from before).

\end{enumerate}

\sphinxstylestrong{Development release (unstable)}
\begin{enumerate}
\item {} 
If you have an installed version of UMEP in your QGIS, uninstall it by going to “Plugins -\textgreater{} Manage and Install Plugins -\textgreater{} Installed -\textgreater{} UMEP” and click \sphinxstylestrong{Uninstall plugin}

\item {} 
To download UMEP from the repository click this \sphinxhref{https://bitbucket.org/fredrik\_ucg/umep/downloads}{link} and download repository

\item {} 
Close QGIS if open

\item {} 
Extract the downloaded zip archive into the folder \sphinxstylestrong{C:\textbackslash{}Users\textbackslash{}.qgis2\textbackslash{}python\textbackslash{}plugins}. If the folder \sphinxstylestrong{plugins} does not exist, install any plugin using \sphinxstyleemphasis{Plugins -\textgreater{} Manage and Install Plugins} and the folder should appear.

\item {} 
Rename the extracted folder to \sphinxstylestrong{UMEP}

\item {} 
Start QGIS. The UMEP plugin should be visible in the QGIS toolbar. If not, go to “Plugins -\textgreater{} Manage and Install Plugins -\textgreater{} All” and search for UMEP. Make sure that you also tick in the box \sphinxstyleemphasis{Show also experimental plugins} in the “Settings”-tab.

\end{enumerate}

Test \sphinxhref{https://bitbucket.org/fredrik\_ucg/umep/downloads/testdata\_UMEP.zip}{datasets} and {\hyperref[\detokenize{Tutorials/Tutorials:tutorials}]{\sphinxcrossref{\DUrole{std,std-ref,std,std-ref}{tutorials}}}} are available to try some of the tools out.


\section{Adding missing Python libraries and other OSGeo functionalities}
\label{\detokenize{Getting_Started:adding-missing-python-libraries-and-other-osgeo-functionalities}}\label{\detokenize{Getting_Started:python-libraries}}
Some of the plugins in the UMEP tool, for example The WATCH data plugin,
requires some Python libraries such as \sphinxstylestrong{pandas} and \sphinxstylestrong{scipy} that
might not been included when you installed QGIS. If so, it is necessary
to install them to make this plugin work. Below are instructions on how
to this for different operation systems. The same procedures can also be
used to obtain other tools and functionalities from the OSGeo
repository. \sphinxstylestrong{Note}: In order for scipy to work you will also need to
make sure \sphinxstyleemphasis{pillow} is installed.
\begin{itemize}
\item {} 
\sphinxstylestrong{Operating System and Installation instructions}:
\begin{enumerate}
\item {} \begin{description}
\item[{Linux}] \leavevmode\begin{itemize}
\item {} 
Linux comes with its own Python installation which QGIS makes use of. This makes it possible to directly use \sphinxstylestrong{pip} (an installation tool of Python libraries) to add missing libraries. Simply open a terminal window and type \sphinxstyleemphasis{sudo pip install pandas} if you want to install this library. In order to install pip open a terminal and type: \sphinxstyleemphasis{sudo easy\_install pip}. You might need to restart QGIS to get it to work.

\item {} 
Or refer to \sphinxhref{http://docs.qgis.org/testing/en/docs/pyqgis\_developer\_cookbook/intro.html\#the-startup-py-file}{startup.py} to modify start up file of QGIS by including the paths to \sphinxstylestrong{pandas} and \sphinxstylestrong{scipy}. An example of startup.py may look like:

\fvset{hllines={, ,}}%
\begin{sphinxVerbatim}[commandchars=\\\{\}]
\PYG{k+kn}{import} \PYG{n+nn}{sys}
\PYG{n}{sys}\PYG{o}{.}\PYG{n}{path}\PYG{o}{.}\PYG{n}{insert}\PYG{p}{(}\PYG{l+m+mi}{1}\PYG{p}{,}\PYG{l+s+s1}{\PYGZsq{}}\PYG{l+s+s1}{/usr/local/lib/python2.7/site\PYGZhy{}packages}\PYG{l+s+s1}{\PYGZsq{}}\PYG{p}{)}
\end{sphinxVerbatim}

\end{itemize}

\end{description}

\item {} \begin{description}
\item[{Windows}] \leavevmode\begin{itemize}
\item {} 
There are two options available:

\end{itemize}
\begin{quote}
\begin{enumerate}
\item {} 
As Windows has no Python installation included, QGIS make use of a separate Python installation added when QGIS was installed on your PC.

\end{enumerate}
\begin{quote}

This results in that pip cannot be used directly. However, if you installed QGIS according to the recommendations in {\hyperref[\detokenize{Getting_Started::doc}]{\sphinxcrossref{\DUrole{doc,doc,doc}{Getting started}}}} you should have a \sphinxstylestrong{OSGeo4W shell} installed where you can use pip to add desired Python libraries. \sphinxstylestrong{OSGeo4W shell} is found in the Windows start menu.

You need to run as an administrator of your PC. To do this, right-click on \sphinxstylestrong{OSGeo4W shell} and choose \sphinxstyleemphasis{run as administrator}. In the command window that appear, write:

\fvset{hllines={, ,}}%
\begin{sphinxVerbatim}[commandchars=\\\{\}]
\PYG{n}{pip} \PYG{n}{install} \PYG{n}{pandas}
\end{sphinxVerbatim}
\end{quote}
\begin{enumerate}
\setcounter{enumii}{1}
\item {} 
Installation of pandas Restart the \sphinxstyleemphasis{installation (64-bit) For Advanced Users} (see Getting started) and choose \sphinxstyleemphasis{Advanced Install}. When you come up to Select Packages search for pandas and click on \sphinxstyleemphasis{Skip} until you see a version number of pandas (see left picture). Finish the installation.

\end{enumerate}
\begin{quote}

\sphinxstylestrong{This method can also be used to include other missing libraries such as gdal etc.}

\sphinxstylestrong{PLEASE NOTICE!}

Due to a recent update of \sphinxstylestrong{netCDF4} library (1.3.0), the \sphinxstylestrong{netCDF4} library has a version conflict related to the \sphinxstylestrong{numpy} version currently used in QGIS 2.18.x. This results in that some plugins in UMEP will fail, e.g. LQf.
We have submitted an issue regarding this to the QGIS community. Meanwhile, we recommend UMEP users to downgrade the netCDF4 library to \sphinxstylestrong{1.2.9}. This is easiest done by opening the \sphinxstylestrong{OSGeo4W shell} and run the two following commands:

\fvset{hllines={, ,}}%
\begin{sphinxVerbatim}[commandchars=\\\{\}]
\PYG{n}{pip} \PYG{n}{uninstall} \PYG{n}{netCDF4}
\PYG{n}{pip} \PYG{n}{install} \PYG{n}{netCDF4}\PYG{o}{==}\PYG{l+m+mf}{1.2}\PYG{o}{.}\PYG{l+m+mi}{9}
\end{sphinxVerbatim}

\begin{figure}[htbp]
\centering
\capstart

\noindent\sphinxincludegraphics{{Pandas}.png}
\caption{\sphinxstylestrong{Installation of pandas}}\label{\detokenize{Getting_Started:id2}}\end{figure}
\end{quote}
\end{quote}

\end{description}

\item {} \begin{description}
\item[{Mac OS X}] \leavevmode\begin{itemize}
\item {} 
Follow the instructions for Linux. \sphinxstylestrong{*Note*}: this approach is tested to be working under Mac OS X 10.11.5.

\end{itemize}

\end{description}

\item {} \begin{description}
\item[{Other Platforms}] \leavevmode\begin{itemize}
\item {} 
Other platforms require the packages to be installed to the QGIS Python path, which differs depending on operating system.
Or refer to \sphinxhref{http://docs.qgis.org/testing/en/docs/pyqgis\_developer\_cookbook/intro.html\#the-startup-py-file}{startup.py}
to modify start up file of QGIS by including the paths to \sphinxstylestrong{pandas} and \sphinxstylestrong{scipy}. An example of startup.py may look like

\fvset{hllines={, ,}}%
\begin{sphinxVerbatim}[commandchars=\\\{\}]
\PYG{k+kn}{import} \PYG{n+nn}{sys}
\PYG{n}{sys}\PYG{o}{.}\PYG{n}{path}\PYG{o}{.}\PYG{n}{insert}\PYG{p}{(}\PYG{l+m+mi}{1}\PYG{p}{,}\PYG{l+s+s1}{\PYGZsq{}}\PYG{l+s+s1}{/usr/local/lib/python2.7/site\PYGZhy{}packages}\PYG{l+s+s1}{\PYGZsq{}}\PYG{p}{)}
\end{sphinxVerbatim}

\end{itemize}

\end{description}

\end{enumerate}

\end{itemize}


\bigskip\hrule\bigskip


\sphinxstylestrong{PLEASE NOTICE!}

Due to a recent update of \sphinxstylestrong{netCDF4} library (1.3.0), the \sphinxstylestrong{netCDF4}
library has a version conflict related to the \sphinxstylestrong{numpy} version
currently used in QGIS 2.18.x. This results in that some plugins in UMEP
will fail, e.g. LQf.

We have submitted an issue regarding this to the QGIS community.
Meanwhile, we recommend UMEP users to downgrade the netCDF4 library to
\sphinxstylestrong{1.2.9}. This is easiest done by opening the \sphinxstylestrong{OSGeo4W shell} and run
the two following commands:

\fvset{hllines={, ,}}%
\begin{sphinxVerbatim}[commandchars=\\\{\}]
\PYG{n}{pip} \PYG{n}{uninstall} \PYG{n}{netCDF4}
\PYG{n}{pip} \PYG{n}{install} \PYG{n}{netCDF4}\PYG{o}{==}\PYG{l+m+mf}{1.2}\PYG{o}{.}\PYG{l+m+mi}{9}
\end{sphinxVerbatim}


\bigskip\hrule\bigskip



\chapter{Pre-Processor}
\label{\detokenize{pre-processor/Pre-Processor:pre-processor}}\label{\detokenize{pre-processor/Pre-Processor::doc}}
This section include manuals for the tools used for preparing input data to the models included in UMEP.

The pre-processing tools are devided into 5 different sections:
\begin{itemize}
\item {} 
Meteorological Data

\item {} 
Spatial Data

\item {} 
Urban Geometry

\item {} 
Urban Land Cover

\item {} 
Urban Morpgology

\end{itemize}

SUEWS Prepare, a separate tool to prepare input data for the SUEWS model is also included.

The tools can be accessed from the left panel.


\section{Meteorological Data: Download data (WATCH)}
\label{\detokenize{pre-processor/Meteorological Data Download data (WATCH):meteorological-data-download-data-watch}}\label{\detokenize{pre-processor/Meteorological Data Download data (WATCH):watch}}\label{\detokenize{pre-processor/Meteorological Data Download data (WATCH)::doc}}\begin{itemize}
\item {} 
Contributors:

\end{itemize}


\begin{savenotes}\sphinxattablestart
\centering
\begin{tabular}[t]{|\X{50}{100}|\X{50}{100}|}
\hline
\sphinxstyletheadfamily 
Name
&\sphinxstyletheadfamily 
Institution
\\
\hline
Andy Gabey
&
Reading
\\
\hline
Ting Sun
&
Reading
\\
\hline
Helen Ward
&
Reading
\\
\hline
Lingbo Xue
&
Reading
\\
\hline
Zhe Zhang
&
Reading
\\
\hline
Tom Kokkonen
&
Helsinki
\\
\hline
Leena Järvi
&
Helsinki
\\
\hline
Sue Grimmond
&
Reading
\\
\hline
\end{tabular}
\par
\sphinxattableend\end{savenotes}
\begin{itemize}
\item {} \begin{description}
\item[{Introduction:}] \leavevmode\begin{enumerate}
\item {} 
Basic meteorological variables are required for most applications in the UMEP processor. If observed data are not available for a particular location, the global \sphinxhref{http://www.eu-watch.org/}{WATCH} forcing datasets (Weedon et al. 2011, 2014) can be used to provide this information.

\item {} 
The WATCH data downloader allows climate reanalysis data to be extracted for a specific location and period of interest, and (optionally) transformed into annual files in a format suitable for models within UMEP.

\end{enumerate}
\begin{itemize}
\item {} 
The {\hyperref[\detokenize{Abbreviations:abbreviations}]{\sphinxcrossref{\DUrole{std,std-ref,std,std-ref}{WFD}}}} dataset is based on 40-year {\hyperref[\detokenize{Abbreviations:abbreviations}]{\sphinxcrossref{\DUrole{std,std-ref,std,std-ref}{ECMWF}}}} Re-analysis data (ERA-40) and is available at half-degree resolution for 1901-2001.

\item {} 
The {\hyperref[\detokenize{Abbreviations:abbreviations}]{\sphinxcrossref{\DUrole{std,std-ref,std,std-ref}{WFDEI}}}} dataset is based on {\hyperref[\detokenize{Abbreviations:abbreviations}]{\sphinxcrossref{\DUrole{std,std-ref,std,std-ref}{ERA}}}}-interim re-analysis data and is available at half-degree resolution for 1979-2012.

\end{itemize}

\end{description}

\end{itemize}


\begin{savenotes}\sphinxattablestart
\centering
\begin{tabular}[t]{|\X{50}{100}|\X{50}{100}|}
\hline
\sphinxstyletheadfamily 
Variables available
&\sphinxstyletheadfamily 
Comments
\\
\hline
Wind speed {[}m s$^{\text{-1}}${]}
&
10 m instantaneous
\\
\hline
Air temperature {[}K{]}
&
2 m instantaneous
\\
\hline
Specific humidity {[}kg kg$^{\text{-1}}${]}
&
2 m instantaneous
\\
\hline
Pressure {[}Pa{]}
&
Instantaneous surface pressure
\\
\hline
Incoming shortwave radiation {[}W m$^{\text{-2}}${]}
&
Average over previous 3 hours in WFDEI and over next 3 hours in WFD, surface flux
\\
\hline
Incoming longwave radiation {[}W m$^{\text{-2}}${]}
&
Average over previous 3 hours in WFDEI and over next 3 hours in WFD, surface flux
\\
\hline
Rainfall rate {[}kg m$^{\text{-2}}$ s$^{\text{-1}}${]}
&
Average over previous 3 hours in WFDEI and over next 3 hours in WFD. CRU and GPCC bias correction options.
\\
\hline
Snowfall rate {[}kg m$^{\text{-2}}$ s$^{\text{-1}}${]}
&
Average over previous 3 hours in WFDEI and over next 3 hours in WFD. CRU and GPCC bias correction options.
\\
\hline
\end{tabular}
\par
\sphinxattableend\end{savenotes}
\begin{itemize}
\item {} 
The current downscaling procedure \sphinxstylestrong{only} deals with WFDEI data; a module for WFD is under development.

\item {} 
All precipitation corrections are currently conducted based on \sphinxstylestrong{CRU} option.

\item {} 
Data is drawn from a subset of the full WATCH dataset that does not cover the entire globe but includes Europe and the majority of Asian countries excluding Russia at this time. More regions may be added in the future. The map below shows current coverage:

\end{itemize}

\begin{figure}[htbp]
\centering
\capstart

\noindent\sphinxincludegraphics{{350px-Watch_masked}.png}
\caption{Available data in WATCH downloader (overlaid on countries)}\label{\detokenize{pre-processor/Meteorological Data Download data (WATCH):id3}}\end{figure}

\begin{sphinxadmonition}{note}{Note:}
Message about missing Python libraries. Follow the instruction at {\hyperref[\detokenize{Getting_Started:python-libraries}]{\sphinxcrossref{\DUrole{std,std-ref,std,std-ref}{link}}}}.
\end{sphinxadmonition}
\begin{itemize}
\item {} \begin{description}
\item[{Obtaining WATCH data via UMEP:}] \leavevmode
\begin{figure}[htbp]
\centering
\capstart

\noindent\sphinxincludegraphics{{Watch_downloader_2}.png}
\caption{Integrated WATCH data downloader: control panel}\label{\detokenize{pre-processor/Meteorological Data Download data (WATCH):id4}}\end{figure}

\end{description}

\item {} \begin{description}
\item[{Running the tool:}] \leavevmode\begin{itemize}
\item {} \begin{description}
\item[{The downloader is separated into two sections:}] \leavevmode\begin{enumerate}
\item {} \begin{description}
\item[{\sphinxstylestrong{Download climate data}: Retrieves WATCH data for all variables for the location and period of interest. This saves a NetCDF (.nc) file that contains all variables at 3 h resolution that can be used directly by ExtremeFinder.}] \leavevmode\begin{itemize}
\item {} 
\sphinxstyleemphasis{Latitude} and \sphinxstyleemphasis{longitude}: WGS84 co-ordinates of the study location. Data is extracted from the WATCH grid cell that contains these co-ordinates.

\item {} 
\sphinxstyleemphasis{Start time} and \sphinxstyleemphasis{End Time}: The time range of data to be downloaded (inclusive; to the nearest month)

\end{itemize}

\end{description}

\item {} \begin{description}
\item[{\sphinxstylestrong{Refine downloaded data}: Before the WATCH data can be loaded into models such as SUEWS, it must be downscaled, separated into annual files and refined. These controls perform the refinement on the .nc file downloaded in part (1) and save the results as a text file that can be loaded into further models. The resulting file contains data at 1 hour intervals, with estimates or placeholders for meteorological variables not present in WATCH.}] \leavevmode\begin{itemize}
\item {} 
\sphinxstyleemphasis{Site height}: Height above sea level of the desired measurement site. This applies adjustments to meteorological parameters based on the height above ground level. Data are available from 1 January 1979 to 31 December 2015.

\item {} 
\sphinxstyleemphasis{UTC offset}: Adjusts the UTC time used in the original WATCH dataset to a local time (e.g., for Beijing time, UTC Offset = 8 h should be specified). \sphinxstylestrong{NOTE:} As of now the tool does not support half hour-timezones.

\item {} 
\sphinxstyleemphasis{Rain hours per 3h}: Rain events in the location of interest may be very short \textendash{} information that is lost because the WATCH data is produced at 3 h intervals, within which it is assumed rain is continuous. This control limits the duration of rain in the 1-hour file to 1, 2 or 3 hours within each 3 hour interval.

\item {} 
\sphinxstyleemphasis{Path to LQF results}: Incorporates results data from the LQF model into the disaggregated data. Note that this feature produces one file per LQF grid cell and year.

\end{itemize}

\end{description}

\end{enumerate}

\end{description}

\end{itemize}

\end{description}

\item {} \begin{description}
\item[{Considerations:}] \leavevmode\begin{itemize}
\item {} 
\sphinxstylestrong{Spatial resolution}: The WATCH data are provided for half-degree grid boxes. In regions with substantial heterogeneity within these grid boxes data at the grid-box scale may be not be representative of your study site (e.g. mountainous regions, urban areas).

\item {} 
\sphinxstylestrong{Temporal resolution}: The data are downloaded at 3 h resolution and are linearly downscaled to 1 h time steps during the refinement step, during which radiation data are corrected for sunrise/sunset.

\end{itemize}

\end{description}

\item {} \begin{description}
\item[{References:}] \leavevmode\begin{itemize}
\item {} 
Kokkonen et al. (2017, in review)

\item {} 
Ward et al. (2017, in review)

\item {} 
Weedon GP, Gomes S, Viterbo P, Shuttleworth WJ, Blyth E, Österle H, Adam JC, Bellouin N, Boucher O and Best MJ (2011) Creation of the WATCH Forcing Data and Its Use to Assess Global and Regional Reference Crop Evaporation over Land during the Twentieth Century. \sphinxhref{http://journals.ametsoc.org/doi/abs/10.1175/2011JHM1369.1}{Journal of Hydrometeorology 12, 823-848}

\item {} 
Weedon GP, Balsamo G, Bellouin N, Gomes S, Best MJ and Viterbo P (2014) The WFDEI meteorological forcing data set: WATCH Forcing Data methodology applied to ERA-Interim reanalysis data. {\color{red}\bfseries{}{}`}Water Resour. Res. 50, 7505-7514       \textless{}\sphinxurl{http://onlinelibrary.wiley.com/doi/10.1002/2014WR015638/abstract}\textgreater{}{}`\_\_\textbar{}

\item {} 
Tan YS (2015) MSc Thesis, University of Reading

\item {} 
Xue L (2016) MSc Thesis, University of Reading

\end{itemize}

\end{description}

\end{itemize}


\section{Meteorological Data: MetPreprocessor}
\label{\detokenize{pre-processor/Meteorological Data MetPreprocessor:meteorological-data-metpreprocessor}}\label{\detokenize{pre-processor/Meteorological Data MetPreprocessor:metpreprocessor}}\label{\detokenize{pre-processor/Meteorological Data MetPreprocessor::doc}}\begin{itemize}
\item {} 
Contributor:

\end{itemize}


\begin{savenotes}\sphinxattablestart
\centering
\begin{tabular}[t]{|\X{50}{100}|\X{50}{100}|}
\hline
\sphinxstyletheadfamily 
Name
&\sphinxstyletheadfamily 
Institution
\\
\hline
Fredrik Lindberg
&
Gothenburg
\\
\hline
\end{tabular}
\par
\sphinxattableend\end{savenotes}
\begin{itemize}
\item {} \begin{description}
\item[{Introduction}] \leavevmode
MetPreprocessor can be used to transform required temporal meteorological data into the format used in UMEP. The following variables are usually required as a minimum: air temperature, relative humidity, barometric pressure, wind speed, incoming shortwave radiation and rainfall; if available, other variables can be supplied as well.

\sphinxstyleemphasis{Input data} can include any number of header lines and should be separated by conventional separators (e.g. comma, space, tab, etc). The \sphinxstyleemphasis{output format} is space-separated and includes time-related variables of year, day of year, hour and minute. The plugin is able to process other input time formats including month, day of month, etc.

\end{description}

\item {} 
\sphinxstylestrong{Dialog box}

\end{itemize}

\begin{figure}[htbp]
\centering
\capstart

\noindent\sphinxincludegraphics{{MetPreProcessor}.jpg}
\caption{Interface for inputting an ascii data file into the correct format for SUEWS}\label{\detokenize{pre-processor/Meteorological Data MetPreprocessor:id1}}\end{figure}
\begin{itemize}
\item {} 
\sphinxstylestrong{Dialog sections}

\end{itemize}


\begin{savenotes}\sphinxattablestart
\centering
\begin{tabular}[t]{|\X{50}{100}|\X{50}{100}|}
\hline

top left
&\begin{description}
\item[{Select an existing text file with meteorological data at a temporal resolution between 5 min and 180 min (3 hours) that is divisible by 5 min.}] \leavevmode\begin{itemize}
\item {} 
Note the model runs at a time step of 5-min.

\item {} 
At least hourly resolution is recommended.

\item {} 
The current version of the SEBE-model (April, 2018) requires hourly data.

\end{itemize}

\end{description}
\\
\hline
middle left
&
Specify time-related columns in the imported data file.
\\
\hline
lower left
&
map extent is specified
\\
\hline
lower left
&\begin{description}
\item[{Perform quality control (recommended)}] \leavevmode\begin{itemize}
\item {} 
Select to perform a simple quality control which will check the input data for unreasonable values of each variable.

\end{itemize}

\end{description}
\\
\hline
right
&
Choose columns from imported data file that correspond to the meteorological variables used in UMEP.
\\
\hline
\end{tabular}
\par
\sphinxattableend\end{savenotes}
\begin{itemize}
\item {} \begin{description}
\item[{Variables included in UMEP meteorological input file}] \leavevmode
if acceptable range is not reasonable (i.e. beyond the limits we have set) please contact

\end{description}

\end{itemize}


\begin{savenotes}\sphinxattablestart
\centering
\begin{tabular}[t]{|\X{4}{100}|\X{10}{100}|\X{25}{100}|\X{18}{100}|\X{43}{100}|}
\hline
\sphinxstyletheadfamily 
No.
&\sphinxstyletheadfamily 
Header name
&\sphinxstyletheadfamily 
Description
&\sphinxstyletheadfamily 
Accepted  range
&\sphinxstyletheadfamily 
Comments
\\
\hline
1
&
iy
&
Year {[}YYYY{]}
&
Not applicable
&\\
\hline
2
&
id
&
Day of year {[}DOY{]}
&
1 to 365 (366 if leap year)
&\\
\hline
3
&
it
&
Hour {[}H{]}
&
0 to 23
&\\
\hline
4
&
imin
&
Minute {[}M{]}
&
0 to 59
&\\
\hline
5
&
qn
&
Net all-wave radiation {[}W m$^{\text{-2}}${]}
&
-200 to 800
&\\
\hline
6
&
qh
&
Sensible heat flux {[}W m$^{\text{-2}}${]}
&
-200 to 750
&\\
\hline
7
&
qe
&
Latent heat flux {[}W m$^{\text{-2}}${]}
&
-100 to 650
&\\
\hline
8
&
qs
&
Storage heat flux {[}W m$^{\text{-2}}${]}
&
-200 to 650
&\\
\hline
9
&
qf
&
Anthropogenic heat flux {[}W m$^{\text{-2}}${]}
&
0 to 1500
&\\
\hline
10
&
U
&
Wind speed {[}m s$^{\text{-1}}${]}
&
0.001 to 60
&\\
\hline
11
&
RH
&
Relative Humidity {[}\%{]}
&
5 to 100
&\\
\hline
12
&
Tair
&
Air temperature {[}°C{]}
&
-30 to 55
&\\
\hline
13
&
pres
&
Surface barometric pressure {[}kPa{]}
&
90 to 107
&\\
\hline
14
&
rain
&
Rainfall {[}mm{]}
&
0 to 30
&
(per 5 min) this should be scaled based on time step used
\\
\hline
15
&
kdown
&
Incoming shortwave radiation {[}W m$^{\text{-2}}${]}
&
0 to 1200
&\\
\hline
16
&
snow
&
Snow {[}mm{]}
&
0 to 300
&
(per 5 min) this should be scaled based on time step used
\\
\hline
17
&
ldown
&
Incoming longwave radiation {[}W m$^{\text{-2}}${]}
&
100 to 600
&\\
\hline
18
&
fcld
&
Cloud fraction {[}tenths{]}
&
0 to 1
&\\
\hline
19
&
wuh
&
External water use {[}m$^{\text{3}}${]}
&
0 to 10
&
(per 5 min) scale based on time step being used
\\
\hline
20
&
xsmd
&
(Observed) soil moisture
&
0.01 to 0.5
&
{[}m$^{\text{3}}$ m$^{\text{-3}}$ or kg kg$^{\text{-1}}${]}
\\
\hline
21
&
lai
&
(Observed) leaf area index {[}m$^{\text{2}}$ m$^{\text{-2}}${]}
&
0 to 15
&\\
\hline
22
&
kdiff
&
Diffuse shortwave radiation {[}W m$^{\text{-2}}${]}
&
0 to 600
&\\
\hline
23
&
kdir
&
Direct shortwave radiation {[}W m$^{\text{-2}}${]}
&
0 to 1200
&
Should be perpendicular to the Sun beam. One way to check this is to compare direct and global radiation and see if kdir is higher than global radiation during clear weather. Then kdir is measured perpendicular to the solar beam.
\\
\hline
24
&
wdir
&
Wind direction {[}°{]}
&
0 to 360
&\\
\hline
\end{tabular}
\par
\sphinxattableend\end{savenotes}
\begin{itemize}
\item {} \begin{description}
\item[{Remarks}] \leavevmode\begin{enumerate}
\item {} 
If decimal time is ticked in, \sphinxstylestrong{day of year column} must be stated and the \sphinxstylestrong{decimal time column} should be numbers between 0 and 1.

\item {} 
If you have problems with importing a data set. Do a time series plot using small points. Check (1) are there any data gaps (there can be no gaps) (2) are the columns lined up throughout the data setes (e.g if variable suddenly changes incorrectly, you may have columns misaligned).

\item {} 
Gapfilling - there are a number of techniques that can be used for this
1. A fast way to get started (you can come back and refine to a more appropriate method)
\begin{enumerate}
\item {} 
Linear fit between one or two missing periods using the data on either

\item {} 
Create diurnal average for each variabel for short periods (e.g. 2 weeks) and use these values to fill missing data

\end{enumerate}

\end{enumerate}

\end{description}

\end{itemize}


\section{SUEWS Prepare}
\label{\detokenize{pre-processor/SUEWS Prepare:suews-prepare}}\label{\detokenize{pre-processor/SUEWS Prepare:suewsprepare}}\label{\detokenize{pre-processor/SUEWS Prepare::doc}}\begin{itemize}
\item {} 
Contributor:

\end{itemize}


\begin{savenotes}\sphinxattablestart
\centering
\begin{tabular}[t]{|\X{50}{100}|\X{50}{100}|}
\hline
\sphinxstyletheadfamily 
Name
&\sphinxstyletheadfamily 
Institution
\\
\hline
Niklas Krave
&
Gothenburg
\\
\hline
Fredrik Lindberg
&
Gothenburg
\\
\hline
Frans Olofson
&
Gothenburg
\\
\hline
Sue Grimmond
&
Reading
\\
\hline
\end{tabular}
\par
\sphinxattableend\end{savenotes}
\begin{itemize}
\item {} \begin{description}
\item[{Introduction:}] \leavevmode\begin{itemize}
\item {} 
The pre-processor SUEWS Prepare generates surface-related input data from geographical data for \sphinxhref{http://urban-climate.net/umep/SUEWS}{SUEWS}, the Surface Urban Energy and Water Balance Scheme. SUEWS (Järvi et al. 2011, 2014; Ward et al. 2016a, b) simulates the urban radiation, energy and water balances using commonly measured/modelled meteorological variables and information about the surface cover. It utilizes an evaporation-interception approach (Grimmond et al. 1991), similar to that used in forests, to model evaporation from urban surfaces. The surface state for each surface type at each time step is calculated from the running water balance of the canopy where the evaporation is calculated from the Penman-Monteith equation. The soil moisture below each surface type (excluding water) is also taken into account.   \textbar{}

\end{itemize}

\end{description}

\item {} \begin{description}
\item[{\sphinxstylestrong{Terminology} :}] \leavevmode\begin{itemize}
\item {} \begin{description}
\item[{Components of the plugin window:}] \leavevmode
\begin{figure}[htbp]
\centering
\capstart

\noindent\sphinxincludegraphics{{SuewsPrepareTerminology}.jpg}
\caption{Some naming conventions used in this document relating to the components of the plugin.}\label{\detokenize{pre-processor/SUEWS Prepare:id1}}\end{figure}

\end{description}

\item {} \begin{description}
\item[{Plugin window:}] \leavevmode\begin{itemize}
\item {} 
Dialog window of the plugin. Any user interface components that are part of the plugin will be a part of the plugin window.

\end{itemize}

\end{description}

\item {} \begin{description}
\item[{Tab:}] \leavevmode\begin{itemize}
\item {} 
The plugin contains many tabs. The tabs can be cycled through to reveal different kinds of information.

\end{itemize}

\end{description}

\item {} \begin{description}
\item[{Widget:}] \leavevmode\begin{itemize}
\item {} 
One tab can contain one or more widgets. One widget contains two boxes, the selection box and the variable box.

\end{itemize}

\end{description}

\item {} \begin{description}
\item[{Variable box:}] \leavevmode\begin{itemize}
\item {} 
Right part of a widget It contains a number of variables. One variable is comprised of a variable title and a variable text box.

\end{itemize}

\end{description}

\item {} \begin{description}
\item[{Variable title:}] \leavevmode\begin{itemize}
\item {} 
The variable title is a short description of the variable.

\end{itemize}

\end{description}

\item {} \begin{description}
\item[{Variable text box:}] \leavevmode\begin{itemize}
\item {} 
The variable text box contains the value of the variable for one site entry.

\end{itemize}

\end{description}

\item {} \begin{description}
\item[{Selection box:}] \leavevmode\begin{itemize}
\item {} 
The selection box is the left part of a widget. It contains a number of user interface components such as buttons and drop down menus.

\end{itemize}

\end{description}

\item {} \begin{description}
\item[{Drop down menu:}] \leavevmode\begin{itemize}
\item {} 
The drop down menu allows a selection from a predetermined range of values.

\end{itemize}

\end{description}

\end{itemize}

\end{description}

\item {} \begin{description}
\item[{\sphinxstylestrong{Terms relating to data used by the plugin} (For more info see developer section below):}] \leavevmode\begin{itemize}
\item {} \begin{description}
\item[{Site Library:}] \leavevmode\begin{itemize}
\item {} 
The site library contains all collected sites (i.e. study areas) and information about those sites.

\end{itemize}

\end{description}

\item {} \begin{description}
\item[{Site code:}] \leavevmode\begin{itemize}
\item {} 
A site code separates site entries of one kind from each other. It needs to be a unique integer number.

\end{itemize}

\end{description}

\item {} \begin{description}
\item[{Identification code:}] \leavevmode\begin{itemize}
\item {} 
The identification code is used when there is a need to separate site entries into categories. If two site entries share the same identification code they belong to the same category.

\end{itemize}

\end{description}

\end{itemize}

\end{description}

\item {} \begin{description}
\item[{\sphinxstylestrong{Using the plugin - The different components of the plugin and the plugin output.}:}] \leavevmode\begin{itemize}
\item {} \begin{description}
\item[{Main window:}] \leavevmode\begin{itemize}
\item {} \begin{description}
\item[{The main window contains all the user interface components of the plugin. Navigation uses tabs, with each providing some of the information needed. The are two categories:}] \leavevmode\begin{itemize}
\item {} 
main settings tab

\item {} 
site library

\end{itemize}

\end{description}

\item {} \begin{description}
\item[{The main window has buttons to specify to:}] \leavevmode\begin{itemize}
\item {} 
indicate the folder where the output will be generated

\item {} 
to start the process of generating the output

\item {} 
to close the main window.

\fvset{hllines={, ,}}%
\begin{sphinxVerbatim}[commandchars=\\\{\}]
\PYG{n}{Note}\PYG{p}{:} \PYG{n}{If} \PYG{n}{the} \PYG{n}{folder} \PYG{n}{selected} \PYG{k}{for} \PYG{n}{the} \PYG{n}{output} \PYG{n}{files} \PYG{n}{already} \PYG{n}{contains}
\PYG{n}{files} \PYG{n}{generated} \PYG{k+kn}{from} \PYG{n+nn}{SUEWS} \PYG{n}{Prepare} \PYG{n}{these} \PYG{n}{files} \PYG{n}{will} \PYG{n}{be} \PYG{o}{*}\PYG{o}{*}\PYG{n}{overwritten}\PYG{o}{*}\PYG{o}{*}
\end{sphinxVerbatim}

\end{itemize}

\end{description}

\item {} \begin{description}
\item[{Main settings tab:}] \leavevmode\begin{itemize}
\item {} 
The main settings tab is where the plugin is provided with inputs from outside sources such as text files and vector layer attributes. Basically anything that is not part of the site library.

\end{itemize}

\begin{figure}[htbp]
\centering
\capstart

\noindent\sphinxincludegraphics{{SUEWSSpatial_Prepare1}.png}
\caption{Plugin window with the main settings tab selected.}\label{\detokenize{pre-processor/SUEWS Prepare:id2}}\end{figure}

\end{description}

\item {} \begin{description}
\item[{Polygon grid:}] \leavevmode\begin{itemize}
\item {} 
The polygon grid is used to provide the plugin with further information through the grid attribute table. Each part of the grid will create a separate entry in the plugins output. The polygon grid can be in any vector file format compatible with QGIS, however, it is recommended to use the shape file format.

\item {} 
To use an existing polygon grid layer in the plugin add the layer to the QGIS interface. This can be done either by dragging and dropping the file into the QGIS program or by using the menu \sphinxstylestrong{Layers}. Any polygon layers added to the QGIS interface can then be selected for use in the plugin from the drop down menu in the main settings tab marked \sphinxstylestrong{Vector polygon grid} If no polygon grid layer is available, there are several opportunities to create these in QGIS. We  recommend to make use of the built-in \sphinxstylestrong{Vector tool} (Vector -\textgreater{} Research tools menu)\textgreater{}

\item {} 
When a relevant polygon grid has been selected for the plugin several separate drop down menus allow for data to be collected from the fields in the polygon grid attribute table. The initial selections in these drops down menus might not be correct and needs to be manually corrected by the user.

\item {} 
The input in the drop down menu marked \sphinxstylestrong{ID field} in the box for polygon grid selection needs to correlate with the polygon layers attribute field for feature ids or any attribute field containing unique integer numbers. The polygon layer should be in a coordinate system that can be related to both lat/lon coordinates as well as meters. The polygon features included in the polygon vector grid can be of any shape and size.

\end{itemize}

\begin{figure}[htbp]
\centering
\capstart

\noindent\sphinxincludegraphics{{SP_Polygon}.jpg}
\caption{\sphinxcode{\sphinxupquote{{}`to do{}`}}}\label{\detokenize{pre-processor/SUEWS Prepare:id3}}\end{figure}

\end{description}

\item {} \begin{description}
\item[{Data for land cover fractions, building morphology and tree morphology :}] \leavevmode\begin{itemize}
\item {} \begin{description}
\item[{To use SUEWS land cover and morphology data for buildings and vegetation are needed. This information can be acquired through other plugins in UMEP. This data can then be added into SUEWSPrepare by two different options:}] \leavevmode\begin{itemize}
\item {} 
Import the data as text

\end{itemize}

\end{description}

\item {} \begin{description}
\item[{To do this click the buttons in the boxes associated with these types of data and follow the import dialogs to select the correct text file. When a file has been selected the file path will be shown in the text boxes above the buttons. The text files on land cover and morphology are generated with the {\hyperref[\detokenize{pre-processor/SUEWS Prepare:Urban_Land_Cover:_Land_Cover_Fraction_(Point)}]{\emph{Land Cover Fraction}}} plugin and the {\hyperref[\detokenize{pre-processor/SUEWS Prepare:Urban_Morphology:_Image_Morphometric_Parameters_Calculator_(Point)}]{\emph{Image Morphometric Calculator}}}, respectively.}] \leavevmode\begin{itemize}
\item {} 
Alternatively, the data need to be available in the attribute table of the polygon layer. If the data are available in this format simply check the check boxes below the buttons to change the interface from buttons into drop down menus. In the drop down menus select the correct attribute fields for the data and the selection is done.

\end{itemize}

\begin{figure}[htbp]
\centering
\capstart

\noindent\sphinxincludegraphics{{SP_landcover}.jpg}
\caption{Box associated with land cover fractions data. The button has been used to import a file containing land cover fraction data.}\label{\detokenize{pre-processor/SUEWS Prepare:id4}}\end{figure}

\begin{figure}[htbp]
\centering
\capstart

\noindent\sphinxincludegraphics{{SP_landcover2}.jpg}
\caption{Box associated with land cover fractions when the checkbox is checked. The drop down menus can be used to import land cover fraction data.}\label{\detokenize{pre-processor/SUEWS Prepare:id5}}\end{figure}

\end{description}

\end{itemize}

\end{description}

\item {} \begin{description}
\item[{Meteorological data:}] \leavevmode\begin{itemize}
\item {} 
The meteorological data have to be imported from a \sphinxstylestrong{text file}. Use the button in the box for meteorological data, follow the dialog and select the correct text file. The meteorological data used in the various UMEP-plugins is format specific and can be generated from other data sources using the {\hyperref[\detokenize{pre-processor/SUEWS Prepare:Meteorological_Data:_MetPreprocessor}]{\emph{MetPreprocessor}}} plugin. There you can also find more information on what parameters are required in the meteorological dataset.

\end{itemize}

\begin{figure}[htbp]
\centering
\capstart

\noindent\sphinxincludegraphics{{SP_met}.jpg}
\caption{Box for meteorological data. The button has been used to import a file containing meteorological data.}\label{\detokenize{pre-processor/SUEWS Prepare:id6}}\end{figure}

\end{description}

\item {} \begin{description}
\item[{Daylight savings time:}] \leavevmode\begin{itemize}
\item {} 
The plugin needs to have access to the correct days in which the switches to and from daylight savings time occurs in the region. The numbers in the text boxes represent the \sphinxhref{https://landweb.modaps.eosdis.nasa.gov/browse/calendar.html}{days of year}. For example, the 21st of January is day of year 21 and the 2nd of February is day of year be 33 and so on. Make sure the days in the text boxes for daylight savings time in the main settings tab are correct for \sphinxhref{https://en.wikipedia.org/wiki/Daylight\_saving\_time\_by\_country}{your region}.

\end{itemize}

\begin{figure}[htbp]
\centering
\capstart

\noindent\sphinxincludegraphics{{SP_DLS}.jpg}
\caption{Box used for setting the start and end of day lights savings time.}\label{\detokenize{pre-processor/SUEWS Prepare:id7}}\end{figure}

\end{description}

\item {} \begin{description}
\item[{Population density:}] \leavevmode\begin{itemize}
\item {} 
This data needs to be added through the polygon grid attribute table. Make sure that the data exist as an attribute field and select it in the drop down menu.

\end{itemize}

\end{description}

\item {} \begin{description}
\item[{Wall area (optional) :}] \leavevmode\begin{itemize}
\item {} 
This data needs to be added through the polygon grid attribute table. Make sure that the data exist as an attribute field and select it in the drop down menu. This can be calculated from a DSM using the {\hyperref[\detokenize{pre-processor/Urban Geometry Wall Height and Aspect:wallheightandaspect}]{\sphinxcrossref{\DUrole{std,std-ref,std,std-ref}{Wall height and aspect calculator}}}}.

\end{itemize}

\begin{figure}[htbp]
\centering
\capstart

\noindent\sphinxincludegraphics{{SUEWSPrepare_wallarea}.png}
\caption{Box for wall area data.}\label{\detokenize{pre-processor/SUEWS Prepare:id8}}\end{figure}

\end{description}

\item {} \begin{description}
\item[{Land use fraction (optional):}] \leavevmode\begin{itemize}
\item {} 
his data needs to be added through a text file. Information needed is land use fractions for impervious and building land cover classes. It is possible to include three impervious and five building classes. The format of the text file should be:

\fvset{hllines={, ,}}%
\begin{sphinxVerbatim}[commandchars=\\\{\}]
\PYG{n}{ID} \PYG{n}{fLUp1} \PYG{n}{fLUp2} \PYG{n}{fLUp3} \PYG{n}{Code\PYGZus{}LUpaved1} \PYG{n}{Code\PYGZus{}LUpaved2} \PYG{n}{Code\PYGZus{}LUpaved3} \PYG{n}{fLUb1} \PYG{n}{fLUb2} \PYG{n}{fLUb3} \PYG{n}{fLUb4} \PYG{n}{fLUb5} \PYG{n}{Code\PYGZus{}LUbuilding1} \PYG{n}{Code\PYGZus{}LUbuilding2} \PYG{n}{Code\PYGZus{}LUbuilding3} \PYG{n}{Code\PYGZus{}LUbuilding4} \PYG{n}{Code\PYGZus{}LUbuilding5}
\PYG{l+m+mi}{1} \PYG{l+m+mi}{0} \PYG{l+m+mf}{0.62} \PYG{l+m+mf}{0.38} \PYG{l+m+mi}{806} \PYG{l+m+mi}{807} \PYG{l+m+mi}{808} \PYG{l+m+mf}{0.90} \PYG{l+m+mf}{0.10} \PYG{l+m+mi}{0} \PYG{l+m+mi}{0} \PYG{l+m+mi}{0} \PYG{l+m+mi}{801} \PYG{l+m+mi}{802} \PYG{l+m+mi}{803} \PYG{l+m+mi}{804} \PYG{l+m+mi}{805}
\PYG{o}{.}\PYG{o}{.}\PYG{o}{.}
\end{sphinxVerbatim}

where \sphinxstyleemphasis{f} is fraction, \sphinxstyleemphasis{LU} is land use and \sphinxstyleemphasis{p} is paved. Fractions must add up to 1 for paved and buildings respectively. A plugin to generate this text file is not yet created.

\begin{figure}[htbp]
\centering
\capstart

\noindent\sphinxincludegraphics{{SUEWSPrepare_landuse}.png}
\caption{Box for land use data.}\label{\detokenize{pre-processor/SUEWS Prepare:id9}}\end{figure}

\end{itemize}

\end{description}

\item {} \begin{description}
\item[{Initial conditions :}] \leavevmode\begin{itemize}
\item {} 
The SUEWS model requires information of initial conditions. SUEWSPRepare generates some basic information used to create a file called \sphinxstylestrong{InitialConditionsXXXX\_YYYY.nml} where XXXX is the ID grid number.

\end{itemize}

\begin{figure}[htbp]
\centering
\capstart

\noindent\sphinxincludegraphics{{SUEWSPrepare_init}.png}
\caption{Box for initial conditions.}\label{\detokenize{pre-processor/SUEWS Prepare:id10}}\end{figure}

\end{description}

\item {} \begin{description}
\item[{Site library tabs :}] \leavevmode\begin{itemize}
\item {} 
The remaining tabs besides the main settings tab all fall under the same category, namely site library tabs. A site library tab represents certain characteristics of an area. A tab can consist of one or more widgets. Each widget has a predetermined layout but will represent different kinds of information. The left part of a widget can be used to select a site entry for the plugin output or to create a new entry to the site library. The right part of a widget will show information about a site through variables.

\end{itemize}

\begin{figure}[htbp]
\centering
\capstart

\noindent\sphinxincludegraphics{{SP_siteLib}.jpg}
\caption{Some of the components of a site library tab.}\label{\detokenize{pre-processor/SUEWS Prepare:id11}}\end{figure}

\end{description}

\item {} \begin{description}
\item[{Selecting a site :}] \leavevmode\begin{itemize}
\item {} 
The selection of a site is done through a drop down menu marked “Code”. The codes in the drop down menu represent the site codes for a site. Each code represents one site entry. Selecting a code will show the values of the variables for the site entry in the variable box the right side of the tab. The code selected will be used in the output of the plugin.

\item {} 
It is possible to use separate codes for each feature in the polygon grid. This requires a field in the polygon grid attribute table that represent the code that is to be used for each feature. If there is such a field click the checkbox marked “Use unique codes for each entry” and select the correct field from the drop down menu.

\end{itemize}
\begin{wrapfigure}{r}{0pt}
\centering
\noindent\sphinxincludegraphics{{SP_siteCode}.jpg}
\caption{Site code selection in a site library tab}\label{\detokenize{pre-processor/SUEWS Prepare:id12}}\end{wrapfigure}
\begin{wrapfigure}{r}{0pt}
\centering
\noindent\sphinxincludegraphics{{Figure12}.png}
\caption{Using more than one site code.}\label{\detokenize{pre-processor/SUEWS Prepare:id13}}\end{wrapfigure}

\end{description}

\item {} \begin{description}
\item[{Creating a new site entry:}] \leavevmode\begin{itemize}
\item {} 
To create a new site entry click the button marked “Edit values”. This will make the text boxes for the variables in the right box available for editing. When you are satisfied with the changes press the button marked “Make changes”. This will show a dialog window that will ask you to provide a site code for your new entry and some description of the site you are adding. After this information is provided you are also able to add an optional url to a picture that represent the site.

\end{itemize}

\end{description}

\item {} \begin{description}
\item[{Plugin Output:}] \leavevmode\begin{itemize}
\item {} 
In the output folder specified, a number of text files needed as input for the SUEWS model is created. These can be used in conjunction with {\hyperref[\detokenize{processor/Urban Energy Balance Urban Energy Balance (SUEWS.BLUEWS, advanced):suewsadvanced}]{\sphinxcrossref{\DUrole{std,std-ref,std,std-ref}{SUEWS/BLUEWS (Advanced)}}}}. Also, see the \sphinxhref{http://urban-climate.net/umep/SUEWS}{SUEWS manual} for more information.

\end{itemize}

\end{description}

\end{itemize}

\end{description}

\end{itemize}

\end{description}

\end{itemize}


\section{Spatial Data: DSM Generator}
\label{\detokenize{pre-processor/Spatial Data DSM Generator:spatial-data-dsm-generator}}\label{\detokenize{pre-processor/Spatial Data DSM Generator:dsmgenerator}}\label{\detokenize{pre-processor/Spatial Data DSM Generator::doc}}\begin{itemize}
\item {} 
Developer:

\end{itemize}


\begin{savenotes}\sphinxattablestart
\centering
\begin{tabular}[t]{|\X{50}{100}|\X{50}{100}|}
\hline
\sphinxstyletheadfamily 
Name
&\sphinxstyletheadfamily 
Institution
\\
\hline
Nils Wallenberg
&
Gothenburg
\\
\hline
\end{tabular}
\par
\sphinxattableend\end{savenotes}
\begin{itemize}
\item {} \begin{description}
\item[{Introduction:}] \leavevmode\begin{itemize}
\item {} 
Digital Surface Models (DSMs) is not always available for the area you want to investigate. The \sphinxstylestrong{DSM Generator} can be used to create or alter a DSM by using information from a polygon building footprint layer where a building height attribute is available. An option to acquire building footprints, and also in some cases building height from \sphinxcode{\sphinxupquote{Open Street Map}} data, is also available.

\end{itemize}

\end{description}

\item {} \begin{description}
\item[{Dialog box:}] \leavevmode
\begin{figure}[htbp]
\centering
\capstart

\noindent\sphinxincludegraphics{{DSMGenerator}.png}
\caption{Dialog for the DSM generator plugin}\label{\detokenize{pre-processor/Spatial Data DSM Generator:id1}}\end{figure}

\end{description}

\item {} 
Dialog sections:

\end{itemize}


\begin{savenotes}\sphinxattablestart
\centering
\begin{tabular}[t]{|\X{50}{100}|\X{50}{100}|}
\hline

top
&
input DEM data is specified
\\
\hline
middle upper
&
input polygon with height data or OSM is specified
\\
\hline
middle
&
map extent is specified
\\
\hline
middle lower
&
to specify the output DSM and output resolution
\\
\hline
bottom
&
to run the calculations
\\
\hline
\end{tabular}
\par
\sphinxattableend\end{savenotes}
\begin{itemize}
\item {} \begin{description}
\item[{Digital Elevation Model:}] \leavevmode\begin{itemize}
\item {} 
A raster file containing elevation values needed to create the DSM

\end{itemize}

\end{description}

\item {} \begin{description}
\item[{Polygon Vector File:}] \leavevmode\begin{itemize}
\item {} 
A polygon vector file including height values of buildings needed to create the DSM

\end{itemize}

\end{description}

\item {} \begin{description}
\item[{Necessary attributes:}] \leavevmode\begin{itemize}
\item {} 
Building height values in meters

\end{itemize}

\end{description}

\item {} \begin{description}
\item[{Use Open Street Map:}] \leavevmode\begin{itemize}
\item {} 
Tick this in if you do not have a polygon layer with building heights. Open Street Map (© OpenStreetMap contributors) data will be used instead. If no building height is found \sphinxstylestrong{building level height} will be used instead. Set to appropriate value, e.g. a three level building with building level height set to 3 will be 3 * 3 = 9 meters high.

\end{itemize}

\end{description}

\item {} \begin{description}
\item[{Save OSM as shapefile:}] \leavevmode\begin{itemize}
\item {} 
Tick this in if you want to save the Open Street Map data as a polygon layer. This can be used if you want to look at what values has been used and if you want to add values manually.

\end{itemize}

\end{description}

\item {} \begin{description}
\item[{Map extent:}] \leavevmode\begin{itemize}
\item {} 
Set either to map canvas extent or extent from layer. Extent have to be smaller or equal to the raster DEM extent specified in the top section.

\end{itemize}

\end{description}

\item {} \begin{description}
\item[{Digital Surface Model:}] \leavevmode\begin{itemize}
\item {} 
Set output for the generated DSM. Also set output resolution.

\end{itemize}

\end{description}

\item {} \begin{description}
\item[{Run:}] \leavevmode\begin{itemize}
\item {} 
Starts the calculations

\end{itemize}

\end{description}

\item {} \begin{description}
\item[{Close:}] \leavevmode\begin{itemize}
\item {} 
Closes the plugin.

\end{itemize}

\end{description}

\item {} \begin{description}
\item[{Output:}] \leavevmode\begin{itemize}
\item {} 
One GeoTIFF is created, a DSM.

\end{itemize}

\end{description}

\item {} \begin{description}
\item[{Remarks:}] \leavevmode\begin{itemize}
\item {} 
The DEM raster and map canvas should be in a projection with meters as units.

\item {} 
Raster elevation data (DEM) can be retrieved from e.g. \sphinxhref{http://www.opendem.info/}{OpenDEM}.

\item {} 
If you use Open Street Map make sure you read \sphinxhref{http://www.openstreetmap.org/copyright}{Open Street Map} © OpenStreetMap contributors.

\end{itemize}

\end{description}

\end{itemize}


\section{Spatial Data: LCZ Converter}
\label{\detokenize{pre-processor/Spatial Data LCZ Converter:spatial-data-lcz-converter}}\label{\detokenize{pre-processor/Spatial Data LCZ Converter:lczconverter}}\label{\detokenize{pre-processor/Spatial Data LCZ Converter::doc}}\begin{itemize}
\item {} 
Contributor:

\end{itemize}


\begin{savenotes}\sphinxattablestart
\centering
\begin{tabular}[t]{|\X{50}{100}|\X{50}{100}|}
\hline
\sphinxstyletheadfamily 
Name
&\sphinxstyletheadfamily 
Institution
\\
\hline
Natalie Theeuwes
&
Reading
\\
\hline
Andy Gabey
&
Reading
\\
\hline
Fredrik Lindberg
&
Gothenburg
\\
\hline
Sue Grimmond
&
Reading
\\
\hline
\end{tabular}
\par
\sphinxattableend\end{savenotes}
\begin{itemize}
\item {} \begin{description}
\item[{Introduction:}] \leavevmode\begin{itemize}
\item {} 
The Local climate zone (LCZ) converter calculates land cover fractions (see land cover reclassifier) on a vector grid based on LCZ raster maps from the \sphinxhref{http://www.wudapt.org/}{WUDAPT portal}. The local climate zone are urban area classified based on the \sphinxhref{http://journals.ametsoc.org/doi/abs/10.1175/BAMS-D-11-00019.1}{Stewart and Oke (2012)} scheme.

\item {} 
The raster LCZ maps can be converted into maps of land cover fraction and morphometric properties. For this conversion we use paved, building and pervious fraction for each LCZ from \sphinxhref{http://onlinelibrary.wiley.com/doi/10.1002/joc.3746/abstract}{Stewart et al. (2014)}. However, what exactly the pervious fraction consists of (grass, trees, bare soil or water) needs to be user-specified. Similarly, morphometric properties for the buildings are specified in this scheme, but the vegetation morphometric properties still need to be specified by the user.

\end{itemize}

\end{description}

\begin{figure}[htbp]
\centering
\capstart

\noindent\sphinxincludegraphics{{700px-LCZ_description}.png}
\caption{The definition of the different local climate zones (LCZ)}\label{\detokenize{pre-processor/Spatial Data LCZ Converter:id1}}\end{figure}

Note: In UMEP we refer to the rural LCZ’s as 101, 102, 103, 104, 105, 106 and 107 instead of A, B, C, D, E, F and G.

\item {} 
Location:
- The Tree LCZ converter is located at
\begin{quote}
\begin{itemize}
\item {} 
UMEP

\end{itemize}
\begin{quote}
\begin{itemize}
\item {} 
Pre-Processor

\end{itemize}
\begin{quote}
\begin{itemize}
\item {} 
Spatial Data

\end{itemize}
\begin{itemize}
\item {} 
LCZ converter

\end{itemize}
\end{quote}
\end{quote}
\end{quote}

\item {} \begin{description}
\item[{Dialog box:}] \leavevmode\begin{itemize}
\item {} 
The first tab in the LCZ converter dialog shows a table. This table includes land cover fractions and morphometric properties for buildings and vegetation for each local climate zone. If the default values in the table are not appropriate for the selected city the user has a choice between editing the table directly or using the “pervious distribution” tab in order to provide approximate values for the distribution between grass, bare soil, trees and water and the height of the vegetation.

\item {} 
Within the “pervious distribution” tab there are two options to change the pervious fraction distribution: Either per LCZ using the “Separate LCZ’s” button or for all LCZ’s together using “Same for all LCZ’s”. When selecting the first option \sphinxstylestrong{make sure to select the LCZ raster first}. Based on the LCZ raster, the dropdown boxes will show the LCZ classes ordered by the frequency of occurrence. Select the classes to specify the pervious distributions for and select the most appropriate pervious land cover options and vegetation heights.

\item {} 
When choosing the “Same for all LCZ’s” option: choose the appropriate pervious land cover fractions and vegetation heights for all urban and all rural LCZ classes.

\end{itemize}

\begin{figure}[htbp]
\centering
\capstart

\noindent\sphinxincludegraphics{{700px-LCZdialog1}.png}
\caption{The \sphinxstyleemphasis{Table}-tab in the \sphinxstyleemphasis{LCZ Converter}}\label{\detokenize{pre-processor/Spatial Data LCZ Converter:id2}}\end{figure}

\begin{figure}[htbp]
\centering
\capstart

\noindent\sphinxincludegraphics{{700px-LCZdialog2}.png}
\caption{The \sphinxstyleemphasis{Pervious distribution}-tab in the \sphinxstyleemphasis{LCZ Converter}: Option 1}\label{\detokenize{pre-processor/Spatial Data LCZ Converter:id3}}\end{figure}

\begin{figure}[htbp]
\centering
\capstart

\noindent\sphinxincludegraphics{{700px-LCZdialog3}.png}
\caption{The \sphinxstyleemphasis{Pervious distribution}-tab in the \sphinxstyleemphasis{LCZ Converter}: Option 2}\label{\detokenize{pre-processor/Spatial Data LCZ Converter:id4}}\end{figure}

\end{description}

\item {} 
Dialog sections:

\end{itemize}


\begin{savenotes}\sphinxattablestart
\centering
\begin{tabular}[t]{|\X{50}{100}|\X{50}{100}|}
\hline

upper
&
Select the LCZ raster layer and the vector grid the land cover fractions should be computed for.
\\
\hline
middle Tab: Pervious distribution
&
Set the distribution of pervious surface fractions for each LCZ separately or all at the same time.
\\
\hline
middle Tab: Table
&
Alters the land cover fractions and building and vegetation heights for each LCZ towards more accurate values.
\\
\hline
lower
&
Specify output and run the calculations.
\\
\hline
\end{tabular}
\par
\sphinxattableend\end{savenotes}
\begin{itemize}
\item {} \begin{description}
\item[{LCZ raster:}] \leavevmode\begin{itemize}
\item {} 
Select the LCZ raster from the \sphinxhref{http://www.wudapt.org}{WUDAPT database}.

\end{itemize}

\end{description}

\item {} \begin{description}
\item[{Vector grid:}] \leavevmode\begin{itemize}
\item {} 
Select your predefined polygon grid (see Vector -\&gt; Research Tools -\&gt; Vector Grid; select polygons not lines)

\end{itemize}

\end{description}

\item {} \begin{description}
\item[{Adjust default parameters:}] \leavevmode\begin{itemize}
\item {} 
Tick this box if you would like to edit the table below with the land use fractions and tree and building heights for each of the local climate zones.

\end{itemize}

\end{description}

\item {} \begin{description}
\item[{Separate LCZ’s:}] \leavevmode\begin{itemize}
\item {} 
Once selected it computes the most common LCZ classes in the Raster grid and allows you to alter the pervious fractions and tree heights in the dropdown boxes to the right for each individual LCZ.

\end{itemize}

\end{description}

\end{itemize}


\begin{savenotes}\sphinxattablestart
\centering
\begin{tabular}[t]{|\X{50}{100}|\X{50}{100}|}
\hline

LCZ’s:
&
List of LCZ’s in the raster, ordered by most frequent occurrence. Select the LCZ(s) for which you would like to specify the pervious fraction.
\\
\hline
Fraction distributions:
&
Select the percentages of each pervious land cover class for the selected LCZ.
\\
\hline
Height of trees:
&
Select the range of tree heights most applicable for that LCZ.
\\
\hline
\end{tabular}
\par
\sphinxattableend\end{savenotes}
\begin{itemize}
\item {} \begin{description}
\item[{Same for all LCZ’s:}] \leavevmode\begin{itemize}
\item {} 
Allows you to alter the pervious fractions and tree heights for all urban and rural classes at the same time.

\end{itemize}

\end{description}

\end{itemize}


\begin{savenotes}\sphinxattablestart
\centering
\begin{tabular}[t]{|\X{50}{100}|\X{50}{100}|}
\hline

Urban:
&
Select the percentages of each pervious land cover class for all urban LCZ’s.
\\
\hline
Rural:
&
Select the percentages of each pervious land cover class for all rural LCZ’s. Note for rural classes you are only able to specify the distribution of tree species.
\\
\hline
Height of trees:
&
Select the range of tree heights most applicable for the urban and rural LCZ’s.
\\
\hline
\end{tabular}
\par
\sphinxattableend\end{savenotes}
\begin{itemize}
\item {} \begin{description}
\item[{Update Table:}] \leavevmode\begin{itemize}
\item {} 
This updates the table from the default values to the user-specified distributions of the pervious fractions. Please check the table, to make sure your changes have taken effect.

\end{itemize}

\end{description}

\item {} \begin{description}
\item[{File Prefix:}] \leavevmode\begin{itemize}
\item {} 
A prefix that will be included in the beginning of the output files.

\end{itemize}

\end{description}

\item {} \begin{description}
\item[{Add results to polygon grid:}] \leavevmode\begin{itemize}
\item {} 
Tick this in if you would like to save the results in the attribute table for your polygon vector grid.

\end{itemize}

\end{description}

\item {} \begin{description}
\item[{Output Folder:}] \leavevmode\begin{itemize}
\item {} 
A specified folder where result will be saved.

\end{itemize}

\end{description}

\item {} \begin{description}
\item[{Run:}] \leavevmode\begin{itemize}
\item {} 
Starts the calculation

\end{itemize}

\end{description}

\item {} \begin{description}
\item[{Close:}] \leavevmode\begin{itemize}
\item {} 
Closes the plugin.

\end{itemize}

\end{description}

\item {} \begin{description}
\item[{Output:}] \leavevmode\begin{itemize}
\item {} \begin{description}
\item[{Three files are saved after a successful run.}] \leavevmode\begin{itemize}
\item {} 
One with the landcover fractions for each grid cell

\item {} 
One with the morphometric properties for the building for each grid cell

\item {} 
One with the morphometric properties for vegetation for each grid cell

\end{itemize}

\end{description}

\end{itemize}

\end{description}

\item {} \begin{description}
\item[{Remarks:}] \leavevmode\begin{itemize}
\item {} 
Rural LCZ’s are marked as 101, 102, etc instead of A, B, etc.

\item {} 
Issues using .sdat rasters has been reported. GeoTiffs are recommended.

\end{itemize}

\end{description}

\item {} \begin{description}
\item[{References:}] \leavevmode\begin{itemize}
\item {} 
Stewart, I.D. and Oke, T.R. 2012. Local Climate Zones for urban temperature studies. Bulletin of the American Meteorological Society, 93: \sphinxhref{http://journals.ametsoc.org/doi/abs/10.1175/BAMS-D-11-00019.1}{1879-1900}.

\item {} 
Stewart, I.D., Oke, T.R., and E.S. Krayenhoff. 2014. Evaluation of the ‘local climate zone’ scheme using temperature observations and model simulations. International Journal of Climatology, 34: \sphinxhref{http://onlinelibrary.wiley.com/doi/10.1002/joc.3746/abstract}{1062-80}.

\end{itemize}

\end{description}

\end{itemize}


\section{Spatial Data: Spatial Data Downloader}
\label{\detokenize{pre-processor/Spatial Data Spatial Data Downloader:spatial-data-spatial-data-downloader}}\label{\detokenize{pre-processor/Spatial Data Spatial Data Downloader:spatialdatadownloader}}\label{\detokenize{pre-processor/Spatial Data Spatial Data Downloader::doc}}\begin{itemize}
\item {} 
Developer:

\end{itemize}


\begin{savenotes}\sphinxattablestart
\centering
\begin{tabular}[t]{|\X{50}{100}|\X{50}{100}|}
\hline

Name
&
Institution
\\
\hline
Andy Gabey
&
Reading
\\
\hline
\end{tabular}
\par
\sphinxattableend\end{savenotes}
\begin{itemize}
\item {} \begin{description}
\item[{Introduction:}] \leavevmode\begin{itemize}
\item {} 
The Spatial Data Downloader downloads geo-datasets useful for UMEP applications. Only the necessary section of the data is downloaded, so that disk use and download time are minimised.

\end{itemize}

\end{description}

\item {} \begin{description}
\item[{Dialog box:}] \leavevmode
\begin{figure}[htbp]
\centering
\capstart

\noindent\sphinxincludegraphics{{650px-Downloader}.png}
\caption{Dialog for the Spatial Data Downloader plugin}\label{\detokenize{pre-processor/Spatial Data Spatial Data Downloader:id1}}\end{figure}

\end{description}

\item {} \begin{description}
\item[{Category and available datasets:}] \leavevmode\begin{itemize}
\item {} 
Each category contains multiple datasets, which are revealed by clicking the category name. To download a dataset, select it from the list, specify the geographic extent and press “Download”

\end{itemize}

\end{description}

\item {} \begin{description}
\item[{Abstract:}] \leavevmode\begin{itemize}
\item {} 
Information about the selected dataset, including citation information.

\end{itemize}

\end{description}

\item {} \begin{description}
\item[{Bounding box:}] \leavevmode\begin{itemize}
\item {} 
The geographic extent of the region to download (maximum download size is 500x500 pixels in the case of raster data). The current QGIS canvas extent can also be used by clicking \sphinxstylestrong{Use canvas extent}

\end{itemize}

\end{description}

\item {} \begin{description}
\item[{Reproject to current project CRS:}] \leavevmode\begin{itemize}
\item {} 
The downloaded data is saved in its original CRS by default. This option reprojects the saved data to the project CRS and performs resampling, the resolution of which is controlled by the “Pixel resolution in CRS units” box.

\end{itemize}

\end{description}

\item {} \begin{description}
\item[{Get data:}] \leavevmode\begin{itemize}
\item {} 
Refreshes the catalogue of available datasets. This is also updated when QGIS starts.

\end{itemize}

\end{description}

\item {} \begin{description}
\item[{Update list:}] \leavevmode\begin{itemize}
\item {} 
Refreshes the catalogue of available datasets. This is also updated when QGIS starts.

\end{itemize}

\end{description}

\item {} \begin{description}
\item[{Close:}] \leavevmode\begin{itemize}
\item {} 
Closes the plugin.

\end{itemize}

\end{description}

\end{itemize}


\section{Spatial Data: Tree Generator}
\label{\detokenize{pre-processor/Spatial Data Tree Generator:spatial-data-tree-generator}}\label{\detokenize{pre-processor/Spatial Data Tree Generator:treegenerator}}\label{\detokenize{pre-processor/Spatial Data Tree Generator::doc}}\begin{itemize}
\item {} 
Developer:

\end{itemize}


\begin{savenotes}\sphinxattablestart
\centering
\begin{tabular}[t]{|\X{50}{100}|\X{50}{100}|}
\hline

Name
&
Institution
\\
\hline
Fredrik Lindberg
&
Gothenburg
\\
\hline
\end{tabular}
\par
\sphinxattableend\end{savenotes}
\begin{itemize}
\item {} \begin{description}
\item[{Introduction:}] \leavevmode\begin{itemize}
\item {} 
Information 3d vegetation is not a common spatial information available. The \sphinxstylestrong{Tree Generator} can be used to create or alter a vegetation CDSM and TDSM (see {\hyperref[\detokenize{Abbreviations:abbreviations}]{\sphinxcrossref{\DUrole{std,std-ref,std,std-ref}{abbreviations}}}}. Be using information from a point layer where the location of the points specifies the tree positions and the attributes sets the shape of the trees, it is possible to produce a the 3d vegetation needed for e.g. Mean radiant temperature modelling (SOLWEIG) or Urban Energy Balance modelling (SUEWS) in UMEP.

\end{itemize}

\end{description}

\item {} \begin{description}
\item[{Dialog box:}] \leavevmode
\begin{figure}[htbp]
\centering
\capstart

\noindent\sphinxincludegraphics{{Treegeneratorsolweig}.png}
\caption{The dialog for the Tree generator}\label{\detokenize{pre-processor/Spatial Data Tree Generator:id1}}\end{figure}

\end{description}

\item {} 
Dialog sections:

\end{itemize}


\begin{savenotes}\sphinxattablestart
\centering
\begin{tabular}[t]{|\X{50}{100}|\X{50}{100}|}
\hline

top
&
input data is specified
\\
\hline
bottom
&
to specify the output and to run the calculations
\\
\hline
\end{tabular}
\par
\sphinxattableend\end{savenotes}
\begin{itemize}
\item {} \begin{description}
\item[{Point vector file:}] \leavevmode\begin{itemize}
\item {} 
A point vector file including the appropriate attributes for generating the vegetation DSMs

\end{itemize}

\end{description}

\item {} 
Necessary attributes:

\end{itemize}


\begin{savenotes}\sphinxattablestart
\centering
\begin{tabular}[t]{|\X{50}{100}|\X{50}{100}|}
\hline

Tree type
&\begin{description}
\item[{Two different tree types (shapes) are currently included:}] \leavevmode\begin{itemize}
\item {} 
1 = conifer

\item {} 
2 = decidouos.

\item {} 
There is also a possibility to remove vegetation by setting tree type = 0 and with an appropriate diameter to remove all vegetation pixels from the DSMs.

\end{itemize}

\end{description}
\\
\hline
Total height
&
This is the total height of the tree from the ground (magl).
\\
\hline
Trunk height
&
This is the height up to the bottom of the canopy (magl).
\\
\hline
Diameter
&
This is the circular diameter of the tree in meter.
\\
\hline
\end{tabular}
\par
\sphinxattableend\end{savenotes}
\begin{itemize}
\item {} \begin{description}
\item[{Bollean building grid exist:}] \leavevmode\begin{itemize}
\item {} 
Tick this in if a boolen building grid exist for your model domain. This can be generated from the SOLWEIG-plugin.

\end{itemize}

\end{description}

\item {} \begin{description}
\item[{Building and Ground DSM:}] \leavevmode\begin{itemize}
\item {} 
A DSM consisting of ground and building heights.

\end{itemize}

\end{description}

\item {} \begin{description}
\item[{Ground DEM:}] \leavevmode\begin{itemize}
\item {} 
A DEM cosisting of ground heights.

\end{itemize}

\end{description}

\item {} \begin{description}
\item[{Building grid:}] \leavevmode\begin{itemize}
\item {} 
A grid where building pixels are 0 and all other pixels are 1.

\end{itemize}

\end{description}

\item {} \begin{description}
\item[{Vegetation Canopy DSM:}] \leavevmode\begin{itemize}
\item {} 
A DSM consisting of pixels with vegetation heights above ground.

\end{itemize}

\end{description}

\item {} \begin{description}
\item[{Vegetation Trunk Zone DSM:}] \leavevmode\begin{itemize}
\item {} 
A DSM (geoTIFF) consisting of pixels with vegetation trunk zone heights above ground.

\end{itemize}

\end{description}

\item {} \begin{description}
\item[{Output Folder:}] \leavevmode\begin{itemize}
\item {} 
A specified folder where the result will be saved.

\end{itemize}

\end{description}

\item {} \begin{description}
\item[{Run:}] \leavevmode\begin{itemize}
\item {} 
starts the calculations

\end{itemize}

\end{description}

\item {} \begin{description}
\item[{Close:}] \leavevmode\begin{itemize}
\item {} 
closes the plugin.

\end{itemize}

\end{description}

\item {} \begin{description}
\item[{Output:}] \leavevmode\begin{itemize}
\item {} 
Two geoTIFFs are created, one CDSM and one TDSM.

\end{itemize}

\end{description}

\item {} \begin{description}
\item[{Remarks:}] \leavevmode\begin{itemize}
\item {} 
All DSMs need to have the same extent and pixel

\item {} 
To ceate a bush, set trunk height to

\item {} 
The SOLWEIG plugin cn be used to create the boolean building grid as well as a TDSM based on a

\end{itemize}

\end{description}

\end{itemize}


\section{Urban Geometry: Sky View Factor Calculator}
\label{\detokenize{pre-processor/Urban Geometry Sky View Factor Calculator:urban-geometry-sky-view-factor-calculator}}\label{\detokenize{pre-processor/Urban Geometry Sky View Factor Calculator:skyviewfactorcalculator}}\label{\detokenize{pre-processor/Urban Geometry Sky View Factor Calculator::doc}}\begin{itemize}
\item {} 
Contributor:

\end{itemize}


\begin{savenotes}\sphinxattablestart
\centering
\begin{tabular}[t]{|\X{50}{100}|\X{50}{100}|}
\hline
\sphinxstyletheadfamily 
Name
&\sphinxstyletheadfamily 
Institution
\\
\hline
Fredrik Lindberg
&
Gothenburg
\\
\hline
Sue Grimmond
&
Reading
\\
\hline
\end{tabular}
\par
\sphinxattableend\end{savenotes}
\begin{itemize}
\item {} \begin{description}
\item[{Introduction:}] \leavevmode\begin{itemize}
\item {} 
The Sky View Factor plugin can be used to generate pixel wise sky view factor (SVF) using ground and building digital surface models (DSM). Optionally, vegetation DSMs could also be used. By definition, SVF is the ratio of the radiation received (or emitted) by a planar surface to the radiation emitted (or received) by the entire hemispheric environment (Watson and Johnson 1987). It is a dimensionless measure between zero and one, representing totally obstructed and free spaces, respectively. The methodology that is used to generate SVF here is described in Lindberg and Grimmond (2010).

\end{itemize}

\end{description}

\item {} \begin{description}
\item[{Dialog box:}] \leavevmode
\begin{figure}[htbp]
\centering
\capstart

\noindent\sphinxincludegraphics{{SVFCalculator}.png}
\caption{The dialog for the Sky View Factor calculator}\label{\detokenize{pre-processor/Urban Geometry Sky View Factor Calculator:id1}}\end{figure}

\end{description}

\item {} 
Dialog sections:

\end{itemize}


\begin{savenotes}\sphinxattablestart
\centering
\begin{tabular}[t]{|\X{50}{100}|\X{50}{100}|}
\hline

top
&
Specify input data
\\
\hline
bottom
&
Specify output data and run calculation
\\
\hline
\end{tabular}
\par
\sphinxattableend\end{savenotes}
\begin{itemize}
\item {} \begin{description}
\item[{Building and Ground DSM:}] \leavevmode\begin{itemize}
\item {} 
A DSM (geoTIFF) consisting of ground and building heights.

\end{itemize}

\end{description}

\item {} \begin{description}
\item[{Vegetation Canopy DSM:}] \leavevmode\begin{itemize}
\item {} 
A DSM (geoTIFF) consisting of pixels with vegetation heights above ground. Pixels where no vegetation is present should be set to zero.

\end{itemize}

\end{description}

\item {} \begin{description}
\item[{Vegetation Trunk Zone DSM:}] \leavevmode\begin{itemize}
\item {} 
A DSM (geoTIFF) consisting of pixels with vegetation trunk zone heights above ground. Pixels where no vegetation is present should be set to zero.

\end{itemize}

\end{description}

\item {} \begin{description}
\item[{Use Vegetation DSMs:}] \leavevmode\begin{itemize}
\item {} 
Tick this box if you want to include vegetation (trees and bushes) in the final SVF.

\end{itemize}

\end{description}

\item {} \begin{description}
\item[{Trunk Zone DSM Exist:}] \leavevmode\begin{itemize}
\item {} 
Tick this box if a trunk zone DSM already exists.

\end{itemize}

\end{description}

\item {} \begin{description}
\item[{Transmissivity of Light Through Vegetation (\%):}] \leavevmode\begin{itemize}
\item {} 
Percentage of light that is penetrating through vegetation. The default value is set to 3 \% according to Konarska et al. (2013).

\end{itemize}

\end{description}

\item {} \begin{description}
\item[{Percentage of Canopy height:}] \leavevmode\begin{itemize}
\item {} 
If a trunk zone vegetation DSM is absent, this can be generated based on the height of the Canopy DSM. The default percentage is set to 25\%.

\end{itemize}

\end{description}

\item {} \begin{description}
\item[{Output Folder:}] \leavevmode\begin{itemize}
\item {} 
Specify folder where results will be saved.

\end{itemize}

\end{description}

\item {} \begin{description}
\item[{Run:}] \leavevmode\begin{itemize}
\item {} 
Starts the calculations.

\end{itemize}

\end{description}

\item {} \begin{description}
\item[{Add Result to Project:}] \leavevmode\begin{itemize}
\item {} 
If ticked, the total SVF raster will be added to the map canvas.

\end{itemize}

\end{description}

\item {} \begin{description}
\item[{Close:}] \leavevmode\begin{itemize}
\item {} 
Closes the plugin.

\end{itemize}

\end{description}

\item {} \begin{description}
\item[{Output:}] \leavevmode\begin{itemize}
\item {} 
16 files (geoTIFF) will be saved if vegetation DSM is used. Otherwise, 5 SVFs are saved.

\end{itemize}

\end{description}

\end{itemize}


\begin{savenotes}\sphinxattablestart
\centering
\begin{tabular}[t]{|\X{50}{100}|\X{50}{100}|}
\hline
\sphinxstyletheadfamily 
File
&\sphinxstyletheadfamily 
Description
\\
\hline
SkyViewFactor.tif
&
Total SVF, i.e. amount of sky that is seen from each pixel.
\\
\hline
SVF different directions
&
Four cardinal points
\\
\hline
SVF based on various fractions
&
Only buildings, only vegetation etc. For a detailed description, see Lindberg and Grimmond (2011).
\\
\hline
\end{tabular}
\par
\sphinxattableend\end{savenotes}
\begin{itemize}
\item {} \begin{description}
\item[{Example:}] \leavevmode
\begin{figure}[htbp]
\centering
\capstart

\noindent\sphinxincludegraphics{{Output_Skyview}.jpg}
\caption{Example of (left) input data - ground and building DSM (grayscale), DSM overlaid with a canopy DSM (yellow to green). Right: the resulting SVF -light highest SVF}\label{\detokenize{pre-processor/Urban Geometry Sky View Factor Calculator:id2}}\end{figure}

\end{description}

\item {} 
Remarks:
\begin{itemize}
\item {} 
All DSMs need to have the same extent and pixel size.

\item {} 
This plugin is computationally intensive i.e. large grids will take a lot of time and very large grids will not be possible to use. Large grids e.g. larger than 4,000,000 pixels should be tiled before.

\end{itemize}

\item {} \begin{description}
\item[{References:}] \leavevmode\begin{itemize}
\item {} 
Konarska J, Lindberg F, Larsson A, Thorsson S, Holmer B (2013). Transmissivity of solar radiation through crowns of single urban trees—application for outdoor thermal comfort modelling. \sphinxhref{http://link.springer.com/article/10.1007/s00704-013-1000-3}{Theoret. Appl. Climatol., 1\textendash{}14}

\item {} 
Lindberg F, Grimmond CSB (2010) Continuous sky view factor maps from high resolution urban digital elevation models. \sphinxhref{http://www.int-res.com/abstracts/cr/v42/n3/p177-183/}{Clim Res 42:177\textendash{}183}

\item {} 
Lindberg, F., Grimmond, C.S.B., 2011a. The influence of vegetation and building morphology on shadow patterns and mean radiant temperatures in urban areas: model development and evaluation. \sphinxhref{http://link.springer.com/article/10.1007/s00704-010-0382-8}{Theoret. Appl. Climatol. 105, 311\textendash{}323}

\item {} 
Watson ID, Johnson GT (1987) Graphical estimation of skyview-factors in urban environments. \sphinxhref{http://onlinelibrary.wiley.com/doi/10.1002/joc.3370070210/abstract}{J Climatol 7: 193\textendash{}197}

\end{itemize}

\end{description}

\end{itemize}


\section{Urban Geometry: Wall Height and Aspect}
\label{\detokenize{pre-processor/Urban Geometry Wall Height and Aspect:urban-geometry-wall-height-and-aspect}}\label{\detokenize{pre-processor/Urban Geometry Wall Height and Aspect:wallheightandaspect}}\label{\detokenize{pre-processor/Urban Geometry Wall Height and Aspect::doc}}\begin{itemize}
\item {} 
Contributor:

\end{itemize}


\begin{savenotes}\sphinxattablestart
\centering
\begin{tabular}[t]{|\X{50}{100}|\X{50}{100}|}
\hline
\sphinxstyletheadfamily 
Name
&\sphinxstyletheadfamily 
Institution
\\
\hline
Fredrik Lindberg
&
Gothenburg
\\
\hline
\end{tabular}
\par
\sphinxattableend\end{savenotes}
\begin{itemize}
\item {} \begin{description}
\item[{Introduction:}] \leavevmode
-The wall height and aspect pre-processor can be used to identify wall pixels and their height from ground and building digital surface models (DSM) by using a filter as presented by Lindberg et al. (2015a). Optionally, wall aspect can also be estimated using a specific linear filter as presented by Goodwin et al. (1999) and further developed by Lindberg et al. (2015b) to obtain the wall aspect. Wall aspect is given in degrees where a north facing wall pixel has a value of zero. The output of this plugin is used in other UMEP plugins such as SEBE (Solar Energy on Building Envelopes) and height to width ratio.

\end{description}

\item {} \begin{description}
\item[{Dialog box}] \leavevmode
\begin{figure}[htbp]
\centering
\capstart

\noindent\sphinxincludegraphics{{WallHeight}.png}
\caption{The dialog for the Wall Height and Aspect calculator}\label{\detokenize{pre-processor/Urban Geometry Wall Height and Aspect:id1}}\end{figure}

\end{description}

\item {} \begin{description}
\item[{Building and Ground DSM:}] \leavevmode
A DSM (geoTIFF) consisting of ground and building heights.

\end{description}

\item {} \begin{description}
\item[{Calculate Wall Aspect:}] \leavevmode
Tick this box if you want to include estimation and output of a wall aspect grid. This calculation is computational intensive and will make your computer work for a while (depending on the size of the input DSM).

\end{description}

\item {} \begin{description}
\item[{Lower Limit for Wall Height (m):}] \leavevmode
This limit gives the lowest height of a building wall.

\end{description}

\item {} \begin{description}
\item[{Output File for Wall Aspect Raster:}] \leavevmode
Name of the output file of the aspect raster.

\end{description}

\item {} \begin{description}
\item[{Output File for Wall Height Raster:}] \leavevmode
Name of the output file of the aspect raster.

\end{description}

\item {} \begin{description}
\item[{Run:}] \leavevmode
Starts the calculations.

\end{description}

\item {} \begin{description}
\item[{Add Result to Project:}] \leavevmode
If ticked, raster(s) will be added to the map canvas.

\end{description}

\item {} \begin{description}
\item[{Close:}] \leavevmode
Closes the plugin.

\end{description}

\item {} \begin{description}
\item[{Output:}] \leavevmode
Two different files (geoTIFF) will be saved if wall aspect is calculated.

\end{description}

\item {} \begin{description}
\item[{Example:}] \leavevmode
\begin{figure}[htbp]
\centering
\capstart

\noindent\sphinxincludegraphics{{Output_Wall_Height}.jpg}
\caption{\sphinxcode{\sphinxupquote{{}`to do{}`}}}\label{\detokenize{pre-processor/Urban Geometry Wall Height and Aspect:id2}}\end{figure}

\end{description}

\item {} \begin{description}
\item[{Remarks:}] \leavevmode\begin{itemize}
\item {} 
This plugin make use of \sphinxstylestrong{Scipy} which in turn make use of \sphinxstylestrong{Pillow}. If this plugin is malfunctioning, try to install/reinstall these packages (see \sphinxhref{http://www.urban-climate.net/umep/UMEP\_Manual\#Adding\_missing\_Python\_libraries\_and\_other\_OSGeo\_functionalities}{here}).

\item {} 
\sphinxstylestrong{NOTE}: The azimuth of the wall is estimated based on a 9 meter linear feature. This implies that coarser pixel resolution gives less pixels and thus a more imprecise measure of wall azimuth as the number of pixels will be lower. It it therefore recommended that use pixel resolution not greater than 2 meter in order to obtain a reasonable result.

\item {} 
Wall pixels will be located ‘inside’ of the building footprint.

\item {} 
The aspect algorithm gives reasonable result but improvements could be made by e.g. using a vector line layer which could be used to populate the wall pixels with aspect values.

\end{itemize}

\end{description}

\item {} \begin{description}
\item[{References:}] \leavevmode\begin{itemize}
\item {} 
Goodwin NR, Coops NC, Tooke TR, Christen A, Voogt JA (2009) Characterizing urban surface cover and structure with airborne lidar technology. \sphinxhref{http://www.tandfonline.com/doi/abs/10.5589/m09-015}{Can J Remote Sens 35:297\textendash{}309}

\item {} 
Lindberg F., Jonsson, P. \& Honjo, T. and Wästberg, D. (2015) Solar energy on building envelopes - 3D modelling in a 2D environment \sphinxhref{http://www.sciencedirect.com/science/article/pii/S0038092X15001164}{Solar Energy 115 369\textendash{}378}

\item {} 
Lindberg F., Grimmond, C.S.B. and Martilli, A. (2015) Sunlit fractions on urban facets - Impact of spatial resolution and approach \sphinxhref{http://www.sciencedirect.com/science/article/pii/S221209551400090X}{Urban Climate DOI: 10.1016/j.uclim.2014.11.006}

\end{itemize}

\end{description}

\end{itemize}


\section{Urban Land Cover: Land Cover Fraction (Grid)}
\label{\detokenize{pre-processor/Urban Land Cover Land Cover Fraction (Grid):urban-land-cover-land-cover-fraction-grid}}\label{\detokenize{pre-processor/Urban Land Cover Land Cover Fraction (Grid):landcoverfraction-grid}}\label{\detokenize{pre-processor/Urban Land Cover Land Cover Fraction (Grid)::doc}}\begin{itemize}
\item {} 
Contributor:

\end{itemize}


\begin{savenotes}\sphinxattablestart
\centering
\begin{tabular}[t]{|\X{50}{100}|\X{50}{100}|}
\hline
\sphinxstyletheadfamily 
Name
&\sphinxstyletheadfamily 
Institution
\\
\hline
Fredrik Lindberg
&
Gothenburg
\\
\hline
Niklas Krave
&
Gothenburg
\\
\hline
\end{tabular}
\par
\sphinxattableend\end{savenotes}
\begin{itemize}
\item {} \begin{description}
\item[{Introduction:}] \leavevmode\begin{itemize}
\item {} 
The Land Cover Fraction (Grid) plugin calculates land cover fractions required for UMEP (see {\hyperref[\detokenize{pre-processor/Urban Land Cover Land Cover Reclassifier:landcoverreclassifier}]{\sphinxcrossref{\DUrole{std,std-ref,std,std-ref}{Land Cover Reclassifier}}}}) from a point location based on a land cover raster grid. A land cover grid suitable for the processor in UMEP can be derived using the Land Cover Classifier. The fraction will vary depending on what angle (wind direction) you are interested in. Thus, this plugin is able to derive the land cover fractions for different directions. It is the same as the Land Cover Fraction (Point) except that this plugin calculates the fractions for each polygon object in polygon vector layer. The polygons should preferable be squares or any other regular shape. To create such a grid, built in functions in QGIS can be used (see \sphinxstyleemphasis{Vector -\textgreater{} Research Tools -\textgreater{} Vector Grid…}).   \textbar{}

\end{itemize}

\end{description}

\item {} 
Location:
- The Land Cover Fraction (Grid) is located at
\begin{quote}
\begin{itemize}
\item {} 
UMEP

\end{itemize}
\begin{quote}
\begin{itemize}
\item {} 
Pre-Processor

\end{itemize}
\begin{quote}
\begin{itemize}
\item {} 
Urban Morphology

\end{itemize}
\begin{itemize}
\item {} 
Land Cover Fraction (Grid)

\end{itemize}
\end{quote}
\end{quote}
\end{quote}

\item {} \begin{description}
\item[{Dialog Box:}] \leavevmode
\begin{figure}[htbp]
\centering
\capstart

\noindent\sphinxincludegraphics{{LandCoverFractionGrid2}.png}
\caption{The dialog for the Land Cover Fraction (Grid) calculator}\label{\detokenize{pre-processor/Urban Land Cover Land Cover Fraction (Grid):id1}}\end{figure}

\end{description}

\item {} 
Dialog sections:

\end{itemize}


\begin{savenotes}\sphinxattablestart
\centering
\begin{tabular}[t]{|\X{10}{100}|\X{90}{100}|}
\hline

upper
&
Sets the parameters for the area of interest where the fractions are calculated. You also set the search interval in degrees and from where the search should take place within each grid.
\\
\hline
middle
&
Specifies the input data regarding polygon layer and the land cover raster grid that should be used.
\\
\hline
lower
&
Specifies output and runs the calculations.
\\
\hline
\end{tabular}
\par
\sphinxattableend\end{savenotes}
\begin{itemize}
\item {} \begin{description}
\item[{Search Throughout the Grid Extent :}] \leavevmode\begin{itemize}
\item {} 
Select if the search should be performed from one side of the grid to the opposite side. Select the other option (Search from Grid Centroid) if the search should start from the centroid of the grid. Setting the \sphinxstylestrong{Search distance} can then allow for the search to extent beyond the grid. See the figure below for illustration. The left performs a search using the grid extent whereas the right illustrates a search from the centroid and extending outside of the grid.

\end{itemize}

\begin{figure}[htbp]
\centering
\capstart

\noindent\sphinxincludegraphics{{Grid_Extent}.png}
\caption{The two search methods for the Search Throughout the Grid Extent option}\label{\detokenize{pre-processor/Urban Land Cover Land Cover Fraction (Grid):id2}}\end{figure}

\end{description}

\item {} \begin{description}
\item[{Wind Direction Search Interval (Degrees):}] \leavevmode\begin{itemize}
\item {} 
This decides the interval in search directions for which the morphometric parameters will be calculated.

\end{itemize}

\end{description}

\item {} \begin{description}
\item[{Vector Polygon Grid:}] \leavevmode\begin{itemize}
\item {} 
Here the grid polygon layer should be specified.

\end{itemize}

\end{description}

\item {} \begin{description}
\item[{ID Field:}] \leavevmode\begin{itemize}
\item {} 
Choose an attribute from the selected polygon layer that will be used to separate the different polygon objects from each other. An attribute field of unique numbers or letters should be used.

\end{itemize}

\end{description}

\item {} \begin{description}
\item[{Add results to polygon grid:}] \leavevmode\begin{itemize}
\item {} 
Tick this in if you would like to save a isotropic results in the attribute table for your polygon vector grid.

\end{itemize}

\end{description}

\item {} \begin{description}
\item[{UMEP Land Cover Grid:}] \leavevmode\begin{itemize}
\item {} 
An integer raster land cover grid (e.g. geoTIFF) consisting of the various land covers specified above.

\end{itemize}

\end{description}

\item {} \begin{description}
\item[{File Prefix:}] \leavevmode\begin{itemize}
\item {} 
A prefix that will be included in the beginning of the output files.

\end{itemize}

\end{description}

\item {} \begin{description}
\item[{Ignore NoData pixels:}] \leavevmode\begin{itemize}
\item {} 
Tick this in if NoData pixels should be ignored and calculation of grid should be performed eventhough NoData pixels exists within that grid. Nodata pixels are set to bare soil (6).

\end{itemize}

\end{description}

\item {} \begin{description}
\item[{Output Folder:}] \leavevmode\begin{itemize}
\item {} 
A specified folder where result will be saved.

\end{itemize}

\end{description}

\item {} \begin{description}
\item[{Run:}] \leavevmode\begin{itemize}
\item {} 
Starts the calculations.

\end{itemize}

\end{description}

\item {} \begin{description}
\item[{Close:}] \leavevmode\begin{itemize}
\item {} 
Closes the plugin.

\end{itemize}

\end{description}

\item {} \begin{description}
\item[{Output:}] \leavevmode\begin{itemize}
\item {} \begin{description}
\item[{Two different files are saved after a successful run.}] \leavevmode\begin{enumerate}
\item {} 
\sphinxstylestrong{anisotropic} results: land cover fractions for each wind direction as specified are included.

\item {} 
\sphinxstylestrong{isotropic} results: all directions are integrated into one value for each land cover fraction.

\end{enumerate}

\end{description}

\item {} 
If the raster data includes no data values within a polygon object, this grid will not be considered in the calculation.

\end{itemize}

\end{description}

\item {} \begin{description}
\item[{Remarks:}] \leavevmode\begin{itemize}
\item {} \begin{description}
\item[{Polygon grids must be squared (or rectangular) and allinged with the CRS used. This will be fixed in future versions so that any shaped grid can be used (see issue \#12 in the \sphinxhref{https://bitbucket.org/fredrik\_ucg/umep/issues}{repository}).}] \leavevmode
\begin{DUlineblock}{0em}
\item[] 
\end{DUlineblock}

\end{description}

\end{itemize}

\end{description}

\end{itemize}


\section{Urban Land Cover: Land Cover Fraction (Point)}
\label{\detokenize{pre-processor/Urban Land Cover Land Cover Fraction (Point):urban-land-cover-land-cover-fraction-point}}\label{\detokenize{pre-processor/Urban Land Cover Land Cover Fraction (Point):landcoverfraction-point}}\label{\detokenize{pre-processor/Urban Land Cover Land Cover Fraction (Point)::doc}}\begin{itemize}
\item {} 
Contributor:

\end{itemize}


\begin{savenotes}\sphinxattablestart
\centering
\begin{tabular}[t]{|\X{50}{100}|\X{50}{100}|}
\hline
\sphinxstyletheadfamily 
Name
&\sphinxstyletheadfamily 
Institution
\\
\hline
Fredrik Lindberg
&
Gothenburg
\\
\hline
Niklas Krave
&
Gothenburg
\\
\hline
\end{tabular}
\par
\sphinxattableend\end{savenotes}
\begin{itemize}
\item {} \begin{description}
\item[{Introduction:}] \leavevmode\begin{itemize}
\item {} 
The Land Cover Fraction (Point) plugin calculates land cover fractions required for UMEP (see {\hyperref[\detokenize{pre-processor/Urban Land Cover Land Cover Reclassifier:landcoverreclassifier}]{\sphinxcrossref{\DUrole{std,std-ref,std,std-ref}{Land Cover Reclassifier}}}}) from a point location based on a land cover raster grid.

\item {} 
A land cover grid suitable for the processor in UMEP can be derived using the Land Cover Classifier. The fraction will vary depending on what angle (wind direction) you are interested in. Thus, this plugin is able to derive the land cover fractions for different directions.

\end{itemize}

\end{description}

\item {} \begin{description}
\item[{Dialog box:}] \leavevmode
\begin{figure}[htbp]
\centering
\capstart

\noindent\sphinxincludegraphics{{Land_Cover_Reclass}.png}
\caption{The dialog for the Land Cover Fraction (Point) calculator}\label{\detokenize{pre-processor/Urban Land Cover Land Cover Fraction (Point):id1}}\end{figure}

\end{description}

\item {} 
Dialog section:

\end{itemize}


\begin{savenotes}\sphinxattablestart
\centering
\begin{tabular}[t]{|\X{10}{100}|\X{90}{100}|}
\hline

upper
&
Select a point on the map canvas by either clicking a location or selecting an existing point from a point vector layer.
\\
\hline
middle
&
Specify the area of interest where the morphometric parameters are calculated. Set the search interval in degrees.
\\
\hline
lower
&
Specify the input data regarding land cover fraction raster as well as specifying output and for running the calculations.
\\
\hline
\end{tabular}
\par
\sphinxattableend\end{savenotes}
\begin{itemize}
\item {} \begin{description}
\item[{Select Point on Canvas  :}] \leavevmode\begin{itemize}
\item {} 
Click to create a point from where the calculations will take place. When you click the button, the plugin will be disabled until you have clicked the map canvas.

\end{itemize}

\end{description}

\item {} \begin{description}
\item[{Use Existing Single Point Vector Layer:}] \leavevmode\begin{itemize}
\item {} 
Select if you want to use a point from a vector layer that already exists and is loaded to the QGIS-project. The Vector point layer drop down list will be enabled and include all point vector layers available.

\end{itemize}

\end{description}

\item {} \begin{description}
\item[{Generate Study Area:}] \leavevmode\begin{itemize}
\item {} 
This button is connected to the Search distance (m). When you click it, a circular polygon layer (Study area) is generated. This is the area that will be used to obtain the land cover fractions.

\end{itemize}

\end{description}

\item {} \begin{description}
\item[{Wind Direction Search Interval (Degrees):}] \leavevmode\begin{itemize}
\item {} 
This decides the interval in search directions for which the morphometric parameters will be calculated.

\end{itemize}

\end{description}

\item {} \begin{description}
\item[{UMEP Land Cover Grid:}] \leavevmode\begin{itemize}
\item {} 
A integer raster land cover grid (e.g. geoTIFF) consisting of the various land covers specified above.

\end{itemize}

\end{description}

\item {} \begin{description}
\item[{File Prefix:}] \leavevmode\begin{itemize}
\item {} 
A prefix that will be included in the beginning of the output files.

\end{itemize}

\end{description}

\item {} \begin{description}
\item[{Output Folder:}] \leavevmode\begin{itemize}
\item {} 
Where the result will be saved.

\end{itemize}

\end{description}

\item {} \begin{description}
\item[{Run:}] \leavevmode\begin{itemize}
\item {} 
Starts the calculations.

\end{itemize}

\end{description}

\item {} \begin{description}
\item[{Close:}] \leavevmode\begin{itemize}
\item {} 
Closes the plugin.

\end{itemize}

\end{description}

\item {} \begin{description}
\item[{Output:}] \leavevmode\begin{itemize}
\item {} \begin{description}
\item[{Two different files are saved after a successful run.}] \leavevmode\begin{enumerate}
\item {} 
\sphinxstylestrong{anisotropic} results: land cover fractions for each wind direction as specified are included.

\item {} 
\sphinxstylestrong{isotropic} results: all directions are integrated into one value for each land cover fraction.

\end{enumerate}

\end{description}

\end{itemize}

\end{description}

\end{itemize}


\section{Urban Land Cover: Land Cover Reclassifier}
\label{\detokenize{pre-processor/Urban Land Cover Land Cover Reclassifier:urban-land-cover-land-cover-reclassifier}}\label{\detokenize{pre-processor/Urban Land Cover Land Cover Reclassifier:landcoverreclassifier}}\label{\detokenize{pre-processor/Urban Land Cover Land Cover Reclassifier::doc}}\begin{itemize}
\item {} 
Contributor:

\end{itemize}


\begin{savenotes}\sphinxattablestart
\centering
\begin{tabular}[t]{|\X{50}{100}|\X{50}{100}|}
\hline
\sphinxstyletheadfamily 
Name
&\sphinxstyletheadfamily 
Institution
\\
\hline
Fredrik Lindberg
&
Gothenburg
\\
\hline
\end{tabular}
\par
\sphinxattableend\end{savenotes}
\begin{itemize}
\item {} \begin{description}
\item[{Introduction:}] \leavevmode\begin{itemize}
\item {} 
The Land Cover Reclassifier is a simple plugin that can be used to create a UMEP land cover raster grid. The land cover fractions included in UMEP are:

\end{itemize}

\end{description}

\end{itemize}


\begin{savenotes}\sphinxattablestart
\centering
\begin{tabular}[t]{|\X{10}{110}|\X{25}{110}|\X{75}{110}|}
\hline

1
&
Paved
&
Paved surfaces (e.g. roads, car parks)
\\
\hline
2
&
Buildings
&
Building surfaces
\\
\hline
3
&
Evergreen Trees
&
Evergreen trees and shrubs
\\
\hline
4
&
Deciduous Trees
&
Deciduous trees and shrubs
\\
\hline
5
&
Grass
&
Grass surfaces
\\
\hline
6
&
Bare soil
&
Bare soil surfaces and unmanaged land
\\
\hline
7
&
Water
&
Open water (e.g. lakes, ponds, rivers, fountain)
\\
\hline
\end{tabular}
\par
\sphinxattableend\end{savenotes}
\begin{itemize}
\item {} 
Dialog box:
\begin{quote}

\begin{figure}[htbp]
\centering
\capstart

\noindent\sphinxincludegraphics{{Landcoverreclassifier}.png}
\caption{The dialog for the Land Cover Reclassifier}\label{\detokenize{pre-processor/Urban Land Cover Land Cover Reclassifier:id1}}\end{figure}
\end{quote}

\item {} 
Dialog sections:

\end{itemize}


\begin{savenotes}\sphinxattablestart
\centering
\begin{tabular}[t]{|\X{10}{100}|\X{90}{100}|}
\hline

upper
&
Select raster land cover dataset to be reclassified into the UMEP land cover classes
\\
\hline
middle
&\begin{description}
\item[{Choose interval values to be classified into a certain UMEP land cover class.}] \leavevmode\begin{itemize}
\item {} 
Not all lines and boxes need to be filled in, but multiple lines are available in case many different intervals are to be classified as the same land cover class.

\end{itemize}

\end{description}
\\
\hline
lower
&
Specify the output file (.tiff) etc.
\\
\hline
\end{tabular}
\par
\sphinxattableend\end{savenotes}
\begin{itemize}
\item {} \begin{description}
\item[{Input raster:}] \leavevmode\begin{itemize}
\item {} 
Any valid raster dataset (float or integer) loaded into QGIS will appear in this dropdown list. Choose the one that includes your land cover information.

\end{itemize}

\end{description}

\item {} \begin{description}
\item[{Land cover classes:}] \leavevmode\begin{itemize}
\item {} 
Fill the interval values that you want to reclassify into a certain cover class. All values not included will appear as 0 in the output land cover raster. This should be avoided.

\end{itemize}

\end{description}

\item {} \begin{description}
\item[{Output file:}] \leavevmode\begin{itemize}
\item {} 
Location and filename (geoTIFF) are specified here.

\end{itemize}

\end{description}

\item {} \begin{description}
\item[{Run:}] \leavevmode\begin{itemize}
\item {} 
Starts the reclassification.

\end{itemize}

\end{description}

\item {} \begin{description}
\item[{Close:}] \leavevmode\begin{itemize}
\item {} 
Closes the plugin.

\end{itemize}

\end{description}

\end{itemize}


\section{Urban Morphology: Morphometric Calculator (Grid)}
\label{\detokenize{pre-processor/Urban Morphology Morphometric Calculator (Grid):urban-morphology-morphometric-calculator-grid}}\label{\detokenize{pre-processor/Urban Morphology Morphometric Calculator (Grid):morphometriccalculator-grid}}\label{\detokenize{pre-processor/Urban Morphology Morphometric Calculator (Grid)::doc}}\begin{itemize}
\item {} \begin{description}
\item[{How to Cite:}] \leavevmode\begin{itemize}
\item {} 
Kent et al. (2017a) unless you are including the impact of vegetation in the roughness calculations then your should cite Kent et al. (2017b).

\item {} 
Kent CW, CSB Grimmond, J Barlow, D Gatey, S Kotthaus, F Lindberg, CH Halios 2017: Evaluation of urban local-scale aerodynamic parameters: implications for the vertical profile of wind and source areas Boundary Layer Meteorology 164 183\textendash{}213 doi: {[}10.1007/s10546-017-0248-z \sphinxurl{https://link.springer.com/article/10.1007/s10546-017-0248-z}{]}

\item {} 
Kent CW, S Grimmond, D Gatey Aerodynamic roughness parameters in cities: inclusion of vegetation Journal of Wind Engineering \& Industrial Aerodynamics \sphinxurl{http://dx.doi.org/10.1016/j.jweia.2017.07.016}

\end{itemize}

\end{description}

\item {} 
Contributor:

\end{itemize}


\begin{savenotes}\sphinxattablestart
\centering
\begin{tabular}[t]{|\X{50}{100}|\X{50}{100}|}
\hline
\sphinxstyletheadfamily 
Name
&\sphinxstyletheadfamily 
Institution
\\
\hline
Christoph Kent
&
Reading
\\
\hline
Fredrik Lindberg
&
Gothenburg
\\
\hline
Brian Offerle
&
previously Indiana University; Tyrens
\\
\hline
Sue Grimmond
&
Reading
\\
\hline
Niklas Krave
&
Gothenburg
\\
\hline
\end{tabular}
\par
\sphinxattableend\end{savenotes}
\begin{itemize}
\item {} \begin{description}
\item[{Introduction:}] \leavevmode\begin{itemize}
\item {} 
The Morphometric Calculator (Grid) pre-processor calculates various morphometric parameters based on digital surface models.

\end{itemize}

\end{description}

\item {} \begin{description}
\item[{Dialog box:}] \leavevmode
\begin{figure}[htbp]
\centering
\capstart

\noindent\sphinxincludegraphics{{Morph_Calc}.png}
\caption{The dialog for the Morphometric Calculator (Grid)}\label{\detokenize{pre-processor/Urban Morphology Morphometric Calculator (Grid):id1}}\end{figure}

\end{description}

\item {} 
Dialog sections:

\end{itemize}


\begin{savenotes}\sphinxattablestart
\centering
\begin{tabular}[t]{|\X{10}{100}|\X{90}{100}|}
\hline

upper
&
Specifies the area of interest where the morphometric parameters are calculated. Set the search interval in degrees and from where the search should take place within each gri
\\
\hline
middle
&
Specifies the input data regarding 3D objects and ground as well as specifying what grid that should be used.
\\
\hline
lower
&
Specifies output and runs the calculations.
\\
\hline
\end{tabular}
\par
\sphinxattableend\end{savenotes}
\begin{itemize}
\item {} \begin{description}
\item[{Search Throughout the Grid Extent:}] \leavevmode\begin{itemize}
\item {} 
Select if the search should be performed from one side of the grid to the opposite side.

\item {} 
Select the other option (Search from Grid Centroid if the search should start from the centroid of the grid. Setting the \sphinxstylestrong{Search distance} can then allow for the search to extent beyond the grid. See the figure below for illustration. The left one performs a search using the grid extent whereas the right illustrates a search from the centroid and extending outside of the grid.

\end{itemize}

\begin{figure}[htbp]
\centering
\capstart

\noindent\sphinxincludegraphics{{Grid_Extent}.png}
\caption{The two search methods for the Search Throughout the Grid Extent option}\label{\detokenize{pre-processor/Urban Morphology Morphometric Calculator (Grid):id2}}\end{figure}

\end{description}

\item {} \begin{description}
\item[{Wind Direction Search Interval (degrees):}] \leavevmode\begin{itemize}
\item {} 
This decides the interval in search directions for which the morphometric parameters will be calculated.

\end{itemize}

\end{description}

\item {} \begin{description}
\item[{Vector Polygon Grid:}] \leavevmode\begin{itemize}
\item {} 
Here the grid polygon layer should be specified.

\end{itemize}

\end{description}

\item {} \begin{description}
\item[{ID Field:}] \leavevmode\begin{itemize}
\item {} 
Choose an attribute from the selected polygon layer that will be used to separated the different polygon objects from each other. An attribute field of unique numbers or letters should be used.

\end{itemize}

\end{description}

\item {} \begin{description}
\item[{Add results to polygon grid:}] \leavevmode\begin{itemize}
\item {} 
Tick this in if you would like to save a isotropic results in the attribute table for your polygon vector grid.

\end{itemize}

\end{description}

\item {} \begin{description}
\item[{Raster DSM (only 3D Objects) Exist:}] \leavevmode\begin{itemize}
\item {} 
Tick this in if a 3D-object DSM without ground heights is available. 3D objects (e.g. buildings) should be metres above ground.

\end{itemize}

\end{description}

\item {} \begin{description}
\item[{Raster DSM (3D Objects and Ground):}] \leavevmode\begin{itemize}
\item {} 
A raster DSM (e.g. geoTIFF) consisting of ground and e.g. building height (metres above sea level).

\end{itemize}

\end{description}

\item {} \begin{description}
\item[{Raster DEM (only Ground):}] \leavevmode\begin{itemize}
\item {} 
A DEM (e.g. geoTIFF) consisting of pixels with ground heights (metres above sea level).

\end{itemize}

\end{description}

\item {} \begin{description}
\item[{Raster DSM (only 3D Objects):}] \leavevmode\begin{itemize}
\item {} 
A DSM (e.g. geoTIFF) consisting of pixels with object (e.g. buildings or vegetation) heights above ground. Pixels where no objects are present should be set to zero.

\end{itemize}

\end{description}

\item {} \begin{description}
\item[{Roughness calculation Method:}] \leavevmode\begin{itemize}
\item {} 
Options to choose methods for roughness calculations regarding zero-plane displacement height (zd) and roughness length (z0) are available.

\end{itemize}

\end{description}

\item {} \begin{description}
\item[{File Prefix:}] \leavevmode\begin{itemize}
\item {} 
A prefix that will be included in the beginning of the output files.

\end{itemize}

\end{description}

\item {} \begin{description}
\item[{Ignore NoData pixels:}] \leavevmode\begin{itemize}
\item {} 
Tick this in if NoData pixels should be ignored and calculation of grid should be performed eventhough NoData pixels exists within that grid. Nodata pixels are set to the average pixel values of the DEM.

\end{itemize}

\end{description}

\item {} \begin{description}
\item[{Output Folder:}] \leavevmode\begin{itemize}
\item {} 
A specified folder where result will be saved.

\end{itemize}

\end{description}

\item {} \begin{description}
\item[{Run:}] \leavevmode\begin{itemize}
\item {} 
Starts the calculations

\end{itemize}

\end{description}

\item {} \begin{description}
\item[{Close:}] \leavevmode\begin{itemize}
\item {} 
Closes the plugin

\end{itemize}

\end{description}

\item {} \begin{description}
\item[{Output:}] \leavevmode\begin{itemize}
\item {} \begin{description}
\item[{Two different files are saved after a successful run.}] \leavevmode\begin{enumerate}
\item {} 
\sphinxstylestrong{Anisotropic} result where the morphometric parameters for each wind direction as selected are included.

\item {} 
\sphinxstylestrong{Isotropic} results where all directions are integrated into one value for each parameter.

\end{enumerate}

\end{description}

\end{itemize}

\end{description}

\item {} \begin{description}
\item[{Remarks:}] \leavevmode\begin{itemize}
\item {} 
All DSMs need to have the same extent and pixel size.

\item {} 
Polygon grids must be squared (or rectangular) and allinged with the CRS used. This will be fixed in future versions so that any shaped grid can be used (see issue \#12 in the \sphinxhref{https://bitbucket.org/fredrik\_ucg/umep/issues}{repository}).

\end{itemize}

\end{description}

\item {} \begin{description}
\item[{References:}] \leavevmode\begin{itemize}
\item {} 
Grimmond CSB and Oke TR (1999) Aerodynamic properties of urban areas derived from analysis of surface form. \sphinxhref{http://journals.ametsoc.org/doi/abs/10.1175/1520-0450(1999)038\%3C1262\%3AAPOUAD\%3E2.0.CO\%3B2}{J Appl Meteorol 38: 1262-1292}

\item {} 
Kent CW, CSB Grimmond, J Barlow, D Gatey, S Kotthaus, F Lindberg, CH Halios 2017: Evaluation of urban local-scale aerodynamic parameters: implications for the vertical profile of wind and source areas Boundary Layer Meteorology 164 183\textendash{}213 \sphinxhref{https://link.springer.com/article/10.1007/s10546-017-0248-z}{doi: 10.1007/s10546-017-0248-z}

\item {} 
Kent CW, S Grimmond, D Gatey Aerodynamic roughness parameters in cities: inclusion of vegetation Journal of Wind Engineering \& Industrial Aerodynamics \sphinxurl{http://dx.doi.org/10.1016/j.jweia.2017.07.016}

\end{itemize}

\end{description}

\end{itemize}


\section{Urban Morphology: Morphometric Calculator (Point)}
\label{\detokenize{pre-processor/Urban Morphology Morphometric Calculator (Point):urban-morphology-morphometric-calculator-point}}\label{\detokenize{pre-processor/Urban Morphology Morphometric Calculator (Point):morphometriccalculator-point}}\label{\detokenize{pre-processor/Urban Morphology Morphometric Calculator (Point)::doc}}\begin{itemize}
\item {} \begin{description}
\item[{How to Cite:}] \leavevmode\begin{itemize}
\item {} 
Kent et al. (2017a) unless you are including the impact of vegetation in the roughness calculations then your should cite Kent et al. (2017b).

\item {} 
Kent CW, CSB Grimmond, J Barlow, D Gatey, S Kotthaus, F Lindberg, CH Halios 2017: Evaluation of urban local-scale aerodynamic parameters: implications for the vertical profile of wind and source areas Boundary Layer Meteorology 164 183\textendash{}213 doi: {[}10.1007/s10546-017-0248-z \sphinxurl{https://link.springer.com/article/10.1007/s10546-017-0248-z}{]}

\item {} 
Kent CW, S Grimmond, D Gatey Aerodynamic roughness parameters in cities: inclusion of vegetation Journal of Wind Engineering \& Industrial Aerodynamics \sphinxurl{http://dx.doi.org/10.1016/j.jweia.2017.07.016}

\end{itemize}

\end{description}

\item {} 
Contributors:

\end{itemize}


\begin{savenotes}\sphinxattablestart
\centering
\begin{tabular}[t]{|\X{50}{100}|\X{50}{100}|}
\hline
\sphinxstyletheadfamily 
Name
&\sphinxstyletheadfamily 
Institution
\\
\hline
Fredrik Lindberg
&
Gothenburg
\\
\hline
Christoph Kent
&
Reading
\\
\hline
Brian Offerle
&
previously Indiana University; Tyrens
\\
\hline
Sue Grimmond
&
Reading
\\
\hline
Niklas Krave
&
Gothenburg
\\
\hline
\end{tabular}
\par
\sphinxattableend\end{savenotes}
\begin{itemize}
\item {} \begin{description}
\item[{Introduction:}] \leavevmode\begin{itemize}
\item {} 
The Morphometric Calculator (Point) plugin calculates various morphometric parameters based on digital surface models. These morphometric parameters are used to describe the roughness of a surface and are included in various local and mesoscale climate models (e.g. Grimmond and Oke 1999). They may vary depending on what angle (wind direction) you are interested in. Thus, this plugin is able to derive the parameters for different directions. Preferably, a ground and 3D-object DSM and DEM should be used as input data. The 3D objects are usually buildings but can also be 3D vegetation (i.e. trees and bushes). It is also possible to derive the parameters from a 3D object DSM with no ground heights.

\end{itemize}

\end{description}

\end{itemize}


\begin{savenotes}\sphinxattablestart
\centering
\begin{tabular}[t]{|\X{40}{100}|\X{60}{100}|}
\hline
\sphinxstyletheadfamily 
Morphometric parameters
&\sphinxstyletheadfamily 
Description
\\
\hline
Mean building height (Z$_{\text{H}}$)
&
Average building height measured from ground level {[}m{]}.
\\
\hline
Standard deviation of building heights (Z$_{\text{H\(\sigma\)}}$).
&
Standard deviation of building heights {[}m{]}.
\\
\hline
Maximum building height (Z$_{\text{Hmax}}$).
&
Height of the tallest building within the study area {[}m{]}
\\
\hline
Plan area index (\(\lambda\)$_{\text{P}}$)
&
Area of building surfaces relative to the total ground area.
\\
\hline
Frontal area index (\(\lambda\)$_{\text{F}}$)
&
Area of building walls normal to wind direction relative to the total ground area.
\\
\hline
Roughness length (Z$_{\text{0}}$)
&
A parameter of some vertical wind profile equations that model the horizontal mean wind speed near the ground; in the log wind profile, it is equivalent to the height at which the wind speed theoretically becomes zero {[}m{]}.
\\
\hline
Zero-plane displacement height (Z$_{\text{d}}$)
&
Height above ground where the wind speed is 0 m s$^{\text{-1}}$ as a result of obstacles to the flow such as trees or buildings {[}m{]}.
\\
\hline
\end{tabular}
\par
\sphinxattableend\end{savenotes}
\begin{itemize}
\item {} \begin{description}
\item[{Dialog box:}] \leavevmode
\begin{figure}[htbp]
\centering
\capstart

\noindent\sphinxincludegraphics{{Morph_Calc_point}.png}
\caption{The dialog for the Morphometric Calculator (Point)}\label{\detokenize{pre-processor/Urban Morphology Morphometric Calculator (Point):id1}}\end{figure}

\end{description}

\item {} 
Dialog sections:

\end{itemize}


\begin{savenotes}\sphinxattablestart
\centering
\begin{tabular}[t]{|\X{10}{100}|\X{90}{100}|}
\hline

upper
&
Select a point on the map canvas by either clicking at a location or by selecting an existing point from a point vector layer.
\\
\hline
middle upper
&\begin{itemize}
\item {} 
Set the parameters for the area of interest where the morphometric parameters are calculated.

\item {} 
Set the search interval in degrees.

\end{itemize}
\\
\hline
middle lower
&
Specify the input data regarding buildings and ground.
\\
\hline
lower
&
Specify output and run the calculations.
\\
\hline
\end{tabular}
\par
\sphinxattableend\end{savenotes}
\begin{itemize}
\item {} \begin{description}
\item[{Select Point on Canvas:}] \leavevmode\begin{itemize}
\item {} 
Click on this button to create a point from where the calculations will take place. When you click button, the plugin will be disabled until you have clicked the map canvas.

\end{itemize}

\end{description}

\item {} \begin{description}
\item[{Use Existing Single Point Vector Layer:}] \leavevmode\begin{itemize}
\item {} 
Select if you want to use a point from a vector layer that already exist and are loaded to the QGIS-project. The Vector point layer dropdown list will be enabled and include all point vector layer available.

\end{itemize}

\end{description}

\item {} \begin{description}
\item[{Generate Study Area:}] \leavevmode\begin{itemize}
\item {} 
This button is connected to the Search distance (m) and when you click the button a circular polygon layer (Study area) is generated. This is the area that will be used to obtain the morphometric parameters.

\end{itemize}

\end{description}

\item {} \begin{description}
\item[{Wind Direction Search Interval (Degrees):}] \leavevmode\begin{itemize}
\item {} 
This decides the interval in search directions for which the morphometric parameters will be calculated.

\end{itemize}

\end{description}

\item {} \begin{description}
\item[{Raster DSM (only Building or Vegetation) Exist:}] \leavevmode\begin{itemize}
\item {} 
Select if a 3D-object DSM without ground heights is available. 3D objects (e.g. buildings) should be metres above ground.

\end{itemize}

\end{description}

\item {} \begin{description}
\item[{Raster DSM (3D Object and Ground):}] \leavevmode\begin{itemize}
\item {} 
A raster DSM (e.g. geoTIFF) consisting of ground and e.g. building height (meters above sea level).

\end{itemize}

\end{description}

\item {} \begin{description}
\item[{Raster DEM (only Ground):}] \leavevmode\begin{itemize}
\item {} 
A DEM (e.g. geoTIFF) consisting of pixels with ground heights (metres above sea level).

\end{itemize}

\end{description}

\item {} \begin{description}
\item[{Raster DSM (only 3D Objects):}] \leavevmode\begin{itemize}
\item {} 
A DSM (e.g. geoTIFF) consisting of pixels with object (e.g. buildings or vegetation) heights above ground. Pixels where no objects are present should be set to zero.

\end{itemize}

\end{description}

\item {} \begin{description}
\item[{Roughness Calculation Method:}] \leavevmode\begin{itemize}
\item {} 
Options to choose methods for roughness calculations regarding zero displacement height (zd) and roughness length (z0) are available.

\end{itemize}

\end{description}

\item {} \begin{description}
\item[{File Prefix:}] \leavevmode\begin{itemize}
\item {} 
A prefix that will be included in the beginning of the output files.

\end{itemize}

\end{description}

\item {} \begin{description}
\item[{Output Folder:}] \leavevmode\begin{itemize}
\item {} 
A specified folder where result will be saved.

\end{itemize}

\end{description}

\item {} \begin{description}
\item[{Run:}] \leavevmode\begin{itemize}
\item {} 
Starts the calculation

\end{itemize}

\end{description}

\item {} \begin{description}
\item[{Close:}] \leavevmode\begin{itemize}
\item {} 
Closes the plugin.

\end{itemize}

\end{description}

\item {} \begin{description}
\item[{Output:}] \leavevmode\begin{itemize}
\item {} \begin{description}
\item[{Two different files are saved after a successful run.}] \leavevmode\begin{enumerate}
\item {} 
\sphinxstylestrong{Anisotropic} result where the morphometric parameters for each wind direction as selected are included.

\item {} 
\sphinxstylestrong{Isotropic} results where all directions are integrated into one value for each parameter.

\end{enumerate}

\end{description}

\end{itemize}

\end{description}

\item {} \begin{description}
\item[{Remarks:}] \leavevmode\begin{itemize}
\item {} 
All DSMs need to have the same extent and pixel size.

\end{itemize}

\end{description}

\item {} \begin{description}
\item[{References:}] \leavevmode\begin{itemize}
\item {} 
Grimmond CSB and Oke TR (1999) Aerodynamic properties of urban areas derived from analysis of surface form. \sphinxhref{http://journals.ametsoc.org/doi/abs/10.1175/1520-0450(1999)038\%3C1262\%3AAPOUAD\%3E2.0.CO\%3B2}{J Appl Meteorol 38: 1262-1292}

\end{itemize}

\end{description}

\end{itemize}


\section{Urban Morphology: Source Area (Point)}
\label{\detokenize{pre-processor/Urban Morphology Source Area (Point):urban-morphology-source-area-point}}\label{\detokenize{pre-processor/Urban Morphology Source Area (Point):sourcearea-point}}\label{\detokenize{pre-processor/Urban Morphology Source Area (Point)::doc}}\begin{itemize}
\item {} \begin{description}
\item[{How to Cite:}] \leavevmode\begin{itemize}
\item {} 
Kent et al. (2017a) unless you are including the impact of vegetation in the roughness calculations then your should cite Kent et al. (2017b).

\item {} 
Kent CW, CSB Grimmond, J Barlow, D Gatey, S Kotthaus, F Lindberg, CH Halios 2017: Evaluation of urban local-scale aerodynamic parameters: implications for the vertical profile of wind and source areas Boundary Layer Meteorology 164 183\textendash{}213 doi: {[}10.1007/s10546-017-0248-z \sphinxurl{https://link.springer.com/article/10.1007/s10546-017-0248-z}{]}

\item {} 
Kent CW, S Grimmond, D Gatey Aerodynamic roughness parameters in cities: inclusion of vegetation Journal of Wind Engineering \& Industrial Aerodynamics \sphinxurl{http://dx.doi.org/10.1016/j.jweia.2017.07.016}

\end{itemize}

\end{description}

\item {} 
Contributors:

\end{itemize}


\begin{savenotes}\sphinxattablestart
\centering
\begin{tabular}[t]{|\X{40}{100}|\X{60}{100}|}
\hline
\sphinxstyletheadfamily 
Name
&\sphinxstyletheadfamily 
Institution
\\
\hline
Christoph Kent
&
Reading
\\
\hline
Fredrik Lindberg
&
Gothenburg
\\
\hline
Brian Offerle
&
previously Indiana University; Fluxsense
\\
\hline
\end{tabular}
\par
\sphinxattableend\end{savenotes}
\begin{itemize}
\item {} \begin{description}
\item[{Introduction:}] \leavevmode\begin{itemize}
\item {} 
The Source Area or \sphinxstyleemphasis{Footprint Model} plugin calculates various morphometric parameters based on digital surface models and source area calculations. Footprint models can be used to determine the likely position and influence of the surface area which is contributing to a turbulent flux measurement at a specific point in time and space with imposed boundary conditions (e.g. meteorological conditions, sources/sinks of passive scalars or surface characteristics). The principle of footprint models is that the measured flux is the integral of all contributing surface elements, with a ‘footprint function’ describing the relative fractional contribution of a discretisized area.                                                                                                                                        \textbar{}

\item {} 
Two footprint models exit in UMEP: (i) the Kormann and Meixner (2001) analytical footprint model; (ii) Kljun et al. (2015).

\item {} 
The mathematical basis of Kormann and Meixner (2001) includes a stationary gradient diffusion formulation, height independent cross-wind dispersion, power law profiles of mean wind velocity and eddy diffusivity and a power law solution of the two-dimensional advection-diffusion equation. The final solution of the footprint function is calculated by fitting the power laws (mean wind and eddy diffusivity) to Monin-Obukhov similarity profiles. As with all models the limitations should be appreciated which include (but are not limited to) assumptions of Monin-Obukhov similarity theory, the use of power law profiles, assumptions of horizontally homogeneous flow and assumptions of stationarity during the meteorological or scalar variable input period (i.e. their averaging period; typically 30 \textendash{} 60 minutes).   \textbar{}

\item {} 
Two footprint models exit in UMEP: (i) the Kormann and Meixner (2001) analytical footprint model; (ii) Kljun et al.

\item {} 
Kljun et al. (2015) is a two-dimensional parameterisation for flux-footprint prediction which builds upon the footprint parameterisation of Kljun et al. (2004b) by providing the width and shape of footprint estimates, as well as explicitly considering surface roughness length. It is developed and evaluated from simulations of the backward Lagrangian stochastic particle dispersion model LPDM-B (Kljun et al., 2002) and demonstrated to be appropriate for a wide range of boundary layer conditions and measurement heights. It can therefore provide footprint estimates for a wide range of real-case applications.

\item {} 
When using pre-determined values of Z$_{\text{d}}$ and Z$_{\text{0}}$, the horizontal wind speed is calculated internally to the respective source area models. This ensures the boundary layer equations used within the models are internally consistent.

\item {} 
A ground and 3D object DSM and a DEM should be used as input data. In addition if vegetation heights above ground level (i.e. trees and bushes) are available (CDSM) this can also be used. However, a CDSM need not be used and it is also possible to only use a 3D object DSM with no ground heights.

\item {} 
Note that the source area calculations are for one iteration. For the determination of roughness parameters, several iterations are recommended until the values converge (see Kent et al. 2017a).

\end{itemize}

\end{description}

\item {} \begin{description}
\item[{Dialog box:}] \leavevmode
\begin{figure}[htbp]
\centering
\capstart

\noindent\sphinxincludegraphics{{Footprint}.png}
\caption{The dialog for the Source Area (Point) calculator}\label{\detokenize{pre-processor/Urban Morphology Source Area (Point):id1}}\end{figure}

\end{description}

\item {} 
Dialog sections:

\end{itemize}


\begin{savenotes}\sphinxattablestart
\centering
\begin{tabular}[t]{|\X{15}{100}|\X{85}{100}|}
\hline

upper
&
Select a point on the map canvas by either clicking at a location or by selecting an existing point from a point vector layer.
\\
\hline
middle upper
&\begin{itemize}
\item {} 
Select if only buildings or both buildings and ground heights are available.

\item {} 
Specify the input data for buildings and ground.

\end{itemize}
\\
\hline
middle upper 2
&\begin{itemize}
\item {} 
Select if vegetation heights are available.

\item {} 
Specify the input data for buildings and ground.

\item {} 
Specify porosity (\%) of vegetation (0\% is impermeable, 100 \% is fully porous)

\end{itemize}
\\
\hline
middle lower
&
Select input parameters to source area model: specify if a file is used, or values from the dialog box.
\\
\hline
lower
&
Specify output options and run calculations.
\\
\hline
\end{tabular}
\par
\sphinxattableend\end{savenotes}
\begin{itemize}
\item {} \begin{description}
\item[{Select Point on Canvas:}] \leavevmode\begin{itemize}
\item {} 
To create a point for where the calculations will take place. When you click the button, the plugin will be disabled until you have clicked the map canvas.

\end{itemize}

\end{description}

\item {} \begin{description}
\item[{Use Existing Single Point Vector Layer:}] \leavevmode\begin{itemize}
\item {} 
Select if you want to use a point from a vector layer that already exist and is loaded in the QGIS-project. The Vector point layer dropdown list will be enabled and include all point vector layer available.

\end{itemize}

\end{description}

\item {} \begin{description}
\item[{Raster DSM (only Building) Exist:}] \leavevmode\begin{itemize}
\item {} 
Select if a 3D-object DSM without ground heights is available. 3D objects (e.g. buildings) should be metres above ground.

\end{itemize}

\end{description}

\item {} \begin{description}
\item[{Raster DSM (3D Objects and Ground):}] \leavevmode\begin{itemize}
\item {} 
A raster DSM (e.g. geoTIFF) consisting of ground and e.g. building height (metres above sea level).

\end{itemize}

\end{description}

\item {} \begin{description}
\item[{Raster DEM (only Ground):}] \leavevmode\begin{itemize}
\item {} 
A DEM (e.g. geoTIFF) consisting of pixels with ground heights (metres above sea level).

\end{itemize}

\end{description}

\item {} \begin{description}
\item[{Vegetation Canopy DSM:}] \leavevmode\begin{itemize}
\item {} 
A CDSM (e.g. geoTIFF) consisting of pixels with vegetation heights above ground. Pixels where no objects are present should be set to zero.

\end{itemize}

\end{description}

\item {} \begin{description}
\item[{Use Input File on Specify Input Parameters:}] \leavevmode\begin{itemize}
\item {} 
An input text file (.txt or .csv) containing the required inputs to the model (see below) with associated time stamps. For example:

\fvset{hllines={, ,}}%
\begin{sphinxVerbatim}[commandchars=\\\{\}]
\PYG{n}{iy} \PYG{n+nb}{id} \PYG{n}{it} \PYG{n}{imin} \PYG{n}{z\PYGZus{}0\PYGZus{}input} \PYG{n}{z\PYGZus{}d\PYGZus{}input} \PYG{n}{z\PYGZus{}m\PYGZus{}input} \PYG{n}{sigv} \PYG{n}{Obukhov} \PYG{n}{ustar} \PYG{n+nb}{dir} \PYG{n}{h} \PYG{n}{por}
\PYG{l+m+mi}{2014} \PYG{l+m+mi}{1} \PYG{l+m+mi}{0} \PYG{l+m+mi}{0} \PYG{l+m+mf}{1.1671} \PYG{l+m+mf}{8.1697} \PYG{l+m+mf}{50.3} \PYG{l+m+mf}{1.4805} \PYG{o}{\PYGZhy{}}\PYG{l+m+mf}{5457.9644} \PYG{l+m+mf}{0.8460} \PYG{l+m+mf}{193.8650} \PYG{l+m+mf}{1000.0000} \PYG{l+m+mf}{60.0000}
\PYG{l+m+mi}{2014} \PYG{l+m+mi}{1} \PYG{l+m+mi}{0} \PYG{l+m+mi}{30} \PYG{l+m+mf}{1.4007} \PYG{l+m+mf}{9.8050} \PYG{l+m+mf}{50.3} \PYG{l+m+mf}{0.9616} \PYG{l+m+mf}{1081.7260} \PYG{l+m+mf}{0.5046} \PYG{l+m+mf}{185.5874} \PYG{l+m+mf}{1000.0000} \PYG{l+m+mf}{60.0000}
\PYG{l+m+mi}{2014} \PYG{l+m+mi}{1} \PYG{l+m+mi}{1} \PYG{l+m+mi}{0} \PYG{l+m+mf}{1.3738} \PYG{l+m+mf}{9.6168} \PYG{l+m+mf}{50.3} \PYG{l+m+mf}{0.9870} \PYG{l+m+mf}{854.9901} \PYG{l+m+mf}{0.4849} \PYG{l+m+mf}{189.0444} \PYG{l+m+mf}{1000.0000} \PYG{l+m+mf}{60.0000}
\PYG{l+m+mi}{2014} \PYG{l+m+mi}{1} \PYG{l+m+mi}{1} \PYG{l+m+mi}{30} \PYG{l+m+mf}{1.2768} \PYG{l+m+mf}{9.3872} \PYG{l+m+mf}{50.3} \PYG{l+m+mf}{1.2345} \PYG{l+m+mf}{1002.2290} \PYG{l+m+mf}{0.5876} \PYG{l+m+mf}{202.3300} \PYG{l+m+mf}{1000.0000} \PYG{l+m+mf}{60.0000}

\PYG{p}{[}\PYG{n}{Header}\PYG{p}{:} \PYG{n}{year}\PYG{p}{,} \PYG{n}{day} \PYG{n}{of} \PYG{n}{year}\PYG{p}{,} \PYG{n}{hour}\PYG{p}{,} \PYG{n}{minutes} \PYG{n}{of} \PYG{n}{averaging} \PYG{n}{period}\PYG{p}{,} \PYG{n}{roughness} \PYG{n}{length} \PYG{k}{for}
\PYG{n}{momentum}\PYG{p}{,}\PYG{n}{zero} \PYG{n}{plane} \PYG{n}{displacement} \PYG{n}{height} \PYG{k}{for} \PYG{n}{momentum}\PYG{p}{,} \PYG{n}{measurement} \PYG{n}{height} \PYG{n}{of} \PYG{n}{sensor}\PYG{p}{,}
\PYG{n}{standard} \PYG{n}{deviation} \PYG{n}{of} \PYG{n}{lateral} \PYG{n}{wind}\PYG{p}{,}\PYG{n}{Obukhov} \PYG{n}{length}\PYG{p}{,} \PYG{n}{friction} \PYG{n}{velocity}\PYG{p}{,} \PYG{n}{wind} \PYG{n}{direction}\PYG{p}{,}
\PYG{n}{boundary} \PYG{n}{layer} \PYG{n}{height}\PYG{p}{,} \PYG{n}{vegetation} \PYG{n}{porosity}\PYG{p}{]}\PYG{o}{.}
\end{sphinxVerbatim}

\end{itemize}

Note In this example, the measurement height of the sensor (z\_m\_input) is 50.3

\end{description}

\item {} 
Conditions for analysis:

\end{itemize}


\begin{savenotes}\sphinxattablestart
\centering
\begin{tabular}[t]{|\X{40}{100}|\X{60}{100}|}
\hline

Parameter/Variable
&
Defintion
\\
\hline
Roughness Length for Momentum
&
First order estimation of roughness length for momentum (Z$_{\text{0}}$) for this wind direction {[}m{]}.
\\
\hline
Zero Displacement Height for Momentum
&
First order estimation of the zero-plane displacement height for momentum (Z$_{\text{d}}$) for this wind direction. {[}m{]}.
\\
\hline
Measurement Height
&
Height of sensor above ground level {[}m{]}.
\\
\hline
Standard Deviation (sigma) of Cross Wind
&
Standard deviation of the wind in the y direction (lateral wind) {[}m s$^{\text{-1}}${]}.
\\
\hline
Obukhov Length
&
Indication of atmospheric stability for use in Monin-Obukhov similarity theory {[}m{]}.
\\
\hline
Friction Velocity
&
Shear stress represented in units of velocity for non-dimensional scaling {[}m s$^{\text{-1}}${]}.
\\
\hline
Wind Direction
&
Prevailing wind direction during averaging period {[}degrees{]}.
\\
\hline
Boundary layer height
&
Height of planetary boundary layer during averaging period {[}m{]}.
\\
\hline
Vegetation porosity
&
Aerodynamic porosity of vegetation, 0\% is impermeable, 100 \% is fully porous {[}\%{]}.
\\
\hline
Maximum Fetch Considered in metres
&
The furthest distance upwind considered in the calculation of the footprint function {[}m{]}.
\\
\hline
\end{tabular}
\par
\sphinxattableend\end{savenotes}
\begin{itemize}
\item {} \begin{description}
\item[{Footprint model:}] \leavevmode\begin{itemize}
\item {} 
Specify the footprint model to use: Kormann and Meixner (2001) or Kljun et al. (2015)

\end{itemize}

\end{description}

\item {} \begin{description}
\item[{Roughness Calculation Method:}] \leavevmode\begin{itemize}
\item {} 
Here, options to choose methods for roughness calculations regarding zero displacement height (z:sub:\sphinxcode{\sphinxupquote{d}}) and roughness length (z:sub:\sphinxcode{\sphinxupquote{0}}) are available.

\end{itemize}


\begin{savenotes}\sphinxattablestart
\centering
\begin{tabular}[t]{|\X{20}{100}|\X{80}{100}|}
\hline

RT
&
Rule of thumb (c.f. Grimmond and Oke 1998)
\\
\hline
Rau
&
Raupach (1994)
\\
\hline
Bot
&
Bottema (1998)
\\
\hline
Mac
&
MacDonald et al. (1998)
\\
\hline
Mho
&
Millward-Hopkins et al. (2011)
\\
\hline
Kan
&
Kanda et al. (2013)
\\
\hline
\end{tabular}
\par
\sphinxattableend\end{savenotes}

\end{description}

\item {} \begin{description}
\item[{File Prefix:}] \leavevmode\begin{itemize}
\item {} 
A prefix that will be included in the beginning of the output files.

\end{itemize}

\end{description}

\item {} \begin{description}
\item[{Output Folder:}] \leavevmode\begin{itemize}
\item {} 
A specified folder where result will be saved.

\end{itemize}

\end{description}

\item {} \begin{description}
\item[{Run:}] \leavevmode\begin{itemize}
\item {} 
Starts the calculations.

\end{itemize}

\end{description}

\item {} \begin{description}
\item[{Close:}] \leavevmode\begin{itemize}
\item {} 
Closes the plugin.

\end{itemize}

\end{description}

\item {} \begin{description}
\item[{Output:}] \leavevmode\begin{itemize}
\item {} \begin{description}
\item[{Two different outputs are generated:}] \leavevmode\begin{enumerate}
\item {} 
A raster grid which represents the fractional contribution of each
pixel in the array to turbulent fluxes measured at the sensor (i.e.
the footprint function). Each pixel of this grid will be of the same
order to the input grid. Because the user can determine the maximum
fetch extent that is considered, each pixel in the footprint function
is weighted as a percentage of the pixel of maximum contribution. If
the footprint model is set to run for more than one time period (i.e.
integrated over time), the footprint functions are summed and
weighted as a percentage of the pixel of maximum contribution.

\item {} 
A text file which specifies the time dimensions of measurements, the
initial aerodynamic and meteorological parameters which were input to
the model and finally the weighted geometry in the footprint and thus
the newly calculated roughness length (z:sub:\sphinxcode{\sphinxupquote{0}}) and displacement
height (z:sub:\sphinxcode{\sphinxupquote{d}}) according to the user specified method. This is of
the form:

\fvset{hllines={, ,}}%
\begin{sphinxVerbatim}[commandchars=\\\{\}]
“iy id it imin z\PYGZus{}0\PYGZus{}input z\PYGZus{}d\PYGZus{}input z\PYGZus{}m\PYGZus{}input sigv Obukhov
 ustar dir fai pai zH zMax zSdev zd z0”

 [Header: year, day of year, hour, minutes of averaging period,
  roughness length for momentum, zero plane displacement height
  for momentum, measurement height of sensor, standard deviation
  of lateral wind, Obukhov length, friction velocity, wind direction,
  building frontal area weighted according to footprint function,
  building plan area weighted according to footprint, average height
  of buildings weighted according to footprint, maximum building height,
  standard deviation of building heights, footprint specific displacement
  height for specified method, footprint specific roughness length for
  specified method]
\end{sphinxVerbatim}

\end{enumerate}

\end{description}

\end{itemize}

\end{description}

\item {} \begin{description}
\item[{Remarks:}] \leavevmode\begin{itemize}
\item {} 
All DSMs need to have the same extent and pixel size.

\item {} 
Make certain that have set the projection correctly
- After you haved opened the the GeoTiff files (in a new project), right click on the layer name
\begin{itemize}
\item {} 
Set Project CRS from this layer
- Now you are ready to start adding the source areas to the image.

\end{itemize}

\end{itemize}

\end{description}

\item {} \begin{description}
\item[{References:}] \leavevmode\begin{itemize}
\item {} \begin{description}
\item[{Footprint Model}] \leavevmode\begin{itemize}
\item {} 
Kormann R and Meixner FX (2001) An analytical footprint model for
non-neutral stratification. \sphinxhref{http://link.springer.com/article/10.1023/A:1018991015119}{Bound-Layer Meteorol, 99,
207-224}.

\item {} 
Kljun N, Calanca P, Rotach MW, Schmid HP (2015) A simple
two-dimensional parameterisation for Flux Footprint Prediction (FFP).
\sphinxhref{http://www.geosci-model-dev.net/8/3695/2015/gmd-8-3695-2015.html}{Geoscientific Model
Development.8(11):3695-713}.

\end{itemize}

\end{description}

\item {} \begin{description}
\item[{Roughness Calculations}] \leavevmode\begin{itemize}
\item {} 
Bottema M and Mestayer PG (1997) Urban roughness mapping\textendash{}validation
techniques and some first results. \sphinxhref{http://www.sciencedirect.com/science/article/pii/S0167610598000142}{J Wind Eng Ind Aerodyn, 74,
163-173}.

\item {} 
Grimmond CSB and Oke TR (1999) Aerodynamic properties of urban areas
derived from analysis of surface form. \sphinxhref{http://journals.ametsoc.org/doi/abs/10.1175/1520-0450(1999)038\%3C1262\%3AAPOUAD\%3E2.0.CO\%3B2}{J Appl Meteorol, 38,
1262-1292}.

\item {} 
Kanda M, Inagaki A, Miyamoto T, Gryschka M and Raasch S (2013) A new
aerodynamic parametrization for real urban surfaces. \sphinxhref{http://link.springer.com/article/10.1007/s10546-013-9818-x}{Bound-Layer
Meteorol, 148,
357-377}.

\item {} 
Macdonald R, Griffiths R and Hall D (1998) An improved method for the
estimation of surface roughness of obstacle arrays. \sphinxhref{http://www.sciencedirect.com/science/article/pii/S1352231097004032}{Atmos Environ,
32,
1857-1864}.

\item {} 
Millward-Hopkins J, Tomlin A, Ma L, Ingham D and Pourkashanian M
(2011) Estimating aerodynamic parameters of urban-like surfaces with
heterogeneous building heights. \sphinxhref{http://link.springer.com/article/10.1007\%2Fs10546-011-9640-2}{Bound-Layer Meteorol, 141,
443-465}.

\item {} 
Raupach M (1994) Simplified expressions for vegetation roughness
length and zero-plane displacement as functions of canopy height and
area index. \sphinxhref{http://link.springer.com/article/10.1007\%2FBF00709229}{Bound-Layer Meteorol, 71,
211-216}.

\end{itemize}

\end{description}

\end{itemize}

\end{description}

\end{itemize}


\chapter{Processor}
\label{\detokenize{processor/Processor:processor}}\label{\detokenize{processor/Processor::doc}}
This section include manuals for the models included in UMEP.

The processing tools are devided into 3 different sections:
\begin{itemize}
\item {} 
Outdoor Thermal Comfort

\item {} 
Solar Radiation

\item {} 
Urban Energy Balance

\end{itemize}

The models can be accessed from the left panel.


\section{Outdoor Thermal Comfort: ExtremeFinder}
\label{\detokenize{processor/Outdoor Thermal Comfort ExtremeFinder:outdoor-thermal-comfort-extremefinder}}\label{\detokenize{processor/Outdoor Thermal Comfort ExtremeFinder:extremefinder}}\label{\detokenize{processor/Outdoor Thermal Comfort ExtremeFinder::doc}}\begin{itemize}
\item {} 
Contributor:

\end{itemize}


\begin{savenotes}\sphinxattablestart
\centering
\begin{tabular}[t]{|\X{50}{100}|\X{50}{100}|}
\hline
\sphinxstyletheadfamily 
Name
&\sphinxstyletheadfamily 
Institution
\\
\hline
Bei Huang
&
Reading
\\
\hline
Andy Gabey
&
Reading
\\
\hline
\end{tabular}
\par
\sphinxattableend\end{savenotes}
\begin{itemize}
\item {} \begin{description}
\item[{Current Options:}] \leavevmode
Identifies extreme high events (e.g. Heat waves) and low events (e.g. Cold Waves). Designed primarily for temperature data (heat waves identified from daily maximum and mean T; cold waves from daily minimum), but can also be used to indicate potential high and low extremes in other meteorological variables.

\end{description}

\item {} \begin{description}
\item[{Data must be provided by the user, and can be:}] \leavevmode\begin{itemize}
\item {} 
Previously-downloaded WATCH data in a NetCDF (.nc) file (this can be obtained from the WATCH downloader)

\item {} 
Other NetCDF (.nc) file containing sub-daily measurements, or daily maximum/mean/minimum values. Must contain a \sphinxstylestrong{‘time}’ dimension, and variable(s) with name(s) matching those being analysed using the ExtremeFinder.

\item {} 
Text (.txt) file, daily T$_{\text{max}}$, T$_{\text{avg}}$ or T$_{\text{min}}$ (\sphinxhref{http://www.urban-climate.net/watch\_data/data\%20set\%20sample.txt}{file sample}: 1979-01-01 to 2009-12-31). Only temperature analysis can be performed using a text file.

\end{itemize}

\end{description}

\item {} \begin{description}
\item[{Method :}] \leavevmode\begin{description}
\item[{—  Basis for thresholds - set into Input.nml (namelist)}] \leavevmode\begin{itemize}
\item {} 
\sphinxhref{http://science.sciencemag.org/content/305/5686/994}{Meehl and Tebaldi (2004)}: 81st, 97.5th

\item {} 
\sphinxhref{http://www.nature.com/ngeo/journal/v3/n6/full/ngeo866.html}{Fischer and Schär (2010)}: 90th

\item {} 
\sphinxhref{https://link.springer.com/article/10.1007\%2Fs00382-013-1714-z}{Vautard et al. (2013)}: 90th

\item {} 
\sphinxhref{https://link.springer.com/article/10.1007/s00382-014-2434-8}{Schoetter et al. (2014)}: 98th

\item {} 
\sphinxhref{http://www.kirj.ee/26593/?tpl=1061\&c\_tpl=1064}{Sirje Keevallik (2015)}: 10th

\item {} 
\sphinxhref{http://onlinelibrary.wiley.com/doi/10.1002/asl.232/abstract}{A. K. Srivastava (2009)}: 3 °C

\item {} 
Busuioc et al. (2010): 5 °C

\end{itemize}

\end{description}

\end{description}

\item {} \begin{description}
\item[{Dialog box:}] \leavevmode
\begin{figure}[htbp]
\centering
\capstart

\noindent\sphinxincludegraphics{{Extremefinder3}.png}
\caption{The interface for the ExtremeFinder plugin}\label{\detokenize{processor/Outdoor Thermal Comfort ExtremeFinder:id1}}\end{figure}

\end{description}

\item {} \begin{description}
\item[{Steps to use:}] \leavevmode\begin{enumerate}
\item {} 
Select climate data: The ExtremeFinder will use all the data available in its analysis. You will be prompted for a text (.txt) or NetCDF (.nc) file:
\begin{itemize}
\item {} 
\sphinxstyleemphasis{NetCDF file}: The latitude, longitude, start and end date boxes will be populated automatically, if the data is available in the NetCDF file.

\item {} 
\sphinxstyleemphasis{Text file}: The latitude, longitude, start and end date boxes must be filled in by the user, as the information is needed in calculations:
\begin{itemize}
\item {} 
\sphinxstyleemphasis{Latitude} (degrees N) and \sphinxstyleemphasis{Longitude} (degrees E) are WGS84 co-ordinates

\item {} 
\sphinxstyleemphasis{Start} and \sphinxstyleemphasis{end date} are inclusive and must match the data extent

\end{itemize}

\end{itemize}

\item {} 
Select the \sphinxstyleemphasis{extreme event type} and the \sphinxstyleemphasis{calculation method}:
\begin{itemize}
\item {} 
Event types are either Extreme \sphinxstyleemphasis{high} (e.g. Heat wave) or \sphinxstyleemphasis{low} (e.g. Cold wave)

\item {} 
There are several different ways to identify extremes, depending on the event type

\item {} 
Choose the \sphinxstyleemphasis{meteorological variable} to analyse for extremes
\begin{itemize}
\item {} 
\sphinxstylestrong{Note:} The methods in the Extreme Finder are based on Tair and may not be appropriate for other variables

\end{itemize}

\end{itemize}

\item {} 
Select Output File: A list of extreme events will be written to the file
\begin{itemize}
\item {} 
Note: this will be overwritten if not a new name

\end{itemize}

\item {} 
Run: Performs the analysis

\end{enumerate}

\end{description}

\item {} \begin{description}
\item[{Output: Extreme events (heat waves used as example below) :}] \leavevmode\begin{enumerate}
\item {} 
Daily T$_{\text{max}}$ (or T$_{\text{avg}}$ / T$_{\text{min}}$) with time (Y= Year, X=Month)
\begin{itemize}
\item {} 
Colour gives Temperature (see key)

\item {} 
Yellow Box Highlights Heatwave (Coldwave) periods This loads the model interface dialog box:
\begin{quote}

\begin{figure}[htbp]
\centering
\capstart

\noindent\sphinxincludegraphics{{350px-TMax1}.jpg}
\caption{Heat/Cold wave periods}\label{\detokenize{processor/Outdoor Thermal Comfort ExtremeFinder:id2}}\end{figure}
\end{quote}

\end{itemize}

\item {} 
Box plot of distribution of heat (cold) wave by year.
\begin{itemize}
\item {} 
whiskers =1.5* IQR

\item {} 
outliers
- any data beyond the whiskers
\begin{quote}

\begin{figure}[htbp]
\centering
\capstart

\noindent\sphinxincludegraphics{{350px-HW_Box}.jpg}
\caption{Box-and-whisker plot of Heat/Cold wave days each year}\label{\detokenize{processor/Outdoor Thermal Comfort ExtremeFinder:id3}}\end{figure}
\end{quote}

\end{itemize}

\item {} 
Number of heat (cold) waves days per year
\begin{quote}

\begin{figure}[htbp]
\centering
\capstart

\noindent\sphinxincludegraphics{{350px-HWDays}.jpg}
\caption{Histogram showing number of Heat/Cold wave days each year}\label{\detokenize{processor/Outdoor Thermal Comfort ExtremeFinder:id4}}\end{figure}
\end{quote}

\end{enumerate}

\end{description}

\end{itemize}


\section{Outdoor Thermal Comfort: SOLWEIG}
\label{\detokenize{processor/Outdoor Thermal Comfort SOLWEIG:outdoor-thermal-comfort-solweig}}\label{\detokenize{processor/Outdoor Thermal Comfort SOLWEIG:solweig}}\label{\detokenize{processor/Outdoor Thermal Comfort SOLWEIG::doc}}\begin{itemize}
\item {} 
Contributor:

\end{itemize}


\begin{savenotes}\sphinxattablestart
\centering
\begin{tabular}[t]{|\X{50}{100}|\X{50}{100}|}
\hline
\sphinxstyletheadfamily 
Name
&\sphinxstyletheadfamily 
Institution
\\
\hline
Fredrik Lindberg
&
Gothenburg
\\
\hline
\end{tabular}
\par
\sphinxattableend\end{savenotes}
\begin{itemize}
\item {} \begin{description}
\item[{Introduction:}] \leavevmode\begin{itemize}
\item {} 
The \sphinxstylestrong{SOLWEIG} plugin can be used to calculate spatial variations of mean radiant temperature (T\textasciitilde{}mrt\textasciitilde{}) and radiant fluxes using digital surface models (DSM) and ground cover information. Optionally, vegetation DSMs could also be used. The methodology that is used to generate shadows originates from Ratti and Richens (1990) and is further developed and described in Lindberg and Grimmond (2011) and Lindberg et al. (2016). The current version of the model is 2016a.

\item {} 
The full manual of the SOLWEIG model can be found \sphinxhref{http://urban-climate.net/umep/SOLWEIG}{here}.

\end{itemize}

\end{description}

\item {} 
Related Preprocessors:
- {\hyperref[\detokenize{pre-processor/Meteorological Data MetPreprocessor:metpreprocessor}]{\sphinxcrossref{\DUrole{std,std-ref,std,std-ref}{Meteorological Data: MetPreprocessor}}}}, {\hyperref[\detokenize{pre-processor/Meteorological Data Download data (WATCH):watch}]{\sphinxcrossref{\DUrole{std,std-ref,std,std-ref}{Meteorological Data: Download data (WATCH)}}}}, {\hyperref[\detokenize{pre-processor/Urban Geometry Sky View Factor Calculator:skyviewfactorcalculator}]{\sphinxcrossref{\DUrole{std,std-ref,std,std-ref}{Urban Geometry: Sky View Factor Calculator}}}}, {\hyperref[\detokenize{pre-processor/Urban Geometry Wall Height and Aspect:wallheightandaspect}]{\sphinxcrossref{\DUrole{std,std-ref,std,std-ref}{Urban Geometry: Wall Height and Aspect}}}}, {\hyperref[\detokenize{pre-processor/Urban Land Cover Land Cover Reclassifier:landcoverreclassifier}]{\sphinxcrossref{\DUrole{std,std-ref,std,std-ref}{Urban Land Cover: Land Cover Reclassifier}}}}

\item {} \begin{description}
\item[{Dialog box:}] \leavevmode
\begin{figure}[htbp]
\centering
\capstart

\noindent\sphinxincludegraphics{{SOLWEIG}.png}
\caption{The dialog for the SOLWEIG model}\label{\detokenize{processor/Outdoor Thermal Comfort SOLWEIG:id1}}\end{figure}

\end{description}

\item {} 
Dialog sections :

\end{itemize}


\begin{savenotes}\sphinxattablestart
\centering
\begin{tabular}[t]{|\X{25}{100}|\X{75}{100}|}
\hline

Spatial data
&
Spatial input data is specified
\\
\hline
Meteorological data
&
Meteorological input data is specified, as a continuous file or specific momentary values.
\\
\hline
Environmental parameters
&
Possibilities to alter emissiveties and albedos for the different urban surfaces.
\\
\hline
Optional settings
&
Here additional setting such as including POIs (Points of Interest) is found.
\\
\hline
Human exposure parameters
&
Settings for calculating mean radiant temperature.
\\
\hline
Output maps
&
Options to choose the geotiffs to be saved for each iteration.
\\
\hline
\end{tabular}
\par
\sphinxattableend\end{savenotes}
\begin{itemize}
\item {} 
Spatial data:

\end{itemize}


\begin{savenotes}\sphinxattablestart
\centering
\begin{tabular}[t]{|\X{25}{100}|\X{75}{100}|}
\hline

Building and Ground DSM
&
A DSM consisting of ground and building heights. This dataset also decides the latitude and longitude used for the calculation of Sun position.
\\
\hline
Vegetation Canopy DSM
&
A DSM consisting of pixels with vegetation heights above ground. Pixels where no vegetation is present should be set to zero.
\\
\hline
Vegetation Trunk Zone DSM
&
A DSM (geoTIFF) consisting of pixels with vegetation trunk zone heights above ground. Pixels where no vegetation is present should be set to zero.
\\
\hline
Use vegetation scheme
&
Tick this box if you want to include vegetation (trees and bushes) in the calculations.
\\
\hline
Trunk Zone DSM Exist
&
Tick this in if a trunk zone DSM already exist.
\\
\hline
Transmissivity of Light Through Vegetation (\%)
&
Percentage of light that is penetrating through vegetation. Default value is set to 3 \% according to Konarska et al. (2013).
\\
\hline
Percent of Canopy Height
&
If a trunk zone vegetation DSM is absent, this can be generated based on the height of the Canopy DSM. The default percentage is set to 25\%.
\\
\hline
Save generated Trunk zone DSM
&
Tick this in if you want to save your TDSM that is generated.
\\
\hline
Use land cover scheme
&
Available since v2015a. Land cover grid should be in the UMEP standard format \sphinxstylestrong{except} for the two tree classes (deciduous and conifer) as the land cover grid should represent what is on the ground surface. UMEP land cover grid can be prepared in the Pre-processor.
\\
\hline
Use land cover grid to produce building grid
&
Tick this in if the building grid should be created from the land cover grid. Otherwise, a DEM including only ground heights must be added. This will then be used to derive a building grid together with the ground and building DSM.
\\
\hline
Save generated building grid
&
Tick this in if you want to save the boolean building grid that is generated.
\\
\hline
SkyViewFactor grids
&
The SOLWEIG model make use of SVFs to calculate T$_{\text{mrt}}$. The zip-file needed can be created with the SkyViewFactor calculator found in the UMEP Pre-processor.
\\
\hline
Wall height raster
&
The SOLWEIG model make use of wall height raster to calculate T$_{\text{mrt}}$. This can be calculated using the Wall height and aspect plugin found in the UMEP Pre-processor
\\
\hline
Wall aspect raster
&
The SOLWEIG model make use of wall height raster to calculate T$_{\text{mrt}}$. This can be calculated using the Wall height and aspect plugin found in the UMEP Pre-processor.
\\
\hline
\end{tabular}
\par
\sphinxattableend\end{savenotes}
\begin{itemize}
\item {} 
Meteorological data:

\end{itemize}


\begin{savenotes}\sphinxattablestart
\centering
\begin{tabular}[t]{|\X{25}{100}|\X{75}{100}|}
\hline

Use continuous meteorological dataset
&
Tick this in if a time series of data should be used. The specific format could be prepared in the UMEP Pre-processor.
\\
\hline
Estimate diffuse and direct components from global radiation
&
Tick this box if diffuse and direct shortwave radiation is unavailable. The Reindl et al. (1990) model is used to calculate diffuse radiation. Direct radiation perpendicular to the solar beam should be considered.
\\
\hline
Settings for one iteration.
&
If a meteorological dataset is not used there is a possibility to run the model for one iteration using the calendar and spin-boxes to set meteorological variables present here. The default values are for a clear Summer day at 1230 in Göteborg, Sweden.
\\
\hline
UTC offset
&
Time zone needs to be specified. Positive numbers moving east (e.g. Stockholm UTC +1).
\\
\hline
\end{tabular}
\par
\sphinxattableend\end{savenotes}
\begin{itemize}
\item {} 
Optional settings:

\end{itemize}


\begin{savenotes}\sphinxattablestart
\centering
\begin{tabular}[t]{|\X{25}{100}|\X{75}{100}|}
\hline

Include POIs
&
By ticking in the option to include POIs (Point of Interest), a vector point layer can be added and full model output are written out to text files for the specific POI. Multiple POIs can be used by including many points in the vector file. See the \sphinxhref{http://www.urban-climate.net/umep/SOLWEIG}{full manual} for more information.
\\
\hline
Adjust sky-emissivity according to Jonsson et al. (2006)
&
Tick this box to include adjustment (0.04) of sky emissivity which was present in the earlier versions of the SOLWEIG model (not recommended).
\\
\hline
Consider human as cylinder instead of box
&
Tick this box to consider man as a cylinder instead of a box according to Holmer at al. (2015).
\\
\hline
\end{tabular}
\par
\sphinxattableend\end{savenotes}
\begin{itemize}
\item {} \begin{description}
\item[{Environmental parameters:}] \leavevmode
Emissivity (ground)\textbar{}\textbar{}Emissivity of ground. Not used if land cover scheme is activated.

\end{description}

\end{itemize}


\begin{savenotes}\sphinxattablestart
\centering
\begin{tabular}[t]{|\X{25}{100}|\X{75}{100}|}
\hline

Albedo (buildings)
&
Albedo of building walls and roofs.
\\
\hline
Albedo (ground)
&
Albedo of ground surfaces. Not used if land cover scheme is active.
\\
\hline
Emissivity (walls)
&
Emissivity of building walls and roofs.
\\
\hline
Emissivity (ground)
&
Emissivity of ground. Not used if land cover scheme is activated.
\\
\hline
\end{tabular}
\par
\sphinxattableend\end{savenotes}
\begin{itemize}
\item {} \begin{description}
\item[{Human exposure parameters :}] \leavevmode
Posture of the human body\textbar{}\textbar{}Choose between standing (default) and sitting.

\end{description}

\end{itemize}


\begin{savenotes}\sphinxattablestart
\centering
\begin{tabular}[t]{|\X{25}{100}|\X{75}{100}|}
\hline

Absorption of shortwave radiation
&
Amount of shortwave radiation that the human body absorb.
\\
\hline
Absorption of longwave radiation
&
Amount of longwave radiation that the human body absorb.
\\
\hline
Posture of the human body
&
Choose between standing (default) and sitting.
\\
\hline
\end{tabular}
\par
\sphinxattableend\end{savenotes}
\begin{itemize}
\item {} \begin{description}
\item[{Output maps:}] \leavevmode
A number of different outputs can be chosen here. All grids will be written out as GeoTIFFs at the location specified as the output folder.

\end{description}

\item {} \begin{description}
\item[{Run:}] \leavevmode
Starts the calculations. As SOLWEIG is a 2.5D model, large grids (i.e. high number of pixels) will take a relatively long time to compute. The model is embedded in a so called worker which means that you can continue working with QGIS while the model runs.

\end{description}

\item {} \begin{description}
\item[{Add Average mean radiant temperature to the map canvas:}] \leavevmode
If ticked, an average T$_{\text{mrt}}$ map will be added to the current

\end{description}

\item {} \begin{description}
\item[{Close:}] \leavevmode
Closes the plugin.

\end{description}

\item {} \begin{description}
\item[{Quick example on how to run SOLWEIG:}] \leavevmode\begin{description}
\item[{presented:}] \leavevmode\begin{enumerate}
\item {} 
Download and extract (unzip) the test dataset (\sphinxhref{https://bitbucket.org/fredrik\_ucg/umep/downloads/testdata\_UMEP.zip}{testdata\_UMEP.zip}).

\item {} 
Add the raster layers (DSM, CDSM and land cover) from the Goteborg folder into a new QGIS session. The coordinate system of the grids is \sphinxstylestrong{Sweref99 1200 (EPSG:3007)}.

\item {} 
In order to run SOLWEIG, some additional datasets must be created based on the raster grids you just added. Open the SkyViewFactor Calculator from the UMEP Pre-processor and calculate SVFs using both your DSM and CDSM. Leave all other settings as default.

\item {} 
Open the Wall height and aspect plugin from the UMEP Pre-processor and calculate both wall height and aspect using the DSM and your input raster. Tick in the box to add them to your project. Leave all other settings as default.

\item {} 
Now you are ready to generate your first T$_{\text{mrt}}$ map. Open SOLWEIG and use the settings as shown in the figure below but replace the paths to the fit your computer environment. When you are finished, press \sphinxstyleemphasis{Run}.

\end{enumerate}

\end{description}

\end{description}

\end{itemize}

\begin{figure}[htbp]
\centering
\capstart

\noindent\sphinxincludegraphics{{SOLWEIGfirsttry}.png}
\caption{Setting for a first try with the SOLWEIG model}\label{\detokenize{processor/Outdoor Thermal Comfort SOLWEIG:id2}}\end{figure}

There is also a meteorological file present in the test dataset that can be used to run the model for a whole day.
\begin{itemize}
\item {} \begin{description}
\item[{Remarks :}] \leavevmode\begin{itemize}
\item {} 
All DSMs need to have the same extent and pixel size.

\item {} 
This plugin is computationally intensive i.e. large grids will take a lot of time and very large grids will not be possible to use. Large grids e.g. larger than 4000000 pixels should preferably be tiled before.

\item {} 
SOLWEIG focus on pedestrian radiation fluxes and it is not recommended to consider fluxes on building roofs.

\end{itemize}

\end{description}

\item {} \begin{description}
\item[{References:}] \leavevmode\begin{itemize}
\item {} 
Holmer, B., Lindberg, F., Rayner, D. and Thorsson, S. 2015: How to transform the standing man from a box to a cylinder \textendash{} a modified methodology to calculate mean radiant temperature in field studies and models, ICUC9 \textendash{} 9 th International Conference on Urban Climate jointly with 12th Symposium on the Urban Environment, BPH5: Human perception and new indicators. Toulouse, July 2015.

\item {} 
Konarska J, Lindberg F, Larsson A, Thorsson S, Holmer B 2013. Transmissivity of solar radiation through crowns of single urban trees—application for outdoor thermal comfort modelling. \sphinxhref{http://link.springer.com/article/10.1007/s00704-013-1000-3}{Theoret. Appl. Climatol., 1\textendash{}14}

\item {} 
Lindberg, F., Grimmond, C.S.B., 2011a. The influence of vegetation and building morphology on shadow patterns and mean radiant temperatures in urban areas: model development and evaluation. \sphinxhref{http://link.springer.com/article/10.1007/s00704-010-0382-8}{Theoret. Appl. Climatol. 105, 311\textendash{}323}

\item {} 
Riendl D.T., Beckman W.A. and Duffie J.A. (1990), Diffuse Fraction Correlations, Solar Energy, Vol. 45, No.1, pp. 1-7.

\end{itemize}

\end{description}

\end{itemize}


\section{Solar Radiation: Daily Shadow Pattern}
\label{\detokenize{processor/Solar Radiation Daily Shadow Pattern:solar-radiation-daily-shadow-pattern}}\label{\detokenize{processor/Solar Radiation Daily Shadow Pattern:dailyshadowpattern}}\label{\detokenize{processor/Solar Radiation Daily Shadow Pattern::doc}}\begin{itemize}
\item {} 
Contributor:

\end{itemize}


\begin{savenotes}\sphinxattablestart
\centering
\begin{tabular}[t]{|\X{50}{100}|\X{50}{100}|}
\hline
\sphinxstyletheadfamily 
Name
&\sphinxstyletheadfamily 
Institution
\\
\hline
Fredrik Lindberg
&
Gothenburg
\\
\hline
\end{tabular}
\par
\sphinxattableend\end{savenotes}
\begin{itemize}
\item {} \begin{description}
\item[{Introduction:}] \leavevmode\begin{itemize}
\item {} 
The \sphinxstylestrong{Shadow generator} plugin can be used to generate pixel wise shadow analysis using ground and building digital surface models (DSM). Optionally, vegetation DSMs could also be used. The methodology that is used to generate shadows originates from Ratti and Richens (1990) and is further developed and described in Lindberg and Grimmond (2011). Position of the Sun is calculated using \sphinxstylestrong{PySolar}, a python library for various Sun related applications ({[}2{]}(\sphinxurl{http://pysolar.org/})).

\end{itemize}

\end{description}

\item {} \begin{description}
\item[{Dialog box :}] \leavevmode
\begin{figure}[htbp]
\centering
\capstart

\noindent\sphinxincludegraphics{{Shadow_generator}.jpg}
\caption{The dialog for the ShadowGenerator}\label{\detokenize{processor/Solar Radiation Daily Shadow Pattern:id1}}\end{figure}

\end{description}

\item {} \begin{description}
\item[{Dialog sections:}] \leavevmode

\begin{savenotes}\sphinxattablestart
\centering
\begin{tabular}[t]{|\X{10}{100}|\X{90}{100}|}
\hline

top
&
input data is specified
\\
\hline
middle
&
setting for positioning the Sun on the hemisphere
\\
\hline
bottom
&
to specify the output and to run the calculations
\\
\hline
\end{tabular}
\par
\sphinxattableend\end{savenotes}

\end{description}

\item {} \begin{description}
\item[{Building and Ground DSM:}] \leavevmode\begin{itemize}
\item {} 
A DSM consisting of ground and building heights. This dataset also decides the latitude and longitude used for the calculation of Sun position.

\end{itemize}

\end{description}

\item {} \begin{description}
\item[{Vegetation Canopy DSM:}] \leavevmode\begin{itemize}
\item {} 
A DSM consisting of pixels with vegetation heights above ground. Pixels where no vegetation is present should be set to zero.

\end{itemize}

\end{description}

\item {} \begin{description}
\item[{Vegetation Trunk Zone DSM:}] \leavevmode\begin{itemize}
\item {} 
A DSM (geoTIFF) consisting of pixels with vegetation trunk zone heights above ground. Pixels where no vegetation is present should be set to zero.

\end{itemize}

\end{description}

\item {} \begin{description}
\item[{Use vegetation DSMs:}] \leavevmode\begin{itemize}
\item {} 
Tick this box if you want to include vegetation (trees and bushes) when shadows are generated.

\end{itemize}

\end{description}

\item {} \begin{description}
\item[{Trunk Zone DSM Exist:}] \leavevmode\begin{itemize}
\item {} 
Tick this in if a trunk zone DSM already exist.

\end{itemize}

\end{description}

\item {} \begin{description}
\item[{Transmissivity of Light Through Vegetation (\%):}] \leavevmode\begin{itemize}
\item {} 
Percentage of light that is penetrating through vegetation. Default value is set to 3 \% according to Konarska et al. (2013).

\end{itemize}

\end{description}

\item {} \begin{description}
\item[{Percent of Canopy Height:}] \leavevmode\begin{itemize}
\item {} 
If a trunk zone vegetation DSM is absent, this can be generated based on the height of the Canopy DSM. The default percentage is set to 25\%.

\end{itemize}

\end{description}

\item {} \begin{description}
\item[{Specify Data:}] \leavevmode\begin{itemize}
\item {} 
The data need to be set in the middle section.

\end{itemize}

\end{description}

\item {} \begin{description}
\item[{Cast Shadows Only Once:}] \leavevmode\begin{itemize}
\item {} 
Tick this box if you only want to cast one shadow. Below this tick box you can set the time that is needed to decide the position of the sun.

\end{itemize}

\end{description}

\item {} \begin{description}
\item[{Time Interval between Casting of each Interval:}] \leavevmode\begin{itemize}
\item {} 
If the above tick box (Cast shadows only once) is not ticked in, a number of shadows is generated based on the interval set.

\end{itemize}

\end{description}

\item {} \begin{description}
\item[{UTC Offset (Hours):}] \leavevmode\begin{itemize}
\item {} 
Time zone needs to be specified. Positive numbers moving east(e.g. Stockholm UTC +1).

\end{itemize}

\end{description}

\item {} \begin{description}
\item[{Output Folder:}] \leavevmode\begin{itemize}
\item {} 
A specified folder where the result will be saved.

\end{itemize}

\end{description}

\item {} \begin{description}
\item[{Run:}] \leavevmode\begin{itemize}
\item {} 
Starts the calculations

\end{itemize}

\end{description}

\item {} \begin{description}
\item[{Add Results to Project:}] \leavevmode\begin{itemize}
\item {} 
If ticked, the shadow raster will be added to the map canvas.

\end{itemize}

\end{description}

\item {} \begin{description}
\item[{Close:}] \leavevmode\begin{itemize}
\item {} 
Closes the plugin.

\end{itemize}

\end{description}

\item {} \begin{description}
\item[{Output:}] \leavevmode\begin{itemize}
\item {} 
If only one shadow image is generated, one geoTIFF will be produced where pixel values of zero indicates shadow and one indicates sunlit. If daily shadow casting is used (Cast shadows only once ticked off), one shadow image for each time step as well as one shadow fraction image is generated. The shadow fraction image is given in percent where 100\% meaning the a pixel is sunlit throughout the day used in the calculation.

\end{itemize}

\end{description}

\item {} 
Example of input data and result:

\end{itemize}

\begin{figure}[htbp]
\centering
\capstart

\noindent\sphinxincludegraphics{{Shadow2}.jpg}
\caption{shadow image in Gothenburg (1 m resolution), Sweden at 1 pm on the 2nd of October 2015 (daylight savings time).}\label{\detokenize{processor/Solar Radiation Daily Shadow Pattern:id2}}\end{figure}
\begin{itemize}
\item {} \begin{description}
\item[{Remarks:}] \leavevmode\begin{itemize}
\item {} 
All DSMs need to have the same extent and pixel

\item {} 
This plugin is computationally intensive i.e. large grids will take a lot of time and very large grids will not be possible to use. Large grids e.g. larger than 4000000 pixels should be tiled before.

\end{itemize}

\end{description}

\item {} \begin{description}
\item[{References :}] \leavevmode\begin{itemize}
\item {} 
Konarska J, Lindberg F, Larsson A, Thorsson S, Holmer B 2013. Transmissivity of solar radiation through crowns of single urban trees—application for outdoor thermal comfort modelling. \sphinxhref{http://link.springer.com/article/10.1007/s00704-013-1000-3}{Theoret. Appl. Climatol., 1\textendash{}14}

\item {} 
Lindberg, F., Grimmond, C.S.B., 2011a. The influence of vegetation and building morphology on shadow patterns and mean radiant temperatures in urban areas: model development and evaluation. \sphinxhref{http://link.springer.com/article/10.1007/s00704-010-0382-8}{Theoret. Appl. Climatol. 105, 311\textendash{}323}

\item {} 
Ratti CF, Richens P (1999) Urban texture analysis with image processing techniques. In: Proceedings of the CAADFutures99, Atalanta, GA

\end{itemize}

\end{description}

\end{itemize}


\section{Solar Radiation: Solar Energy on Building Envelopes (SEBE)}
\label{\detokenize{processor/Solar Radiation Solar Energy on Building Envelopes (SEBE):solar-radiation-solar-energy-on-building-envelopes-sebe}}\label{\detokenize{processor/Solar Radiation Solar Energy on Building Envelopes (SEBE)::doc}}\begin{itemize}
\item {} 
Contributor:

\end{itemize}


\begin{savenotes}\sphinxattablestart
\centering
\begin{tabular}[t]{|\X{50}{100}|\X{50}{100}|}
\hline
\sphinxstyletheadfamily 
Name
&\sphinxstyletheadfamily 
Institution
\\
\hline
Fredrik Lindberg
&
Gothenburg
\\
\hline
Dag Wäsrberg
&
Ramböll
\\
\hline
\end{tabular}
\par
\sphinxattableend\end{savenotes}
\begin{itemize}
\item {} \begin{description}
\item[{Introduction:}] \leavevmode
The \sphinxstylestrong{SEBE} plugin (Solar Energy on Building Envelopes) can be used to calculate pixel wise potential solar energy using ground and building digital surface models (DSM). SEBE is also able to estimate irradiance on building walls. Optionally, vegetation DSMs could also be used. The methodology that is used to generate irradiance is presented in Lindberg et al. (2015).

\end{description}

\item {} \begin{description}
\item[{Related Preprocessors :}] \leavevmode
{\hyperref[\detokenize{pre-processor/Meteorological Data MetPreprocessor:metpreprocessor}]{\sphinxcrossref{\DUrole{std,std-ref,std,std-ref}{Meteorological Data: MetPreprocessor}}}}, {\hyperref[\detokenize{pre-processor/Meteorological Data Download data (WATCH):watch}]{\sphinxcrossref{\DUrole{std,std-ref,std,std-ref}{Meteorological Data: Download data (WATCH)}}}}, {\hyperref[\detokenize{pre-processor/Urban Geometry Wall Height and Aspect:wallheightandaspect}]{\sphinxcrossref{\DUrole{std,std-ref,std,std-ref}{Urban Geometry: Wall Height and Aspect}}}}, {\hyperref[\detokenize{pre-processor/Urban Land Cover Land Cover Reclassifier:landcoverreclassifier}]{\sphinxcrossref{\DUrole{std,std-ref,std,std-ref}{Urban Land Cover: Land Cover Reclassifier}}}}

\end{description}

\item {} \begin{description}
\item[{Dialog box :}] \leavevmode\begin{itemize}
\item {} \begin{description}
\item[{Consists of}] \leavevmode\begin{itemize}
\item {} 
top section where input data is specified

\item {} 
bottom section for specifying the output and for running the calculations

\end{itemize}

\end{description}

\end{itemize}

\end{description}

\end{itemize}

\begin{figure}[htbp]
\centering
\capstart

\noindent\sphinxincludegraphics{{SEBE1}.png}
\caption{The dialog for the SEBE model}\label{\detokenize{processor/Solar Radiation Solar Energy on Building Envelopes (SEBE):id1}}\end{figure}
\begin{itemize}
\item {} \begin{description}
\item[{Building and Ground DSM:}] \leavevmode
A DSM consisting of ground and building heights. This dataset also decides the latitude and longitude used for the calculation of the Sun position.

\end{description}

\item {} \begin{description}
\item[{Vegetation Canopy DSM:}] \leavevmode
A DSM consisting of pixels with vegetation heights above ground. Pixels where no vegetation is present should be set to zero.

\end{description}

\item {} \begin{description}
\item[{Vegetation Trunk Zone DSM:}] \leavevmode
A DSM (geoTIFF) consisting of pixels with vegetation trunk zone heights above ground. Pixels where no vegetation is present should be set to zero.

\end{description}

\item {} \begin{description}
\item[{Use Vegetation DSMs:}] \leavevmode
Tick this box if you want to include vegetation (trees and bushes) into the analysis.

\end{description}

\item {} \begin{description}
\item[{Trunk Zone DSM Exist:}] \leavevmode
Tick this in if a trunk zone DSM already exist.

\end{description}

\item {} \begin{description}
\item[{Transmissivity of Light Through Vegetation (\%):}] \leavevmode
Percentage of light that is penetrating through vegetation. Default value is set to 3 \% according to Konarska et al. (2013).

\end{description}

\item {} \begin{description}
\item[{Percent of Canopy Height:}] \leavevmode
If a trunk zone vegetation DSM is absent, this can be generated based on the height of the Canopy DSM. The default percentage is set to 25\%.

\end{description}

\item {} \begin{description}
\item[{Wall Height Raster:}] \leavevmode
A raster of the same size and extent as the ground and building DSM including information of the wall pixels and its height in meters above ground should be specified here. Non wall pixels should be set to zero. This raster is used to estimate irradiance on building walls and can be generated using the Wall Height and Aspect plugin located at UMEP  -\textgreater{} Pre-processing  -\textgreater{} Urban Geometry  -\textgreater{} Wall Height and Aspect.

\end{description}

\item {} \begin{description}
\item[{Wall Aspect Raster:}] \leavevmode
A raster of the same size and extent as the ground and building DSM including information of the wall pixels and its aspect, i.e. angle, should be specified here. For example a wall facing towards the south has a value of 180°. Non wall pixels should be set to zero. This raster are used to estimate irradiance on building walls and can be generated using the Wall Height and Aspect plugin located at UMEP  -\textgreater{} Pre-processing  -\textgreater{} Urban Geometry  -\textgreater{} Wall Height and Aspect.

\end{description}

\item {} \begin{description}
\item[{Albedo:}] \leavevmode
This parameter specifies the reflectivity of shortwave radiation of all surfaces (ground, roofs, walls and vegetation). It should be a value between 0 and 1. The default value is set to 0.15.

\end{description}

\item {} \begin{description}
\item[{UTC Offset (Hours):}] \leavevmode
Time zone needs to be specified. Positive numbers increase when moving east (e.g. Stockholm UTC +1).

\end{description}

\item {} \begin{description}
\item[{Estimate Diffuse and Direct Shortwave Components from Global Radiation:}] \leavevmode
Tick this in if only global radiation is present. Diffuse and direct shortwave components will then be estimated from global radiation based on the statistical model presented by Reindl et al. (1990). If air temperature and relative humidity is present, the statistical model will perform better but it is able to estimate the components using only global shortwave radiation.

\end{description}

\item {} 
Input Meteorological File:
\begin{quote}

Input meteorological data specifically formatted to be used in UMEP. This specific format can be created using UMEP  -\textgreater{} Pre-processing  -\textgreater{} Meteorological data  -\textgreater{} Prepare existing data. A dataset with \sphinxstylestrong{hourly} time resolution should be used for SEBE, preferably at least \sphinxstylestrong{one year in length}. The time should be in {\hyperref[\detokenize{Abbreviations:abbreviations}]{\sphinxcrossref{\DUrole{std,std-ref,std,std-ref}{LST}}}} for the specific location to be modelled. Multiple years can also be used to improve the model outcome. Model output is dependent on the meteorological input data so if a short dataset is used, potential solar energy would be valid for that particular time period only.
\begin{itemize}
\item {} 
Mandatory data is global shortwave radiation, but the model will perform best if also diffuse and direct components are available.

\item {} 
The direct radiation component used as input in the SOLWEIG model is not the direct shortwave radiation on a horizontal surface but on a surface perpendicular to the light source. Hence, the relationship between global radiation and the two separate components are:
\begin{itemize}
\item {} 
\sphinxstyleemphasis{Global radiation = direct radiation * sin(h) + diffuse radiation}

\item {} 
where h is the sun altitude. Since diffuse and direct components of short wave radiation is not common data, it is also possible to calculate diffuse and direct shortwave radiation (see above).

\end{itemize}

\end{itemize}
\end{quote}

\item {} \begin{description}
\item[{Save Sky Irradiance Distribution:}] \leavevmode
When the box is ticked in, it is possible to save the radiation distribution from the sky vault calculated from the meteorological file. SEBE first distributes the radiation on 145 sky patches on the sky vault and then generates shadows on the DSMs based on these patches, i.e. the core loop in the model iterates 145 times. For more detailed information on this, see Lindberg et al. (2015).

\end{description}

\item {} \begin{description}
\item[{Output Folder:}] \leavevmode
A specified folder where result will be saved should be specified here. One raster showing irradiance on ground and building roofs named Energyyearroof.tif is saved as well as a text file of wall irradiance (Energyyearwall.txt). Also, the ground and building DSM is saved in the output folder to be used later in a SEBE visualization plugin (UMEP  -\textgreater{} Post-processing  -\textgreater{} Solar Energy  -\textgreater{} SEBE (Visualisation)).

\end{description}

\item {} \begin{description}
\item[{Run:}] \leavevmode
This starts the calculations.

\end{description}

\item {} \begin{description}
\item[{Add Roof and Ground Irradiance Result Raster to Project:}] \leavevmode
If this is ticked in, \sphinxstylestrong{Energyyearroof.tif} will be loaded into to the map canvas.

\end{description}

\item {} \begin{description}
\item[{Close:}] \leavevmode
This button closes the plugin.

\end{description}

\item {} \begin{description}
\item[{Output:}] \leavevmode
As mentioned earlier, three mandatory datasets are save is the model was successful. The geoTIFF \sphinxstylestrong{Energyyearroof.tif} show pixel wise total irradiance in kWh. \sphinxstylestrong{Energyyearwall.txt} show total wall irradiance for each wall column. The column voxel is decided based on the pixel resolution of the input data. Also, the ground and building DSM is saved in the output folder for later use. If the vegetation DSMs were added, one additional file (\sphinxstylestrong{Vegetationdata.txt}) including information of vegetation height and location are also saved. This file is also be used in the SBEB visualization plugin.

\end{description}

\item {} 
Example of input data and result:

\end{itemize}

\begin{figure}[htbp]
\centering
\capstart

\noindent\sphinxincludegraphics{{SEBE2}.jpg}
\caption{Input DSM (left) and irradiance image (right) in Gothenburg using data from 1977.}\label{\detokenize{processor/Solar Radiation Solar Energy on Building Envelopes (SEBE):id2}}\end{figure}
\begin{itemize}
\item {} \begin{description}
\item[{Remarks:}] \leavevmode\begin{itemize}
\item {} 
All DSMs need to have the same extent and pixel

\item {} 
This plugin is computationally intensive i.e. large grids will take a lot of time and very large grids will not be possible to use. Large grids e.g. larger than 4000000 pixels should be tiled before.

\end{itemize}

\end{description}

\item {} \begin{description}
\item[{References :}] \leavevmode\begin{itemize}
\item {} 
Konarska J, Lindberg F, Larsson A, Thorsson S, Holmer B 2013. Transmissivity of solar radiation through crowns of single urban trees—application for outdoor thermal comfort modelling. Theoret. Appl. Climatol., 1\textendash{}14 \sphinxhref{http://link.springer.com/article/10.1007/s00704-013-1000-3}{Link to Paper}

\item {} 
Lindberg, F., Jonsson, P. \& Honjo, T. and Wästberg, D. (2015) Solar energy on building envelopes - 3D modelling in a 2D environment. Solar Energy. 115 (2015) 369\textendash{}378 \sphinxhref{http://www.sciencedirect.com/science/article/pii/S0038092X15001164}{Link to Paper}

\item {} 
Reindl DT, Beckman WA, Duffie JA (1990) Diffuse fraction correlation. Sol Energy 45:1\textendash{}7. \sphinxhref{http://www.sciencedirect.com/science/article/pii/0038092X9090060P}{Link to paper}

\end{itemize}

\end{description}

\end{itemize}


\section{Urban Energy Balance: GQ$_{\text{F}}$}
\label{\detokenize{processor/Urban Energy Balance GQ:urban-energy-balance-gqf}}\label{\detokenize{processor/Urban Energy Balance GQ:gqf}}\label{\detokenize{processor/Urban Energy Balance GQ::doc}}\begin{itemize}
\item {} 
Contributor:

\end{itemize}


\begin{savenotes}\sphinxattablestart
\centering
\begin{tabular}[t]{|\X{50}{100}|\X{50}{100}|}
\hline
\sphinxstyletheadfamily 
Name
&\sphinxstyletheadfamily 
Institution
\\
\hline
Andy Gabey
&
Reading
\\
\hline
Izzy Capel Timms
&
Reading
\\
\hline
Sue Grimmond
&
Reading
\\
\hline
\end{tabular}
\par
\sphinxattableend\end{savenotes}
\begin{itemize}
\item {} \begin{description}
\item[{How to Cite:}] \leavevmode\begin{itemize}
\item {} 
Gabey A, S Grimmond, I Capel-Timms 2018: Anthropogenic Heat Flux: advisable spatial resolutions when input data are scarce Theoretical and Applied Climatology \sphinxurl{https://doi.org/10.1007/s00704-018-2367-y}

\item {} 
Lindberg F, CSB Grimmond, A Gabey, B Huang, CW Kent, T Sun, NE Theeuwes, L Järvi, H Ward, I Capel-Timms, YY Chang, P Jonsson, N Krave, DW Liu, D Meyer, KFG Olofson, JG Tan, D Wästberg, L Xue, Z Zhang 2018: Urban multiscale environmental predictor (UMEP) - An integrated tool for city-based climate services Environmental Modelling and Software 99, 70\textendash{}87 10.1016/j.envsoft.2017.09.020

\end{itemize}

\end{description}

\item {} \begin{description}
\item[{Introduction:}] \leavevmode
\sphinxhref{http://urban-climate.net/umep/GQF\_Manual}{See separate manual}

\end{description}

\item {} \begin{description}
\item[{Dialog box:}] \leavevmode
\begin{figure}[htbp]
\centering
\capstart

\noindent\sphinxincludegraphics{{GQF}.png}
\caption{Diolag for the GQF model}\label{\detokenize{processor/Urban Energy Balance GQ:id1}}\end{figure}

\end{description}

\item {} \begin{description}
\item[{Dialog sections:}] \leavevmode\begin{itemize}
\item {} \begin{description}
\item[{The model run is configured using the dialog box:}] \leavevmode\begin{itemize}
\item {} 
\sphinxstyleemphasis{Start date} and \sphinxstyleemphasis{end date}: The first and final dates for which the model should be run.

\item {} 
\sphinxstyleemphasis{Output areas}: Two options are currently available: Local authority areas and 1km grid. These select the spatial units of the model calculations.

\item {} 
\sphinxstyleemphasis{Include QF components}: The components of anthropogenic heat flux for the model to include in calculations.

\item {} 
\sphinxstyleemphasis{Output path}: A directory that houses model outputs.

\end{itemize}

\end{description}

\end{itemize}

\end{description}

\item {} \begin{description}
\item[{Model outputs ;}] \leavevmode\begin{itemize}
\item {} \begin{description}
\item[{\sphinxstylestrong{Example map}}] \leavevmode\begin{itemize}
\item {} 
The total anthropogenic heat flux for the first time step is displayed in QGIS to demonstrate model output and the output areas. In order for these areas to be displayed correctly, the coordinate reference system must be selected. The QGIS “Select CRS” screen will appear, and EPSG 27700 (British National Grid) must be chosen.

\item {} 
The layer displaying model output also contains the other contributions to QF (e.g. car transport). These can be visualised using standard QGIS methods of styling the layer according to the selected component, or inspecting the layer attributes table.

\end{itemize}

\end{description}

\item {} \begin{description}
\item[{\sphinxstylestrong{CSV files}}] \leavevmode\begin{itemize}
\item {} 
A CSV file is generated for each of the 19 contributions to QF (e.g. car travel, wastewater heating) and the total QF. Each file contains a column per output area (shown in the example map) and a row per time step. These are labelled accordingly. The filenames are abbreviated where necessary for compatibility, with the following convention used:

\end{itemize}

\end{description}

\end{itemize}

\end{description}

\end{itemize}


\begin{savenotes}\sphinxattablestart
\centering
\begin{tabular}[t]{|\X{20}{100}|\X{80}{100}|}
\hline

El
&
Electricity
\\
\hline
Gas
&
Gas
\\
\hline
Dm
&
Domestic use
\\
\hline
Id
&
Industrial use
\\
\hline
Tspt
&
Transport
\\
\hline
Unre
&
Unrestricted electricity (non-Economy 7)
\\
\hline
Eco7
&
Economy 7 electricity
\\
\hline
Everything
&
Grand total QF across all sources
\begin{itemize}
\item {} 
A “pickled” Python data object containing the results is also saved in the local temporary folder for future use with other UMEP components.

\end{itemize}
\\
\hline
\end{tabular}
\par
\sphinxattableend\end{savenotes}
\begin{itemize}
\item {} \begin{description}
\item[{References  :}] \leavevmode\begin{itemize}
\item {} 
Iamarino M, Beevers S \& Grimmond CSB (2012) High-resolution (space, time) anthropogenic heat emissions: London 1970-2025 \sphinxhref{http://doi.wiley.com/10.1002/joc.2390}{International J. of Climatology 32, 11, 1754-1767}

\item {} 
Gabey A, S Grimmond, I Capel-Timms 2018: Anthropogenic Heat Flux: advisable spatial resolutions when input data are scarce Theoretical and Applied Climatology \sphinxurl{https://doi.org/10.1007/s00704-018-2367-y}

\item {} 
Lindberg F, CSB Grimmond, A Gabey, B Huang, CW Kent, T Sun, NE Theeuwes, L Järvi, H Ward, I Capel-Timms, YY Chang, P Jonsson, N Krave, DW Liu, D Meyer, KFG Olofson, JG Tan, D Wästberg, L Xue, Z Zhang 2018: Urban multiscale environmental predictor (UMEP) - An integrated tool for city-based climate services Environmental Modelling and Software 99, 70\textendash{}87 10.1016/j.envsoft.2017.09.020

\end{itemize}

\end{description}

\end{itemize}


\section{Urban Energy Balance: LQ$_{\text{F}}$}
\label{\detokenize{processor/Urban Energy Balance LQ:urban-energy-balance-lqf}}\label{\detokenize{processor/Urban Energy Balance LQ:lqf}}\label{\detokenize{processor/Urban Energy Balance LQ::doc}}\begin{itemize}
\item {} 
Contributor:

\end{itemize}


\begin{savenotes}\sphinxattablestart
\centering
\begin{tabular}[t]{|\X{50}{100}|\X{50}{100}|}
\hline
\sphinxstyletheadfamily 
Name
&\sphinxstyletheadfamily 
Institution
\\
\hline
Andy Gabey
&
Reading
\\
\hline
Izzy Capel Timms
&
Reading
\\
\hline
Sue Grimmond
&
Reading
\\
\hline
Sam Jackson
&
Reading
\\
\hline
XY Ao
&
SIMS
\\
\hline
Bei Huang
&
Tsinghua Unviersity
\\
\hline
\end{tabular}
\par
\sphinxattableend\end{savenotes}
\begin{itemize}
\item {} \begin{description}
\item[{Introduction  :}] \leavevmode\begin{itemize}
\item {} 
\sphinxhref{http://urban-climate.net/umep/LQF\_Manual}{See separate manual}

\end{itemize}

\end{description}

\item {} \begin{description}
\item[{References  :}] \leavevmode\begin{itemize}
\item {} 
Allen, L., Lindberg, F. and Grimmond, C. (2011) Global to city scale urban anthropogenic heat flux: model and variability. \sphinxhref{http://onlinelibrary.wiley.com/doi/10.1002/joc.2210/abstract}{International Journal of Climatology 31:13, 1990-2005.}

\item {} 
Lindberg, F., Grimmond, C., Yogeswaran, N., Kotthaus, S. and Allen, L. (2013a) Impact of city changes and weather on anthropogenic heat flux in Europe 1995\textendash{}2015. \sphinxhref{http://www.sciencedirect.com/science/article/pii/S2212095513000059}{Urban Climate 4, 1-15.}

\item {} 
Gabey A, S Grimmond, I Capel-Timms 2018: Anthropogenic Heat Flux: advisable spatial resolutions when input data are scarce Theoretical and Applied Climatology \sphinxurl{https://doi.org/10.1007/s00704-018-2367-y}

\item {} 
Lindberg F, CSB Grimmond, A Gabey, B Huang, CW Kent, T Sun, NE Theeuwes, L Järvi, H Ward, I Capel-Timms, YY Chang, P Jonsson, N Krave, DW Liu, D Meyer, KFG Olofson, JG Tan, D Wästberg, L Xue, Z Zhang 2018: Urban multiscale environmental predictor (UMEP) - An integrated tool for city-based climate services Environmental Modelling and Software 99, 70\textendash{}87 \sphinxurl{https://10.1016/j.envsoft.2017.09.020}

\end{itemize}

\end{description}

\end{itemize}


\section{Urban Energy Balance: Urban Energy Balance (SUEWS, simple)}
\label{\detokenize{processor/Urban Energy Balance Urban Energy Balance (SUEWS, simple):urban-energy-balance-urban-energy-balance-suews-simple}}\label{\detokenize{processor/Urban Energy Balance Urban Energy Balance (SUEWS, simple):suewssimple}}\label{\detokenize{processor/Urban Energy Balance Urban Energy Balance (SUEWS, simple)::doc}}\begin{itemize}
\item {} 
Contributor:

\end{itemize}
\begin{quote}


\begin{savenotes}\sphinxattablestart
\centering
\begin{tabular}[t]{|\X{50}{100}|\X{50}{100}|}
\hline

Name
&
Institution
\\
\hline
Fredrik Lindberg
&
Gothenburg
\\
\hline
Sue Grimmond
&
Reading
\\
\hline
\end{tabular}
\par
\sphinxattableend\end{savenotes}
\begin{itemize}
\item {} \begin{description}
\item[{Introduction:}] \leavevmode\begin{itemize}
\item {} 
SUEWS can be run as a standalone or via UMEP (see \sphinxhref{http://urban-climate.net/umep/SUEWS}{SUEWS Manual}).

\item {} 
This plugin makes it possible to run a simplified version of the Surface Urban Energy and Water Balance Scheme (SUEWS). For a full version of the model, the SUEWS/BLUEWS (Advanced) plugin can be used. It is also available as a separate program.

\item {} 
SUEWS (Järvi et al. 2011, 2014, Ward et al. 2016a, b) simulates the urban radiation, energy and water balances using commonly measured/modeled meteorological variables and information about the surface cover. It utilizes an evaporation-interception approach (Grimmond et al. 1991), similar to that used in forests, to model evaporation from urban surfaces.

\item {} 
The model uses seven surface types: paved, buildings, evergreen trees/shrubs, deciduous trees/shrubs, grass, bare soil and water. The surface state for each surface type at each time step is calculated from the running water balance of the canopy where the evaporation is calculated from the Penman-Monteith equation. The soil moisture below each surface type (excluding water) is taken into account.

\item {} \begin{description}
\item[{The model distributed with this manual can be run in two standard ways:}] \leavevmode\begin{itemize}
\item {} 
For an individual area

\item {} 
For multiple areas that are contiguous. There is no requirement for the areas to be of any particular shape but here we refer to them as ‘grids’.

\end{itemize}

\end{description}

\item {} 
Model applicability: Local scale \textendash{} so forcing data should be above the height of the roughness elements (trees, buildings). SUEWS Simple is designed to be executed for a single location but the model is also able to be executed on a grid.

\end{itemize}

\end{description}

\end{itemize}
\end{quote}
\begin{itemize}
\item {} \begin{description}
\item[{Related Preprocessors:}] \leavevmode\begin{itemize}
\item {} 
\sphinxcode{\sphinxupquote{MetaPreprocessor}}, {\hyperref[\detokenize{pre-processor/Meteorological Data Download data (WATCH):watch}]{\sphinxcrossref{\DUrole{std,std-ref,std,std-ref}{Meteorological Data: Download data (WATCH)}}}}, {\hyperref[\detokenize{pre-processor/Urban Land Cover Land Cover Reclassifier:landcoverreclassifier}]{\sphinxcrossref{\DUrole{std,std-ref,std,std-ref}{Urban Land Cover: Land Cover Reclassifier}}}}, {\hyperref[\detokenize{pre-processor/Urban Land Cover Land Cover Fraction (Point):landcoverfraction-point}]{\sphinxcrossref{\DUrole{std,std-ref,std,std-ref}{Urban Land Cover: Land Cover Fraction (Point)}}}}, {\hyperref[\detokenize{pre-processor/Urban Morphology Morphometric Calculator (Point):morphometriccalculator-point}]{\sphinxcrossref{\DUrole{std,std-ref,std,std-ref}{Urban Morphology: Morphometric Calculator (Point)}}}}, {\hyperref[\detokenize{pre-processor/Urban Morphology Source Area (Point):sourcearea-point}]{\sphinxcrossref{\DUrole{std,std-ref,std,std-ref}{Urban Morphology: Source Area (Point)}}}}

\end{itemize}

\end{description}

\item {} \begin{description}
\item[{Dialog Box:}] \leavevmode
\begin{figure}[htbp]
\centering
\capstart

\noindent\sphinxincludegraphics{{SuewsSimple}.png}
\caption{Dialog for the SUEWS Simple plugin}\label{\detokenize{processor/Urban Energy Balance Urban Energy Balance (SUEWS, simple):id1}}\end{figure}

\end{description}

\item {} \begin{description}
\item[{Dialog sections:}] \leavevmode

\begin{savenotes}\sphinxattablestart
\centering
\begin{tabular}[t]{|\X{10}{100}|\X{90}{100}|}
\hline

far right
&
provides some tips and tricks for running the model.
\\
\hline
other four
&
to specify user-defined input data, either manually or by using the appropriate UMEP-plugin in the per-processor.
\\
\hline
bottom
&
to make some additional settings as well as running the model.
\\
\hline
\end{tabular}
\par
\sphinxattableend\end{savenotes}

\end{description}

\item {} \begin{description}
\item[{Prepared dataset:}] \leavevmode
SUEWS Simple comes with a prepared dataset that can be used for testing. This can be utilized by pressing \sphinxstylestrong{Add settings from test dataset}. This dataset is a fictitious dataset from the central parts of London.

\end{description}

\item {} \begin{description}
\item[{Building Morphology:}] \leavevmode\begin{itemize}
\item {} 
The three site specific building morphology parameters needed are usually derived from Digital Surface Models DSMs. However, they also can be entered manually.
\begin{itemize}
\item {} 
To use an already generated text file from the Image Morphometric Calculator (Point) plugin.

\item {} 
To open the plugin from SUEWS Simple and generate the data.

\end{itemize}

\item {} 
If an already generated text file is used, the \sphinxstylestrong{isotropic file} should be used (see Image Morphometric Calculator (Point)).

\end{itemize}

\end{description}

\item {} \begin{description}
\item[{Tree Morphology:}] \leavevmode\begin{itemize}
\item {} 
Three site specific tree morphology parameters need to be specified. These can be derived from a Canopy DSMs that include vegetation heights. This can be entered manually or from the Image Morphometric Calculator (Point) plugin. When the plugin is used there are two options:
\begin{itemize}
\item {} 
To use an already generated text file from the Image Morphometric Calculator (Point) plugin.

\item {} 
To open the plugin from SUEWS Simple and generate the data.

\end{itemize}

\item {} 
If an already generated text file is used, the \sphinxstylestrong{isotropic file} should be used (see Image Morphometric Calculator (Point)).

\end{itemize}

\end{description}

\item {} \begin{description}
\item[{Land Cover Fractions :}] \leavevmode
Land cover fractions should add up to a total of 1. Values can be derived from a UMEP land cover dataset which can be generated via the Land Cover Reclassifier plugin in UMEP. The values can be entered manually or directly from the Land Cover Fraction (Point) plugin. If the plugin is used, there are two options:
\begin{itemize}
\item {} 
To use an already generated text file from the Land Cover Fraction (Point) plugin.

\item {} 
To open the plugin from SUEWS Simple and generate the data.

\end{itemize}

\end{description}

\item {} \begin{description}
\item[{Initial Conditions:}] \leavevmode
The initial conditions are entered here. These relate to time of year, days since rain, soil moisture state and daily mean air temperature at the beginning of a model run. The state of the leaf cycle sets a rough estimate of leaf area index based on season. To adjust this in more detail, the SUEWS, BLUEWS (Advanced) plugin should be used.

\end{description}

\item {} \begin{description}
\item[{Meteorological File:}] \leavevmode
The location and filename (.txt) of the meteorological file should be specified here. The format used in most UMEP-related plugins where meteorological data is required can be generated using the Metdata Processor in UMEP. For details, see the help section in the Metdata Processor or the SUEWS manual (Ward et al. 2016a).

\end{description}

\item {} \begin{description}
\item[{Output Folder:}] \leavevmode\begin{itemize}
\item {} 
Specify a folder where you would like all the model results to be saved to. Make sure that you have write capabilities to the specified folder.

\item {} 
\sphinxstyleemphasis{Note if you put it within the UMEP plugin folder\textendash{} be careful that you do not lose any results if you update the plugin by deleting it first.}

\end{itemize}

\end{description}

\item {} \begin{description}
\item[{Year:}] \leavevmode
Specify what year you are running.

\end{description}

\item {} \begin{description}
\item[{Latitude:}] \leavevmode
Specify the latitude in decimal degrees. Positive numbers indicate Northern Hemisphere.

\end{description}

\item {} \begin{description}
\item[{Longitude:}] \leavevmode
Specify the longitude in decimal degrees. Positive numbers are to the West.

\end{description}

\item {} \begin{description}
\item[{Population Density:}] \leavevmode
Specify the population density in people/ha (hectare) around the area of interest.

\end{description}

\item {} \begin{description}
\item[{Show Basic Plots of Model Results:}] \leavevmode
Tick this box in if you would like to generate some simple plots of the result from a model run. This requires that the matplotlib library is added to your QGIS installation.

\end{description}

\item {} \begin{description}
\item[{Add Settings from Test Dataset:}] \leavevmode
This is recommended if you want to try the model for the first time. This uses a year long dataset from London, UK.

\end{description}

\item {} \begin{description}
\item[{Run:}] \leavevmode
Button starts the model. All inputs must be set prior to this button being available.

\end{description}

\item {} \begin{description}
\item[{Close:}] \leavevmode\begin{itemize}
\item {} 
Button closes the plugin.

\end{itemize}

\end{description}

\item {} \begin{description}
\item[{References:}] \leavevmode\begin{itemize}
\item {} 
Järvi L, Grimmond CSB \& Christen A (2011) The Surface Urban Energy and Water Balance Scheme (SUEWS): Evaluation in Los Angeles and Vancouver \sphinxhref{http://www.sciencedirect.com/science/article/pii/S0022169411006937}{J. Hydrol. 411, 219-237.}

\item {} 
Järvi L, Grimmond CSB, Taka M, Nordbo A, Setälä H \&Strachan IB (2014) Development of the Surface Urban Energy and Water balance Scheme (SUEWS) for cold climate cities, Geosci. Model Dev. 7, 1691-1711, \sphinxhref{http://www.geosci-model-dev.net/7/1691/2014/}{doi:10.5194/gmd-7-1691-2014}.                                                                                                                                                                                                                                                                        \textbar{}

\item {} 
Ward HC, L Järvi, S Onomura, F Lindberg, CSB Grimmond (2016a) \sphinxhref{http://urban-climate.net/umep/SUEWS}{SUEWS Manual}: Version 2016a

\item {} 
Ward HC. S Kotthaus, L Järvi, CSB Grimmond (2016b) Surface Urban Energy and Water Balance Scheme (SUEWS): development and evaluation at two UK sites \sphinxhref{:File:SUEWS\_UKEvaluationPaper\_Revised\_v1-03.pdf}{Urban Climate (in press)}.

\end{itemize}

\end{description}

\end{itemize}


\section{Urban Energy Balance: Urban Energy Balance (SUEWS/BLUEWS, advanced)}
\label{\detokenize{processor/Urban Energy Balance Urban Energy Balance (SUEWS.BLUEWS, advanced):urban-energy-balance-urban-energy-balance-suews-bluews-advanced}}\label{\detokenize{processor/Urban Energy Balance Urban Energy Balance (SUEWS.BLUEWS, advanced):suewsadvanced}}\label{\detokenize{processor/Urban Energy Balance Urban Energy Balance (SUEWS.BLUEWS, advanced)::doc}}\begin{itemize}
\item {} 
Contributor:

\end{itemize}


\begin{savenotes}\sphinxattablestart
\centering
\begin{tabular}[t]{|\X{50}{100}|\X{50}{100}|}
\hline
\sphinxstyletheadfamily 
Name
&\sphinxstyletheadfamily 
Institution
\\
\hline
Fredrik Lindberg
&
Gothenburg
\\
\hline
\end{tabular}
\par
\sphinxattableend\end{savenotes}
\begin{itemize}
\item {} \begin{description}
\item[{Introduction:}] \leavevmode\begin{itemize}
\item {} 
This plugin makes it possible to run the Surface Urban Energy and Water Balance Scheme (SUEWS). SUEWS is also available as a separate program and a simplified version within UMEP (SUEWS Simple).

\item {} 
SUEWS (Järvi et al. 2011, 2014, Ward et al. 2016a, b) simulates the urban radiation, energy and water balances using commonly measured/modeled meteorological variables and information about the surface cover. It utilizes an evaporation-interception approach (Grimmond et al. 1991), similar to that used in forests, to model evaporation from urban surfaces.

\item {} 
The model uses seven surface types: paved, buildings, evergreen trees/shrubs, deciduous trees/shrubs, grass, bare soil and water. The surface state for each surface type at each time step is calculated from the running water balance of the canopy where the evaporation is calculated from the Penman-Monteith equation. The soil moisture below each surface type (excluding water) is taken into account.

\item {} 
Model applicability: Local scale \textendash{} so forcing data should be above the height of the roughness elements (trees, buildings)

\end{itemize}

\end{description}

\item {} \begin{description}
\item[{Related Preprocessors:}] \leavevmode\begin{itemize}
\item {} 
{\hyperref[\detokenize{pre-processor/Meteorological Data MetPreprocessor:metpreprocessor}]{\sphinxcrossref{\DUrole{std,std-ref,std,std-ref}{Meteorological Data: MetPreprocessor}}}}, {\hyperref[\detokenize{pre-processor/Meteorological Data Download data (WATCH):watch}]{\sphinxcrossref{\DUrole{std,std-ref,std,std-ref}{Meteorological Data: Download data (WATCH)}}}}, {\hyperref[\detokenize{pre-processor/Urban Land Cover Land Cover Reclassifier:landcoverreclassifier}]{\sphinxcrossref{\DUrole{std,std-ref,std,std-ref}{Urban Land Cover: Land Cover Reclassifier}}}}, {\hyperref[\detokenize{pre-processor/Urban Land Cover Land Cover Fraction (Point):landcoverfraction-point}]{\sphinxcrossref{\DUrole{std,std-ref,std,std-ref}{Urban Land Cover: Land Cover Fraction (Point)}}}}, {\hyperref[\detokenize{pre-processor/Urban Land Cover Land Cover Fraction (Grid):landcoverfraction-grid}]{\sphinxcrossref{\DUrole{std,std-ref,std,std-ref}{Urban Land Cover: Land Cover Fraction (Grid)}}}}, {\hyperref[\detokenize{pre-processor/Urban Morphology Morphometric Calculator (Point):morphometriccalculator-point}]{\sphinxcrossref{\DUrole{std,std-ref,std,std-ref}{Urban Morphology: Morphometric Calculator (Point)}}}}, {\hyperref[\detokenize{pre-processor/Urban Morphology Morphometric Calculator (Grid):morphometriccalculator-grid}]{\sphinxcrossref{\DUrole{std,std-ref,std,std-ref}{Urban Morphology: Morphometric Calculator (Grid)}}}}, {\hyperref[\detokenize{pre-processor/Urban Morphology Source Area (Point):sourcearea-point}]{\sphinxcrossref{\DUrole{std,std-ref,std,std-ref}{Urban Morphology: Source Area (Point)}}}}

\end{itemize}

\end{description}

\item {} \begin{description}
\item[{Dialog box :}] \leavevmode
\begin{figure}[htbp]
\centering
\capstart

\noindent\sphinxincludegraphics{{SUEWSAdvanced_SuewsAdvanced}.png}
\caption{The dialog for SUEWS Advanced}\label{\detokenize{processor/Urban Energy Balance Urban Energy Balance (SUEWS.BLUEWS, advanced):id1}}\end{figure}

\end{description}

\item {} \begin{description}
\item[{Dialog sections:}] \leavevmode\begin{itemize}
\item {} 
When you run the plugin, you will see the dialog shown above. To use this plugin, all input data needs to be prepared beforehand. This can be done using the various plugins in the pre-processor in UMEP (see {\hyperref[\detokenize{Introduction:toolapplications}]{\sphinxcrossref{\DUrole{std,std-ref,std,std-ref}{Tool Applications}}}}). The settings available in this plugin is used for specifying the settings for a specific model run. You should consult the manual (\sphinxurl{http://suews-docs.readthedocs.io}) for instructions and information on what settings to use.

\item {} 
For extensive models run it is recommended to execute the model outside of QGIS (see manual).

\item {} 
The interface above creates a so-called namelist (\sphinxstylestrong{RunControl.nml}) that is used be the model for general settings. After running the model, this file can be found in the suewsmodel directory in the UMEP plugin directory as well as in the input folder specified.

\item {} 
It is always recommended to make use of a spin-up period to adjust e.g. water availbility in the model. This can be dome by e.g. start your model before (months-years) the time period of interest. one option to use a spin-up procedure using exisitng data is availabe. This requires a full year meteorological dataset. This option will run the model for one year and then run the same year using the state from the first model run.

\end{itemize}

\end{description}

\item {} \begin{description}
\item[{References:}] \leavevmode\begin{itemize}
\item {} 
Järvi L, Grimmond CSB \& Christen A (2011) The Surface Urban Energy and Water Balance Scheme (SUEWS): Evaluation in Los Angeles and Vancouver \sphinxhref{http://www.sciencedirect.com/science/article/pii/S0022169411006937}{J. Hydrol. 411, 219-237.}

\item {} 
Järvi L, Grimmond CSB, Taka M, Nordbo A, Setälä H \&Strachan IB (2014) Development of the Surface Urban Energy and Water balance Scheme (SUEWS) for cold climate cities, Geosci. Model Dev. 7, 1691-1711, \sphinxhref{http://www.geosci-model-dev.net/7/1691/2014/}{doi:10.5194/gmd-7-1691-2014}.

\item {} 
Ward HC. S Kotthaus, L Järvi, CSB Grimmond (2016b) Surface Urban Energy and Water Balance Scheme (SUEWS): development and evaluation at two UK sites \sphinxhref{https://www.sciencedirect.com/science/article/pii/S2212095516300256}{link)}.

\end{itemize}

\end{description}

\end{itemize}


\chapter{Post-Processor}
\label{\detokenize{post_processor/Post-Processor:post-processor}}\label{\detokenize{post_processor/Post-Processor::doc}}
This section include manuals for the tools used for preparing input data to the models included in UMEP.

The tools can be accessed from the left panel.
\begin{quote}


\section{Benchmark System}
\label{\detokenize{post_processor/Benchmark System:benchmark-system}}\label{\detokenize{post_processor/Benchmark System:benchmark}}\label{\detokenize{post_processor/Benchmark System::doc}}\begin{itemize}
\item {} 
Contributor:

\end{itemize}


\begin{savenotes}\sphinxattablestart
\centering
\begin{tabular}[t]{|\X{50}{100}|\X{50}{100}|}
\hline
\sphinxstyletheadfamily 
Name
&\sphinxstyletheadfamily 
Institution
\\
\hline
Ting Sun
&
Reading
\\
\hline
Sue Grimmond
&
Reading
\\
\hline
\end{tabular}
\par
\sphinxattableend\end{savenotes}
\begin{itemize}
\item {} \begin{description}
\item[{Overview :}] \leavevmode\begin{itemize}
\item {} 
\sphinxstylestrong{Note}: the current version runs in a command-line interface (CLI) driven by Python and the GUI-based version is under construction.

\item {} 
The Benchmark System for SUEWS (BSS) can be used with SUEWS to assess the model performance between different configurations and model generations. BSS is written in Python and shipped with an example namelist and an MS Excel spreadsheet for header lookup between different SUEWS versions.

\end{itemize}

\end{description}

\item {} \begin{description}
\item[{Benchmark results:}] \leavevmode\begin{itemize}
\item {} \begin{description}
\item[{Two types of metrics are provided: :}] \leavevmode\begin{itemize}
\item {} 
overall performance score: a score between 0 and 100 with larger score denoting better overall performance

\item {} 
specific statistics: a range of statistics, including Mean absolute error (MAE), root mean square error (RMSE), standard deviation (Std), etc., to indicate detailed performance in specific variables.

\end{itemize}

\end{description}

\item {} \begin{description}
\item[{The users can use the overall performance score to get the performance overview of all configurations (Fig. 1a) and specific statistics to examine the performance details (Fig. 1b).}] \leavevmode
\begin{figure}[htbp]
\centering
\capstart

\noindent\sphinxincludegraphics{{300px-BSS-result}.png}
\caption{BSS results for (a) the overall performance and (b) a specific statistics (e.g., RMSE)}\label{\detokenize{post_processor/Benchmark System:id1}}\end{figure}

\end{description}

\end{itemize}

\end{description}

\item {} \begin{description}
\item[{Usage:}] \leavevmode\begin{itemize}
\item {} 
To use BSS, in addition to the mandatory BSS files (i.e., Benchmark\_SUEWS.py, benchmark.nml and head-2016to2017.xlsx), the SUEWS output results are required to be placed in a separate folder (e.g., “input”) that contains the sub-folders of results produced by different configurations. A sample layout of the BSS test case refers to Fig. 2. It must be noted that the output files to be benchmarked should be of consistent temporal organisation (i.e., identical length and resolution) while the headers of different files are not necessarily to be identical as BSS will handle the header inconsistency automatically. Besides, two sub-folders, “base” and “ref”, which contain the baseline results to be tested against and reference results to be compared with, respectively, must exist otherwise the BSS will stop.   \textbar{}

\item {} \begin{description}
\item[{When the SUEWS output files are prepared, the namelist (i.e., benchmark.nml) needs to be set for the benchmarking. The benchmark namelist is fairly self-explanatory and consists two sections, “file” and “benchmark”, to play with. One tip is about the variable list (i.e., var\_list): if one non-string value is set (e.g., 123, 3.2, etc.), all valid variables will be included in the benchmarking. Then the user can execute the Benchmark\_SUEWS.py script and a PDF file with benchmark results will be generated (e.g., benchmark.pdf in Fig. 2).}] \leavevmode
\begin{figure}[htbp]
\centering
\capstart

\noindent\sphinxincludegraphics{{300px-BSS-file-layout}.png}
\caption{Required file organisation by BSS.}\label{\detokenize{post_processor/Benchmark System:id2}}\end{figure}

\end{description}

\end{itemize}

\end{description}

\item {} \begin{description}
\item[{Namelist: benchmark.nml:}] \leavevmode\begin{itemize}
\item {} 
The benchmark namelist is fairly self-explanatory and consists two sections, “file” and “benchmark”, to play with.

\item {} 
One tip is about the variable list (i.e., var\_list): if one non-string value is set (e.g., 123, 3.2, etc.), all valid variables will be included in the benchmarking.

\item {} 
A sample namelist is as follows:

\fvset{hllines={, ,}}%
\begin{sphinxVerbatim}[commandchars=\\\{\}]
\PYGZam{}file
  input\PYGZus{}dir = \PYGZsq{}input\PYGZsq{}
  output\PYGZus{}pdf = \PYGZsq{}benchmark\PYGZsq{}
/
\PYGZam{}benchmark
  list\PYGZus{}var=\PYGZsq{}QN\PYGZsq{} \PYGZsq{}QS\PYGZsq{} \PYGZsq{}QE\PYGZsq{} \PYGZsq{}QH\PYGZsq{}
  list\PYGZus{}metric=\PYGZsq{}MAE\PYGZsq{} \PYGZsq{}MBE\PYGZsq{} \PYGZsq{}RMSE\PYGZsq{}
  method\PYGZus{}score=1 ! not used yet
/
\end{sphinxVerbatim}

\end{itemize}

\end{description}

\end{itemize}


\section{Outdoor Thermal Comfort: SOLWEIG Analyzer}
\label{\detokenize{post_processor/Outdoor Thermal Comfort SOLWEIG Analyzer:outdoor-thermal-comfort-solweig-analyzer}}\label{\detokenize{post_processor/Outdoor Thermal Comfort SOLWEIG Analyzer:solweiganalyzer}}\label{\detokenize{post_processor/Outdoor Thermal Comfort SOLWEIG Analyzer::doc}}\begin{itemize}
\item {} 
Contributor:

\end{itemize}


\begin{savenotes}\sphinxattablestart
\centering
\begin{tabular}[t]{|\X{50}{100}|\X{50}{100}|}
\hline
\sphinxstyletheadfamily 
Name
&\sphinxstyletheadfamily 
Institution
\\
\hline
Fredrik Lindberg
&
Gothenburg
\\
\hline
\end{tabular}
\par
\sphinxattableend\end{savenotes}
\begin{itemize}
\item {} \begin{description}
\item[{Introduction:}] \leavevmode
The \sphinxstylestrong{SOLWEIG Analyzer} plugin can be used to make basic analysis of model results generated by the SOLWEIG plugin.

\end{description}

\item {} 
Dialog box  :

\end{itemize}

\begin{figure}[htbp]
\centering
\capstart

\noindent\sphinxincludegraphics{{SOLWEIGAnalyzer}.png}
\caption{Dailog for SOLWEIG Analyser}\label{\detokenize{post_processor/Outdoor Thermal Comfort SOLWEIG Analyzer:id1}}\end{figure}
\begin{itemize}
\item {} 
Dialog sections  :

\end{itemize}


\begin{savenotes}\sphinxattablestart
\centering
\begin{tabular}[t]{|\X{10}{100}|\X{90}{100}|}
\hline
\sphinxstyletheadfamily 
top
&\sphinxstyletheadfamily 
Input data is specified
\\
\hline
left
&
Plotting of time series derived from Points of Interest during model calculations in SOLWEIG
\\
\hline
bottom
&
Analysis of spatial results from model calculations in SOLWEIG
\\
\hline
\end{tabular}
\par
\sphinxattableend\end{savenotes}
\begin{itemize}
\item {} \begin{description}
\item[{Load model result:}] \leavevmode
The directory where results from a previous model run in SOLWEIG is located.

\end{description}

\item {} 
Point of Interest data:

\end{itemize}


\begin{savenotes}\sphinxattablestart
\centering
\begin{tabular}[t]{|\X{25}{100}|\X{75}{100}|}
\hline

POIs available
&
Here, a list of all available POI files are listed. Specify one of the POIs. If no variable is available, then no data if found in the Model output folder.
\\
\hline
Variable
&
Specify one of the available variables to plot.
\\
\hline
Add another POI / variable
&
Tick this in to add another variable in the plot.
\\
\hline
Scatterplot
&
Tick this in to generate a scatterplot between the two variables specified above.
\\
\hline
Plot
&
Plot the data selected above
\\
\hline
\end{tabular}
\par
\sphinxattableend\end{savenotes}
\begin{itemize}
\item {} 
Spatial data  :

\end{itemize}


\begin{savenotes}\sphinxattablestart
\centering
\begin{tabular}[t]{|\X{25}{100}|\X{75}{100}|}
\hline

Variable to visualize
&
Select a listed variable to process. If no variable is available, then no data if found in the Model output folder.
\\
\hline
Show animation
&
This produces a time-related animation of the selected variable scaled based on the settings to the left in the GUI.
\\
\hline
Exclude building pixels
&
Tick this in to exclude building pixels using a building grid generated from the SOLWEIG run. The building grid must have the same extent and pixel resolution as the generated output maps in the model output folder.
\\
\hline
Diurnal average
&
Tick this in to include all grids for the selected variable to derive a pixelwise average.
\\
\hline
Daytime average
&
Tick this in to include all daytime grids for the selected variable to derive a pixelwise average.
\\
\hline
Nightime average
&
Tick this in to include all nighttime grids for the selected variable to derive a pixelwise average.
\\
\hline
Maximum
&
Tick this in to get the pixelwise maximum for the selected variable.
\\
\hline
Minimum
&
Tick this in to get the pixelwise minimum for the selected variable.
\\
\hline
Average of specific time of day
&
The average of the selected time of day for the variable selected is generated. If ‘Not Specified’ is highlighted, no grid will be generated.
\\
\hline
Maximum of specific time of day
&
The maximum of the selected time of day for the variable selected is generated. If ‘Not Specified’ is highlighted, no grid will be generated.
\\
\hline
Minimum of specific time of day
&
The minimum of the selected time of day for the variable selected is generated. If ‘Not Specified’ is highlighted, no grid will be generated.
\\
\hline
T$_{\text{mrt}}$: Percent of time above threshold (degC)
&
If T$_{\text{mrt}}$ is the selected variable, this box become active and calculates the percent of time that T$_{\text{mrt}}$ for each is above the threshold specified to the right.
\\
\hline
T$_{\text{mrt}}$: Percent of time below threshold (degC)
&
If T$_{\text{mrt}}$ is the selected variable, this box become active and calculates the percent of time that T$_{\text{mrt}}$ for each is below the threshold specified to the right.
\\
\hline
Output folder
&
Directory where the results specified above will be saved.
\\
\hline
Add analysis to map canvas
&
All analysis specified above will be added to the map canvas if this box is ticked in.
\\
\hline
Generate
&
Starts the spatial processing/analysis.
\\
\hline
\end{tabular}
\par
\sphinxattableend\end{savenotes}
\begin{itemize}
\item {} \begin{description}
\item[{Close:}] \leavevmode
This closes the plugin.

\end{description}

\end{itemize}


\section{Solar Radiation: SEBE (Visualisation)}
\label{\detokenize{post_processor/Solar Radiation SEBE (Visualisation):solar-radiation-sebe-visualisation}}\label{\detokenize{post_processor/Solar Radiation SEBE (Visualisation):sebevisualisation}}\label{\detokenize{post_processor/Solar Radiation SEBE (Visualisation)::doc}}\begin{itemize}
\item {} 
Contributor:

\end{itemize}


\begin{savenotes}\sphinxattablestart
\centering
\begin{tabular}[t]{|\X{50}{100}|\X{50}{100}|}
\hline
\sphinxstyletheadfamily 
Name
&\sphinxstyletheadfamily 
Institution
\\
\hline
Niklas Krave
&
Gothenburg
\\
\hline
\end{tabular}
\par
\sphinxattableend\end{savenotes}
\begin{itemize}
\item {} \begin{description}
\item[{Introduction:}] \leavevmode
The \sphinxstylestrong{SEBE (Visualisation)} plugin can be used to visulise 3D output from model results generated by the SEBE plugin.

\end{description}

\item {} 
Dialog box  :

\end{itemize}

\begin{figure}[htbp]
\centering
\capstart

\noindent\sphinxincludegraphics{{SEBEvisualisation}.png}
\caption{The dialog for SEBE Visualisation}\label{\detokenize{post_processor/Solar Radiation SEBE (Visualisation):id1}}\end{figure}
\begin{itemize}
\item {} 
Dialog sections  :

\end{itemize}


\begin{savenotes}\sphinxattablestart
\centering
\begin{tabular}[t]{|\X{50}{100}|\X{50}{100}|}
\hline

top
&
Canvas for visualisation
\\
\hline
bottom
&
Input data and settings
\\
\hline
\end{tabular}
\par
\sphinxattableend\end{savenotes}
\begin{itemize}
\item {} \begin{description}
\item[{Select input folder:}] \leavevmode
The directory where results from a previous model run in SEBE is located.

\end{description}

\item {} \begin{description}
\item[{Area of visualisation:}] \leavevmode
When this pushbutton is clicked, a recanglge can be drawn on the map canvas. This is the area that will be visulised.

\end{description}

\item {} \begin{description}
\item[{Visulise:}] \leavevmode
When this button is clicked, the selected rectangular area will be visulised in the SEBE (visualisation) canvas at the top of the GUI.

\end{description}

\item {} \begin{description}
\item[{Close:}] \leavevmode
This closes the plugin.

\end{description}

\end{itemize}


\section{Urban Energy Balance: SUEWS Analyser}
\label{\detokenize{post_processor/Urban Energy Balance SUEWS Analyser:urban-energy-balance-suews-analyser}}\label{\detokenize{post_processor/Urban Energy Balance SUEWS Analyser:suewsanalyser}}\label{\detokenize{post_processor/Urban Energy Balance SUEWS Analyser::doc}}\begin{itemize}
\item {} 
Contributor:

\end{itemize}


\begin{savenotes}\sphinxattablestart
\centering
\begin{tabular}[t]{|\X{50}{100}|\X{50}{100}|}
\hline
\sphinxstyletheadfamily 
Name
&\sphinxstyletheadfamily 
Institution
\\
\hline
Fredrik Lindberg
&
Gothenburg
\\
\hline
\end{tabular}
\par
\sphinxattableend\end{savenotes}
\begin{itemize}
\item {} \begin{description}
\item[{Introduction:}] \leavevmode
The \sphinxstylestrong{SUEWS Analyzer} plugin can be used to make basic analysis of model results generated by the \sphinxstyleemphasis{SUEWS Simple} and \sphinxstyleemphasis{SUEWS Advanced} plugins.

\end{description}

\item {} 
Dialog box  :

\end{itemize}

\begin{figure}[htbp]
\centering
\capstart

\noindent\sphinxincludegraphics{{SUEWSAnalyzer}.png}
\caption{The dialog for SUEWSAnalyzer}\label{\detokenize{post_processor/Urban Energy Balance SUEWS Analyser:id1}}\end{figure}
\begin{itemize}
\item {} 
Dialog sections  :

\end{itemize}


\begin{savenotes}\sphinxattablestart
\centering
\begin{tabular}[t]{|\X{10}{100}|\X{90}{100}|}
\hline

top
&
Model results to be analyzer is specified.
\\
\hline
left
&
Plotting of time series derived from Points of Interest during model calculations in SUEWS
\\
\hline
bottom
&
Analysis of spatial results from model calculations in SUEWS
\\
\hline
\end{tabular}
\par
\sphinxattableend\end{savenotes}
\begin{description}
\item[{Load model result:}] \leavevmode
A namelist (\sphinxstyleemphasis{RunControl.nml}) used for the model run should be specified. This can be located in the \sphinxstylestrong{suewsmodel} directory found as a sub-directory of the \sphinxstylestrong{UMEP}-plugin directory. Note that the namelist includes information on the last model run performed.

\end{description}
\begin{itemize}
\item {} 
Point data:

\end{itemize}


\begin{savenotes}\sphinxattablestart
\centering
\begin{tabular}[t]{|\X{25}{100}|\X{75}{100}|}
\hline

Grid
&
Here, a list of all available modeled grids are listed. Specify one of the grid IDs. If no grid is available, then no data if found in the model output folder.
\\
\hline
Year to investigate
&
Specify one of the available years to plot.
\\
\hline
Plot basic data
&
Tick this in to plot a summary of the most essential output variables.
\\
\hline
Time Period (DOY)
&
Specify the time period to plot.
\\
\hline
Variable
&
Specify one of the available variables to plot.
\\
\hline
Include another variable
&
Tick this in to add another variable in the plot.
\\
\hline
Grid
&
Here, a list of all available modeled grids are listed. Specify one of the grid IDs. If no grid is available, then no data if found in the model output folder.
\\
\hline
Variable
&
Specify one of the available variables to plot.
\\
\hline
Scatterplot
&
Tick this in to generate a scatterplot between the two variables specified above.
\\
\hline
Plot
&
Plot the data selected above
\\
\hline
\end{tabular}
\par
\sphinxattableend\end{savenotes}

Spatial data:


\begin{savenotes}\sphinxattablestart
\centering
\begin{tabular}[t]{|\X{25}{100}|\X{75}{100}|}
\hline

Variable to analyze
&
Select a listed variable to process. If no variable is available, then no data if found in the Model output folder.
\\
\hline
Year to investigate
&
Specify one of the available years to plot.
\\
\hline
Time Period (DOY)
&
Specify the time period to plot.
\\
\hline
Average
&
Tick this in to derive a grid-wise average.
\\
\hline
Maximum
&
Tick this in to derive a grid-wise maximum.
\\
\hline
Minimum
&
Tick this in to derive a grid-wise minimum.
\\
\hline
Median
&
Tick this in to derive a grid-wise median value.
\\
\hline
IQR
&
Tick this in to derive a grid-wise interquartile range.
\\
\hline
Diurnal
&
Tick this in to include diurnal (all) data.
\\
\hline
Daytime
&
Tick this in to include daytime data.
\\
\hline
Nightime
&
Tick this in to include nighttime data.
\\
\hline
Vector polygon grid used in the SUEWS model
&
Specify the grid that was used to generate the input data to the SUEWS model run of interest.
\\
\hline
ID
&
Specify the attribute ID used to generate the input data to the SUEWS model run of interest.
\\
\hline
Add result to polygon grid
&
Tick this box to add the results in the attribute table of the grid specified.
\\
\hline
Save of GeoTIFF
&
Tick this in to generate a raster grid from the analyze settings specified.
\\
\hline
Irregular grid (not squared)
&
Tick this in if a grid is irregular i.e. not squared and aligned north to south.
\\
\hline
Pixel resolution (m)
&
When a irregular grid is used, a pixel resolution in meters must be specified.
\\
\hline
Output filename
&
Name of the GeoTIFF to be saved.
\\
\hline
Add analysis to map canvas
&
All analysis specified above will be added to the map canvas if this box is ticked in.
\\
\hline
Generate
&
Starts the spatial processing/analysis.
\\
\hline
\end{tabular}
\par
\sphinxattableend\end{savenotes}
\begin{itemize}
\item {} \begin{description}
\item[{Close:}] \leavevmode
This closes the plugin.

\end{description}

\end{itemize}
\end{quote}


\chapter{People Involved \& Acknowledgements}
\label{\detokenize{People_Involved___Acknowledgements:people-involved-acknowledgements}}\label{\detokenize{People_Involved___Acknowledgements:id1}}\label{\detokenize{People_Involved___Acknowledgements::doc}}

\begin{savenotes}\sphinxattablestart
\centering
\begin{tabular}[t]{|\X{17}{100}|\X{20}{100}|\X{23}{100}|\X{40}{100}|}
\hline
\sphinxstyletheadfamily 
Group
&\sphinxstyletheadfamily 
Institution
&\sphinxstyletheadfamily 
Team
&\sphinxstyletheadfamily 
Acknowledgement
\\
\hline
\sphinxhref{http://www.met.reading.ac.uk/userpages/xv904931.php}{Sue Grimmond}
&
\sphinxstyleemphasis{University of Reading, UK}
&\begin{itemize}
\item {} 
Christoph W. Kent

\item {} 
Helen Ward

\item {} 
Ting Sun

\item {} 
Izzy Capel-Timms

\item {} 
Andy Gabey

\item {} 
Bei HUANG

\end{itemize}
&\begin{itemize}
\item {} 
Met Office/Newton Fund CSSP - China

\item {} 
NERC TRUC

\item {} 
NERC ClearfLo

\item {} 
EPSRC LoHCool

\item {} 
EPSRC PhD Studentships

\item {} 
EUf7 Bridge

\item {} 
EUf7 emBRACE

\item {} 
H2020 UrbanFluxes

\item {} 
NERC Case Studentship

\end{itemize}
\\
\hline
\sphinxhref{https://gvc.gu.se/english/personnel?languageId=100001\&userId=xlinfr}{Fredrik Lindberg}
&
\sphinxstyleemphasis{University of Gothenburg, Sweden}
&\begin{itemize}
\item {} 
Frans Olofsson

\item {} 
Niklas Krave

\item {} 
Shiho Onomura

\item {} 
Nils Wallenberg

\end{itemize}
&\begin{itemize}
\item {} 
H2020 UrbanFluxes

\item {} 
FORMAS Climplan

\end{itemize}
\\
\hline
\sphinxhref{https://tuhat.helsinki.fi/portal/en/persons/leena-jarvi(198f2cdc-762e-4456-9170-284c1507429a).html}{Leena Järvi}
&
\sphinxstyleemphasis{University of Helsinki, Finland}
&
Tom Kokkonen
&\begin{itemize}
\item {} 
Maj and Tor Nessling foundation

\item {} 
Academy of Finland

\item {} 
EUf7 Bridge

\end{itemize}
\\
\hline
\sphinxhref{https://scholar.google.com/citations?user=NwIDutIAAAAJ\&hl=en}{Jian Guo Tan}
&
\sphinxstyleemphasis{Shanghai Institute of Meteorological Sciences, SMS, CMA, China}
&\begin{itemize}
\item {} 
Yuan Yong Chang

\item {} 
Dongwei Liu

\item {} 
XY Ao

\end{itemize}
&\\
\hline
\end{tabular}
\par
\sphinxattableend\end{savenotes}


\chapter{How to Contribute}
\label{\detokenize{How_to_Contribute:how-to-contribute}}\label{\detokenize{How_to_Contribute:id1}}\label{\detokenize{How_to_Contribute::doc}}
UMEP is an an open source tool that we are keen to get others inputs and
contributions. There are two main ways to contribute:
\begin{enumerate}
\item {} 
Submit comments or issues to the
\sphinxhref{https://bitbucket.org/fredrik\_ucg/umep/issues}{repository}

\item {} 
Participate in \sphinxhref{http://urban-climate.net/umep/DevelopmentGuidelines}{Coding or adding new
features}.

\item {} 
Create new tutorials for the UMEP-plugin.

\end{enumerate}


\section{Reporting a Bug}
\label{\detokenize{How_to_Contribute:reporting-a-bug}}\begin{itemize}
\item {} 
As a good citizen of the open source community please report bugs. If
it is a UMEP plugin related issue - report this to the
\sphinxhref{https://bitbucket.org/fredrik\_ucg/umep/issues}{https://bitbucket.org/fredrik\_ucg/umep/issues UMEP
plugin}. You can
see if your bug is already reported. In order for the UMEP team to
solve your issue as easy as fast as possible, please provide a full
description of the problem including steps to repeat it. The more
info given, the easier it is for us to solve your issues.

\item {} 
\sphinxstylestrong{Please have a look at Known Issues and FAQ (found below) before
submitting an issue to the repository}.

\item {} 
A bug may also be caused by \sphinxhref{http://hub.qgis.org/issues}{QGIS}. By
reporting bugs (and also maybe helping out to solve them) is
essential to the open source community. At {[}www.qgis.org
www.qgis.org{]} you can find out more on what you can do to get
involved.

\item {} 
QGIS: how to report a QGIS issue:
\sphinxurl{http://qgis.org/en/site/getinvolved/development/index.html\#qgis-bugreporting}.

\end{itemize}


\chapter{DevelopmentGuidelines}
\label{\detokenize{DevelopmentGuidelines:developmentguidelines}}\label{\detokenize{DevelopmentGuidelines:id1}}\label{\detokenize{DevelopmentGuidelines::doc}}

\section{Contributing with python code to the UMEP plugin in QGIS}
\label{\detokenize{DevelopmentGuidelines:contributing-with-python-code-to-the-umep-plugin-in-qgis}}
Here, instructions and recommendations on how to make your own plugin to
go with the UMEP plugin will be described as well as tips and tricks on
how to make use of the GIS capabilities included in QGIS to go with your
plugin.

Gary Sherman (creator of QGIS) has produced a number of useful tools for
developers that can be used to make your own contribution to the
software. Below, a number of tool are listed that come very handy when
you want to create your own plugin. There is also a number of resources
online that is very useful. One is the \sphinxhref{http://docs.qgis.org/testing/en/docs/pyqgis\_developer\_cookbook/}{PyQGIS Developer
Cookbook}
available from the QGIS webpage and another is a book written by Gary
Sherman called The \sphinxhref{http://locatepress.com/ppg}{PyQGIS Programmer’s
Guide}.
\begin{enumerate}
\item {} 
Start off by creating a plugin using the \sphinxstylestrong{PluginBuilder} in QGIS.
This is a plugin that sets up all necessary files and folders for
your plugin.

\item {} 
Another useful plugin is the \sphinxstylestrong{PluginReloader} which makes it
possible to reload a plugin in QGIS without restarting the software.

\item {} 
pb\_tool is another useful program to use for installing your plugin
in the QGIS plugin folder as well as cleaning etc.

\end{enumerate}

Please use the python libraries that comes with the QGIS installation as
much as possible without including external libraries when developing
your plugin. All libraries are not included when a common installation
of QGIS is installed. Go to our \sphinxhref{http://urban-climate.net/umep/UMEP\_Manual\#UMEP:\_Getting\_Started}{Getting
started}
section for further instructions.

External Python libraries currently included in the UMEP plugin
(Utilities folder):
\begin{itemize}
\item {} 
f90nml - for reading and writing of fortran namelists

\item {} 
Pysolar - for calculation of Sun positions

\end{itemize}

To use the f90nml library located in the Utilities folder:

\fvset{hllines={, ,}}%
\begin{sphinxVerbatim}[commandchars=\\\{\}]
\PYG{k+kn}{from} \PYG{n+nn}{.}\PYG{n+nn}{.}\PYG{n+nn}{Utilities} \PYG{k}{import} \PYG{n}{f90nml}
\end{sphinxVerbatim}

The \sphinxstylestrong{pandas} library is not install by default so a simple install
pandas cannot be used use instead a try statement:

\fvset{hllines={, ,}}%
\begin{sphinxVerbatim}[commandchars=\\\{\}]
\PYG{k}{try}\PYG{p}{:}
    \PYG{k+kn}{import} \PYG{n+nn}{pandas}
\PYG{k}{except} \PYG{n+ne}{Exception}\PYG{p}{,} \PYG{n}{e}\PYG{p}{:}
    \PYG{n}{QMessageBox}\PYG{o}{.}\PYG{n}{critical}\PYG{p}{(}\PYG{k+kc}{None}\PYG{p}{,} \PYG{l+s+s1}{\PYGZsq{}}\PYG{l+s+s1}{Error}\PYG{l+s+s1}{\PYGZsq{}}\PYG{p}{,} \PYG{l+s+s1}{\PYGZsq{}}\PYG{l+s+s1}{The WATCH data download/extract feature requires the pandas package }\PYG{l+s+s1}{\PYGZsq{}}
                           \PYG{l+s+s1}{\PYGZsq{}}\PYG{l+s+s1}{to be installed. Please consult the FAQ in the manual for further information}\PYG{l+s+s1}{\PYGZsq{}}\PYG{p}{)}
    \PYG{k}{return}
\end{sphinxVerbatim}

The same goes for \sphinxstylestrong{matplotlib} and other libraries that you are
uncertain of.


\subsection{SUEWS wrapper}
\label{\detokenize{DevelopmentGuidelines:suews-wrapper}}
The main file of the python wrapper for the SUEWS model is called
SUEWSwrapper.py. To change version of SUEWS when running the wrapper,
simple go in to SUEWSwrapper.py and activate the line which is calling
the appropriate wrapper (e.g. SUEWSwrapper\_v2016a) and comment out
other versions.

In SUEWS v2017b - this is no longer used/needed


\subsection{f2py}
\label{\detokenize{DevelopmentGuidelines:f2py}}
A possibility to make use of fortran subroutines in python. See
\sphinxhref{http://docs.scipy.org/doc/numpy-dev/f2py/}{here} for documetation.
\begin{enumerate}
\item {} 
When developing the fortran code for compiling with f2py kind=1d0
will not work for function parameters to be passed in from python.
Should use KIND=8.

\end{enumerate}


\subsubsection{Setting up Windows machine for running f2py}
\label{\detokenize{DevelopmentGuidelines:setting-up-windows-machine-for-running-f2py}}
Please note that the steps below contain some links that over time could
change, however, the basics should remain the same. This has been tested
and established for an Intel 64-bit machine running NT operating system.
This is still to be tested on other versions of Windows operating
system. If you carry out on both OS, please add to the list below or
report to team.

The process can also be applied to a 32-bit machine, but the choice of
python2.7 and set up for MinGW-w64 will be different. This has not been
tested, and perhaps shouldn’t be encouraged if SUEWS is being developed
and tested on/for 64-bit architecture.

Before starting any of the required steps it is recommended that any
version of Python, mingw, and Cygwin be removed from the machine if
possible. If you already have Python 2.7 and MinGW installed, or have
followed the procedure below then go to the instructions on developing
fortran-python interface.


\subsubsection{System Setup}
\label{\detokenize{DevelopmentGuidelines:system-setup}}\begin{enumerate}
\item {} 
Download Python 2.7.XX : download from:
\sphinxhref{http://www.python.org/downloads}{here} and install to directory
C:\textbackslash{}Python27\textless{}!\textendash{}

\item {} 
Add to PATH environment variable:
\begin{quote}

C:\textbackslash{}Python27

C:\textbackslash{}Python27\textbackslash{}libs

C:\textbackslash{}Python\textbackslash{}Scripts
\end{quote}

\item {} 
Add to C\_INCLUDE\_PATH environment variable:
\begin{quote}

C:\textbackslash{}Python27\textbackslash{}include

C:\textbackslash{}Python27\textbackslash{}libs
\end{quote}

\end{enumerate}

\sphinxstylestrong{Note:} The \sphinxstyleemphasis{C:\textbackslash{}Python27\textbackslash{}Scripts} directory is required for use of
pip in subsequent steps.
\begin{enumerate}
\item {} 
Install latest NumPY package to get latest f2py
\begin{quote}

(i) Download numpy-1.11.0rc1+mkl-cp27-cp27m-win\_amd64.whl from
\sphinxhref{http://www.lfd.uci.edu/~gohlke/pythonlibs/\#numpy}{here}

(ii) Open command prompt and change directory to where numpy wheel is
installed (e.g. Downloads folder). Run the following command:

\fvset{hllines={, ,}}%
\begin{sphinxVerbatim}[commandchars=\\\{\}]
\PYG{n}{Pip} \PYG{n}{install} \PYG{n}{numpy}\PYG{o}{\PYGZhy{}}\PYG{l+m+mf}{1.11}\PYG{o}{.}\PYG{l+m+mi}{0}\PYG{n}{rc1}\PYG{o}{+}\PYG{n}{mkl}\PYG{o}{\PYGZhy{}}\PYG{n}{cp27}\PYG{o}{\PYGZhy{}}\PYG{n}{cp27m}\PYG{o}{\PYGZhy{}}\PYG{n}{win\PYGZus{}amd64}\PYG{o}{.}\PYG{n}{whl}
\end{sphinxVerbatim}

(iii) Check the directories for \sphinxstylestrong{numpy} and
\sphinxstylestrong{numpy-1.11.0rc1.dist-info} are under
\sphinxstylestrong{C:\textbackslash{}Python27\textbackslash{}Lib\textbackslash{}site-packages}. Dates and times of directory
should indicate these are new from wheel.
\end{quote}

\item {} 
Download Mingw-w64 from
\sphinxhref{https://sourceforge.net/projects/mingw\_w64}{here}

\sphinxstylestrong{Note:} This is suitable for 64 and 32 bit architecture.

\item {} 
Run mingw-w64-install.exe (found in directory to which you have
downloaded it to).
\begin{quote}
\begin{enumerate}
\item {} 
In the install procedure set:

\end{enumerate}

\sphinxstylestrong{Version:} 5.3.0 (or what ever is latest)

\sphinxstylestrong{Architecture:} x86\_64 (for 64 bit machine)

\sphinxstylestrong{Threads:} posix

\sphinxstylestrong{Exception:} dwarf

\sphinxstylestrong{Build revision:} 0

(ii) Set the destination folder to: \sphinxstylestrong{C:\textbackslash{}mingw-w64\_x86} when
prompted.
\end{quote}

\item {} 
Add to environment variable:
\begin{enumerate}
\item {} 
\sphinxstylestrong{C\_INCLUDE\_PATH: C:\textbackslash{}mingw-w64\_x86\textbackslash{}mingw64\textbackslash{}include}

\item {} 
\sphinxstylestrong{PATH: C:\textbackslash{}mingw-w64\_x86\textbackslash{}mingw64\textbackslash{}bin}

\end{enumerate}

\item {} 
This step is required to create/replace the import library found
under directory \sphinxstylestrong{C:\textbackslash{}Python27\textbackslash{}libs}. The import library is
\sphinxstylestrong{libpython27.a}.
\begin{quote}

(i) Download the pexports binary \sphinxstylestrong{pexports-0.47-mingw32-bin.tar.xz}
from
\sphinxhref{http://www.sourceforge.net/projects/mingw/files/MinGW/Extension/pexports/pexports-0.47/}{here}.

\sphinxstylestrong{Note:} pexports-0.47 could change for subsequent versions.
\begin{enumerate}
\setcounter{enumii}{1}
\item {} 
Unpack the tar file and put pexports.exe in \sphinxstylestrong{C:\textbackslash{}Python27\textbackslash{}libs}.

\end{enumerate}
\begin{enumerate}
\item {} 
Open a command prompt and run the following command:

\end{enumerate}

\fvset{hllines={, ,}}%
\begin{sphinxVerbatim}[commandchars=\\\{\}]
\PYG{n}{pexports} \PYG{n}{C}\PYG{p}{:}\PYGZbs{}\PYG{n}{Windows}\PYGZbs{}\PYG{n}{System32}\PYGZbs{}\PYG{n}{python27}\PYG{o}{.}\PYG{n}{dll} \PYG{o}{\PYGZgt{}} \PYG{n}{USERDIR}\PYGZbs{}\PYG{n}{python27}\PYG{o}{.}\PYG{k}{def}
\end{sphinxVerbatim}

\sphinxstylestrong{Note:} \sphinxstyleemphasis{USERDIR} is the user directory you put the file in. As it is
an intermediary step and a temporary file, the user directory you use
shouldn’t matter, however, don’t try to put it in Windows\textbackslash{}System32
directory, or any other directory in the system Path.
\end{quote}

\item {} 
Create the import library \sphinxstylestrong{libpython27.a} for helping the linker of
MinGW link to the correct python DLL.

(i) Open Command Prompt and change directory to where \sphinxstyleemphasis{python27.def} was
created in step 8 (i.e.USERDIR*).
\begin{enumerate}
\setcounter{enumii}{1}
\item {} 
Run command:

\end{enumerate}

\fvset{hllines={, ,}}%
\begin{sphinxVerbatim}[commandchars=\\\{\}]
dlltool \textendash{}D python27.dll \textendash{}d python.def \textendash{}l libpython27.a
\end{sphinxVerbatim}

(iii) Move resulting \sphinxstyleemphasis{libpython27.a} into \sphinxstylestrong{C:\textbackslash{}Python27\textbackslash{}libs} ,
replacing any existing version of this file in the directory

\end{enumerate}


\subsubsection{Making Fortran-Python Interface ‘dll’ (.pyd file) with F2PY}
\label{\detokenize{DevelopmentGuidelines:making-fortran-python-interface-dll-pyd-file-with-f2py}}\begin{itemize}
\item {} 
This is shown using a makefile (named \sphinxstyleemphasis{Makefile}) that is called from
the command line as follows:

\end{itemize}

\fvset{hllines={, ,}}%
\begin{sphinxVerbatim}[commandchars=\\\{\}]
mingw32\PYGZhy{}make  \textendash{}f  Makefile
\end{sphinxVerbatim}

\sphinxstylestrong{Note:} This should be called from within the directory that the
Makefile and source code is in.
\begin{itemize}
\item {} 
Basic Makefile:

\end{itemize}

\fvset{hllines={, ,}}%
\begin{sphinxVerbatim}[commandchars=\\\{\}]
CC = gnu95
CCO= x86\PYGZus{}64\PYGZhy{}w64\PYGZhy{}mingw32\PYGZhy{}gfortran
FFLAGS = \PYGZhy{}fPIC
TARGET = INTENDED\PYGZus{}NAME\PYGZus{}OF\PYGZus{}PYD
MODULES = nameOfModules.o

main:  NAMEOFMAINPROGRAMFILE.f95  \PYGZdl{}(MODULES)
 f2py.py \textendash{}c \textendash{}\PYGZhy{}fcompiler=\PYGZdl{}(CC) \textendash{}\PYGZhy{}compiler=mingw32 \textendash{}m \PYGZdl{}(TARGET) NAMEOFMAINPROGRAMFILE.f95 \PYGZdl{}(MODULES)

\PYGZdl{}(MODULES): nameOfModules.f95
\PYGZdl{}(CCO) \textendash{}c  \PYGZdl{}(FFLAGS)  nameOfModules.f95

cleanall:
   \PYGZhy{}del \PYGZdl{}(MODULES)
\end{sphinxVerbatim}

\sphinxstylestrong{Note:} A .pyd file should have been created on the completion of
compilation from command called in step 1.
\begin{itemize}
\item {} 
Create a directory to store all created .pyd files in (e.g.
C:\textbackslash{}PythonPYD) and add to PATH environment variable.

\end{itemize}

This ensures the .pyd files are picked up and used by python scripts.


\subsubsection{Distributing f2py Modules for Windows}
\label{\detokenize{DevelopmentGuidelines:distributing-f2py-modules-for-windows}}
\sphinxstylestrong{Note:} all the .dll files (including those used to make python
library from MinGW) need to be packaged up so that a machine without MinGW can use the
developed python libraries.

The .dll files to include are: NEED TO LIST THEM


\subsubsection{Importing and using in Python}
\label{\detokenize{DevelopmentGuidelines:importing-and-using-in-python}}\begin{enumerate}
\item {} 
Import the module into python script in the same way you would import
any other module:
\begin{quote}
\begin{itemize}
\item {} 
If your module is called \sphinxstyleemphasis{SolweigShadow}, for example, then {\color{red}\bfseries{}*}import

\end{itemize}

SolweigShadow as SS* will enable you to access the functions of the
module by \sphinxstyleemphasis{SS.functionName()}.

\sphinxstylestrong{Note:} The parentheses are needed regardless of whether the function
has parameter inputs/outputs.
\end{quote}

\item {} 
To see what functions are available for the imported module, use the
command

\fvset{hllines={, ,}}%
\begin{sphinxVerbatim}[commandchars=\\\{\}]
\PYG{n+nb}{print} \PYG{n}{hw}\PYG{o}{.}\PYG{n+nv+vm}{\PYGZus{}\PYGZus{}doc\PYGZus{}\PYGZus{}}
\end{sphinxVerbatim}

\end{enumerate}


\subsection{Upcoming Developments}
\label{\detokenize{DevelopmentGuidelines:upcoming-developments}}

\begin{savenotes}\sphinxattablestart
\centering
\begin{tabular}[t]{|\X{25}{100}|\X{25}{100}|\X{25}{100}|\X{25}{100}|}
\hline
\sphinxstyletheadfamily 
U/S
&\sphinxstyletheadfamily 
Topic
&\sphinxstyletheadfamily 
Status
&\sphinxstyletheadfamily 
Lead
\\
\hline
SUEWS
&
Snow
&
Completed
&
Univ Helsinki
\\
\hline
SUEWS
&
Convective boundary layer development
&
Completed
&
Göteborg Univ
\\
\hline
SUEWS/SOLWEIG
&
Mean radiant temperature model
&
Active
&
Göteborg Univ
\\
\hline
SUEWS
&
Storage Heat flux - ESTM
&
Completed
&
Göteborg Univ /Reading
\\
\hline
SUEWS
&
Storage Heat flux - AnOHM
&
Active
&
Reading/Tsinghua
\\
\hline
SUEWS
&
Anthropogenic Heat fluxes
&
Actve
&
Reading
\\
\hline
Multi
&
Benchmark
&
Active
&
Reading
\\
\hline
Wind
&
Pedestrian wind speed
&
Active
&
Göteborg Univ/Reading
\\
\hline
Multi
&
Downscaling data *download WATCH
&
Active
&
Lingbo Xue (Reading)/TS
\\
\hline
Multi
&
Downscaling data *precip mass check
&
Active
&
TK/ LX (LJ/TS)
\\
\hline
Multi
&
Downscaling data *precip intensity
&
Active
&
AG/ LX
\\
\hline
SUEWS/SOLWEIG
&
Radiation coupling
&
Active
&
Göteborg Univ/Reading
\\
\hline
\end{tabular}
\par
\sphinxattableend\end{savenotes}


\subsection{Benchmarking system}
\label{\detokenize{DevelopmentGuidelines:benchmarking-system}}
\sphinxhref{http://urban-climate.net/umep/Benchmark}{Benchmark}


\subsection{Coding Guidelines}
\label{\detokenize{DevelopmentGuidelines:coding-guidelines}}

\subsubsection{SUEWS}
\label{\detokenize{DevelopmentGuidelines:suews}}
If you are interested in contributing to the code please contact Sue
Grimmond.
\begin{enumerate}
\item {} 
Code written in Fortran \textendash{} currently Fortran 95

\item {} 
Variables
\begin{itemize}
\item {} 
Names should be defined at least in one place in the code \textendash{}
ideally when defined

\item {} 
Implicit None should be used in all subroutines

\item {} 
Variable name should include units. e.g. Temp\_C, Temp\_K

\item {} 
Output variable attributes should be provided in the TYPE
structure defined in the ctrl\_output module as follows:
\begin{quote}

\fvset{hllines={, ,}}%
\begin{sphinxVerbatim}[commandchars=\\\{\}]
: TYPE varAttr
: CHARACTER(len = 15) :: header ! short name in headers
: CHARACTER(len = 12) :: unit   ! unit
: CHARACTER(len = 14) :: fmt    ! output format
: CHARACTER(len = 50) :: longNm ! long name for detailed description
: CHARACTER(len = 1)  :: aggreg ! aggregation method
: CHARACTER(len = 10) :: group  ! group: datetime, default, ESTM, Snow, etc.
: INTEGER             :: level  ! output priority level: 0 for highest (defualt output)
: END TYPE varAttr
\end{sphinxVerbatim}
\end{quote}

\end{itemize}

\item {} 
Code should be written generally

\item {} 
Data set for testing should be provided

\item {} 
Demonstration that the model performance has improved when new code
has been added or that any deterioration is warranted.

\item {} 
Additional requirements for modelling need to be indicated in the
manual

\item {} 
All code should be commented in the program (with initials of who
made the changes \textendash{} name specified somewhere and institution)

\item {} 
The references used in the code and in the equations will be
collected to a webpage

\item {} 
Current developments that are being actively worked on

\end{enumerate}


\subsection{How to setup your development environment on Windows}
\label{\detokenize{DevelopmentGuidelines:how-to-setup-your-development-environment-on-windows}}

\subsubsection{gfortran with NetBeans}
\label{\detokenize{DevelopmentGuidelines:gfortran-with-netbeans}}\begin{enumerate}
\item {} 
Go to Cygwin and install 64-bit. You need to make sure that you
install gfortran, g++, gdb, make and gcc. I am not really sure what
is needed so I tend to install too many packages rather that too few.
Install in c:\textbackslash{}cygwin64

\item {} 
Go to your Environment Variables in advanced system settings in
windows and include

C:\textbackslash{}cygwin64\textbackslash{}bin;C:\textbackslash{}cygwin64\textbackslash{}usr\textbackslash{}bin;C:\textbackslash{}cygwin64\textbackslash{}usr\textbackslash{}local\textbackslash{}bin;C:\textbackslash{}cygwin64\textbackslash{}lib;C:\textbackslash{}cygwin64\textbackslash{}usr\textbackslash{}lib
in your Path.

\item {} 
Install NetBeans from www.netbeans.org. You only need to download the
C/C++ version.

\item {} 
If you don’t have the correct Java, follow the link presented to you
and install correct version.

\item {} 
Copy your code to a folder of your choice.

\item {} 
Create a new project (C/C++ from Existing Source) and use you folder
as the project folder. Keep all other settings.

\item {} 
You are ready to work.

\end{enumerate}

NOTE: Another nice thing to do is to use gfortran from your cluster on
your windows PC. Do the following:
\begin{itemize}
\item {} 
In Netbeans, go to Tools\textgreater{}Options\textgreater{}C/C++ and click Edit next to
localhost. Click Add… and write metcl2. Just keep on clicking until
you need to give your username and password for the cluster.

\item {} 
Now you should be able to run GNU on the cluster from your windows
PC.

\end{itemize}


\subsubsection{Python and PyCharm (Not so good alternative)}
\label{\detokenize{DevelopmentGuidelines:python-and-pycharm-not-so-good-alternative}}\begin{enumerate}
\item {} 
Install python 2.7.X, 64 bit from python.org (Windows x86-64 MSI
installer). Install with default settings.

\item {} 
Visit JetBrain, Pycharm website and obtain a student account (go to
\sphinxstylestrong{Discounted and Complimentary Licenses},
\sphinxurl{https://www.jetbrains.com/pycharm/buy/}). Click on \sphinxstylestrong{For Students and
Teachers}, go to bottom of the page and click \sphinxstylestrong{Apply Now}. Choose
either a student or a teacher status. You will get an email where you
activate your license.

\item {} 
Create a folder which you can use as a project folder. Copy the
python code (*.py) from the suews repository and put it the folder.
If you don’t have access to the repository talk to Fredrik Lindberg.

\item {} 
Download PyCharm professional
(\sphinxurl{https://www.jetbrains.com/pycharm/download/}) and install.

\item {} 
Start PyCharm and activate license using your JetBrains account.

\item {} 
Create a new project (Pure python) and choose the created folder (3)
as your project folder and use your python installation as
interpreter. Click ok in the next message box.

\item {} 
Go to File\textgreater{}Settings \textgreater{}Project Interpreter. Add a new package by
clicking the green plus sign. Search for numpy and install package.
If you get errors, you probably need correct version of Visual
studio. There is an address of a website where you can download it in
the error message when you tried to install numpy.

\item {} 
Also install matplotlib (used for plotting)

\item {} 
Run mainfileLondon.py to do stuff.

\end{enumerate}


\subsubsection{Python and PyCharm (good alternative)}
\label{\detokenize{DevelopmentGuidelines:python-and-pycharm-good-alternative}}\begin{enumerate}
\item {} 
Go to qgis.org and click on download. Choose the installation for
advanced users (64-bit). Choose the \sphinxstylestrong{advanced desktop installation}
and make sure that \sphinxstylestrong{qgis-ltr} is included. Keep other default
settings. This give you a python installation with everything you
need (pretty much). IF you are missing python libraries after the
installation, you can restart the installation file and add more
components.

\item {} 
If you haven’t installed PyCharm, follow set 2 through 5 above.

\item {} 
Create a .bat-file (e.g. PyCharmWithQgis.bat) with the following
content (put it in your folder created earlier and edit it so that
the paths on line 1 and 5 is correct):
\begin{quote}

\fvset{hllines={, ,}}%
\begin{sphinxVerbatim}[commandchars=\\\{\}]
SET OSGEO4W\PYGZus{}ROOT=C:\PYGZbs{}OSGeo4W64
SET QGISNAME=qgis
SET QGIS=\PYGZpc{}OSGEO4W\PYGZus{}ROOT\PYGZpc{}\PYGZbs{}apps\PYGZbs{}\PYGZpc{}QGISNAME\PYGZpc{}
SET QGIS\PYGZus{}PREFIX\PYGZus{}PATH=\PYGZpc{}QGIS\PYGZpc{}
SET PYCHARM= “C:\PYGZbs{}Program   Files   (x86)\PYGZbs{}JetBrains\PYGZbs{}PyCharm   2017.3.5\PYGZbs{}bin\PYGZbs{}pycharm.exe
CALL \PYGZpc{}OSGEO4W\PYGZus{}ROOT\PYGZpc{}\PYGZbs{}bin\PYGZbs{}o4w\PYGZus{}env.bat
SET PATH=\PYGZpc{}PATH\PYGZpc{};\PYGZpc{}QGIS\PYGZpc{}\PYGZbs{}bin
SET PYTHONPATH=\PYGZpc{}QGIS\PYGZpc{}\PYGZbs{}python;\PYGZpc{}PYTHONPATH\PYGZpc{}
start  “PyCharm   aware   of   QGIS”\PYGZbs{}  /B \PYGZpc{}PYCHARM\PYGZpc{} \PYGZpc{}*
\end{sphinxVerbatim}
\end{quote}

\item {} 
Run the bat-file.

\end{enumerate}


\subsubsection{How to make standalone application using py2exe (this is not used, see below)}
\label{\detokenize{DevelopmentGuidelines:how-to-make-standalone-application-using-py2exe-this-is-not-used-see-below}}\begin{enumerate}
\item {} 
In PyCharm, add the pip package (if not already there). See bullet
point 6. Above.7.

\item {} 
Go to \sphinxurl{http://www.lfd.uci.edu/~gohlke/pythonlibs/} and download the
appropriate py2exe package (.whl).

\item {} 
Open a command prompt and go to the folder where you download the
py2exe package and write:

\fvset{hllines={, ,}}%
\begin{sphinxVerbatim}[commandchars=\\\{\}]
\PYG{n}{pip} \PYG{n}{name\PYGZus{}of\PYGZus{}whl\PYGZus{}file}
\end{sphinxVerbatim}

\item {} 
Create a file called setup.py in your working directory with the
following code:
\begin{quote}

\fvset{hllines={, ,}}%
\begin{sphinxVerbatim}[commandchars=\\\{\}]
\PYG{k+kn}{from} \PYG{n+nn}{distutils}\PYG{n+nn}{.}\PYG{n+nn}{core} \PYG{k}{import} \PYG{n}{setup}
\PYG{k+kn}{import} \PYG{n+nn}{py2exe}
\end{sphinxVerbatim}

\fvset{hllines={, ,}}%
\begin{sphinxVerbatim}[commandchars=\\\{\}]
\PYG{n}{setup}\PYG{p}{(}\PYG{n}{console}\PYG{o}{=}\PYG{p}{[}\PYG{l+s+s1}{\PYGZsq{}}\PYG{l+s+s1}{Suews\PYGZus{}wrapper\PYGZus{}v2015a.py}\PYG{l+s+s1}{\PYGZsq{}}\PYG{p}{]}\PYG{p}{)}
\end{sphinxVerbatim}
\end{quote}

\item {} 
From a command prompt (can use terminal in PyCharm) write:

\end{enumerate}

python setup.py install
\begin{enumerate}
\item {} 
Then write:
\begin{quote}

\fvset{hllines={, ,}}%
\begin{sphinxVerbatim}[commandchars=\\\{\}]
\PYG{n}{python} \PYG{n}{setup}\PYG{o}{.}\PYG{n}{py} \PYG{n}{py2exe}
\end{sphinxVerbatim}
\end{quote}

\item {} 
All files and folders needed are now created in a subfolder call
dist. You also have to add the SUEWS executable and all files needed
to run the model.

\end{enumerate}


\subsubsection{How to make standalone application using Pyinstaller (use this)}
\label{\detokenize{DevelopmentGuidelines:how-to-make-standalone-application-using-pyinstaller-use-this}}\begin{enumerate}
\item {} 
Add the pip package (see above)

\item {} 
You need to add the path to where pip.exe is located (usually
C:\textbackslash{}Python27\textbackslash{}Scripts\textbackslash{}). If you don’t know how to add a path in your
environment settings you can temporarily add one in a command prompt
by writing:
\begin{quote}

\fvset{hllines={, ,}}%
\begin{sphinxVerbatim}[commandchars=\\\{\}]
\PYG{n}{path} \PYG{o}{\PYGZpc{}}\PYG{n}{PATH}\PYG{o}{\PYGZpc{}}\PYG{p}{;}\PYG{n}{C}\PYG{p}{:}\PYGZbs{}\PYG{n}{Folder\PYGZus{}where\PYGZus{}pipexecutable\PYGZus{}is\PYGZus{}located}
\end{sphinxVerbatim}
\end{quote}

\item {} 
In the same command prompt, write:
\begin{quote}

\fvset{hllines={, ,}}%
\begin{sphinxVerbatim}[commandchars=\\\{\}]
\PYG{n}{pip} \PYG{n}{install} \PYG{n}{pyinstaller}
\end{sphinxVerbatim}
\end{quote}

\item {} 
Locate yourself where you have your script and write:
\begin{quote}

\fvset{hllines={, ,}}%
\begin{sphinxVerbatim}[commandchars=\\\{\}]
\PYG{n}{pyinstaller} \PYG{n}{suews\PYGZus{}wrapper\PYGZus{}v3}\PYG{o}{.}\PYG{n}{py}
\end{sphinxVerbatim}
\end{quote}

\end{enumerate}


\subsection{SUEWS Prepare Developer}
\label{\detokenize{DevelopmentGuidelines:suews-prepare-developer}}
This is for advanced users regarding SUEWS Prepare plugin in UMEP. The
information in should help with translating the plugin, adding new tabs
or adding new variables.
\begin{itemize}
\item {} \begin{description}
\item[{most important files for making changes to the plugin}] \leavevmode\begin{itemize}
\item {} 
excel documents SUEWS\_init.xlsx, SUEWS\_SiteLibrary.xls and SUEWS\_SiteSelect.xlsx.

\end{itemize}

\end{description}

\item {} \begin{description}
\item[{files are located}] \leavevmode\begin{itemize}
\item {} 
as a part of the plugin in the folder named “Input” (by default in C:\textbackslash{}Users\textbackslash{}your\_username.qgis2\textbackslash{}python\textbackslash{}plugins\textbackslash{}SUEWSPrepare\textbackslash{}Input).

\end{itemize}

\end{description}

\item {} \begin{description}
\item[{SUEWS Prepare uses these files}] \leavevmode\begin{itemize}
\item {} 
for example to generate the amount of site library tabs and the contents of those tabs.

\end{itemize}

\end{description}

\item {} \begin{description}
\item[{Take care}] \leavevmode\begin{itemize}
\item {} 
any changes made to these documents will be lost if they are replaced (e.g. reinstalling or updating the plugin). This can be prevented by making backups of the excel documents before reinstalling or updating.

\end{itemize}

\end{description}

\item {} \begin{description}
\item[{SUEWS\_init.xlsx}] \leavevmode\begin{itemize}
\item {} 
This file handles the amount of site library tabs in the plugin, the name of these tabs and their connection to other excel sheets and text documents. Each sheet represents one tab.

\end{itemize}

\end{description}

\item {} \begin{description}
\item[{SUEWS\_SiteLibrary.xls}] \leavevmode\begin{itemize}
\item {} 
This file contains all the different information connected to different site. Each excel sheet is connected to a different kind of information like vegetation and water data and each line in a sheet represents a different area or site. This information is used to determine what kind of information and variable will be present in a widget of a site library tab.

\begin{figure}[htbp]
\centering
\capstart

\noindent\sphinxincludegraphics{{Figure14}.png}
\caption{Empty widget not connected to any sheet from the SUEWS\_SiteLibrary document.}\label{\detokenize{DevelopmentGuidelines:id4}}\end{figure}

\end{itemize}

\end{description}

\end{itemize}


\subsubsection{SUEWS\_SiteSelect.xlsx}
\label{\detokenize{DevelopmentGuidelines:suews-siteselect-xlsx}}\begin{itemize}
\item {} \begin{description}
\item[{A detailed look at the SUEWS\_init document}] \leavevmode\begin{itemize}
\item {} 
This file contains an example of one line of output from the plugin. It is used by the plugin to check the order of the outputs. It can be considered the least important and useful for developers.

\end{itemize}

\end{description}

\item {} \begin{description}
\item[{Modifying the plugin}] \leavevmode\begin{itemize}
\item {} 
How to work with the excel documents to make changes to existing information inside the plugin such as titles. This could be required for translation or to fix spelling errors.

\end{itemize}

\end{description}

\item {} \begin{description}
\item[{\sphinxstylestrong{Changes available through SUEWS\_init.xlsx}}] \leavevmode\begin{itemize}
\item {} 
The SUEWS\_init determines the number of site library tabs as well as the number of widgets in these tabs and where the widgets will fetch their content. The document contains a number of sheets and every sheet represents one site library tab. The names of the sheets will determine the title of the site library tab. The first one is an example of how the layout of a working sheet should look.

\end{itemize}

\begin{figure}[htbp]
\centering
\capstart

\noindent\sphinxincludegraphics{{Figure15}.png}
\caption{Example of the layout of a sheet in the SUEWS\_init document.}\label{\detokenize{DevelopmentGuidelines:id5}}\end{figure}
\begin{itemize}
\item {} 
Each row of a sheet represents a new widget. Every column of the row is used to determine the specific characteristics of the widget.

\end{itemize}

\end{description}

\end{itemize}


\begin{savenotes}\sphinxattablestart
\centering
\begin{tabular}[t]{|\X{5}{30}|\X{25}{30}|}
\hline

1
&
The content of a widget such as variables are determined by a sheet from the document SUEWS\_SiteLibrary (See {\hyperref[\detokenize{DevelopmentGuidelines:XLS}]{\emph{\#XLS}}}). The first column of a row in a sheet in SUEWS\_init makes the connection between a widget and a sheet in SUEWS\_SiteLibrary. This means that the content of the first column will be the name of a sheet in SUEWS\_SiteLibrary.
\\
\hline
2
&
As well as being connected to a sheet in SUEWS\_SiteLibrary each tab needs to be connected to a text document. This text document will basically be a copy of the site library sheet and will be part of the plugin output. All available text documents are located in the folder named “Output” in the plugin directory and will have the same name as the sheets in SUEWS\_SiteLibrary. The contents of the second column will be the full name of a text document including the file extension, for example “SUEWS\_Veg.txt”.
\\
\hline
3
&
determines the title of the widget’s variable box.
\\
\hline
4
&
optional and determines if there is an identification code for the widget. The identification code is an integer number is used when multiple widgets share a site library sheet but shouldn’t share the same site entries. If an identification code is added the widget will only fetch site entries that match the code. If no identification code is need the column is left blank.
\\
\hline
5
&
determines if there is a default site entry that should be selected in the widget’s drop down menu when the plugin is initiated. If the site code of a site entry (see {\hyperref[\detokenize{DevelopmentGuidelines:XLS}]{\emph{\#XLS}}}) is added to the fifth column this site entry will be automatically selected in the widget on plugin start up.
\\
\hline
6
&
When the plugin generates an output each widget will provide the selected site code in the widget as part of the output. The sixth column is the index of the site code in the plugin output. It should not be changed without careful consideration as there is a risk of the site code overwriting other information in the output if it is.
\\
\hline
\end{tabular}
\par
\sphinxattableend\end{savenotes}

\begin{figure}[htbp]
\centering
\capstart

\noindent\sphinxincludegraphics{{Figure16}.png}
\caption{Red outline illustrates the title for the widget variable box.}\label{\detokenize{DevelopmentGuidelines:id6}}\end{figure}
\begin{itemize}
\item {} \begin{description}
\item[{Change the variables in the variable box of a widget}] \leavevmode\begin{itemize}
\item {} 
The content of a widget is decided by what sheet in the document SUEWS\_SiteLibrary.xls it is connected to. This connection is created by the information in the first column of a sheet in SUEWS\_init. To make changes edit the text in the first column to match the name of the sheet you want to fetch information from. Example: Let’s say for the purposes of this example that we want the content of the tab named “Paved” to have the same content of the tab named “Evergreen”. To do this we must change the connection in the paved sheet of SUEWS\_init to match that of the evergreen sheet. In the evergreen sheet we can see it’s connected to a sheet in SUEWS\_SiteLibrary called SUEWS\_Veg. If we change the text of the first column in the paved sheet to match this, the content of the tab will change to the same as the evergreen tab. \sphinxcode{\sphinxupquote{{}`PICTURE? this needs attention{}`}}

\end{itemize}

\end{description}

\end{itemize}


\begin{savenotes}\sphinxattablestart
\centering
\begin{tabular}[t]{|\X{5}{30}|\X{25}{30}|}
\hline

2
&
is like the first a kind of connection but instead of a sheet it’s to a text document. The text file is close to a copy of the sheet a tab is connected to. If the sheet connection is changed the text file connection should be changed as well. Example: If we did the changes to the paved tab in the example above. In the current state of the paved sheet any changes (for example adding a new site) made would write to the wrong text file. Therefore we also need to change the second column to match the correct text file. In this case to “SUEWS\_Veg.txt”
\\
\hline
\end{tabular}
\par
\sphinxattableend\end{savenotes}
\begin{itemize}
\item {} 
\sphinxstylestrong{Change the title of the variable box in the widget}

\end{itemize}


\begin{savenotes}\sphinxattablestart
\centering
\begin{tabular}[t]{|\X{5}{30}|\X{25}{30}|}
\hline

3
&
title of the variable box in the widget. “Variable box” is referring to the box on the right hand side of the widget that contains the variables from the site library. If the title needs to be changed simply edit the text in the third column of the correct sheet and the new title of the box will match that. Example: Following the above examples, the title “Building surface characteristics” no longer matches the content of the variable box in the paved tab. Replace the text in the third column to “Vegetation surface characteristics” and our title will now make more sense.   \sphinxcode{\sphinxupquote{{}`PICTURE? this needs attention{}`}}
\\
\hline
\end{tabular}
\par
\sphinxattableend\end{savenotes}
\begin{itemize}
\item {} \begin{description}
\item[{Change the default parameters for a widget}] \leavevmode\begin{itemize}
\item {} 
fourth and fifth columns are optional information and decide if there are any default parameters for a widget. The number in the fourth column decides if there is an identification code for the tab. This identification code is used to exclude entries from the site library. Many tabs might link to the same site library sheet and if there is an identification code only the entries that match the code will be shown in the widget. If there is a number in the fifth column the tab will try to match this number against the site codes (not to be confused with the identification code). The side codes are the codes that fill out the drop down box in the widget marked “code” and each code represent one site library entry. If there exist a default site code for a tab this code will be selected in the drop down menu on the plugin start up. Example: Let’s keep making changes to the paved sheet. Right now the identification code for the sheet is “1” and the default site code is “661”. If we change the identification code (fourth column) to “4” a different set of site entries will be available for selection in the widget. One of the site codes that are now available is “662”. By changing the content of the fifth tab to “662” this will now be the default site code for the widget.   \sphinxcode{\sphinxupquote{{}`PICTURE? this needs attention{}`}}

\end{itemize}
\begin{itemize}
\item {} 
Change the order of the widget site code in the final output of the plugin

\end{itemize}
\begin{itemize}
\item {} 
A widget’s contribution to the final output of the plugin will be the selected site code in the widget. This code will be placed somewhere on a predetermined place in a long list of variables. The sixth column in a SUEWS\_init sheet represents this position in the final output. To change a widget’s output order edit the number in the sixth column. Take care to make sure changing the position doesn’t overwrite any other information. The order of the final output is also closely tied to the document SUEWS\_SiteSelect, see more {[}{[}\#XLSX{]}.

\end{itemize}

\end{description}

\item {} \begin{description}
\item[{Editing a tab name}] \leavevmode\begin{itemize}
\item {} 
The name of the tabs in the SUEWS Prepare main window correspond to the names of the sheets in the excel document SUEWS\_init. To edit a tab name simply change the name of the sheet.

\item {} 
Example: After all the changes made to the paved sheet in SUEWS\_init the name “paved” as a description of the tab no longer fit. By renaming the sheet to “vegetation” the tab will have a more fitting name.  \sphinxcode{\sphinxupquote{{}`this needs attention PICTURE?{}`}}

\end{itemize}

\end{description}

\item {} \begin{description}
\item[{Changes available through SUEWS\_SiteLibrary.xls}] \leavevmode\begin{itemize}
\item {} 
What can be made through the SUEWS\_SiteLibrary.xls.

\item {} 
The SUEWS\_SiteLibrary document is what defines the variables inside a tab. This document defines the titles and tooltips for the variables as well as the values for the variables on different sites.
\begin{quote}

\begin{figure}[htbp]
\centering
\capstart

\noindent\sphinxincludegraphics{{Figure17}.png}
\caption{Different rows of a site library sheet highlighted with different color. For the variable title row and the site entry rows the different purposes of the columns have been illustrated.}\label{\detokenize{DevelopmentGuidelines:id7}}\end{figure}
\end{quote}

\end{itemize}

\end{description}

\item {} \begin{description}
\item[{Variable index}] \leavevmode\begin{itemize}
\item {} 
first row of a site library sheet is an index of the variables in the sheet.

\end{itemize}

\end{description}

\item {} \begin{description}
\item[{Variable and metadata titles}] \leavevmode\begin{itemize}
\item {} 
second row contains the titles of the variables. The first cell is always the title “Code”. After all the variable titles follows a blank cell. The cells that follows will be titles for metadata, it is also possible that there is no metadata for the sheet. The row always end with the titles “Photo”, “LC\_previous” and “LC\_code” in that order.

\end{itemize}

\end{description}

\item {} \begin{description}
\item[{Variable tooltips}] \leavevmode\begin{itemize}
\item {} 
third row contains tooltips or longer descriptions of the variable titles.

\end{itemize}

\end{description}

\item {} \begin{description}
\item[{Site entries}] \leavevmode\begin{itemize}
\item {} 
A site entry represents one complete set of values for all the variables in the sheet. One row represents one site entry. The first cell of a site entry always contains the site code. This code is used to differentiate between different site entries and needs to be a unique integer number for the sheet. The following cells contain values for different variables until an exclamation mark marks the end of variables. If there are any metadata descriptions these will be in the cells following the exclamation mark. The last three cells are in order: a photo url if there is one otherwise the cell is left blank, a blank cell and lastly the identification code if there is one (otherwise the cell is left blank). The two last rows: The two last rows of the sheet contains a single “  -9” in the first cell. These rows are used by the plugin to signify the end of the data in the sheet and nothing below these rows will be read.

\end{itemize}

\end{description}

\item {} \begin{description}
\item[{Change the title of a variable}] \leavevmode\begin{itemize}
\item {} 
To change the title of a variable, first navigate to the correct sheet in SUEWS\_SiteLibrary. The titles of all variables are decided by the text in the second row. Replace the text in a column to change the name of a single variable or for example translation purposes replace every word in the second row with its translation.

\end{itemize}

\end{description}

\item {} \begin{description}
\item[{Change the tooltip of a variable}] \leavevmode\begin{itemize}
\item {} \begin{description}
\item[{The tooltip of a variable is a longer description than the title that shows up when the user hovers over the variable text box.}] \leavevmode
\begin{figure}[htbp]
\centering
\capstart

\noindent\sphinxincludegraphics{{Figure18}.png}
\caption{Tooltip of a variable.}\label{\detokenize{DevelopmentGuidelines:id8}}\end{figure}

\end{description}

\item {} 
The third row of a SUEWS\_SiteLibrary sheet defines the tooltip of a variable. To changes it, replace the text for the relevant column in the third row.

\end{itemize}

\end{description}

\item {} \begin{description}
\item[{Changes available through SUEWS\_SiteSelect.xlsx}] \leavevmode\begin{itemize}
\item {} 
The document SUEWS\_SiteSelect.xlsx is mainly connected to the final output of the plugin. Most developers won’t need to make any changes to it. Developers mainly concerned with the layout of the SUEWS Prepare plugin will not need to be concerned about SUEWS\_SiteSelect.

\end{itemize}

\end{description}

\item {} \begin{description}
\item[{Change the order of the final output}] \leavevmode\begin{itemize}
\item {} 
The second row of the sheet SUEWS\_SiteSelect contains text strings that are used by the plugin to identify a variables place in the final output of the plugin. Changing the order of the strings in the second row will similarly affect the final output.

\end{itemize}

\end{description}

\end{itemize}


\subsubsection{Adding to the plugin}
\label{\detokenize{DevelopmentGuidelines:adding-to-the-plugin}}
How to make additions to the plugin (e.g. adding new tabs). Earlier information will be useful when adding to the plugin. i.e. read earlier sections before reading this one.
\begin{itemize}
\item {} \begin{description}
\item[{Adding a new tab to the plugin}] \leavevmode\begin{itemize}
\item {} 
As discussed ({\hyperref[\detokenize{DevelopmentGuidelines:XLSX}]{\emph{\#XLSX}}}) the excel document SUEWS\_init.xlsx is closely tied to how the plugin generates tabs. The plugin will generate tabs according to the number of sheets in this excel document and according to the information in the sheets. A single sheet represents one new tab. Every row in a sheet represents a widget that will be added to the tab. Every column in a sheet contains certain information that decides the specifics for a widget such as what variables will be added. The first sheet of the excel document is an example sheet that can be used as a quick reference for the content of the columns. For a more detailed description see {\hyperref[\detokenize{DevelopmentGuidelines:XLSX}]{\emph{\#XLSX}}}.

\end{itemize}

\end{description}

\item {} \begin{description}
\item[{To add a new tab to the plugin:}] \leavevmode\begin{enumerate}
\item {} 
Create a new sheet in the SUEWS\_init document. The order of the sheets will match the order of the tabs in the plugin. Do not place the sheet first in the excel document as this is used as a placeholder for the example sheet. The name of the sheet will become the title of the tab.

\item {} 
Add the name of a sheet from the SUEWS\_SiteLibrary document to the first column. This will be what decides the content of the first widget in the tab. See {\hyperref[\detokenize{DevelopmentGuidelines:ADD}]{\emph{\#ADD}}} if there is a need to create a new sheet for the tab.

\item {} 
Add the name of a text file that will receive the output of the widget to the second column. In most cases the text file should have the same name as the sheet from step two. Make sure to add the file extension, for example .txt, to the second column as well.

\item {} 
Add a title for the widget in the third column. This title should describe what the variables in the widget represent, for example “Paved surface characteristics”.   *

\item {} 
The content of the fourth column is optional. This column contains a code that can be used if multiple tabs share a sheet from SUEWS\_SiteLibrary. The code is used to identify what site entries belong to which tab and widget. (See {\hyperref[\detokenize{DevelopmentGuidelines:XLS}]{\emph{\#XLS}}}) Leave the column empty if no identification code is needed.

\item {} 
The content of the fifth column is optional. This column can be used if there is a site entry in the sheet from step two that should be selected in the widget’s drop down menu by default. Enter the site code of a site entry in the fifth column to make it the default. Note that the site code is not the same as an identification code (See {\hyperref[\detokenize{DevelopmentGuidelines:XLS}]{\emph{\#XLS}}}). Leave the column empty if there is no default site.

\item {} 
The sixth column represents the index of the widget output in the order of the plugins final output. The widget output will be the site code selected in the drop down menu. Make sure that the index doesn’t overwrite an already existing output. The easiest way to make sure of this is to check the document SUEWS\_SiteSelect for the index of the last variable and use the index after the last variable.

\item {} 
To add more widgets to the tab, follow the instructions from step 2 and forward again on the following rows of the sheet.

\end{enumerate}

\end{description}

\item {} \begin{description}
\item[{\textless{}div id=”ADD” Adding a new set of site variables to the plugin\textless{}/div\textgreater{}}] \leavevmode\begin{itemize}
\item {} 
As discussed in {\hyperref[\detokenize{DevelopmentGuidelines:XLS}]{\emph{\#XLS}}} the variables of a site (and consequently the variables that appear in a widget connected to this site) are generated from the excel document SUEWS\_SiteLibrary. One sheet represents the variables of a type of site and can be connected to multiple widgets and tabs. A new site sheet must fulfil certain conditions. The first row of the sheet should be an index of the variables in the sheet that ranges from one to the amount of variables. The second row should       contain the titles for the variables and the first column should always be “Code”. Furthermore the second row should always end with the titles “Photo”, “LC\_previous” and “LC\_code” in that order. The third row should contain longer tooltips or descriptions of the variables. The rows following the third row should each represent one site entry. Lastly the sheet should end with two rows that just contains “  -9” in the first column. For a more detailed description see {\hyperref[\detokenize{DevelopmentGuidelines:XLS}]{\emph{\#XLS}}}.

\item {} 
There are two options when adding site entries; it can be done manually directly in the sheet or through the plugin when the sheet has been connected to a widget. (See Section 6.1 and 3.3.2)

\item {} 
When adding a site entry manually certain conditions must be followed:
\begin{itemize}
\item {} 
The first entry should be the site code for the entry. This needs to be an integer unique for the sheet.

\item {} 
The column following the last variable needs to contain an exclamation mark designating the end of the variables.

\item {} 
If there are metadata titles for the sheet the information for these should be entered in the columns following the exclamation mark. Metadata is optional for all site entries.

\item {} 
For the last three mandatory titles; the “Photo” column can contain a url link to a picture representing the site entry. The “LC\_previous” column can be left blank. The “LC\_code” column can contain an identification code if this is needed for the site entry.

\end{itemize}

\item {} 
Each new sheet needs a matching text document located in the “Output” directory of the plugin. This text document needs to mimic most of the excel sheet. Instead of columns separating the variables the text document should use tab indents and each line in the text document represents a row in the sheet. The first line of the text document should be an index of the variables. The second line should be the variable titles. The text document should not contain the variable tooltips therefore       the site entries should start on the third line of the text document as opposed to the fourth row of the excel sheet. Any site entries added manually to the excel sheet needs to be manually entered to the text document as well. The two last lines of the text document should just contain a single “  -9”. To add a new site library sheet use the methodology above and follow these steps:
\begin{enumerate}
\item {} 
Create a new sheet in the excel document SUEWS\_SiteLibrary.

\item {} 
If you know how many variables the sheet will contain start numbering the first row of the sheet from 1 in the first column to the amount of variables in the last column. Otherwise fill in this row when all the variable titles have been entered. The numbering should end where an exclamation mark would be entered for a site entry.

\item {} 
Add the variable titles in the second row. Start with “Code” in the first column. Leave a column blank where the exclamation mark for a site entry would be entered. If there are any metadata descriptors relevant for the site add the title for these after the blank column. Examples of this could be “City”, “Area” or “Description”. After adding any metadata descriptors add “Photo”, “LC\_previous” and “LC\_code” in the last three columns of the row in that order.

\item {} 
Add the tooltips of the variables to the columns in the third row. These should be longer descriptions of what the variable represents.

\item {} 
OPTIONAL: Add any site entries manually to the sheet. Use a new row for each site entry. The other option is to use the plugin to add all the entries. One benefit of using the plugin is that the site entries will be added automatically to the text document as well as the excel sheet.

\item {} 
Add “  -9” to the first column of the two last rows of the excel sheet.

\item {} 
Create a text document in the “Output” directory of the plugin. Name it after the excel sheet if possible.

\item {} 
Make a copy of the variable index in the first row of the sheet as the first line of the text document. Use tab indents as a replacement for columns.

\item {} 
Make a copy of the variable titles in the first row of the sheet as the second line of the text document. Use tab indents as a replacement for columns.

\item {} 
Copy any manually added site entries in the sheet to the text document. Each site entry is a new line in the text document. Use tab indents as a replacement for columns.

\item {} 
End the text document with two lines, both only containing a single “  -9”.

\end{enumerate}

\end{itemize}

\end{description}

\end{itemize}


\chapter{Known Issues}
\label{\detokenize{Known_Issues:known-issues}}\label{\detokenize{Known_Issues:id1}}\label{\detokenize{Known_Issues::doc}}\begin{itemize}
\item {} 
QGIS (27/September/2017) \sphinxstylestrong{had} an issue using gdal which causes
QGIS to create a minidump when the software is closed. This issue has
now been fixed (issue
\#\sphinxhref{https://hub.qgis.org/issues/13061}{13061}). Other issues found
should be reported to our
\sphinxhref{https://bitbucket.org/fredrik\_ucg/umep/issues}{repository}.

\item {} 
UMEP plugin is not compatible with matplotlib versions 2.x. Use
instead 1.5.x. (23/August/2017)

\item {} 
Mac users might have issue pointing at non-existing directories. Work
around is to manually create directories before starting any
UMEP-process.

\item {} 
Only use standard English alpha-numeric characters (e.g. no space, å,
\% etc.)

\item {} 
Issues has been reported using .sdat rasters. GeoTiff are
recommended.

\end{itemize}


\chapter{FAQ (Frequently Asked Questions)}
\label{\detokenize{FAQ:faq-frequently-asked-questions}}\label{\detokenize{FAQ:faq}}\label{\detokenize{FAQ::doc}}\begin{itemize}
\item {} \begin{description}
\item[{How do I upgrade the plugin?}] \leavevmode
When a new LTR version is released it will be available from the repository. In QGIS to check for updates, go to \sphinxstyleemphasis{Plugins\textgreater{}Manage and Install Plugins…}. If the UMEP plugin is in bold, a new version is available. On how to upgrade to the development version, see {\hyperref[\detokenize{Getting_Started::doc}]{\sphinxcrossref{\DUrole{doc,doc,doc}{Getting started}}}}.

\end{description}

\item {} \begin{description}
\item[{How do I uninstall the plugin?}] \leavevmode
Go to \sphinxstyleemphasis{Plugins\textgreater{}Manage and Install Plugins…}. Locate the UMEP plugin and click \sphinxstyleemphasis{Uninstall}.

\end{description}

\item {} \begin{description}
\item[{How do I install other python packages (e.g. pandas) as well as other libraries not included in the Desktop Express Install of QGIS?}] \leavevmode
Follow the instruction from this {\hyperref[\detokenize{Getting_Started:python-libraries}]{\sphinxcrossref{\DUrole{std,std-ref,std,std-ref}{link}}}}.

\end{description}

\item {} \begin{description}
\item[{MY new raster is just black after using e.g. the \sphinxstyleemphasis{Wall Height and Aspect} plugin. What is wrong?}] \leavevmode
Probably nothing. Is is just QGIS that scales the a loaded raster by excluding outliers and if you have large areas with e.g. zeros (which you have in the resulting raster from this plugin) it looks like there is only zeros in your new raster. Go to properties of your new raster layers and reclassify your values that should visualized.

\end{description}

\item {} \begin{description}
\item[{Can the UMEP-plugin be used when \sphinxstylestrong{Nodata}-values are present in the input rasters?}] \leavevmode
Yes, it can but we strongly recommend you to reclassify Nodata values to e.g. 0 before using them in UMEP. Here is a forum discussion that can help: \sphinxurl{https://gis.stackexchange.com/questions/12418/redefining-nodata-value-into-zero-in-qgis}

\end{description}

\item {} \begin{description}
\item[{Why is UMEP having problems saving output files?}] \leavevmode
Check that your path contains only English characters. For Mac users: the UMEP graphical interface will occasionally want to create a folder instead of selecting a folder. In this case in \sphinxstyleemphasis{Save As:} write the folder name you would like to save your output, press \sphinxstyleemphasis{Save}, when it asks \sphinxstyleemphasis{“…folder name…” already exists. Do you want to replace it?} press \sphinxstyleemphasis{Replace}.

\end{description}

\item {} \begin{description}
\item[{How is frontal area index calculated in \sphinxstyleemphasis{Image Morphometric Parameters} plugins?}] \leavevmode
Our method is only using one line through the center of the grid for each wind direction. This is because we rotate the DSM and hence it is only the center line that includes height information. We do this since we are using a pure raster-based approach and if we were to instead rotate the search direction vector we would end up with different lengths for each wind direction. If you want to investigate a certain wind direction I suggest that you use a section of wind directions; e.g. 45 degrees.

\end{description}

\item {} \begin{description}
\item[{How do I report a bug?}] \leavevmode
Report it at the \sphinxhref{http://bitbucket.org/fredrik\_ucg/umep/issues/}{repository}

\end{description}

\item {} \begin{description}
\item[{What can UMEP do?}] \leavevmode
{\hyperref[\detokenize{Introduction:pluginarchitecture}]{\sphinxcrossref{\DUrole{std,std-ref,std,std-ref}{Tool Architecture}}}} provides an overview

\end{description}

\item {} \begin{description}
\item[{Who has developed this?}] \leavevmode
{\hyperref[\detokenize{People_Involved___Acknowledgements::doc}]{\sphinxcrossref{\DUrole{doc,doc,doc}{People}}}} involved in development

\end{description}

\item {} \begin{description}
\item[{What are the development guidelines?}] \leavevmode
\sphinxcode{\sphinxupquote{ContributeCoding}}

\end{description}

\item {} 
How can I uninstall QGIS?
\begin{quote}

Uninstalling QGIS on a Windows PC is not done via the Control Panel as most other software. To uninstall completely, start the OSGeo4W setup (found in your start menu) and choose \sphinxstyleemphasis{Advanced install}. Continue until you come up to the window where you can add, remove and upgrade the different packages in your QGIS installation. Click on the small wheel with two arrows next to \sphinxstyleemphasis{Desktop} until \sphinxstyleemphasis{Uninstall} is seen. This removes shortcuts and most of the files related to QGIS. However, not all OSGeo products are removed. IF you want remove everything, open your File Explorer and remove the folder manually where you installed the OSGEO products (usually under \sphinxstyleemphasis{C:\textbackslash{}OSGeo4W64}).
\begin{quote}

\begin{figure}[htbp]
\centering
\capstart

\noindent\sphinxincludegraphics{{Uninstall}.png}
\caption{QGIS installation dialog (Advanced)}\label{\detokenize{FAQ:id1}}\end{figure}
\end{quote}
\end{quote}

\item {} \begin{description}
\item[{How do I ask other questions?}] \leavevmode
There is an email list. Or you can ask them at the \sphinxhref{http://bitbucket.org/fredrik\_ucg/umep/issues/}{repository}

\end{description}

\end{itemize}


\chapter{Abbreviations}
\label{\detokenize{Abbreviations:abbreviations}}\label{\detokenize{Abbreviations:id1}}\label{\detokenize{Abbreviations::doc}}\begin{description}
\end{description}

\sphinxstylestrong{CDSM} - Canopy Digital Surface Model
A vegetation raster grid where vegetation heights is given in meter above ground level. Pixels with no vegetation should be zero.

\sphinxstylestrong{CRS}  - Coordinate Reference System

\sphinxstylestrong{CRU}  - Climatic Research Unit

\sphinxstylestrong{DEM}  - Digtial Elevation Model
Here, same definition as DTM.

\sphinxstylestrong{DSM}  - Digital Surface Model
A raster grid including both buildings and ground given in meter above sea level.

\sphinxstylestrong{DTM}  - Digtial Terrain Model
A raster grid including only ground heights given in meter above sea level.

\sphinxstylestrong{ECMWF} - European Centre for Medium-Range Weather Forecasts

\sphinxstylestrong{GIS} - Geographical Information System

\sphinxstylestrong{LCZ} - Local Climate Zone

\sphinxstylestrong{LST} - Local Standard Time

\sphinxstylestrong{LTR}  - Long term release

\sphinxstylestrong{LUCY} - Large scale Urban Consumption of energy model

\sphinxstylestrong{m agl} - metres above ground level

\sphinxstylestrong{m asl} - metres above sea level

\sphinxstylestrong{OTF} - On the Fly
Used in QGIS when different geodatasets with different coordinate systems are projected in the same automatically.

\sphinxstylestrong{Q}$_{\text{F}}$ - Anthropogenic heat flux

\sphinxstylestrong{SEBE} - Solar Energy on Building Envelopes

\sphinxstylestrong{SOLWEIG} - Solar and longwave environmental irradiance geometry model

\sphinxstylestrong{SUEWS} - Surface urban energy and water balance scheme

\sphinxstylestrong{SVF} - Sky View Factor

\sphinxstylestrong{TDSM} - Trunk zone Digital Surface Model
A raster grid specifying the height up to the buttom of a vegetation canopy in meter above ground level. Pixels with no trunk height should be zero.

\sphinxstylestrong{UMEP} - Urban Multi-scale Environmental Predictor

\sphinxstylestrong{WFDEI} - WATCH Forcing Data methodology applied to ERA-Interim data

\sphinxstylestrong{WDF} - WATCH Forcing Data methodology applied to ERA-40 data

\sphinxstylestrong{WUDAPT} - The World Urban Database and Access Portal Tools


\chapter{Introduction}
\label{\detokenize{Tutorials/Tutorials:introduction}}\label{\detokenize{Tutorials/Tutorials:tutorials}}\label{\detokenize{Tutorials/Tutorials::doc}}
To help users getting started with UMEP, the community is working on
setting up tutorials and instructions for different parts of the UMEP
tool. The tutorials are available are found in the table below.


\begin{savenotes}\sphinxattablestart
\centering
\begin{tabular}[t]{|\X{20}{100}|\X{20}{100}|\X{30}{100}|\X{30}{100}|}
\hline
\sphinxstyletheadfamily 
Topic
&\sphinxstyletheadfamily 
Parts of UMEP
&\sphinxstyletheadfamily 
Name
&\sphinxstyletheadfamily 
Application
\\
\hline
Source Area Footprint
&
Pre-Processor
&
{\hyperref[\detokenize{Tutorials/Footprint::doc}]{\sphinxcrossref{\DUrole{doc,doc,doc}{Footprint modelling}}}}
&
Interpretation of eddy covariance flux source areas
\\
\hline
Urban energy balance
&
Processor
&
{\hyperref[\detokenize{Tutorials/IntroductionToSuews::doc}]{\sphinxcrossref{\DUrole{doc,doc,doc}{Urban Energy Balance - SUEWS Introduction}}}}
&
Energy, water and radiation fluxes for one location
\\
\hline
Urban energy balance
&
Pre-Processor and Processor
&
{\hyperref[\detokenize{processor/Urban Energy Balance Urban Energy Balance (SUEWS.BLUEWS, advanced):suewsadvanced}]{\sphinxcrossref{\DUrole{std,std-ref,std,std-ref}{Urban Energy Balance: Urban Energy Balance (SUEWS/BLUEWS, advanced)}}}}
&
Energy, water and radiation fluxes for one location
\\
\hline
Urban energy balance
&
Pre-Processor, Processor and Post-processor
&
{\hyperref[\detokenize{Tutorials/SuewsSpatial:suewsspatial}]{\sphinxcrossref{\DUrole{std,std-ref,std,std-ref}{Urban Energy Balance - SUEWS Spatial}}}}
&
Energy, water and radiation fluxes for a spatial grid
\\
\hline
Urban energy balance
&
Pre-Processor, Processor and Post-processor
&
{\hyperref[\detokenize{Tutorials/SuewsWUDAPT:suewswudapt}]{\sphinxcrossref{\DUrole{std,std-ref,std,std-ref}{Urban Energy Balance - SUEWS and WUDAPT}}}}
&
Making use of \sphinxhref{http://www.wudapt.org/}{WUDAPT} local climate zones in \sphinxhref{http://suews-docs.readthedocs.io}{SUEWS}
\\
\hline
Solar Energy
&
Processor and Post-Processor
&
{\hyperref[\detokenize{Tutorials/SEBE::doc}]{\sphinxcrossref{\DUrole{doc,doc,doc}{Solar Energy - Introduction to SEBE}}}}
&
Amount of solar energy received on building facets
\\
\hline
Outdoor thermal comfort
&
Pre-Processor and Processor
&
{\hyperref[\detokenize{Tutorials/IntroductionToSolweig:introductiontosolweig}]{\sphinxcrossref{\DUrole{std,std-ref,std,std-ref}{Thermal Comfort - Introduction to SOLWEIG}}}}
&
Mean radiation temperature modelling in complex urban settings
\\
\hline
Anthropogenic heat
&
Processor
&
{\hyperref[\detokenize{Tutorials/GQF::doc}]{\sphinxcrossref{\DUrole{doc,doc,doc}{Anthropogenic heat - GQF}}}}
&
Anthropogenic heat modelling in London using GQF (uses the GreaterQF methodology)
\\
\hline
Anthropogenic heat
&
Processor
&
{\hyperref[\detokenize{Tutorials/LQF::doc}]{\sphinxcrossref{\DUrole{doc,doc,doc}{Anthropogenic heat - LQF}}}}
&
Anthropogenic heat modelling in London using LQF (uses the LUCY methodology)
\\
\hline
\end{tabular}
\par
\sphinxattableend\end{savenotes}


\section{Footprint modelling}
\label{\detokenize{Tutorials/Footprint:footprint-modelling}}\label{\detokenize{Tutorials/Footprint:footprint}}\label{\detokenize{Tutorials/Footprint::doc}}

\subsection{Introduction}
\label{\detokenize{Tutorials/Footprint:introduction}}
Each meteorological instrument has a ‘\sphinxstyleemphasis{source area}’ (sometimes
referred to as \sphinxstyleemphasis{footprint}), the area that influences the measurement.
The shape and location of that area is a function of the meteorological
variable the sensor measures and the method of operation of the sensor.

For turbulent heat fluxes measured with a sonic anemometer, extensive
effort has been directed to try and model the ‘probable source area
location’ (Leclerc and Foken 2014). Numerous models exist, but the
Kormann and Meixner (2001) and Kljun et al. (2015) models are used in
UMEP. Both models require input of information about the wind direction,
stability, turbulence characteristics (friction velocity, variance of
the lateral or crosswind wind velocity) and roughness parameters. Kljun
et al. (2015) requires the boundary layer height.

\begin{figure}[htbp]
\centering
\capstart

\noindent\sphinxincludegraphics[width=1100\sphinxpxdimen]{{Footprint_ReadingSourceArea}.png}
\caption{Example result (click on image for larger image)}\label{\detokenize{Tutorials/Footprint:id1}}\end{figure}


\subsection{Initial Practical steps}
\label{\detokenize{Tutorials/Footprint:initial-practical-steps}}\begin{itemize}
\item {} 
If \sphinxstylestrong{QGIS} is not on your computer you will {\hyperref[\detokenize{Getting_Started:getting-started}]{\sphinxcrossref{\DUrole{std,std-ref,std,std-ref}{need to install
it}}}}

\item {} 
Then install the {\hyperref[\detokenize{Getting_Started:getting-started}]{\sphinxcrossref{\DUrole{std,std-ref,std,std-ref}{UMEP}}}}.plugin

\item {} 
Start the QGIS software

\item {} 
If not visible on the desktop use the \sphinxstylestrong{Start} button to find the
software (i.e. Find QGIS under your applications)

\item {} 
Select \sphinxstylestrong{QGIS 2.18.x Desktop} (or the latest version installed). Do not use QGIS3 at this point.

\end{itemize}

When you open it on the top toolbar you will see \sphinxstylestrong{UMEP}.

\begin{figure}[htbp]
\centering
\capstart

\noindent\sphinxincludegraphics[width=960\sphinxpxdimen]{{Footprint_UMEP_location}.png}
\caption{Location of footprint plugin (click on image for larger image)}\label{\detokenize{Tutorials/Footprint:id2}}\end{figure}
\begin{itemize}
\item {} 
If UMEP is not on your machine, add the UMEP plugin by go to \sphinxstyleemphasis{Plugins\textgreater{}Manage and Install Plugins} in QGIS
and search for UMEP. Click \sphinxstylestrong{Install plugin}. Here you can also see if there is newer versions of your added plugins.

\item {} 
Preferably, read through the section in the \sphinxhref{SourceArea(Point)}{online manual}
BEFORE using the model, so you are familiar with it’s operation and terminology used.

\end{itemize}


\subsubsection{Data for Tutorial}
\label{\detokenize{Tutorials/Footprint:data-for-tutorial}}
Use the appropriate data:
\begin{itemize}
\item {} 
Reading: \sphinxhref{https://github.com/Urban-Meteorology-Reading/Urban-Meteorology-Reading.github.io/tree/master/other\%20files/DataReading.zip}{download}

\item {} 
London: \sphinxhref{https://github.com/Urban-Meteorology-Reading/Urban-Meteorology-Reading.github.io/tree/master/other\%20files/DataSmallAreaLondon.zip}{download}

\end{itemize}


\subsubsection{Prior to Starting}
\label{\detokenize{Tutorials/Footprint:prior-to-starting}}\begin{enumerate}
\item {} 
Download the \sphinxstylestrong{Data needed for the Tutorial}. You can use either Reading of the London dataset.

\item {} 
Load the Raster data ({\hyperref[\detokenize{Abbreviations:abbreviations}]{\sphinxcrossref{\DUrole{std,std-ref,std,std-ref}{DEM, DSM,
CDSM}}}})
\sphinxstyleemphasis{files \textendash{} DOES A CDSM EXIST? Yes for London, No for Reading}
\begin{itemize}
\item {} 
Go to: \sphinxstyleemphasis{Layer \textgreater{} Add layer \textgreater{} Add Raster Layer}. Locate downloaded files add add them to your QGIS project.

\end{itemize}

\end{enumerate}

\begin{figure}[htbp]
\centering
\capstart

\noindent\sphinxincludegraphics[width=960\sphinxpxdimen]{{Footprint_Add_Raster_Layer}.png}
\caption{Loading a raster layer to QGIS (click on image for larger image)}\label{\detokenize{Tutorials/Footprint:id3}}\end{figure}
\begin{itemize}
\item {} 
Have a look at the \sphinxstylestrong{layers} (see lower left) - if you untick the
box layer names from the top you can see the different layers.

\end{itemize}

\begin{figure}[htbp]
\centering
\capstart

\noindent\sphinxincludegraphics[width=1070\sphinxpxdimen]{{Footprint_ReadingMap}.png}
\caption{The Reading data loaded into QGIS (click on image for larger image)}\label{\detokenize{Tutorials/Footprint:id4}}\end{figure}


\subsection{Source Area Modelling}
\label{\detokenize{Tutorials/Footprint:source-area-modelling}}
To access the Source area model or Footprint model, go to \sphinxstyleemphasis{UMEP -\textgreater{} Pre-processor \textgreater{} Urban Morphology \textgreater{} Source Area Model (Point)}.
\begin{enumerate}
\item {} 
Click on Select Point on Canvas \textendash{} then select a point (e.g. where an Eddy Covariance (EC) tower site is located)

\item {} 
Select the appropriate surface elevation data file names

\item {} 
Choose the model you wish to run (Kormann and Meixner 2001 or Kljun et al. 2015)

\item {} 
Some initial parameters values are given - think about what would be appropriate values for your site and period   of interest. The manual has   more information (e.g. definitions) of the input variables.
\begin{itemize}
\item {} 
The values are dependent on the meteorological conditions and the surface surrounding the site. The former   obviously vary on an hour to hour basis (or shorter time periods), whereas the others are dependent on the wind direction and the fetch characteristics.

\end{itemize}

\item {} 
Add a prefix and an output folder.

\item {} 
Tick “add the integrated source area to your project”. This will provide visual information of the location of the source area (footprint)

\item {} 
Click \sphinxstylestrong{Run}. If you get an error/warning message (model is unable to execute your request, as the maximum fetch exceeds the extent of your grid for your point of interest. measure the distance to the limit of your raster maps
\begin{itemize}
\item {} 
To allow the model to work for the dataset with your point of interest you need to adjust the maximum fetch distance.

\item {} 
Locate the Measure tool.

\item {} 
Measure the distance to the point of interest to the boundary of the DSM data set.

\item {} 
Adjust the maximum fetch.

\item {} 
Click Run and wait for the calculations to finish.

\end{itemize}

\end{enumerate}

\begin{figure}[htbp]
\centering
\capstart

\noindent\sphinxincludegraphics[width=960\sphinxpxdimen]{{Footprint_SourceAreaModel}.png}
\caption{Snapshot of the Footprint plugin using Reading data (click on image for larger image)}\label{\detokenize{Tutorials/Footprint:id5}}\end{figure}

The \sphinxstylestrong{output is a source area} grid showing the cumulative percentage of source area influencing the flux at the point of interest.
\begin{itemize}
\item {} 
To display the legend correctly in the layer window: Double-click on the source area grid and then click OK without doing any changes. The source area display is showing up to 98\% of the cumulative area.

\item {} 
Other output: A text file giving both the input setting variables and the output morphometric parameters calculated based on the source area generated. More information is provided in the manual, row titled: “Output”

\end{itemize}

It is possible to input a text file to generate a source area based on morphometric parameters (e.g. an hourly dataset). However, for now you can moodify the input variables set in the interface. Format of file is given in the manual.


\subsection{Iterative process}
\label{\detokenize{Tutorials/Footprint:iterative-process}}
To work with a site with no value known \sphinxstyleemphasis{a priori}.
\begin{enumerate}
\item {} 
Use the {\hyperref[\detokenize{pre-processor/Urban Morphology Morphometric Calculator (Point):morphometriccalculator-point}]{\sphinxcrossref{\DUrole{std,std-ref,std,std-ref}{Urban Morphology: Morphometric Calculator (Point)}}}} tool in the UMEP plugin to select a
point to get the initial parameter values:
\begin{enumerate}
\item {} 
UMEP-\textgreater{} Pre-Processor -\textgreater{} Urban Morphology -\textgreater{} Image Morphometric
Calculator (Point)

\item {} 
Open the output files

\end{enumerate}

\item {} 
\sphinxstylestrong{Anisotropic} file \textendash{} has the values in, e.g., 5 degree \sphinxstylestrong{sectors}
\textendash{} i.e. what you selected. This is appropriate if the area is very
inhomogeneous.

\item {} 
\sphinxstylestrong{Isotropic} file - has the \sphinxstylestrong{average value} for the area

\item {} 
Use these values to populate the source area model window.

\end{enumerate}


\subsubsection{Roughness parameters}
\label{\detokenize{Tutorials/Footprint:roughness-parameters}}
In the UMEP plugin the roughness length and zero plane displacement
length can be calculated with a morphometric method based on the Rule of
Thumb (Grimmond and Oke 1999) as the default. There are other methods
available: Bottema (1995), Kanda et al. (2013), Macdonald et al. (1998),
Millward-Hopkins et al. (2011) and Raupach (1994, 1995). Many of these
have been developed for urban roughness elements. The Raupach method was
originally intended for forested areas but has also been found to
perform well for urban areas.

With wind profile and eddy covariance anemometric data and the source
area model, appropriate parameters can be determined and morphometric
methods assessed (e.g. Kent et al. 2017).


\subsection{References}
\label{\detokenize{Tutorials/Footprint:references}}\begin{itemize}
\item {} 
Bottema M 1995: Parameterisation of aerodynamic roughness parameters
in relation to air pollutant removal ef?ciency of streets. Air
Pollution Engineering and Management, H. Power et al., Eds.,
Computational Mechanics, 235\textendash{}242.

\item {} 
Grimmond CSB and TR Oke 1999: Aerodynamic properties of urban areas
derived, from analysis of surface form. \sphinxhref{http://journals.ametsoc.org/doi/full/10.1175/1520-0450\%281999\%29038\%3C1262\%3AAPOUAD\%3E2.0.CO\%3B2}{Journal of Applied
Climatology 38:9,
1262-1292}

\item {} 
Kanda M, Inagaki A, Miyamoto T, Gryschka M, Raasch S 2013: A new
aerodynamic parameterization for real urban surfaces. \sphinxhref{http://link.springer.com/article/10.1007/s10546-013-9818-x?no-access=true}{Boundary-
Layer Meteorol 148:357\textendash{}377.
doi:10.1007/s10546-013-9818-x}

\item {} 
Kent CW, Grimmond CSB, Barlow J, Gatey D, Kotthaus S, Lindberg F,
Halios CH 2017: Evaluation of Urban Local-Scale Aerodynamic
Parameters: Implications for the Vertical Profile of Wind Speed and
for Source Areas. Boundary-Layer Meteorol 164:183-213.

\item {} 
Kljun N, Calanca P, Rotach MW, Schmid HP 2015: A simple
two-dimensional parameterisation for Flux Footprint Prediction (FFP).
\sphinxhref{http://www.geosci-model-dev.net/8/3695/2015/}{Geoscientific Model
Development.8(11):3695-713.}

\item {} 
Kormann R, Meixner FX 2001: An analytical footprint model for
non-neutral stratification. Bound.-Layer Meteorol., 99, 207\textendash{}224
\sphinxurl{http://www.sciencedirect.com/science/article/pii/S2212095513000497\#b0145}

\item {} 
Kotthaus S and Grimmond CSB 2014: Energy exchange in a dense urban
environment \textendash{} Part II: Impact of spatial heterogeneity of the
surface. Urban Climate 10, 281\textendash{}307
\sphinxurl{http://www.sciencedirect.com/science/article/pii/S2212095513000497}

\item {} 
Leclerc MY and Foken TK 2014: Footprints in Micrometeorology and
Ecology. \sphinxhref{http://www.springer.com/us/book/9783642545443}{Springer, xix, 239 p.
E-book}

\item {} 
Macdonald, R. W., R. F. Griffiths, and D. J. Hall, 1998: An improved
method for estimation of surface roughness of obstacle arrays.
\sphinxhref{http://www.sciencedirect.com/science/article/pii/S1352231097004032}{Atmos. Environ., 32,
1857\textendash{}1864}

\item {} 
Millward-Hopkins JT, Tomlin AS, Ma L, Ingham D, Pourkashanian M 2011:
Estimating aerodynamic parameters of urban-like surfaces with
heterogeneous building heights. \sphinxhref{http://link.springer.com/article/10.1007\%2Fs10546-011-9640-2}{Boundary-Layer Meteorol 141:443\textendash{}465.
doi:10.1007/s10546-011-9640-2}

\item {} 
Raupach MR 1994: Simpli?ed expressions for vegetation roughness
length and zero-plane displacement as functions of canopy height and
area index. \sphinxhref{http://link.springer.com/article/10.1007\%2FBF00709229}{Bound.-Layer Meteor., 71, 211\textendash{}216.
doi:10.1007/BF0070922}

\item {} 
Raupach MR 1995: Corrigenda. \sphinxhref{http://link.springer.com/article/10.1007/BF00709356}{Bound.-Layer Meteor., 76,
303\textendash{}304.}

\end{itemize}


\subsubsection{Contributors to the material covered}
\label{\detokenize{Tutorials/Footprint:contributors-to-the-material-covered}}
\sphinxstylestrong{University of Reading:} Christoph Kent, Simone Kotthaus, Sue Grimmond
\sphinxstylestrong{University of Gothenburg:} Fredrik Lindberg Background work also
comes from: UBC (Andreas Christen); Germany: Kormann and Meixner (2001);
Japan: Kanda et al. (2013); UK: Millward-Hopkins et al. (2011),
Macdonald et al. (1998); Australia: Raupach (1994, 1995); Netherlands:
Bottema (1995)

Authors of this document: Kent, Grimmond (2016). Lindberg

\sphinxhref{https://bitbucket.org/fredrik\_ucg/umep/}{UMEP Repository}


\section{Urban Energy Balance - SUEWS Introduction}
\label{\detokenize{Tutorials/IntroductionToSuews:urban-energy-balance-suews-introduction}}\label{\detokenize{Tutorials/IntroductionToSuews:introductiontosuews}}\label{\detokenize{Tutorials/IntroductionToSuews::doc}}

\subsection{Introduction}
\label{\detokenize{Tutorials/IntroductionToSuews:introduction}}
In this tutorial you will use a land-surface model,
\sphinxhref{http://suews-docs.readthedocs.io}{SUEWS} to simulate energy
exchanges in a city (London is the test case).

SUEWS (Surface Urban Energy and Water Balance Scheme) allows the energy
and water balance exchanges for urban areas to be modelled (Järvi et al.
2011, 2014, Ward et al. 2016a). The model is applicable at the
neighbourhood scale (e.g. 10$^{\text{2}}$ to 10$^{\text{4}}$ m). The fluxes
calculated are applicable to height of about 2-3 times the mean height
of the roughness elements; i.e. above the \sphinxhref{http://glossary.ametsoc.org/wiki/Roughness\_sublayer}{roughness sublayer
(RSL)}. The use
of SUEWS within Urban Multi-scale Environmental Predictor (UMEP)
provides an introduction to the model and the processes simulated, the
parameters used and the impact on the resulting fluxes.

Tools such as this, once appropriately assessed for an area, can be used
for a broad range of applications. For example, for climate services
(e.g. \sphinxurl{http://www.wmo.int/gfcs/}). Running a model can allow analyses,
assessments, and long-term projections and scenarios. Most applications
require not only meteorological data but also information about the
activities that occur in the area of interest (e.g. agriculture,
population, road and infrastructure, and socio-economic variables).

Model output may be needed in many formats depending on a users’ needs.
Thus, the format must be useful, while ensuring the science included
within the model is appropriate. The figure below provides an overview of
{\hyperref[\detokenize{index:index-page}]{\sphinxcrossref{\DUrole{std,std-ref,std,std-ref}{UMEP}}}}, a city based climate
service tool (CBCST). Within UMEP there are a number of models which can
predict and diagnose a range of meteorological processes. In this
activity we are concerned with SUEWS, initially the central components
of the model. See \sphinxhref{http://suews-docs.readthedocs.io}{manual} or
published papers for more detailed information of the model.

\begin{figure}[htbp]
\centering
\capstart

\noindent\sphinxincludegraphics[width=378\sphinxpxdimen]{{SUEWSIntro_UMEP_overview}.png}
\caption{Overview of the climate service tool UMEP (from Lindberg et al. 2018)}\label{\detokenize{Tutorials/IntroductionToSuews:id1}}\end{figure}

SUEWS can be run in a number of different ways:
\begin{enumerate}
\item {} 
Within UMEP via the Simple selection. This is useful for becoming
familiar with the model (Part 1)

\item {} 
Within UMEP via the Advanced selection. This can be used to exploit
the full capabilities of the model (Part 2)

\item {} 
SUEWS standalone (see
\sphinxhref{http://suews-docs.readthedocs.io}{manual})

\item {} 
Within other larger scale models (e.g. WRF).

\end{enumerate}


\subsection{SUEWS Simple Objectives}
\label{\detokenize{Tutorials/IntroductionToSuews:suews-simple-objectives}}
This tutorial introduces SUEWS and demonstartes how to run the model within {\hyperref[\detokenize{index:index-page}]{\sphinxcrossref{\DUrole{std,std-ref,std,std-ref}{UMEP (Urban
Multi-scale Environmental Predictor)}}}}. {\hyperref[\detokenize{Abbreviations:abbreviations}]{\sphinxcrossref{\DUrole{std,std-ref,std,std-ref}{Help with
Abbreviations}}}}


\subsubsection{Steps}
\label{\detokenize{Tutorials/IntroductionToSuews:steps}}\begin{enumerate}
\item {} 
An introduction to the model and how it is designed.

\item {} 
Different kinds of input data that are needed to run the model

\item {} 
How to run the model

\item {} 
How to examine the model output

\end{enumerate}


\subsection{Initial Steps}
\label{\detokenize{Tutorials/IntroductionToSuews:initial-steps}}
UMEP is a python plugin used in conjunction with
\sphinxhref{http://www.qgis.org}{QGIS}. To install the software and the UMEP
plugin see the {\hyperref[\detokenize{Getting_Started:getting-started}]{\sphinxcrossref{\DUrole{std,std-ref,std,std-ref}{getting started}}}} section in the UMEP manual.

As UMEP is under development, some documentation may be missing and/or
there may be instability. Please report any issues or suggestions to our
\sphinxhref{https://bitbucket.org/fredrik\_ucg/umep/}{repository}.


\subsection{SUEWS Model Inputs}
\label{\detokenize{Tutorials/IntroductionToSuews:suews-model-inputs}}
Details of the model inputs and outputs are provided in the \sphinxhref{http://suews-docs.readthedocs.io}{SUEWS
manual}. As this tutorial is
concerned with a \sphinxstylestrong{simple application} only the most critical
parameters are shown. Other versions allow many other parameters to be
modified to more appropriate values if applicable. The table below
provides an overview of the parameters that can be modified in the
Simple application of SUEWS.


\begin{savenotes}\sphinxattablestart
\centering
\begin{tabulary}{\linewidth}[t]{|T|T|T|}
\hline
\sphinxstyletheadfamily 
Type
&\sphinxstyletheadfamily 
Definition
&\sphinxstyletheadfamily 
Reference/Comments
\\
\hline&
\sphinxstylestrong{Building/ Tree
Morphology}
&\\
\hline
Mean height of
Building/Trees (m)
&&
Grimmond and Oke
(1999)
\\
\hline
Frontal area index
&
Area of the front
face of a roughness
element exposed to
the wind relative to
the plan area.
&
Grimmond and Oke
(1999), Fig 2
\\
\hline
Plan area index
&
Area of the roughness
elements relative to
the total plan area.
&
Grimmond and Oke
(1999), Fig 2
\\
\hline&
\sphinxstylestrong{Land cover
fraction}
&
Should sum to 1
\\
\hline
Paved
&
Roads, sidewalks,
parking lots,
impervious surfaces
that are not
buildings
&\\
\hline
Buildings
&
Buildings
&
Same as the plan area
index of buildings in
the morphology
section.
\\
\hline
Evergreen trees
&
Trees/shrubs that
retain their
leaves/needles all
year round
&
Tree plan area index
will be the sum of
evergreen and
deciduous area. Note:
this is the same as
the plan area index
of vegetation in the
morphology section.
\\
\hline
Deciduous trees
&
Trees/shrubs that
lose their leaves
&
Same as above
\\
\hline
Grass
&
Grass
&\\
\hline
Bare soil
&
Bare soil \textendash{} non
vegetated but water
can infilitrate
&\\
\hline
Water
&
River, ponds,
swimming pools,
fountains
&\\
\hline&
\sphinxstylestrong{Initial
conditions}
&
What is the state of
the conditions when
the model run begins?
\\
\hline
Days since rain
(days)
&
This will influence
irrigation behaviour
in the model. If
there has been rain
recently then it will
be longer before
irrigiation occurs.
&
If this is a period
or location when no
irrigation is
permitted/occurring
then this is not
critical as the model
will calculate from
this point going
forward.
\\
\hline
Daily mean
temperature (°C)
&
Influences irrigation
and anthropogenic
heat flux
&\\
\hline
Soil mositure status
(\%)
&
This will influence
both evaporation and
runoff processes
&
If close to 100\%
then there is plenty
of water for
evaporation but also
a higher probability
of flooding if
intense precipitation
occurs.
\\
\hline&
\sphinxstylestrong{Other}
&\\
\hline
Year
&
What days are
weekdays/weekends
&\\
\hline
Latitude (°)
&
Solar related
calculations
&\\
\hline
Longitude (°)
&
Solar related
calculations
&\\
\hline
UTC (h)
&
Time zone
&
Influences solar
related calculations
\\
\hline
\end{tabulary}
\par
\sphinxattableend\end{savenotes}


\subsection{How to Run SuewsSimple from the UMEP-plugin}
\label{\detokenize{Tutorials/IntroductionToSuews:how-to-run-suewssimple-from-the-umep-plugin}}\begin{enumerate}
\item {} 
Open SuewsSimple from \sphinxstyleemphasis{UMEP -\textgreater{} Processor -\textgreater{} Urban Energy Balance -\textgreater{}
Urban Energy Balance, SUEWS (Simple)}. The GUI that opens looks quite
extensive but it is actually not that complicated to start a basic
model run (figure below). Some additional information about the plugin is
found in the left window. As you can read, a \sphinxstylestrong{test dataset} from
observations for London, UK (\sphinxhref{http://www.sciencedirect.com/science/article/pii/S2212095513000503}{Kotthaus and Grimmond
2014},
\sphinxhref{http://www.sciencedirect.com/science/article/pii/S2212095516300256}{Ward et al.
2016a})
is included in within the plugin.

\end{enumerate}

\begin{figure}[htbp]
\centering
\capstart

\noindent\sphinxincludegraphics[width=1107\sphinxpxdimen]{{SUEWSIntro_Interface}.png}
\caption{The interface for SUEWS, simple version (click on image to make it larger).}\label{\detokenize{Tutorials/IntroductionToSuews:id2}}\end{figure}
\begin{enumerate}
\item {} 
To make use of this dataset click on \sphinxstylestrong{Add settings from test
dataset} (see near bottom of the box). The land cover fractions and
all other settings originate from Kotthaus and Grimmond (2014). They
used a source area model to obtain the different input parameters
(their \sphinxhref{http://www.sciencedirect.com/science/article/pii/S2212095513000497}{Fig. 7 in Kotthaus and Grimmond,
2014}).

\item {} 
Before you start the model, change the location of the output data to
any location of your choice. Also, make notes on the settings such as
\sphinxstyleemphasis{Year} etc.

\item {} 
Do a model run and explore the results by clicking \sphinxstylestrong{Run}. A command
window appears, when SUEWS performs the calculations using the
settings from the interface. Once the calculations are done, some of
the results are shown in two summary plots.

\end{enumerate}

\begin{figure}[htbp]
\centering
\capstart

\noindent\sphinxincludegraphics[width=900\sphinxpxdimen]{{SUEWSIntro_SuewsSimplefig1}.png}
\caption{Model output from SUEWS (simple) using the default settings and data (click on image to make it larger).}\label{\detokenize{Tutorials/IntroductionToSuews:id3}}\end{figure}

\begin{figure}[htbp]
\centering
\capstart

\noindent\sphinxincludegraphics[width=900\sphinxpxdimen]{{SUEWSIntro_SuewsSimplefig2}.png}
\caption{Model output from SUEWS (simple) using the default settings and data (click on image to make it larger).}\label{\detokenize{Tutorials/IntroductionToSuews:id4}}\end{figure}


\subsection{Model results}
\label{\detokenize{Tutorials/IntroductionToSuews:model-results}}
The graphs in the upper figure are the monthly mean energy (left) and water
balance (right). The lower graphs show the radiation fluxes,
energy fluxes, and water related outputs throughout the year. This plot
includes a lot of data and it might be difficult to examine it in
detail.

To zoom into the plot: use the tools in the top left corner, to zoom to
a period of interest. For example, the Zoom in to about the last ten
days in March (figure below). This was a period with clear relatively
weather.

\begin{figure}[htbp]
\centering
\capstart

\noindent\sphinxincludegraphics[width=900\sphinxpxdimen]{{SUEWSIntro_SuewsSimplefig2zoom}.png}
\caption{Zoom in on end of March from the daily plot (click on image to make it larger).}\label{\detokenize{Tutorials/IntroductionToSuews:id5}}\end{figure}


\subsection{Saving a Figure}
\label{\detokenize{Tutorials/IntroductionToSuews:saving-a-figure}}
Use the disk tool in the upper left corner.
\begin{enumerate}
\item {} 
.jpg

\item {} 
.pdf

\item {} 
.tif (Recommended)

\item {} 
.png

\end{enumerate}


\subsection{Output data Files}
\label{\detokenize{Tutorials/IntroductionToSuews:output-data-files}}
In the output folder (you selected earlier) you will find (at least)
three files:
\begin{enumerate}
\item {} 
\sphinxstylestrong{Kc98\_2012\_60.txt} \textendash{} provides the 60 min model results for site
“KC1” for the year 2012

\item {} 
\sphinxstylestrong{Kc\_FilesChoices.txt} \textendash{} this indicates all options used in the
model run see the SUEWS Manual for interpretation of content (this is
for when you are doing large number of runs so you know exactly what
options were used in each run)

\item {} 
\sphinxstylestrong{Kc98\_DailyState.txt} \textendash{} this provides the daily mean state (see
SUEWS manual for detailed explanation). This allows you to see, for
example, the daily state of the LAI (leaf area index).

\item {} 
\sphinxstylestrong{Kc\_OutputFormat.txt} \textendash{} provides detailed information about the
output files such as extended descriptions for each column including
units.

\end{enumerate}

If you open these files in a text editor. To understand the header
variables read the \sphinxhref{http://suews-docs.readthedocs.io}{SUEWS
manual}.


\subsection{Sensitivity to land surface fractions}
\label{\detokenize{Tutorials/IntroductionToSuews:sensitivity-to-land-surface-fractions}}\begin{wrapfigure}{r}{0pt}
\centering
\noindent\sphinxincludegraphics{{SUEWSIntro_LCFs}.png}
\caption{Land cover fractions (click on image to make it larger).}\label{\detokenize{Tutorials/IntroductionToSuews:id6}}\end{wrapfigure}

The previous results are for a densely build-up area in
London, UK. In order to test the sensitivity of SUEWS to some surface
properties you can think about changing some of the surface properties
in the SUEWS Simple. For example, change the land cover fraction by:
\begin{enumerate}
\item {} 
Change the land cover fractions as seen in the figure. Feel free to
select other values as long as all the fractions \sphinxstyleemphasis{add up to 1.0}.

\item {} 
Save the output to a different folder by selecting \sphinxstyleemphasis{output folder}.

\item {} 
Click \sphinxstyleemphasis{Run}.

\end{enumerate}


\subsection{References}
\label{\detokenize{Tutorials/IntroductionToSuews:references}}\begin{itemize}
\item {} 
Grimmond CSB and Oke 1999: Aerodynamic properties of urban areas
derived, from analysis of surface form. \sphinxhref{http://journals.ametsoc.org/doi/abs/10.1175/1520-0450(1999)038\%3C1262\%3AAPOUAD\%3E2.0.CO\%3B2}{Journal of Applied
Climatology 38:9,
1262-1292}

\item {} 
Grimmond et al. 2015: Climate Science for Service Partnership: China,
Shanghai Meteorological Servce, Shanghai, China, August 2015.

\item {} 
Järvi L, Grimmond CSB \& Christen A 2011: The Surface Urban Energy and
Water Balance Scheme (SUEWS): Evaluation in Los Angeles and Vancouver
\sphinxhref{http://www.sciencedirect.com/science/article/pii/S0022169411006937}{J. Hydrol. 411,
219-237}

\item {} 
Järvi L, Grimmond CSB, Taka M, Nordbo A, Setälä H \&Strachan IB 2014:
Development of the Surface Urban Energy and Water balance Scheme
(SUEWS) for cold climate cities, , \sphinxhref{http://www.geosci-model-dev.net/7/1691/2014/}{Geosci. Model Dev. 7,
1691-1711}

\item {} 
Kormann R, Meixner FX 2001: An analytical footprint model for
non-neutral stratification. \sphinxhref{http://www.sciencedirect.com/science/article/pii/S2212095513000497\#b0145}{Bound.-Layer Meteorol., 99,
207\textendash{}224}

\item {} 
Kotthaus S and Grimmond CSB 2014: Energy exchange in a dense urban
environment \textendash{} Part II: Impact of spatial heterogeneity of the
surface. \sphinxhref{http://www.sciencedirect.com/science/article/pii/S2212095513000497}{Urban Climate 10,
281\textendash{}307}

\item {} 
Onomura S, Grimmond CSB, Lindberg F, Holmer B, Thorsson S 2015:
Meteorological forcing data for urban outdoor thermal comfort models
from a coupled convective boundary layer and surface energy balance
scheme. Urban Climate. 11:1-23 \sphinxhref{http://www.sciencedirect.com/science/article/pii/S2212095514000856}{(link to
paper)}

\item {} 
Ward HC, L Järvi, S Onomura, F Lindberg, A Gabey, CSB Grimmond 2016
SUEWS Manual V2016a, \sphinxurl{http://urban-climate.net/umep/SUEWS} Department
of Meteorology, University of Reading, Reading, UK

\item {} 
Ward HC, Kotthaus S, Järvi L and Grimmond CSB 2016b: Surface Urban
Energy and Water Balance Scheme (SUEWS): Development and evaluation
at two UK sites. \sphinxhref{http://www.sciencedirect.com/science/article/pii/S2212095516300256}{Urban Climate
http://dx.doi.org/10.1016/j.uclim.2016.05.001}

\item {} 
Ward HC, S Kotthaus, CSB Grimmond, A Bjorkegren, M Wilkinson, WTJ
Morrison, JG Evans, JIL Morison, M Iamarino 2015b: Effects of urban
density on carbon dioxide exchanges: observations of dense urban,
suburban and woodland areas of southern England. \sphinxhref{http://dx.doi.org/10.1016/j.envpol.2014.12.031}{Env Pollution 198,
186-200}

\end{itemize}

Authors this document: Lindberg and Grimmond (2016)


\subsection{Definitions and Notation}
\label{\detokenize{Tutorials/IntroductionToSuews:definitions-and-notation}}
To help you find further information about the acronyms they are
classified by \sphinxstylestrong{T}: Type of term: \sphinxstylestrong{C}: computer term, \sphinxstylestrong{S}: science
term, \sphinxstylestrong{G}: GIS term.


\begin{savenotes}\sphinxattablestart
\centering
\begin{tabulary}{\linewidth}[t]{|T|T|T|T|}
\hline
\sphinxstyletheadfamily &\sphinxstyletheadfamily 
Definition
&\sphinxstyletheadfamily 
T
&\sphinxstyletheadfamily 
Reference/Comme
nt
\\
\hline
DEM
&
Digital
elevation model
&
G
&\\
\hline
DSM
&
Digital surface
model
&
G
&\\
\hline
FAI (?:sub:\sphinxcode{\sphinxupquote{F}})
&
Frontal area
index
&
S
&
Grimmond and
Oke (1999)
\\
\hline
GUI
&
Graphical User
Interface
&
C
&\\
\hline
LAI
&
Leaf Area Index
&
S
&\\
\hline
PAI (?:sub:\sphinxcode{\sphinxupquote{P}})
&
Plan area index
&
S
&\\
\hline
png
&
Portable
Network
Graphics
&
C
&
format for
saving
plots/figures
\\
\hline
QGIS
&&
G
&
www.qgis.org
\\
\hline
SUEWS
&
Surface Urban
Energy and
Water Balance
Scheme
&
S
&\\
\hline
Tif
&
Tagged Image
File Format
&
C
&
format for
saving
plots/figures
\\
\hline
UI
&
user interface
&
C
&\\
\hline
UMEP
&
Urban
Multi-scale
Environmental
predictor
&
C
&\\
\hline
z$_{\text{0}}$
&
Roughness
length for
momentum
&
S
&
Grimmond and
Oke (1999)
\\
\hline
z$_{\text{d}}$
&
Zero plane
displacement
length for
momentum
&
S
&
Grimmond and
Oke (1999)
\\
\hline
\end{tabulary}
\par
\sphinxattableend\end{savenotes}


\subsection{Further explanation}
\label{\detokenize{Tutorials/IntroductionToSuews:further-explanation}}

\subsubsection{Morphometric Methods to determine Roughness parameters:}
\label{\detokenize{Tutorials/IntroductionToSuews:morphometric-methods-to-determine-roughness-parameters}}
For more and overview and details see \sphinxhref{http://journals.ametsoc.org/doi/abs/10.1175/1520-0450\%281999\%29038\%3C1262\%3AAPOUAD\%3E2.0.CO\%3B2}{Grimmond and Oke
(1999)}
and \sphinxhref{https://link.springer.com/article/10.1007\%2Fs10546-017-0248-z}{Kent et al.
(2017a)}.
This uses the height and spacing of roughness elements (e.g. buildings,
trees) to model the roughness parameters. For more details see \sphinxhref{https://link.springer.com/article/10.1007\%2Fs10546-017-0248-z}{Kent et
al.
(2017a)},
\sphinxhref{http://www.sciencedirect.com/science/article/pii/S0167610516307346?via\%3Dihub}{Kent et al.
(2017b)}
and {[}Kent et al. (2017c){]}. UMEP has tools for doing this: \sphinxstyleemphasis{Pre-processor
-\textgreater{} Urban Morphology}


\subsubsection{Source Area Model}
\label{\detokenize{Tutorials/IntroductionToSuews:source-area-model}}
For more details see \sphinxhref{http://www.sciencedirect.com/science/article/pii/S2212095513000497}{Kotthaus and Grimmond
(2014b)}
and \sphinxhref{https://link.springer.com/article/10.1007\%2Fs10546-017-0248-z}{Kent et al.
(2017a)}.
The \sphinxhref{https://link.springer.com/article/10.1023\%2FA\%3A1018991015119}{Kormann and Meixner
(2001)}
model is used to determine the probable area that a turbulent flux
measurement was impacted by. This is a function of wind direction,
stability, turbulence characteristics (friction velocity, variance of
the lateral wind velocity) and roughness parameters.


\section{Urban Energy Balance - SUEWS Advanved}
\label{\detokenize{Tutorials/SuewsAdvanced:urban-energy-balance-suews-advanved}}\label{\detokenize{Tutorials/SuewsAdvanced:suewsadvanced}}\label{\detokenize{Tutorials/SuewsAdvanced::doc}}

\subsection{Introduction}
\label{\detokenize{Tutorials/SuewsAdvanced:introduction}}
The tutorial {\hyperref[\detokenize{Tutorials/IntroductionToSuews:introductiontosuews}]{\sphinxcrossref{\DUrole{std,std-ref}{Urban Energy Balance - SUEWS Introduction}}}} should be completed first. This tutorial is designed to work with QGIS 2.18.


\subsubsection{Objectives}
\label{\detokenize{Tutorials/SuewsAdvanced:objectives}}\begin{enumerate}
\item {} 
To explore the link between QGIS and SUEWS to include new
site-specific information

\item {} 
To examine how it affects the energy fluxes

\end{enumerate}


\subsubsection{Overview of steps}
\label{\detokenize{Tutorials/SuewsAdvanced:overview-of-steps}}\begin{enumerate}
\item {} 
Initially become familiar with SUEWS advanced which is a
plugin that makes it possible for you to set all parameters that can
be manipulated in SUEWS as well as execute the model on mutiple grids ({\hyperref[\detokenize{Tutorials/SuewsSpatial:suewsspatial}]{\sphinxcrossref{\DUrole{std,std-ref}{Urban Energy Balance - SUEWS Spatial}}}}).

\item {} 
Derive new surface information

\item {} 
Run the model

\end{enumerate}


\subsection{How to Run from the UMEP-plugin}
\label{\detokenize{Tutorials/SuewsAdvanced:how-to-run-from-the-umep-plugin}}
\sphinxstylestrong{How to run SUEWS Advanced:}
\begin{enumerate}
\item {} 
Open the plugin which is located at \sphinxstyleemphasis{UMEP -\textgreater{} Processor -\textgreater{} Urban Energy
Balance -\textgreater{} Urban Energy Balance, SUEWS/BLUEWS (Advanced)}. This has
most of the general settings (e.g. activate the snow module etc.)
which are related to
\sphinxhref{http://suews-docs.readthedocs.io/en/latest/input\_files/RunControl/RunControl.html}{RunControl.nml}.

\item {} 
Use the Input folder:
\begin{itemize}
\item {} 
\sphinxstyleemphasis{C:/Users/your\_user\_name/.qgis2/python/plugins/UMEP/suewsmodel/Input}

\end{itemize}

\item {} 
Create or enter an \sphinxstylestrong{Output directory} of your choice.

\item {} 
From the \sphinxstylestrong{Input folder} - confirm the data are in there.

\item {} 
Fill in the following. File Code: should be Kc, UTC: should be 0
(London).

\item {} 
Run

\item {} 
Make sure that output files are created.

\item {} 
You can now close the \sphinxstylestrong{SUEWS/BLUEWS (Advanced)}-plugin again.

\end{enumerate}

\begin{figure}[htbp]
\centering
\capstart

\noindent\sphinxincludegraphics{{SUEWSAdvanced_SuewsAdvanced}.png}
\caption{Interface for SUEWS Advanced version.}\label{\detokenize{Tutorials/SuewsAdvanced:id1}}\end{figure}


\subsubsection{Sensitivity Test}
\label{\detokenize{Tutorials/SuewsAdvanced:sensitivity-test}}
The default dataset included in \sphinxstylestrong{Suews Simple} has parameters
calculated from a {\hyperref[\detokenize{pre-processor/Urban Morphology Source Area (Point):sourcearea-point}]{\sphinxcrossref{\DUrole{std,std-ref,std,std-ref}{source area
model}}}}
to obtain the appropriate values for the input parameters. Roughness
parameters such as roughness length (z$_{\text{0}}$) and zero plane
displacement length (z$_{\text{d}}$) are calculated using \sphinxhref{MorphometricCalculator(Point)}{morphometric
models}.
Now you will explore the differences in fluxes using the default
settings or using input parameters from the geodata included in the test
datasets available for this tutorial. Download the zip-file (see below)
and extract the files to a suitable location where you both have reading
and writing capabilities.

Data for the tutorial can be downloaded
\sphinxhref{https://github.com/Urban-Meteorology-Reading/Urban-Meteorology-Reading.github.io/tree/master/other\%20files/DataSmallAreaLondon.zip}{here}


\begin{savenotes}\sphinxattablestart
\centering
\begin{tabulary}{\linewidth}[t]{|T|T|}
\hline

\sphinxstylestrong{Geodata}
&
\sphinxstylestrong{Name}
\\
\hline
Ground and building DSM
&
DSM\_LondonCity\_1m.tif (m asl)
\\
\hline
Vegetation DSM
&
CDSM\_LondonCity\_1m.tif (m agl)
\\
\hline
DEM (digital elevation model)
&
DEM\_LondonCity\_1m.tif (masl)
\\
\hline
Land cover
&
LC\_londoncity\_UMEP\_32631
\\
\hline
\end{tabulary}
\par
\sphinxattableend\end{savenotes}

They are all projected in UTM 31N (EPSG:32631). The three surface models
originate from a LiDAR dataset. The land cover data is a mixture of
Ordnance Survey and the LiDAR data.
\begin{enumerate}
\item {} 
Open the geodatasets. Go to \sphinxstyleemphasis{Layer \textgreater{} Add layer \textgreater{} Add Raster Layer}.
Locate the files you downloaded before (see above).

\item {} 
A QGIS style file (.qml) is available for the land cover grid. It can
found in \sphinxstyleemphasis{C:Usersyour\_user\_name.qgis2pythonpluginsUMEP\textbackslash{}
LandCoverReclassifier\textbackslash{}}. Load it in the \sphinxstyleemphasis{Layer \textgreater{} Properties \textgreater{} Style
\textgreater{} Style} (lower left) \sphinxstylestrong{Load file}.

\item {} 
Click Apply before you close so that the names of the classes also
load. You can also get the properties of a layer by right-click on a
layer in the Layers-window.

\item {} 
If you have another land cover dataset you can use the
{\hyperref[\detokenize{pre-processor/Urban Land Cover Land Cover Reclassifier:landcoverreclassifier}]{\sphinxcrossref{\DUrole{std,std-ref,std,std-ref}{LandCoverReclassifier}}}}
in the UMEP pre-processor to populate with the correct values
suitable for the UMEP plugin environment.

\item {} 
Now take a moment and investigate the different geodatasets. What is
the sparial (pixel) resolution? How is ground represented in the
CDSM?

\end{enumerate}


\subsection{Generating data from the geodatasets}
\label{\detokenize{Tutorials/SuewsAdvanced:generating-data-from-the-geodatasets}}\begin{enumerate}
\item {} 
Make certain that you have the geodatafiles open. The file at the top
(left hand side (LHS)) of the list is the one that is shown in the
centre (figure below). You can swap their order using the LHS box.

\item {} 
Open SUEWS Simple.

\item {} 
Begin by adding the test dataset again.

\item {} 
Update the building morphology parameters (top left panel in Suews
Simple).

\item {} 
To generate new values, click on Open tool.

\item {} 
This is another plugin within UMEP that can be used to generate
morphometric parameters

\begin{figure}[htbp]
\centering
\capstart

\noindent\sphinxincludegraphics[width=605\sphinxpxdimen]{{SUEWSAdvanced_QGIS_SuewsSimple}.png}
\caption{QGIS where Suews Simple and Image Morphometric Parameters (Point) is opened.}\label{\detokenize{Tutorials/SuewsAdvanced:id2}}\end{figure}

\item {} 
First, clear the map canvas from your two other plugin windows, e.g.
as figure above.

\item {} 
If you use the default test data in SUEWS Simple - you can overwrite
is as you go.

\item {} 
Locate the eddy covariance tower position on the Strand building,
King’s College London. To find the position, consult Figure 1 (KSS)
in \sphinxhref{http://www.sciencedirect.com/science/article/pii/S2212095513000503}{Kotthaus and Grimmond
(2014)}.

\item {} 
Use Select point on canvas and put a point at that location (left).

\item {} 
Generate a study area. Use 500 m search distance, 5 degree interval
and click Generate study area.

\item {} 
A circular area will be considered. Enter the DSM and DEM files (i.e.
the files you currently have in the viewer)

\item {} 
Click Run.

\begin{figure}[htbp]
\centering
\capstart

\noindent\sphinxincludegraphics{{SUEWSAdvanced_SUEWS_MorphometricParametersBuild}.png}
\caption{Figure 3. Settings for Image Morphometric Parameters for buildings.}\label{\detokenize{Tutorials/SuewsAdvanced:id3}}\end{figure}

\item {} 
In the folder you specified two additional files will be present (i)
isotropic - averages of the morphometric parameters (ii) anisotropic
- values for each wind sector you specified (5 degrees).

\item {} 
Close this plugin

\item {} 
Click on Fetch file from… in the building morphology panel

\item {} 
Choose the isotropic file (just generated).

\item {} 
Do the same for vegetation (upper left panel, right).

\item {} 
Instead of locating the point again you can use the existing point.

\item {} 
You still need to generate a separate study area for the vegetation
calculation.

\item {} 
Examine the CDSM (vegetation file) in your map canvas. As you can
see, this data has no ground heights (ground = 0). Therefore, this
time Tick in the box Raster DSM (only buildings) exist.

\item {} 
Enter the CDSM as your Raster DSM (only buildings).

\begin{figure}[htbp]
\centering
\capstart

\noindent\sphinxincludegraphics[scale=0.75]{{SUEWSAdvanced_SUEWS_MorphometricParametersVeg}.png}
\caption{Settings for Image Morphometric Parameters for vegetation}\label{\detokenize{Tutorials/SuewsAdvanced:id4}}\end{figure}

\item {} 
A warning appears that your vegetation fractions between the
morphology dataset and land cover dataset are large. You can ignore
this for now since the land cover dataset also will change.

\item {} 
Repeat the same procedure for land cover using the Land Cover
Fraction (Point) plugin.

\item {} 
Enter the meteorological file, Year etc. This should be the same as
for the first run you made.

\item {} 
Now you are ready to run the model. Click Run.

\end{enumerate}

If you get an error window (figure below). This error is generate by SUEWS as the sum
of the land cover fractions is not 1. If you calculate carefully, one
part of a thousand is missing (this is probably a rounding error during
data extraction). To fix this issue: add 0.001 to e.g. bare soil. Now
run again.

\begin{figure}[htbp]
\centering
\capstart

\noindent\sphinxincludegraphics[scale=1.0]{{SUEWSAdvanced_Modelrununsuccessful}.png}
\caption{Possible error window from running SUEWS with new settings.}\label{\detokenize{Tutorials/SuewsAdvanced:id5}}\end{figure}

\begin{figure}[htbp]
\centering
\capstart

\noindent\sphinxincludegraphics[width=800\sphinxpxdimen]{{SUEWSAdvanced_SuewsSimpleGeodata}.png}
\caption{The settings for running with geodata derived parameters (old version of GUI).}\label{\detokenize{Tutorials/SuewsAdvanced:id6}}\end{figure}

You are now familiar with the Suews Simple plugin. Your next task is to
choose another location within the geodataset domain, generate data and
run the model. If you choose an area where the fraction of buildings and
paved surfaces are low, consider lowering the population density to get
more realistic model outputs. Compare the results for the different
area.


\subsection{References}
\label{\detokenize{Tutorials/SuewsAdvanced:references}}\begin{itemize}
\item {} 
Grimmond CSB and Oke 1999: Aerodynamic properties of urban areas
derived, from analysis of surface form. \sphinxhref{http://journals.ametsoc.org/doi/abs/10.1175/1520-0450(1999)038\%3C1262\%3AAPOUAD\%3E2.0.CO\%3B2}{Journal of Applied
Climatology 38:9,
1262-1292}

\item {} 
Grimmond et al. 2015: Climate Science for Service Partnership: China,
Shanghai Meteorological Servce, Shanghai, China, August 2015.

\item {} 
Järvi L, Grimmond CSB \& Christen A 2011: The Surface Urban Energy and
Water Balance Scheme (SUEWS): Evaluation in Los Angeles and Vancouver
\sphinxhref{http://www.sciencedirect.com/science/article/pii/S0022169411006937}{J. Hydrol. 411,
219-237}

\item {} 
Järvi L, Grimmond CSB, Taka M, Nordbo A, Setälä H \&Strachan IB 2014:
Development of the Surface Urban Energy and Water balance Scheme
(SUEWS) for cold climate cities, , \sphinxhref{http://www.geosci-model-dev.net/7/1691/2014/}{Geosci. Model Dev. 7,
1691-1711}

\item {} 
Kormann R, Meixner FX 2001: An analytical footprint model for
non-neutral stratification. \sphinxhref{http://www.sciencedirect.com/science/article/pii/S2212095513000497\#b0145}{Bound.-Layer Meteorol., 99,
207-224}

\item {} 
Kotthaus S and Grimmond CSB 2014: Energy exchange in a dense urban
environment - Part II: Impact of spatial heterogeneity of the
surface. \sphinxhref{http://www.sciencedirect.com/science/article/pii/S2212095513000497}{Urban Climate 10,
281â\texteuro{}“307}

\item {} 
Onomura S, Grimmond CSB, Lindberg F, Holmer B, Thorsson S 2015:
Meteorological forcing data for urban outdoor thermal comfort models
from a coupled convective boundary layer and surface energy balance
scheme. Urban Climate. 11:1-23 \sphinxhref{http://www.sciencedirect.com/science/article/pii/S2212095514000856}{(link to
paper)}

\item {} 
Ward HC, L Järvi, S Onomura, F Lindberg, A Gabey, CSB Grimmond 2016
SUEWS Manual V2016a, \sphinxurl{http://urban-climate.net/umep/SUEWS} Department
of Meteorology, University of Reading, Reading, UK

\item {} 
Ward HC, Kotthaus S, Järvi L and Grimmond CSB 2016b: Surface Urban
Energy and Water Balance Scheme (SUEWS): Development and evaluation
at two UK sites. \sphinxhref{http://www.sciencedirect.com/science/article/pii/S2212095516300256}{Urban Climate
http://dx.doi.org/10.1016/j.uclim.2016.05.001}

\item {} 
Ward HC, S Kotthaus, CSB Grimmond, A Bjorkegren, M Wilkinson, WTJ
Morrison, JG Evans, JIL Morison, M Iamarino 2015b: Effects of urban
density on carbon dioxide exchanges: observations of dense urban,
suburban and woodland areas of southern England. \sphinxhref{http://dx.doi.org/10.1016/j.envpol.2014.12.031}{Env Pollution 198,
186-200}

\end{itemize}

Authors of this document: Lindberg and Grimmond (2016)


\subsection{Definitions and Notation}
\label{\detokenize{Tutorials/SuewsAdvanced:definitions-and-notation}}
To help you find further information about the acronyms they are
classified by \sphinxstylestrong{T}: Type of term: \sphinxstylestrong{C}: computer term, \sphinxstylestrong{S}: science
term, \sphinxstylestrong{G}: GIS term.


\begin{savenotes}\sphinxattablestart
\centering
\begin{tabulary}{\linewidth}[t]{|T|T|T|T|}
\hline
\sphinxstyletheadfamily &\sphinxstyletheadfamily 
Definition
&\sphinxstyletheadfamily 
T
&\sphinxstyletheadfamily 
Reference/Comme
nt
\\
\hline
DEM
&
Digital
elevation model
&
G
&\\
\hline
DSM
&
Digital surface
model
&
G
&\\
\hline
FAI (?:sub:\sphinxcode{\sphinxupquote{F}})
&
Frontal area
index
&
S
&
Grimmond and
Oke (1999),
their figure 2
\\
\hline
GUI
&
Graphical User
Interface
&
C
&\\
\hline
LAI
&
Leaf Area Index
&
S
&\\
\hline
PAI (?:sub:\sphinxcode{\sphinxupquote{P}})
&
Plan area index
&
S
&\\
\hline
png
&
Portable
Network
Graphics
&
C
&
format for
saving
plots/figures
\\
\hline
QGIS
&&
G
&
www.qgis.org
\\
\hline
SUEWS
&
Surface Urban
Energy and
Water Balance
Scheme
&
S
&\\
\hline
Tif
&
Tagged Image
File Format
&
C
&
format for
saving
plots/figures
\\
\hline
UI
&
user interface
&
C
&\\
\hline
UMEP
&
Urban
Multi-scale
Environmental
predictor
&
C
&\\
\hline
z$_{\text{0}}$
&
Roughness
length for
momentum
&
S
&
Grimmond and
Oke (1999)
\\
\hline
z$_{\text{d}}$
&
Zero plane
displacement
length for
momentum
&
S
&
Grimmond and
Oke (1999)
\\
\hline
\end{tabulary}
\par
\sphinxattableend\end{savenotes}


\subsection{Further explanation}
\label{\detokenize{Tutorials/SuewsAdvanced:further-explanation}}

\subsubsection{Morphometric Methods to determine Roughness parameters:}
\label{\detokenize{Tutorials/SuewsAdvanced:morphometric-methods-to-determine-roughness-parameters}}
For more and overview and details see \sphinxhref{http://journals.ametsoc.org/doi/abs/10.1175/1520-0450(1999)038\%3C1262\%3AAPOUAD\%3E2.0.CO\%3B2}{Grimmond and Oke
(1999)}.
This uses the height and spacing of roughness elements (e.g. buildings,
trees) to model the roughness parameters. UMEP has tools for doing this:
\sphinxstyleemphasis{Pre-processor -\textgreater{} Urban Morphology}


\subsubsection{Source Area Model}
\label{\detokenize{Tutorials/SuewsAdvanced:source-area-model}}
For more details see Kotthaus and Grimmond (2014b). The Kormann and
Meixner (2001) model is used to determine the probable area that a
turbulent flux measurement was impacted by. This is a function of wind
direction, stability, turbulence characteristics (friction velocity,
variance of the lateral wind velocity) and roughness parameters.


\section{Urban Energy Balance - SUEWS Spatial}
\label{\detokenize{Tutorials/SuewsSpatial:urban-energy-balance-suews-spatial}}\label{\detokenize{Tutorials/SuewsSpatial:suewsspatial}}\label{\detokenize{Tutorials/SuewsSpatial::doc}}

\subsection{Introduction}
\label{\detokenize{Tutorials/SuewsSpatial:introduction}}
In this tutorial you will generate input data for the
\sphinxhref{http://suews-docs.readthedocs.io}{SUEWS} model and simulate spatial
(and temporal) variations of energy exchanges within a small area on Manhattan
(New York City) with regards to a heat wave event.

Tools such as this, once appropriately assessed for an area, can be used
for a broad range of applications. For example, for climate services
(e.g. \sphinxurl{http://www.wmo.int/gfcs/} , {\color{red}\bfseries{}{}`}Baklanov et al. 2018 \textless{}\sphinxurl{https://doi.org/10.1016/j.uclim.2017.05.004}\textgreater{}’). Running a model can allow analyses,
assessments, and long-term projections and scenarios. Most applications
require not only meteorological data but also information about the
activities that occur in the area of interest (e.g. agriculture,
population, road and infrastructure, and socio-economic variables).

This tutorial make use of local high resolution detailed spatial data. If this kind of data is unavailable, other datasets such as local climate zones (LCZ) from the \sphinxhref{http://www.wudapt.org/}{WUDAPT} database could be used. The tutorial {\hyperref[\detokenize{Tutorials/SuewsWUDAPT:suewswudapt}]{\sphinxcrossref{\DUrole{std,std-ref,std,std-ref}{Urban Energy Balance - SUEWS and WUDAPT}}}} is available if you want to know more about using LCZz in SUEWS. However, it is strongly recommended to go through this tutorial before moving on to the WUDAPT/SUEWS tutorial.

Model output may be needed in many formats depending on a users’ needs.
Thus, the format must be useful, while ensuring the science included
within the model is appropriate. The figure below provides an overview of
{\hyperref[\detokenize{index:index-page}]{\sphinxcrossref{\DUrole{std,std-ref,std,std-ref}{UMEP}}}}, a city based climate service tool (CBCST) used in this tutorial. Within UMEP there are a number
of models which can predict and diagnose a range of meteorological processes.

\begin{figure}[htbp]
\centering
\capstart

\noindent\sphinxincludegraphics[width=0.800\linewidth]{{SUEWSIntro_UMEP_overview}.png}
\caption{Overview of the climate service tool UMEP (from Lindberg et al. 2018)}\label{\detokenize{Tutorials/SuewsSpatial:id5}}\end{figure}

\begin{sphinxadmonition}{note}{Note:}
This tutorial is currently designed to work with QGIS 2.18. It is recommended that you have a look at the tutorials {\hyperref[\detokenize{Tutorials/IntroductionToSuews:introductiontosuews}]{\sphinxcrossref{\DUrole{std,std-ref}{Urban Energy Balance - SUEWS Introduction}}}} and {\hyperref[\detokenize{processor/Urban Energy Balance Urban Energy Balance (SUEWS.BLUEWS, advanced):suewsadvanced}]{\sphinxcrossref{\DUrole{std,std-ref}{Urban Energy Balance: Urban Energy Balance (SUEWS/BLUEWS, advanced)}}}} before you go through this tutorial.
\end{sphinxadmonition}


\subsection{Objectives}
\label{\detokenize{Tutorials/SuewsSpatial:objectives}}
To perform and analyse energy exchanges within a small area on Manhattan, NYC.


\subsubsection{Steps to be preformed}
\label{\detokenize{Tutorials/SuewsSpatial:steps-to-be-preformed}}\begin{enumerate}
\item {} 
Pre-process the data and create input datasets for the SUEWS model

\item {} 
Run the model

\item {} 
Analyse the results

\item {} 
Perform simple mitigation measures to see how it affects the model results (optional)

\end{enumerate}


\subsection{Initial Steps}
\label{\detokenize{Tutorials/SuewsSpatial:initial-steps}}
UMEP is a python plugin used in conjunction with
\sphinxhref{http://www.qgis.org}{QGIS}. To install the software and the UMEP
plugin see the {\hyperref[\detokenize{Getting_Started:getting-started}]{\sphinxcrossref{\DUrole{std,std-ref,std,std-ref}{getting started}}}} section in the UMEP manual.

As UMEP is under development, some documentation may be missing and/or
there may be instability. Please report any issues or suggestions to our
\sphinxhref{https://bitbucket.org/fredrik\_ucg/umep/}{repository}.


\subsubsection{Loading and analyzing the spatial data}
\label{\detokenize{Tutorials/SuewsSpatial:loading-and-analyzing-the-spatial-data}}
All geodata used in this tutorial originates from open datasets available from various sources, foremost from the City of New York. Information about the data is found in the table below.

\begin{sphinxadmonition}{note}{Note:}
You can download the all the data from \sphinxhref{https://github.com/Urban-Meteorology-Reading/Urban-Meteorology-Reading.github.io/blob/master/other\%20files/SUEWSSpatial\_Tutorialdata.zip}{here}. Unzip and place in a folder where you have read and write access to.
\end{sphinxadmonition}


\begin{savenotes}\sphinxattablestart
\centering
\sphinxcapstartof{table}
\sphinxcaption{Spatial data used for this tuorial}\label{\detokenize{Tutorials/SuewsSpatial:id6}}
\sphinxaftercaption
\begin{tabular}[t]{|\X{10}{100}|\X{10}{100}|\X{40}{100}|\X{40}{100}|}
\hline

\sphinxstylestrong{Geodata}
&
\sphinxstylestrong{Year}
&
\sphinxstylestrong{Source}
&
\sphinxstylestrong{Description}
\\
\hline
Digital surface model (DSM)
&
2013 (Lidar), 2016 (building polygons)
&
United States Geological Survey (USGS). New York CMGP Sandy 0.7m NPS Lidar and NYC Open Data Portal. \sphinxhref{https://data.cityofnewyork.us}{link}
&
A raster grid including both buildings and ground given in meter above sea level.
\\
\hline
Digital elevation model (DEM)
&
2013
&
United States Geological Survey (USGS). New York CMGP Sandy 0.7m NPS Lidar. \sphinxhref{https://data.cityofnewyork.us}{link}
&
A raster grid including only ground heights given in meter above sea level.
\\
\hline
Digital canopy model (CDSM)
&
2013 (August)
&
United States Geological Survey (USGS). New York CMGP Sandy 0.7m NPS Lidar. \sphinxhref{https://coast.noaa.gov/htdata/lidar1\_z/geoid12b/data/4920/}{link}
&
A vegetation raster grid where vegetation heights is given in meter above ground level. Vegetation lower than 2.5 meter Pixels with no vegetation should be zero.
\\
\hline
Land cover (UMEP formatted)
&
2010
&
New York City Landcover 2010 (3ft version). University of Vermont Spatial Analysis Laboratory and New York City Urban Field Station. \sphinxhref{https://opendata.cityofnewyork.us/}{link}
&
A raster grid including: 1. Paved surfaces, 2. Building surfaces, 3. Evergreen trees and shrubs, 4. Deciduous trees and shrubs, 5. Grass surfaces, 6. Bare soil, 7. Open water
\\
\hline
Population density (residential)
&
2010
&
2010 NYC Population by Census Tracts, Department of City Planning (DCP). \sphinxhref{https://data.cityofnewyork.us}{link})
&
People per census tract converted to pp/ha. Converted from vector to raster.
\\
\hline
Land use
&
2018
&
NYC Department of City Planning, Technical Review Division. \sphinxhref{https://zola.planning.nyc.gov}{link}
&
Used to redistribute population during daytime (see text). Converted from vector to raster
\\
\hline
\end{tabular}
\par
\sphinxattableend\end{savenotes}
\begin{itemize}
\item {} 
Start by loading all the raster datasets into an empty QGIS project.

\end{itemize}

The order in the \sphinxstyleemphasis{Layers Panel} decides what layer that will be visible. Here you can also choose not to show a layer in the tick box. You can adjust layers according to your likeing by right-click on a layer in the Layers Panel and choose \sphinxstyleemphasis{Properties}. Note for example that that CDSM (vegetation) is given as height above ground (meter) and that all non-vegetated pixels are set to zero. This makes it hard to get an overview between all 3D objects (buildings and trees).
\begin{itemize}
\item {} 
Right-click on your \sphinxstylestrong{CDSM} layer and go to \sphinxstyleemphasis{Properties \textgreater{} Style} and choose \sphinxstylestrong{Singleband pseudocolor} with a min value of 0 and max of 30. Choose also a nice color scheme of your liking.

\item {} 
Go to \sphinxstyleemphasis{Transparency} and  add and additional no data value of 0. Click ok.

\item {} 
Now put your \sphinxstylestrong{CDSM} layer at the top and your \sphinxstylestrong{DSM} layer second in your \sphinxstyleemphasis{Layers Panel}. Now you can see both buislings and vegetation 3D object in your map canvas.

\end{itemize}

\begin{figure}[htbp]
\centering
\capstart

\noindent\sphinxincludegraphics[width=0.800\linewidth]{{SUEWSSpatial_dataview}.png}
\caption{DSM and CDSM visible at the same time (click for larger image)}\label{\detokenize{Tutorials/SuewsSpatial:id7}}\end{figure}

The land cover grid comes with a specific QGIS style file.
\begin{itemize}
\item {} 
Right-click on the land cover layer (\sphinxstylestrong{landcover\_2010\_nyc}) and choose \sphinxstyleemphasis{Properties}. Down to the left you see a \sphinxstyleemphasis{Style}-button. Choose \sphinxstyleemphasis{Load Style} and open \sphinxstylestrong{landcoverstyle.qml} and click OK.

\item {} 
Make only your land cover class layer visible to examine the spatial variability of the different land cover classes.

\end{itemize}

The land cover grid has allready been classified into the seven different classes used in most UMEP applications (see table 1). If you have a land cover dataset that is not UMEP formatted you can make use of the \sphinxstyleemphasis{Land Cover Reclassifier} found at \sphinxstyleemphasis{UMEP \textgreater{} Pre-processor \textgreater{} Urban Land Cover \textgreater{} Land Cover Reclassifier} in the menubar to reclassify your data.

Furthermore, a polygon grid (500 m times 500 m) for defining the study area and individual grids are included (Grid\_500m.shp). Such grid can be produced directly in QGIS (e.g. \sphinxstyleemphasis{Vector \textgreater{} Research Tools \textgreater{} Vector Grid}) or an external grid can also be used.
\begin{itemize}
\item {} 
Load the vector layer \sphinxstylestrong{Grid\_500m.shp} into your QGIS project.

\item {} 
In the \sphinxstyleemphasis{Style} tab in layer \sphinxstyleemphasis{Properties}, choose a \sphinxstyleemphasis{No Brush} fill style to be able to see the spatial data within each grid.

\item {} 
Also, add the label IDs for the grid to the map canvas in \sphinxstyleemphasis{Properties \textgreater{} Labels} to make it easier to identify the different grid squares later on in this tutorial.

\end{itemize}

As you can see the grid does not cover the whole extent of the raster grids. This is to reduce computation time so that this tutorial will not extent for too long. One grid cell will take approximately 20 seconds to model using SUEWS using meteorological forcing data for a full year.


\subsubsection{Meteorlogical forcing data}
\label{\detokenize{Tutorials/SuewsSpatial:meteorlogical-forcing-data}}
Meteorological forcing data is one mandatory input dataset for most of the models included in UMEP. UMEP make use of a specific formatted dataset as decribed in the table below. Some of the variables are optional and if not available shuold be put to -999.


\begin{savenotes}\sphinxattablestart
\centering
\sphinxcapstartof{table}
\sphinxcaption{Variables included in UMEP meteorological input file.}\label{\detokenize{Tutorials/SuewsSpatial:id8}}
\sphinxaftercaption
\begin{tabular}[t]{|\X{3}{100}|\X{6}{100}|\X{25}{100}|\X{18}{100}|\X{48}{100}|}
\hline
\sphinxstyletheadfamily 
No.
&\sphinxstyletheadfamily 
Header
&\sphinxstyletheadfamily 
Description
&\sphinxstyletheadfamily 
Accepted  range
&\sphinxstyletheadfamily 
Comments
\\
\hline
1
&
iy
&
Year {[}YYYY{]}
&
Not applicable
&\\
\hline
2
&
id
&
Day of year {[}DOY{]}
&
1 to 365 (366 if leap year)
&\\
\hline
3
&
it
&
Hour {[}H{]}
&
0 to 23
&\\
\hline
4
&
imin
&
Minute {[}M{]}
&
0 to 59
&\\
\hline
5
&
qn
&
Net all-wave radiation {[}W m$^{\text{-2}}${]}
&
-200 to 800
&\\
\hline
6
&
qh
&
Sensible heat flux {[}W m$^{\text{-2}}${]}
&
-200 to 750
&\\
\hline
7
&
qe
&
Latent heat flux {[}W m$^{\text{-2}}${]}
&
-100 to 650
&\\
\hline
8
&
qs
&
Storage heat flux {[}W m$^{\text{-2}}${]}
&
-200 to 650
&\\
\hline
9
&
qf
&
Anthropogenic heat flux {[}W m$^{\text{-2}}${]}
&
0 to 1500
&\\
\hline
10
&
U
&
Wind speed {[}m s$^{\text{-1}}${]}
&
0.001 to 60
&\\
\hline
11
&
RH
&
Relative Humidity {[}\%{]}
&
5 to 100
&\\
\hline
12
&
Tair
&
Air temperature {[}°C{]}
&
-30 to 55
&\\
\hline
13
&
pres
&
Surface barometric pressure {[}kPa{]}
&
90 to 107
&\\
\hline
14
&
rain
&
Rainfall {[}mm{]}
&
0 to 30
&
(per 5 min) this should be scaled based on time step used
\\
\hline
15
&
kdown
&
Incoming shortwave radiation {[}W m$^{\text{-2}}${]}
&
0 to 1200
&\\
\hline
16
&
snow
&
Snow {[}mm{]}
&
0 to 300
&
(per 5 min) this should be scaled based on time step used
\\
\hline
17
&
ldown
&
Incoming longwave radiation {[}W m$^{\text{-2}}${]}
&
100 to 600
&\\
\hline
18
&
fcld
&
Cloud fraction {[}tenths{]}
&
0 to 1
&\\
\hline
19
&
wuh
&
External water use {[}m$^{\text{3}}${]}
&
0 to 10
&
(per 5 min) scale based on time step being used
\\
\hline
20
&
xsmd
&
(Observed) soil moisture
&
0.01 to 0.5
&
{[}m$^{\text{3}}$ m$^{\text{-3}}$ or kg kg$^{\text{-1}}${]}
\\
\hline
21
&
lai
&
(Observed) leaf area index {[}m$^{\text{2}}$ m$^{\text{-2}}${]}
&
0 to 15
&\\
\hline
22
&
kdiff
&
Diffuse shortwave radiation {[}W m$^{\text{-2}}${]}
&
0 to 600
&\\
\hline
23
&
kdir
&
Direct shortwave radiation {[}W m$^{\text{-2}}${]}
&
0 to 1200
&
Should be perpendicular to the Sun beam. One way to check this is to compare direct and global radiation and see if kdir is higher than global radiation during clear weather. Then kdir is measured perpendicular to the solar beam.
\\
\hline
24
&
wdir
&
Wind direction {[}°{]}
&
0 to 360
&\\
\hline
\end{tabular}
\par
\sphinxattableend\end{savenotes}

The meteorological dataset used in this tutorial (\sphinxstylestrong{MeteorologicalData\_NYC\_2010.txt}) is obtained from NOAA (meteorology) and NREL (radiation) and consist of \sphinxstyleemphasis{tab-separated} hourly data of air temperature, relative humidity, incoming shortwave radiation, pressure, precipitation and wind speed of the full year of 2010. There are also other possibilities within UMEP to acquire meteorological forcing data. The pre-processor plugin {\hyperref[\detokenize{pre-processor/Meteorological Data Download data (WATCH):watch}]{\sphinxcrossref{\DUrole{std,std-ref,std,std-ref}{Meteorological Data: Download data (WATCH)}}}} can be used to download the variables needed from the global \sphinxhref{http://www.eu-watch.org/}{WATCH} forcing datasets (Weedon et al. 2011, 2014).
\begin{itemize}
\item {} 
Open the meterological dataset (\sphinxstylestrong{MeteorologicalData\_NYC\_2010.txt}) in a text editor of your choice. As you can see it does not include all the variables shown in the table above. However, these variables are the mandatory ones that are required to run SUEWS. In order to format (and make a quality check) the data provided into UMEP standard, you will make use of the {\hyperref[\detokenize{pre-processor/Meteorological Data MetPreprocessor:metpreprocessor}]{\sphinxcrossref{\DUrole{std,std-ref,std,std-ref}{Meteorological Data: MetPreprocessor}}}}.

\item {} 
Open MetDataPreprocessor (\sphinxstyleemphasis{UMEP\textgreater{} Pre-Prpcessor -\textgreater{} Meteorological Data \textgreater{} MetPreprocessor}).

\item {} 
Load \sphinxstylestrong{MeteorologicalData\_NYC\_2010.txt} and make make the settings as shown below. Name your new dataset \sphinxstylestrong{NYC\_metdata\_UMEPformatted.txt}.

\end{itemize}

\begin{figure}[htbp]
\centering
\capstart

\noindent\sphinxincludegraphics[width=0.800\linewidth]{{SUEWSSpatial_MetPreprocessor}.png}
\caption{The settings for formatting met data into UMEP format (click for a larger image)}\label{\detokenize{Tutorials/SuewsSpatial:id9}}\end{figure}
\begin{itemize}
\item {} 
Close the Metdata preprocessor and open your newly fomatted datset in a text editor of your choice. Now you see that the forcing data is structured into the UMEP pre-defing format.

\item {} 
Close your text file and move one to the next section of this tutorial.

\end{itemize}


\subsection{Preparing input data for the SUEWS model}
\label{\detokenize{Tutorials/SuewsSpatial:preparing-input-data-for-the-suews-model}}
One key feature of UMEP is to facilitate the preparation of input data for the various models included. SUEWS requires a number of input information to model the urban energy balance. I plugin called \sphinxstyleemphasis{SUEWS Prepare} has been developed for this purpose. This tutorial make use of high resolution data but there are also possibilities to make use of \sphinxhref{http://www.wudapt.org/}{WUDAPT} datasets in-conjuction to the \sphinxstyleemphasis{LCZ Converter} (\sphinxstyleemphasis{UMEP \textgreater{} Pre-Processor \textgreater{} Spatial data \textgreater{} LCZ Converter}).
\begin{itemize}
\item {} 
Open SUEWS Prepare (\sphinxstyleemphasis{UMEP \textgreater{} Pre-Processor \textgreater{} SUEWS prepare}).

\end{itemize}

\begin{figure}[htbp]
\centering
\capstart

\noindent\sphinxincludegraphics[width=0.800\linewidth]{{SUEWSSpatial_Prepare1}.png}
\caption{The dialog for the SUEWS Prepare plugin (click for a larger image).}\label{\detokenize{Tutorials/SuewsSpatial:id10}}\end{figure}

Here you can see all the various settings that can be made. You will focus on the \sphinxstyleemphasis{Main Settings} tab where the mandatory settings are made. The other tabs include the settings for e.g. different land cover classes, human activities etc.

There are 10 frames included in the \sphinxstyleemphasis{Main Settings} tab where 8 need to be filled in for this tutorial:
\begin{enumerate}
\item {} 
\sphinxstylestrong{Polygon grid}

\item {} 
\sphinxstylestrong{Building morphology}

\item {} 
\sphinxstylestrong{Tree morphology}

\item {} 
\sphinxstylestrong{Land cover fractions}

\item {} 
\sphinxstylestrong{Meteorological data}

\item {} 
\sphinxstylestrong{Population density}

\item {} 
\sphinxstylestrong{Daylight savings and UTC}

\item {} 
\sphinxstylestrong{Initial conditions}

\end{enumerate}

The two optional frames (\sphinxstyleemphasis{Land use fractions} and \sphinxstyleemphasis{Wall area}) should be used if the ESTM model should be used which is a model scheme used to estimate the storage energy term (Q$_{\text{S}}$). You will use another modelling scheme (\sphinxstyleemphasis{OHM}) and therefore, these two tabs could be ignored for now.
\begin{itemize}
\item {} 
Close \sphinxstyleemphasis{SUEWS Prepare}

\end{itemize}


\subsubsection{Building morphology}
\label{\detokenize{Tutorials/SuewsSpatial:building-morphology}}
First you will calculate roughness paprmeters based on the building geometry within your grids.
\begin{itemize}
\item {} 
Open \sphinxstyleemphasis{UMEP \textgreater{} Pre-Processor \textgreater{} Urban Morphology \textgreater{} Morphometric Calculator (Grid)}.

\item {} 
Use the settings as in the figure below and press \sphinxstyleemphasis{Run}.

\item {} 
When calculation ids done, close the plugin.

\end{itemize}

\begin{figure}[htbp]
\centering
\capstart

\noindent\sphinxincludegraphics[width=0.800\linewidth]{{SUEWSSpatial_IMCGBuilding}.png}
\caption{The settings for calculating building morphology.}\label{\detokenize{Tutorials/SuewsSpatial:id11}}\end{figure}

This operation should have produced 17 different text files; 16 (\sphinxstyleemphasis{anisotrophic}) that include morphometric parameters from each 5 degree section for each grid and one file (\sphinxstyleemphasis{isotropic}) that includes averaged values for each of the 16 grids. You can open \sphinxstylestrong{build\_IMPGrid\_isotropic.txt} and compare the different values for a park grid (3054) and an urban grid (3242). Header abbreviations is explained {\hyperref[\detokenize{pre-processor/Urban Morphology Morphometric Calculator (Grid):morphometriccalculator-grid}]{\sphinxcrossref{\DUrole{std,std-ref,std,std-ref}{here}}}}.


\subsubsection{Tree morphology}
\label{\detokenize{Tutorials/SuewsSpatial:tree-morphology}}
Now you will calculate roughness paprmeters based on the vegetation (trees and bushes) within your grids. As you noticed there is only one surface data for vegetation present (\sphinxstylestrong{CDSM\_nyc}) and if you examine your land cover grid (\sphinxstylestrong{landcover\_2010\_nyc}) you can see that there is only one class of high vegetation (\sphinxstyleemphasis{Deciduous trees}) present with our model domain. Therefore, you will not separate between evergreen and deciduous vegetation in this tutorial. As shown in table 1, the tree surface model represents height above ground.
\begin{itemize}
\item {} 
Again, Open \sphinxstyleemphasis{UMEP \textgreater{} Pre-Processor \textgreater{} Urban Morphology \textgreater{} Morphometric Calculator (Grid)}.

\item {} 
Use the settings as in the figure below and press \sphinxstyleemphasis{Run}.

\item {} 
When calculation ids done, close the plugin.

\end{itemize}

\begin{figure}[htbp]
\centering
\capstart

\noindent\sphinxincludegraphics[width=0.800\linewidth]{{SUEWSSpatial_IMCGVeg}.png}
\caption{The settings for calculating vegetation morphology.}\label{\detokenize{Tutorials/SuewsSpatial:id12}}\end{figure}


\subsubsection{Land cover fractions}
\label{\detokenize{Tutorials/SuewsSpatial:land-cover-fractions}}
Moving on to land cover fraction calculations for each grid.
\begin{itemize}
\item {} 
Open \sphinxstyleemphasis{UMEP \textgreater{} Pre-Processor \textgreater{} Urban Land Cover \textgreater{} Land Cover Fraction (Grid)}.

\item {} 
Use the settings as in the figure below and press \sphinxstyleemphasis{Run}.

\item {} 
When calculation ids done, close the plugin.

\end{itemize}

\begin{figure}[htbp]
\centering
\capstart

\noindent\sphinxincludegraphics[width=0.800\linewidth]{{SUEWSSpatial_LCF}.png}
\caption{The settings for calculating land cover fractions}\label{\detokenize{Tutorials/SuewsSpatial:id13}}\end{figure}


\subsubsection{Population density}
\label{\detokenize{Tutorials/SuewsSpatial:population-density}}
Population density will be used to estimate the anthropogenic heat release (Q$_{\text{F}}$) in SUEWS. There is a possibility to make use of both night-time and daytime population densities to make the model more dynamic. You have two different raster grids for night-time (\sphinxstylestrong{pop\_nighttime\_perha}) and daytime (\sphinxstylestrong{pop\_daytime\_perha}), respectively. This time you will make use of a built-in function to QGIS to accuire the population density for each grid.
\begin{itemize}
\item {} 
Go to \sphinxstyleemphasis{Plugins \textgreater{} Manage and Install Plugins} and make sure that the \sphinxstyleemphasis{Zonal statistics plugin} is ticked in. This is a build-in plugin which comes with the QGIS installation.

\item {} 
Close the \sphinxstyleemphasis{Plugin maanager} and open \sphinxstyleemphasis{Raster \textgreater{} Zonal Statistics \textgreater{} Zonal Statistics}.

\item {} 
Choose your \sphinxstylestrong{pop\_daytime\_perha} layer as \sphinxstyleemphasis{Raster layer*} and your \sphinxstylestrong{Grid\_500m} and polygon layer. Use a \sphinxstyleemphasis{Output column prefix} of \sphinxstylestrong{PPday} and chose only to calculate \sphinxstyleemphasis{Mean}. Click OK.

\item {} 
Run the tool again but this time use the night-time dataset.

\end{itemize}


\subsubsection{SUEWS Prepare}
\label{\detokenize{Tutorials/SuewsSpatial:suews-prepare}}
Now you are ready to organise all input data into the SUEWS input format.
\begin{itemize}
\item {} 
Open \sphinxstyleemphasis{SUEWS Prepare}

\item {} 
In the \sphinxstyleemphasis{Polygon grid} frams, choose your polygon grid (\sphinxstylestrong{Grid\_500m}) and choose \sphinxstylestrong{id} as your \sphinxstyleemphasis{ID field}

\item {} 
In the \sphinxstyleemphasis{Building morphology} frame, fetch the file called \sphinxstylestrong{build\_IMPGrid\_isotropic.txt}.

\item {} 
In the \sphinxstyleemphasis{Land cover fractions} frame, fetch the file called \sphinxstylestrong{lc\_LCFG\_isotropic.txt}.

\item {} 
In the \sphinxstyleemphasis{Tree morphology} frame, fetch the file called \sphinxstylestrong{veg\_IMPGrid\_isotropic.txt}.

\item {} 
In the \sphinxstyleemphasis{Meteorological data} frame, fetch your UMEP formatted met forcing data text file.

\item {} 
In the \sphinxstyleemphasis{Population density} frame, choose the appropriate attributes created in the previous section for daytime and night-time population density.

\item {} 
In the \sphinxstyleemphasis{Daylight savings and UTC} frame, leave start and end of the daylight saving as they are and choose \sphinxstyleemphasis{-5}.

\item {} 
In the \sphinxstyleemphasis{Initial conditions} frame, choose \sphinxstylestrong{Winter (0\%)} in the \sphinxstyleemphasis{Leaf Cycle}, 100\% \sphinxstyleemphasis{Soil moisture state} and \sphinxstylestrong{nyc} as a \sphinxstyleemphasis{File code*}.

\item {} 
In the \sphinxstyleemphasis{Anthropogenic} tab, change the code to 771.

\item {} 
Choose an empty directory as your \sphinxstyleemphasis{Output folder} in the main tab.

\item {} 
Press \sphinxstyleemphasis{Generate}

\item {} 
When processing is finished, close \sphinxstyleemphasis{SUEWS Prepare}.

\end{itemize}


\subsection{Running the SUEWS model in UMEP}
\label{\detokenize{Tutorials/SuewsSpatial:running-the-suews-model-in-umep}}
To perform modelling energy fluxes for multiple grids, {\hyperref[\detokenize{processor/Urban Energy Balance Urban Energy Balance (SUEWS.BLUEWS, advanced):suewsadvanced}]{\sphinxcrossref{\DUrole{std,std-ref,std,std-ref}{Urban Energy Balance: Urban Energy Balance (SUEWS/BLUEWS, advanced)}}}} can be used.
\begin{itemize}
\item {} 
Open \sphinxstyleemphasis{UMEP \textgreater{} Processor \textgreater{} Urban Energy Balance \textgreater{} SUEWS/BLUEWS, Advanced}. Here you can change some of the run control settings in SUEWS. SUEWS can also be executed outside of UMEP and QGIS (see \sphinxhref{http://suews-docs.readthedocs.io}{SUEWS Manual}. This is recommended when modelling long time series (multiple years) of large model domains (many grid points).

\item {} 
Change the OHM option to {[}1{]}. This allows the anthropogenic energy to be partitioned also into the storage energy term.

\item {} 
Leave the rest of the combobox settings at the top as default and tick in both the \sphinxstyleemphasis{Use snow module} and the \sphinxstyleemphasis{Obtain temporal resolution…} box.

\item {} 
Set the {\color{red}\bfseries{}*}Temporal resolution of output (minutes) to 60.

\item {} 
Locate the directory where you save your output from \sphinxstyleemphasis{SUEWSPrepare} earlier and choose an output folder of your choice.

\item {} 
Also, Tick in the box \sphinxstyleemphasis{Apply spin-up using…}. This will force the model to run twice using the conditions from the first run as initial conditions for the second run.

\item {} 
Click \sphinxstyleemphasis{Run}. This computation will take a while so just have patience.

\end{itemize}


\subsection{Analysing model reults}
\label{\detokenize{Tutorials/SuewsSpatial:analysing-model-reults}}
UMEP also comes with a tool to make basic analysis of any modelling performed with the SUEWS model. The {\hyperref[\detokenize{post_processor/Urban Energy Balance SUEWS Analyser:suewsanalyser}]{\sphinxcrossref{\DUrole{std,std-ref,std,std-ref}{Urban Energy Balance: SUEWS Analyser}}}} tool is availble from the post-processing section in UMEP.
\begin{itemize}
\item {} 
Open \sphinxstyleemphasis{UMEP \textgreater{} Post-Processor \textgreater{} Urban Energy Balance \textgreater{} SUEWS Analyzer}. There are two main sections in this tool. The \sphinxstyleemphasis{Plot data}-section can be used to make temporal analysis as well as making simple comparisins between two grids or variables. This \sphinxstyleemphasis{Spatial data}-section can be used to make aggregated maps of the output variables from the SUEWS model. This requires that you have loaded the same polygon grid into your QGIS project that was used when you prepared the input data for SUEWS using \sphinxstyleemphasis{SUEWS Prepare} earlier in this tutorial.

\end{itemize}

\begin{figure}[htbp]
\centering
\capstart

\noindent\sphinxincludegraphics[width=0.800\linewidth]{{SUEWSAnalyzer}.png}
\caption{The dialog for the SUEWS Analyzer tool.}\label{\detokenize{Tutorials/SuewsSpatial:id14}}\end{figure}

To access the output data from the a model run, the \sphinxstylestrong{RunControl.nml} file for that particular run must be located. If your run has been made through UMEP, this file can be found in your output folder. Otherwise, this file can be located in the same folder from where the model was executed.
\begin{itemize}
\item {} 
In the top panel of \sphinxstyleemphasis{SUEWS Analyzer}, load the \sphinxstylestrong{RunControl.nml} located in the output folder.

\end{itemize}

You will start by plotting basic data for grid 3242 which is one of the most dense urban area in the World.
\begin{itemize}
\item {} 
In the left panel, choose grid \sphinxstyleemphasis{3242} and year \sphinxstyleemphasis{2010}. Tick in \sphinxstyleemphasis{plot basic data} and click \sphinxstyleemphasis{Plot}. This will display some of the most essential variables such as radiation balance and budget etc. You can use the tools such as the zoom to examine a shorter time period more in detail.

\end{itemize}

\begin{figure}[htbp]
\centering
\capstart

\noindent\sphinxincludegraphics[width=0.800\linewidth]{{SUEWSSpatial_basicplot_grid3242}.png}
\caption{Basic plot for grid 3242. Click on image for enlargement.}\label{\detokenize{Tutorials/SuewsSpatial:id15}}\end{figure}

Notice e.g. the high Q$_{\text{F}}$ values during winter as well as the low Q$_{\text{E}}$ values throughout the year.
\begin{itemize}
\item {} 
Close the plot and make the same kind of plot for grid 3054 which is a grid mainly within Central Park. Consider the differences between the plot generated for grid 3242. Close the plot when you are done.

\end{itemize}

In the left panel, there is also possibilities to examine two different variables in time, either from the same grid or between two different grid points. There is also possible to examine different parameters through scatterplots.

The right panel in SUEWS Analyzer can be used to perform basic spatial analysis on your model results by producing aggragated maps etc. using different variables and time spans. Sensible heat (Q$_{\text{H}}$) is a suitable variable to visualise warm areas as it is a variable the show the amount of the available energy that will be partitioned into heat.
\begin{itemize}
\item {} 
Make the settings as shown in the figure below but change the location where you will save your data on your own system.

\end{itemize}

\begin{figure}[htbp]
\centering
\capstart

\noindent\sphinxincludegraphics[width=0.800\linewidth]{{SUEWSSpatial_Analyzer}.png}
\caption{The dialog for the SUEWS Analyzer tool to produce a mean Q$_{\text{H}}$ for each grid. Click on image for enlargement.}\label{\detokenize{Tutorials/SuewsSpatial:id16}}\end{figure}

Note that the warmest areas are located in the most dense urban environments and the coolest are found where either vegetation and/or water bodies are present. During 2010 there was a 3-day heat-wave event in the region around NYC that lasted from 5 to 8 July 2010 (Day of Year: 186-189).
\begin{itemize}
\item {} 
Make a similar average map of Q$_{\text{H}}$ as above but choose only the heat wave period. Save it as a separate geoTiff.

\end{itemize}


\subsection{The influence of mitigation measures on the urban energy balance (optional)}
\label{\detokenize{Tutorials/SuewsSpatial:the-influence-of-mitigation-measures-on-the-urban-energy-balance-optional}}
There different ways of manipulating the data using UMEP as well directly changing the input data in SUEWS to examine the influence of migitagion measrues on the UEB. The most detailed way would be to directly changing the surface data by e.g. increasing the number of street trees. This can be done by e.g. using the {\hyperref[\detokenize{pre-processor/Spatial Data Tree Generator:treegenerator}]{\sphinxcrossref{\DUrole{std,std-ref,std,std-ref}{Spatial Data: Tree Generator}}}}-plugin in UMEP. This method would require that you go through the workflow of this tutorial again before you do your new model run. Another way is to directly manipulate input data to SUEWS at grid point level. This can done by e.g. changing the land cover fractions in \sphinxstylestrong{SUEWS\_SiteSelect.txt}, the file that includes all grid-specific information used in SUEWS.
\begin{itemize}
\item {} 
Make a copy of your whole input folder created from SUEWSPRepare earlier and rename it to e.g. \sphinxstyleemphasis{Input\_mitigation}.

\item {} 
In that folder remove all the files beginning with \sphinxstyleemphasis{InitialConditions} \sphinxstylestrong{except} the one called \sphinxstylestrong{InitialConditionsnyc\_2010.nml}.

\item {} 
Open \sphinxstylestrong{SUEWS\_SiteSelect.txt} in Excel (or similar software).

\item {} 
Now increace the fraction of decidious trees (\sphinxstyleemphasis{Fr\_DecTr}) for grid 3242 and 3243 by 0.2. As the total land cover fraction has to be 1 you also need to reduce the paved fraction (\sphinxstyleemphasis{Fr\_Paved}) by the same amount.

\item {} 
Save and close. Remember to keep the format (tab-separated text).

\item {} 
Create an empty folder called \sphinxstyleemphasis{Output\_mitigation}

\item {} 
Open {\hyperref[\detokenize{Tutorials/SuewsAdvanced::doc}]{\sphinxcrossref{\DUrole{doc,doc,doc}{Urban Energy Balance - SUEWS Advanved}}}} and make the same settings as before but change to inout and output folders.

\item {} 
Run the model.

\item {} 
When finished, create a similar average Q$_{\text{H}}$ map for the heat event and compare the two maps. You can do a difference map by using the Raster Calculator in QGIS (\sphinxstyleemphasis{Raster\textgreater{}Raster Calculator…}).

\end{itemize}

Tutorial finished.


\section{Urban Energy Balance - SUEWS and WUDAPT}
\label{\detokenize{Tutorials/SuewsWUDAPT:urban-energy-balance-suews-and-wudapt}}\label{\detokenize{Tutorials/SuewsWUDAPT:suewswudapt}}\label{\detokenize{Tutorials/SuewsWUDAPT::doc}}

\subsection{Introduction}
\label{\detokenize{Tutorials/SuewsWUDAPT:introduction}}
\begin{sphinxadmonition}{note}{Note:}
This tutorial is not ready for use. Work in progress.
\end{sphinxadmonition}

In this tutorial you will generate input data for the
\sphinxhref{http://suews-docs.readthedocs.io}{SUEWS} model and simulate spatial
(and temporal) variations of energy exchanges within an area in New York City using local climate zones derived within the \sphinxhref{http://www.wudapt.org/}{WUDAPT} project. The World Urban Database and Access Portal Tools project is a community-based project to gather a census of cities around the world.

Model output may be needed in many formats depending on a users’ needs.
Thus, the format must be useful, while ensuring the science included
within the model is appropriate. The figure below provides an overview of
{\hyperref[\detokenize{index:index-page}]{\sphinxcrossref{\DUrole{std,std-ref,std,std-ref}{UMEP}}}}, a city based climate
service tool (CBCST) used in this tutorial. Within UMEP there are a number
of models which can predict and diagnose a range of meteorological processes.

\begin{sphinxadmonition}{note}{Note:}
This tutorial is currently designed to work with QGIS 2.18. It is strongly recommended that you goo through the {\hyperref[\detokenize{Tutorials/SuewsSpatial:suewsspatial}]{\sphinxcrossref{\DUrole{std,std-ref}{Urban Energy Balance - SUEWS Spatial}}}} tutorial before you go through this tutrial.
\end{sphinxadmonition}


\subsection{Objectives}
\label{\detokenize{Tutorials/SuewsWUDAPT:objectives}}
To prepare input data for the SUEWS model using a WUDAPT dataset and analyse energy exchanges within an area in New York City.


\subsubsection{Steps to be preformed}
\label{\detokenize{Tutorials/SuewsWUDAPT:steps-to-be-preformed}}\begin{enumerate}
\item {} 
Download, pre-process the WUDAPT data to create input datasets for the \sphinxhref{http://suews-docs.readthedocs.io}{SUEWS} model

\item {} 
Run the model

\item {} 
Analyse the results

\end{enumerate}


\subsection{Initial Steps}
\label{\detokenize{Tutorials/SuewsWUDAPT:initial-steps}}
UMEP is a python plugin used in conjunction with
\sphinxhref{http://www.qgis.org}{QGIS}. To install the software and the UMEP
plugin see the {\hyperref[\detokenize{Getting_Started:getting-started}]{\sphinxcrossref{\DUrole{std,std-ref,std,std-ref}{getting started}}}} section in the UMEP manual.

As UMEP is under development, some documentation may be missing and/or
there may be instability. Please report any issues or suggestions to our
\sphinxhref{https://bitbucket.org/fredrik\_ucg/umep/}{repository}.


\subsubsection{Loading and analyzing the spatial data}
\label{\detokenize{Tutorials/SuewsWUDAPT:loading-and-analyzing-the-spatial-data}}
All geodata used in this tutorial originates from open datasets available from various sources, foremost from the City of New York. Information about the data is found in the table below.

\begin{sphinxadmonition}{note}{Note:}
You can download the all the data from \sphinxhref{https://github.com/Urban-Meteorology-Reading/Urban-Meteorology-Reading.github.io/blob/master/other\%20files/SUEWSSpatial\_Tutorialdata.zip}{here}. Unzip and place in a folder where you have read and write access to.
\end{sphinxadmonition}
\begin{itemize}
\item {} 
Start by loading all the raster datasets into an empty QGIS project.

\end{itemize}

The order in the \sphinxstyleemphasis{Layers Panel} decides what layer that will be visible. Here you can also choose not to show a layer in the tick box. You can adjust layers according to your likeing by right-click on a layer in the Layers Panel and choose \sphinxstyleemphasis{Properties}. Note for example that that CDSM (vegetation) is given as height above ground (meter) and that all non-vegetated pixels are set to zero. This makes it hard to get an overview between all 3D objects (buildings and trees).
\begin{itemize}
\item {} 
Right-click on your \sphinxstylestrong{CDSM} layer and go to \sphinxstyleemphasis{Properties \textgreater{} Style} and choose \sphinxstylestrong{Singleband pseudocolor} with a min value of 0 and max of 30. Choose also a nice color scheme of your liking.

\item {} 
Go to \sphinxstyleemphasis{Transparency} and  add and additional no data value of 0. Click ok.

\item {} 
Now put your \sphinxstylestrong{CDSM} layer at the top and your \sphinxstylestrong{DSM} layer second in your \sphinxstyleemphasis{Layers Panel}. Now you can see both buislings and vegetation 3D object in your map canvas.

\end{itemize}

\begin{figure}[htbp]
\centering
\capstart

\noindent\sphinxincludegraphics[width=1073\sphinxpxdimen]{{SUEWSSpatial_dataview}.png}
\caption{DSM and CDSM visible at the same time (click for larger image)}\label{\detokenize{Tutorials/SuewsWUDAPT:id3}}\end{figure}

The land cover grid comes with a specific QGIS style file.
\begin{itemize}
\item {} 
Right-click on the land cover layer (\sphinxstylestrong{landcover\_2010\_nyc}) and choose \sphinxstyleemphasis{Properties}. Down to the left you see a \sphinxstyleemphasis{Style}-button. Choose \sphinxstyleemphasis{Load Style} and open \sphinxstylestrong{landcoverstyle.qml} and click OK.

\item {} 
Make only your land cover class layer visible to examine the spatial variability of the different land cover classes.

\end{itemize}

The land cover grid has allready been classified into the seven different classes used in most UMEP applications (see table 1). If you have a land cover dataset that is not UMEP formatted you can make use of the \sphinxstyleemphasis{Land Cover Reclassifier} found at \sphinxstyleemphasis{UMEP \textgreater{} Pre-processor \textgreater{} Urban Land Cover \textgreater{} Land Cover Reclassifier} in the menubar to reclassify your data.

Furthermore, a polygon grid (500 m times 500 m) for defining the study area and individual grids are included (Grid\_500m.shp). Such grid can be produced directly in QGIS (e.g. \sphinxstyleemphasis{Vector \textgreater{} Research Tools \textgreater{} Vector Grid}) or an external grid can also be used.
\begin{itemize}
\item {} 
Load the vector layer \sphinxstylestrong{Grid\_500m.shp} into your QGIS project.

\item {} 
In the \sphinxstyleemphasis{Style} tab in layer \sphinxstyleemphasis{Properties}, choose a \sphinxstyleemphasis{No Brush} fill style to be able to see the spatial data within each grid.

\item {} 
Also, add the label IDs for the grid to the map canvas in \sphinxstyleemphasis{Properties \textgreater{} Labels} to make it easier to identify the different grid squares later on in this tutorial.

\end{itemize}

As you can see the grid does not cover the whole extent of the raster grids. This is to reduce computation time so that this tutorial will not extent for too long. One grid cell will take approximately 20 seconds to model using SUEWS using meteorological forcing data for a full year.


\subsection{Preparing input data for the SUEWS model}
\label{\detokenize{Tutorials/SuewsWUDAPT:preparing-input-data-for-the-suews-model}}
\sphinxstylestrong{NOT READY}

One key feature of UMEP is to facilitate the preparation of input data for the various models included. SUEWS requires a number of input information to model the urban energy balance. I plugin called \sphinxstyleemphasis{SUEWS Prepare} has been developed for this purpose. This tutorial make use of high resolution data but there are also possibilities to make use of \sphinxhref{http://www.wudapt.org/}{WUDAPT} datasets in-conjuction to the \sphinxstyleemphasis{LCZ Converter} (\sphinxstyleemphasis{UMEP \textgreater{} Pre-Processor \textgreater{} Spatial data \textgreater{} LCZ Converter}).


\subsubsection{Population density}
\label{\detokenize{Tutorials/SuewsWUDAPT:population-density}}
Population density will be used to estimate the anthropogenic heat release (Q$_{\text{F}}$) in SUEWS. There is a possibility to make use of both night-time and daytime population densities to make the model more dynamic. You have two different raster grids for night-time (\sphinxstylestrong{pop\_nighttime\_perha}) and daytime (\sphinxstylestrong{pop\_daytime\_perha}), respectively. This time you will make use of a built-in function to QGIS to accuire the population density for each grid.
\begin{itemize}
\item {} 
Go to \sphinxstyleemphasis{Plugins \textgreater{} Manage and Install Plugins} and make sure that the \sphinxstyleemphasis{Zonal statistics plugin} is ticked in. This is a build-in plugin which comes with the QGIS installation.

\item {} 
Close the \sphinxstyleemphasis{Plugin maanager} and open \sphinxstyleemphasis{Raster \textgreater{} Zonal Statistics \textgreater{} Zonal Statistics}.

\item {} 
Choose your \sphinxstylestrong{pop\_daytime\_perha} layer as \sphinxstyleemphasis{Raster layer*} and your \sphinxstylestrong{Grid\_500m} and polygon layer. Use a \sphinxstyleemphasis{Output column prefix} of \sphinxstylestrong{PPday} and chose only to calculate \sphinxstyleemphasis{Mean}. Click OK.

\item {} 
Run the tool again but this time use the night-time dataset.

\end{itemize}


\subsection{Running the SUEWS model in UMEP}
\label{\detokenize{Tutorials/SuewsWUDAPT:running-the-suews-model-in-umep}}
To examine energy fluxes for multiple grids, {\hyperref[\detokenize{processor/Urban Energy Balance Urban Energy Balance (SUEWS.BLUEWS, advanced):suewsadvanced}]{\sphinxcrossref{\DUrole{std,std-ref,std,std-ref}{Urban Energy Balance: Urban Energy Balance (SUEWS/BLUEWS, advanced)}}}} will be used.
\begin{itemize}
\item {} 
Open \sphinxstyleemphasis{UMEP \textgreater{} Processor \textgreater{} Urban Energy Balance \textgreater{} SUEWS/BLUEWS, Advanced}. Here you can change some of the run control settings in SUEWS. SUEWS can also be executed outside of UMEP and QGIS (see \sphinxhref{http://suews-docs.readthedocs.io}{SUEWS Manual}. This is recommended when modelling long time series (multiple years) of large model domains (many grid points).

\item {} 
Leave all the combobox settings at the top as default and tick in both the \sphinxstyleemphasis{Use snow module} and the \sphinxstyleemphasis{Obtain temporal resolution…} box.

\item {} 
Set the {\color{red}\bfseries{}*}Temporal resolution of output (minutes) to 60.

\item {} 
Locate the directory where you save your output from \sphinxstyleemphasis{SUEWSPrepare} earlier and choose an output folder of your choice.

\item {} 
Click \sphinxstyleemphasis{Run}. This computation will take a while so just have patience.

\end{itemize}


\subsection{Analysing model reults}
\label{\detokenize{Tutorials/SuewsWUDAPT:analysing-model-reults}}
Here I am.


\section{Thermal Comfort - Introduction to SOLWEIG}
\label{\detokenize{Tutorials/IntroductionToSolweig:thermal-comfort-introduction-to-solweig}}\label{\detokenize{Tutorials/IntroductionToSolweig:introductiontosolweig}}\label{\detokenize{Tutorials/IntroductionToSolweig::doc}}
\begin{sphinxadmonition}{note}{Note:}
Click \sphinxcode{\sphinxupquote{here}} to find a tutorial designed for ICUC10 workshop participants.
\end{sphinxadmonition}


\subsection{Introduction}
\label{\detokenize{Tutorials/IntroductionToSolweig:introduction}}
In this tutorial you will use a model \sphinxstylestrong{SOlar and LongWave Environmental
Irradiance Geometry model (SOLWEIG)} to estimate the mean radiant
temperature (T$_{\text{mrt}}$).

SOLWEIG is a model that simulates spatial variations of 3D radiation
fluxes and the T$_{\text{mrt}}$ in complex urban settings. It is also able
to model spatial variations of shadow patterns. T$_{\text{mrt}}$ is one of
the key meteorological variables governing human energy balance and the
thermal comfort of people. It is derived from summing all the radiative
(shortwave and longwave) fluxes (both direct and reflected) to which the
human body is exposed. In SOLWEIG, T$_{\text{mrt}}$ is derived by modelling
shortwave and longwave radiation fluxes in six directions (upward,
downward and from the four cardinal points) and angular factors.

The model requires \sphinxstylestrong{meteorological} forcing data (global shortwave
radiation (K$_{\text{down}}$), air temperature (T$_{\text{a}}$), relative humidity (RH)),
urban geometry (DSMs), and geographic information (latitude, longitude
and elevation). To determine T$_{\text{mrt}}$, continuous maps of sky view
factors are required. Both vegetation and ground cover information can
be added to increase the accuracy of the model output. Below
a schematic flowchart of SOLWEIG in shown. The {\hyperref[\detokenize{OtherManuals/SOLWEIG:solweigmanual}]{\sphinxcrossref{\DUrole{std,std-ref,std,std-ref}{full
manual}}}} provides more
detail.

\begin{figure}[htbp]
\centering
\capstart

\noindent\sphinxincludegraphics{{SOLWEIG_flowchart}.png}
\caption{Overview of SOLWEIG}\label{\detokenize{Tutorials/IntroductionToSolweig:id1}}\end{figure}


\subsection{Objectives}
\label{\detokenize{Tutorials/IntroductionToSolweig:objectives}}
To introduce SOLWEIG and how to run the model within {\hyperref[\detokenize{index:index-page}]{\sphinxcrossref{\DUrole{std,std-ref,std,std-ref}{UMEP (Urban
Multi-scale Environmental Predictor)}}}}.

Help with Abbreviations can be found {\hyperref[\detokenize{Abbreviations:abbreviations}]{\sphinxcrossref{\DUrole{std,std-ref,std,std-ref}{here}}}}


\subsubsection{Steps}
\label{\detokenize{Tutorials/IntroductionToSolweig:steps}}\begin{enumerate}
\item {} 
Generation of the different kinds of input data that are needed to
run the model

\item {} 
How to run the model

\item {} 
How to examine the model output

\item {} 
Add additional information (vegetation and ground cover) to improve
the model outcome and to examine the effect of climate sensitive
design

\end{enumerate}


\subsection{Initial Practical steps}
\label{\detokenize{Tutorials/IntroductionToSolweig:initial-practical-steps}}
UMEP is a python plugin used in conjunction with
\sphinxhref{http://www.qgis.org}{QGIS}. To install the software and the UMEP
plugin see the {\hyperref[\detokenize{Getting_Started:getting-started}]{\sphinxcrossref{\DUrole{std,std-ref,std,std-ref}{getting
started}}}}
section in the UMEP manual.

As UMEP is under constant development, some documentation may be missing
and/or there may be instability. Please report any issues or suggestions
to our \sphinxhref{https://bitbucket.org/fredrik\_ucg/umep/}{repository}.


\subsubsection{Data for this exercise}
\label{\detokenize{Tutorials/IntroductionToSolweig:data-for-this-exercise}}
The UMEP tutorial datasets can be downloaded from our here repository
\sphinxhref{https://github.com/Urban-Meteorology-Reading/Urban-Meteorology-Reading.github.io/tree/master/other\%20files/Goteborg\_SWEREF99\_1200.zip}{here}
\begin{itemize}
\item {} 
Download, extract and add the raster layers (DSM, CDSM, DEM and land
cover) from the \sphinxstylestrong{Goteborg folder} into a new QGIS session (see
below).
\begin{itemize}
\item {} 
Create a new project

\item {} 
Examine the geodata by adding the layers (\sphinxstyleemphasis{DSM\_KRbig},
\sphinxstyleemphasis{CDSM\_KRbig}, \sphinxstyleemphasis{DEM\_KRbig} and \sphinxstyleemphasis{landcover}) to your project (\sphinxstylestrong{*Layer
\textgreater{} Add Layer \textgreater{} Add Raster Layer}).

\end{itemize}

\item {} 
Coordinate system of the grids is Sweref99 1200 (EPSG:3007). If you
look at the lower right hand side you can see the CRS used in the
current QGIS project

\item {} 
Examine the different datasets before you move on.

\item {} 
To add a legend to the \sphinxstylestrong{land cover} raster you can load
\sphinxstylestrong{landcoverstyle.qml} found in the test dataset. Right click on the
land cover (\sphinxstyleemphasis{Properties -\textgreater{} Style (lower left) -\textgreater{} Load Style}).

\end{itemize}


\subsection{SOLWEIG Model Inputs}
\label{\detokenize{Tutorials/IntroductionToSolweig:solweig-model-inputs}}
Details of the model inputs and outputs are provided in the {\hyperref[\detokenize{OtherManuals/SOLWEIG:solweigmanual}]{\sphinxcrossref{\DUrole{std,std-ref,std,std-ref}{SOLWEIG
manual}}}}. As this tutorial is
concerned with a \sphinxstylestrong{simple application} only the most critical
parameters are used. Many other parameters can be modified to more
appropriate values if applicable. The table below provides an overview
of the parameters that can be modified in the Simple application of
SOLWEIG.

Data requreiments:
R: required, O: Optional, N : not needed,
S: Spatial, M: Meteorological,


\begin{savenotes}\sphinxattablestart
\centering
\sphinxcapstartof{table}
\sphinxcaption{Input data and parameters}\label{\detokenize{Tutorials/IntroductionToSolweig:id2}}
\sphinxaftercaption
\begin{tabular}[t]{|\X{30}{100}|\X{30}{100}|\X{5}{100}|\X{5}{100}|\X{30}{100}|}
\hline

\sphinxstylestrong{Data}
&
\sphinxstylestrong{Definition}
&
\sphinxstylestrong{Use}
&
\sphinxstylestrong{Type}
&
\sphinxstylestrong{Description}
\\
\hline
Ground and building digital surface model (DSM)
&
High resolution surface model of ground and building heights
&
R
&
S
&
Given in metres above sea level (m asl)
\\
\hline
Digital elevation model (DEM)
&
High resolution surface model of the ground
&
R*
&
S
&
R* if land cover is absent to identify buildings. Given in m asl. Must be same resolution as the DSM.
\\
\hline
Digital canopy surface model (CDSM)
&
High resolution surface model of 3D vegetation
&
O
&
S
&
Given in metres above ground level (m agl). Must be same resolution as the DSM.
\\
\hline
Digital trunk zone surface model (TDSM)
&
High resolution surface model of trunk zone heights (underneath tree canopy)
&
O
&
S
&
Given in m agl. Must be same resolution as the DSM.
\\
\hline
Land (ground) cover information (LC)
&
High resolution surface model of ground cover
&
O
&
S
&
Must be same resolution as the DSM. Five different ground covers are currently available (building, paved, grass, bare soil and water)
\\
\hline
UMEP formatted meteorological data
&
Meteorological data from one nearby observation station, preferably at 1-2 m above ground.
&
R
&
M
&
Any time resolution can be given.
\\
\hline
Latitude (°)
&
Solar related calculations
&
R
&
O
&
Obtained from the ground and building DSM coordinate system
\\
\hline
Longitude (°)
&
Solar related calculations
&
R
&
O
&
Obtained from the ground and building DSM coordinate system
\\
\hline
\sphinxhref{https://en.wikipedia.org/wiki/Coordinated\_Universal\_Time}{UTC (h)}
&
Time zone
&
R
&
O
&
Influences solar related calculations. Set in the interface of the model.
\\
\hline
Human exposure parameters
&
Absorption of radiation and posture
&
R
&
O
&
Set in the interface of the model.
\\
\hline
Environmental parameters
&
e.g. albedos and emissivites of surrounding urban fabrics
&
R
&
O
&
Set in the interface of the model.
\\
\hline
\end{tabular}
\par
\sphinxattableend\end{savenotes}

Meterological input data should be in UMEP format. You can use the
{\hyperref[\detokenize{pre-processor/Meteorological Data MetPreprocessor:metpreprocessor}]{\sphinxcrossref{\DUrole{std,std-ref,std,std-ref}{Meterological Preprocessor}}}}
to prepare your input data. There is also a possibility to use a single point in time in the plugin.

Requred meteorological data is:
\begin{enumerate}
\item {} 
Air temperature (°C)

\item {} 
Relative humidity (\%)

\item {} 
Incoming shortwave radiation (W m$^{\text{2}}$)

\end{enumerate}

The model performance will increase if also diffure and direct beam solar radiation is
available but the mdoel can also calculate these variables.


\subsection{How to Run SOLWEIG from the UMEP-plugin}
\label{\detokenize{Tutorials/IntroductionToSolweig:how-to-run-solweig-from-the-umep-plugin}}\begin{enumerate}
\item {} 
Open SOLWEIG from \sphinxstyleemphasis{UMEP -\textgreater{} Processor -\textgreater{} Outdoor Thermal Comfort -\textgreater{}
Mean radiant temperature (SOLWEIG)}.
\begin{itemize}
\item {} 
Some additional information about the plugin is found in the lower
left window. You will make use of a test dataset from observations
for Gothenburg, Sweden.

\end{itemize}
\begin{quote}

\begin{figure}[htbp]
\centering
\capstart

\noindent\sphinxincludegraphics[width=1070\sphinxpxdimen]{{SOLWEIG_Interface}.png}
\caption{Dialog for the SOLWEIG model (click on image for larger image)}\label{\detokenize{Tutorials/IntroductionToSolweig:id3}}\end{figure}
\end{quote}

\item {} 
To be able to run the model some additional spatial datasets needs to
be created.
\begin{itemize}
\item {} 
Close the SOLWEIG plugin and open \sphinxstyleemphasis{UMEP -\textgreater{} Pre-Processor -\textgreater{} Urban
geometry -\textgreater{} Sky View Factor}.

\item {} 
To run SOLWEIG various sky view factor (SVF) maps for both
vegetation and buildings must be created (see \sphinxhref{http://link.springer.com/article/10.1007/s00704-010-0382-8}{Lindberg and
Grimmond
(2011)}
for details).

\item {} 
You can create all SVFs needed (vegetation and buildings) at the
same time. Use the settings as shown below. Use an appropriate
output folder for your computer.

\end{itemize}
\begin{quote}

\begin{figure}[htbp]
\centering
\capstart

\noindent\sphinxincludegraphics[width=487\sphinxpxdimen]{{SOLWEIG_SVF_solweig}.png}
\caption{Settings for the SkyViewFactorCalculator.}\label{\detokenize{Tutorials/IntroductionToSolweig:id4}}\end{figure}
\end{quote}
\begin{itemize}
\item {} 
When the calculation is done, map will appear in the map canvas.
This is the ‘total’ SVF i.e., including both buildings and
vegetation. Examine the dataset.

\item {} 
Where are the highest and lowest values found?

\item {} 
Look in your output folder and find a zip-file containing all the
necessary SVF maps needed to run the SOLWEIG-model.

\end{itemize}

\item {} 
Another preprocessing plugin needed is to create the building wall
heights and aspect. Open \sphinxstyleemphasis{UMEP -\textgreater{} Pre-Processor -\textgreater{} Urban geometry -\textgreater{}
Wall height and aspect} and use the settings as shown below.
\begin{quote}

\begin{figure}[htbp]
\centering
\capstart

\noindent\sphinxincludegraphics[width=505\sphinxpxdimen]{{SOLWEIG_wallgeight_solweig}.png}
\caption{Settings for the Wall height and aspect plugin.}\label{\detokenize{Tutorials/IntroductionToSolweig:id5}}\end{figure}
\end{quote}

\item {} 
Re-open the SOLWEIG plugin and use the settings shown below. You will
use the GUI to set one point in time (i.e. a summer hour in
Gothenburg, Sweden) hence, no input meteorological file is needed for
now. No information on vegetation and ground cover is added for this
first try. Click \sphinxstylestrong{Run}.
\begin{quote}

\begin{figure}[htbp]
\centering
\capstart

\noindent\sphinxincludegraphics[width=1078\sphinxpxdimen]{{SOLWEIG_Tmrt1_solweig}.png}
\caption{The settings for your first SOLWEIG run (click on image for larger image).}\label{\detokenize{Tutorials/IntroductionToSolweig:id6}}\end{figure}
\end{quote}

\item {} 
Examine the output (Average T$_{\text{mrt}}$ (°C). What is the main
driver to the spatial variations in T$_{\text{mrt}}$?

\item {} 
Add 3D vegetation information by ticking in \sphinxstyleemphasis{Use vegetation scheme
(Lindberg, Grimmond 2011)} and add \sphinxstylestrong{CDSM\_Krbig} as the \sphinxstyleemphasis{Vegetation
Canopy DSM}. As no TDSM exists we estimate the it by using 25\% of the
canopy height. Leave the tranmissivity as 3\%. Tick in \sphinxstyleemphasis{Save generated
Trunk Zone DSM} (a tif file, \sphinxstylestrong{TDSM.tif}, will be generated in the
specified output folder and used in a later section: \sphinxstylestrong{Climate
sensitive planning}). Also tick in \sphinxstyleemphasis{Save generated building grid} as
this will be needed later in this tutorial. Leave the other setting
as before (Step 4) except for changing your output directory
Otherwise, results from your first run will be overwritten. Run the
model again and compare the result with your first run.

\item {} 
Add your last spatial dataset, the \sphinxstylestrong{land cover} grid by ticking in
\sphinxstyleemphasis{Use land cover scheme (Lindberg et al. 2016)}. Run and compare the
result again with the previous runs.

\end{enumerate}


\subsection{Using meteorolgical data and POIs}
\label{\detokenize{Tutorials/IntroductionToSolweig:using-meteorolgical-data-and-pois}}
SOLWEIG is also able to run a continuous dataset of meteorological data.
You will make use of a single summer day as well as a winter day for
Gothenburg, Sweden. The GUI is also able to derive full model output
(all calculated variables) from certain points of interest (POIs).
\begin{enumerate}
\item {} 
First you need to create a point vector layer to store the POIs. Go
to \sphinxstyleemphasis{Layer -\textgreater{} Create Layer -\textgreater{} New Shape file}. Choose \sphinxstyleemphasis{Point} as
\sphinxstyleemphasis{Type} and add a new text field called \sphinxstylestrong{name}. Name the new layer
\sphinxstylestrong{POI\_Kr.shp}. Specify the coordinate system as SWEREF99 12 00
(EPSG: 3007).

\item {} 
Now you should add two points within the study area. To add points to
the layer it has to be editable and Add Feature should be activated.
\begin{quote}

\begin{figure}[htbp]
\centering
\capstart

\noindent\sphinxincludegraphics[width=411\sphinxpxdimen]{{SOLWEIG_AddPoint}.png}
\caption{Setting to add points}\label{\detokenize{Tutorials/IntroductionToSolweig:id7}}\end{figure}
\end{quote}

Two points should be added and the attributes should be id=\sphinxstylestrong{1} and
name=\sphinxstylestrong{courtyard} for the right point and id=\sphinxstylestrong{2} and
name=\sphinxstylestrong{park} for the left point. See figure below for the locations of
the two points.
\begin{quote}

\begin{figure}[htbp]
\centering
\capstart

\noindent\sphinxincludegraphics[width=846\sphinxpxdimen]{{SOLWEIG_Pointskr}.png}
\caption{Location of the two POIs}\label{\detokenize{Tutorials/IntroductionToSolweig:id8}}\end{figure}
\end{quote}

When you are
finished, save layer edits (box in-between the two marked boxes in
Figure 6). Close the editing by pressing Toggle editing (the pencil).

\item {} 
Now open the SOLWEIG plugin. Use both the vegetation and land cover
schemes as before. This time, tick in \sphinxstyleemphasis{Include POI(s)}, select your
point layer and use the ID attribute as \sphinxstyleemphasis{ID field}.

\item {} 
Tick in \sphinxstyleemphasis{Use continuous meteorological dataset} and choose
\sphinxstylestrong{gbg19970606\_2015a.txt} as \sphinxstyleemphasis{Input meteorological file}. Also, tick
in to save T$_{\text{mrt}}$ as \sphinxstyleemphasis{Output maps}. Run the model again.

\end{enumerate}


\subsection{Examine your output with SOLWEIG Analyzer}
\label{\detokenize{Tutorials/IntroductionToSolweig:examine-your-output-with-solweig-analyzer}}
To perform a first set of analysis of your result you can make use of
the SOLWEIG Analyzer plug-in.
\begin{enumerate}
\item {} 
Open the Analyzer located in \sphinxstyleemphasis{UMEP -\textgreater{} Post-Processor -\textgreater{} Outdoor
Thermal Comfort -\textgreater{} SOLWEIG Analyzer}. Here you can analyze both data
from your POIs as well as perform statistical analysis based on saved
output maps. Start by locating your output folder in the top section
(\sphinxstyleemphasis{Load Model Result}).
\begin{quote}

\begin{figure}[htbp]
\centering
\capstart

\noindent\sphinxincludegraphics[width=875\sphinxpxdimen]{{SOLWEIG_SOLWEIGAnalyzer}.png}
\caption{Dialog for the SOLWEIG Analyzer plug-in}\label{\detokenize{Tutorials/IntroductionToSolweig:id9}}\end{figure}
\end{quote}

\item {} 
Firstly you will compare differences in T$_{\text{mrt}}$ for the two
locations (courtyard and park). This can done using the left frame
(\sphinxstyleemphasis{Point of Interest data}). Specify \sphinxstyleemphasis{courtyard} as \sphinxstyleemphasis{POI} and \sphinxstyleemphasis{Mean
Radiant Temperature} in the two top scroll down lists. Then tick in
\sphinxstyleemphasis{Include another POI/variable} and chose \sphinxstyleemphasis{park} and \sphinxstyleemphasis{Mean Radiant
Temperature} below. Click \sphinxstyleemphasis{Plot}. What explains the differences?

\item {} 
Now lets us move on to analyse the output maps generated from our
last model run. In the right frame, specify \sphinxstyleemphasis{Mean Radiant
Temperature} as \sphinxstyleemphasis{Variable to visualize}. Start by clicking \sphinxstyleemphasis{Show
Animation}. Now the output maps of T$_{\text{mrt}}$ generated before are
displayed in a sequence.

\item {} 
Next step is to generate some statistical maps from the last model
run. Specify \sphinxstyleemphasis{Mean Radiant Temperature} as \sphinxstyleemphasis{Variable to visualize}
and tick in to \sphinxstyleemphasis{Exclude building pixels}. Choose the building grid
that you saved earlier in this tutorial. If it is not in the
drop-down list you need to add this layer (\sphinxstylestrong{buildings}) to your
project. Tick in \sphinxstyleemphasis{T:sub:{}`mrt{}`: Percent of time above threshold
(degC)} and specify 55.0 as threshold. Specify an output folder and
tick also in \sphinxstyleemphasis{Add analysis to map canvas} before you generate the
result. The resulting map show the time that a pixel has been above
55 degC based on the whole analysis time i.e. 24 hours. This type of
maps can be used to identify areas prone to e.g. heat stress

\end{enumerate}


\subsection{Climate sensitive planning}
\label{\detokenize{Tutorials/IntroductionToSolweig:climate-sensitive-planning}}
Vegetation is one effective measure to reduce areas prone to heat
related health issues. In this section you make use of the Tree
Generator plugin to see the effect of adding more vegetation into our
study area. The municipality in Gothenburg have identified a “hot spot”
south of the german church and they want to see the effect of planting
three new trees in that area.


\subsubsection{The Tree Generator}
\label{\detokenize{Tutorials/IntroductionToSolweig:the-tree-generator}}
The Tree Generator plugin make use of a point vector file including the
necessary attributes to generate/add/remove vegetation suitable for
either mean radiant temperature modelling with SOLWEIG or urban energy
balance modelling with SUEWS.
\begin{enumerate}
\item {} 
Create a point vector shape file named (\sphinxstylestrong{TreesKR.shp}) as described
in the previous section adding five attributes (\sphinxstyleemphasis{id, ttype, trunk,
totheight, diameter}). The attributes should all be decimal (float)
numbers (see table below). The location of the three new trees are
shown in figure below. The values for all three vegetation units should
be \sphinxstylestrong{ttype=2, trunk=4, totheight=15, diameter=10}.
\begin{quote}

\begin{figure}[htbp]
\centering
\capstart

\noindent\sphinxincludegraphics[width=846\sphinxpxdimen]{{SOLWEIG_File_TreesKR}.png}
\caption{Location of the three new vegetation units.}\label{\detokenize{Tutorials/IntroductionToSolweig:id10}}\end{figure}
\end{quote}

\item {} 
Add your created trunk zone dsm (TDSM.tif) that was created
previously (located in your output directory).

\item {} 
Open the TreeGenerator (UMEP -\textgreater{} PreProcessor -\textgreater{} TreeGenerator) and
use the settings as shown in figure below.
\begin{quote}

\begin{figure}[htbp]
\centering
\capstart

\noindent\sphinxincludegraphics[width=574\sphinxpxdimen]{{SOLWEIG_Treegeneratorsolweig}.png}
\caption{The settings for the Tree Generator}\label{\detokenize{Tutorials/IntroductionToSolweig:id11}}\end{figure}
\end{quote}

\item {} 
As the vegetation DSMs have been changed, the SVFs has to be
recalculated. This time use the two generated vegetation DSMs.

\item {} 
Now re-run SOLWEIG using the same settings as before but now use the
new vegetation surface models as well as the new SVFs generated in
the previous step.

\item {} 
Generate a new, updated threshold map based on the new results and
compare the differences.

\end{enumerate}

The table below show the input variables needed for each tree point.


\begin{savenotes}\sphinxattablestart
\centering
\begin{tabular}[t]{|*{3}{\X{1}{3}|}}
\hline
\sphinxstyletheadfamily 
Attribute name
&\sphinxstyletheadfamily 
Name
&\sphinxstyletheadfamily 
Description
\\
\hline
ttype
&
Tree type
&
Two shapes are
available:
\begin{itemize}
\item {} 
conifer = 1 and

\item {} 
deciduous = 2.

\item {} 
To remove
vegetation set
ttype = 0.

\end{itemize}
\\
\hline
trunk
&
Trunk zone height (m
agl)
&
Height of the trunk
zone.
\\
\hline
totheight
&
Total tree height (m
agl)
&
Maximum height of the
vegetation unit
\\
\hline
diameter
&
Canopy diameter (m)
&
Circular diameter of
the vegetation unit
\\
\hline
\end{tabular}
\par
\sphinxattableend\end{savenotes}


\section{Solar Energy - Introduction to SEBE}
\label{\detokenize{Tutorials/SEBE:solar-energy-introduction-to-sebe}}\label{\detokenize{Tutorials/SEBE:sebe}}\label{\detokenize{Tutorials/SEBE::doc}}

\subsection{Introduction}
\label{\detokenize{Tutorials/SEBE:introduction}}
Lindberg et al.’s (2015) solar radiation model SEBE (Solar Energy on
Building Envelopes) allows estimates of solar irradiance on ground
surfaces, building roofs and walls. It uses a shadow casting algorithm
with a digital surface model (\sphinxstylestrong{DSM}) and the solar position to
generate pixel-level information of shadow or sunlit areas. The shadow
casting algorithm (Ratti and Richens 1999) has been developed to
incorporate walls (Lindberg et al. 2015). This is of interest for a
broad range of applications, for example solar energy potential and
thermal comfort.

SEBE is incorporated in {\hyperref[\detokenize{index:index-page}]{\sphinxcrossref{\DUrole{std,std-ref,std,std-ref}{UMEP(Urban Multi-scale Environmental
Predictor)}}}}, a plugin for
\sphinxhref{http://www.qgis.org}{QGIS}. As SEBE was initially developed to
estimate solar energy potential on building roofs, the Digital Surface
Models (DSMs) used need to include roof structures, such as tilted
roofs, chimneys etc. Methods to produce accurate ground and building
DSMs for SEBE include the use of LiDAR technology and 3D roof structure
objects in vector format.

In this exercise you will apply the model in \sphinxhref{https://en.wikipedia.org/wiki/Gothenburg}{Gothenburg,
Sweden} and \sphinxhref{https://en.wikipedia.org/wiki/Covent\_Garden}{Covent
Garden}, London to
investigate solar energy potential, how it changes between buildings,
with seasons, with the effects of vegetation etc.


\subsection{Initial Practical steps}
\label{\detokenize{Tutorials/SEBE:initial-practical-steps}}\begin{itemize}
\item {} 
If \sphinxstylestrong{QGIS} is not on your computer you will {\hyperref[\detokenize{Getting_Started:getting-started}]{\sphinxcrossref{\DUrole{std,std-ref,std,std-ref}{need to install it}}}}

\item {} 
Then install the {\hyperref[\detokenize{Getting_Started:getting-started}]{\sphinxcrossref{\DUrole{std,std-ref,std,std-ref}{**UMEP** plugin}}}}

\item {} 
Start the QGIS software

\item {} 
\sphinxstyleemphasis{Windows:} If not visible on the desktop use the \sphinxstylestrong{Start} button to
find the software (i.e. Find QGIS under your applications)

\item {} 
Select \sphinxstylestrong{QGIS 2.18 Desktop} (or the latest version installed). Do not use QGIS3 at this point.

\end{itemize}

When you open it on the top toolbar you will see \sphinxstylestrong{UMEP}.

\begin{figure}[htbp]
\centering
\capstart

\noindent\sphinxincludegraphics[width=1066\sphinxpxdimen]{{SEBE_Interfacelocation}.png}
\caption{Location of SEBE in UMEP}\label{\detokenize{Tutorials/SEBE:id1}}\end{figure}
\begin{itemize}
\item {} 
If it UMEP not on our machine, download and install the {\hyperref[\detokenize{Getting_Started:getting-started}]{\sphinxcrossref{\DUrole{std,std-ref,std,std-ref}{**UMEP**
plugin}}}}

\item {} 
Read through the section in the online
manual BEFORE using the model, so you are familiar with it’s operation and
terminology used.

\end{itemize}


\subsubsection{Data for Tutorial}
\label{\detokenize{Tutorials/SEBE:data-for-tutorial}}
\begin{figure}[htbp]
\centering
\capstart

\noindent\sphinxincludegraphics[width=356\sphinxpxdimen]{{SEBE_Gothenburg}.png}
\caption{Central Gothenburg study area (red square).
The Open layers plugin in QGIS was used to generate
this snapshot.}\label{\detokenize{Tutorials/SEBE:id2}}\end{figure}

Geodata and meteorological data for \sphinxstylestrong{Gothenburg, Sweden}.
\begin{itemize}
\item {} 
Data are projected in SWEREF99 1200 (EPSG:3007) the national
coordinate system of Sweden.

\end{itemize}

Data requreiments:
S: Spatial, M: Meteorological,


\begin{savenotes}\sphinxattablestart
\centering
\sphinxcapstartof{table}
\sphinxcaption{Input data and parameters}\label{\detokenize{Tutorials/SEBE:id3}}
\sphinxaftercaption
\begin{tabular}[t]{|\X{20}{100}|\X{20}{100}|\X{10}{100}|\X{50}{100}|}
\hline

\sphinxstylestrong{Name}
&
\sphinxstylestrong{Definition}
&
\sphinxstylestrong{Type}
&
\sphinxstylestrong{Description}
\\
\hline
krbig\_dsm.asc
&
Ground and building DSM
&
S
&
Raster dataset: derived from a 3D vector roof structure dataset and a digital elevation model (DEM)
\\
\hline
krbig\_cdsm.asc
&
Vegetation canopy DSM
&
S
&
Raster dataset: drived from a LiDAR dataset
\\
\hline
kr\_buildings.shp
&
Building footprint polygon layer
&
S
&
Vector dataset
\\
\hline
GBG\_typicalweatheryear\_1977.txt
&
Meteorological forcing data
&
M
&
Meteorological data, hourly time resolution for 1977 Gothenburg, Sweden.
\\
\hline
\end{tabular}
\par
\sphinxattableend\end{savenotes}

\sphinxhref{https://github.com/Urban-Meteorology-Reading/Urban-Meteorology-Reading.github.io/tree/master/other\%20files/Goteborg\_SWEREF99\_1200.zip}{Download link for datasets in Gothenburg,
Sweden}

\sphinxhref{https://github.com/Urban-Meteorology-Reading/Urban-Meteorology-Reading.github.io/tree/master/other\%20files/DataCoventGarden.zip}{Download link for datasets in London Covent
Garden}

\sphinxhref{https://www.google.co.uk/maps/@51.5117012,-0.1231273,356m/data=!3m1!1e3}{Google map link to Convent Garden}


\subsection{Steps}
\label{\detokenize{Tutorials/SEBE:steps}}\begin{enumerate}
\item {} 
Start with the Gothenbrug data. If the data are zipped - unzip the data first.

\item {} 
Examine the geodata by adding the layers to your project.

\item {} 
Use \sphinxstyleemphasis{Layer \textgreater{} Add Layer \textgreater{} Add Raster
Layer} to open the .asc raster files
and \sphinxstyleemphasis{Layer \textgreater{} Add Layer \textgreater{} Add Vector Layer}. The Vector layer is a
shape file which consists of multiple files. It is the
\sphinxstylestrong{kr\_building.shp} that should be used to load the vector layer into
QGIS.

\item {} 
You will need to indicate the co-ordinate system
(\sphinxhref{https://docs.qgis.org/2.18/en/docs/gentle\_gis\_introduction/coordinate\_reference\_systems.html}{CRS})
that is associated with these data. If you look at the lower right
hand side you can see the CRS used in the current QGIS project.
\begin{itemize}
\item {} 
You can use the filter to find this then.

\item {} 
Select SWEREF99 1200 as CRS and the files will load into the map
canvas.

\item {} 
Do this for all of the geodata files.

\end{itemize}

\item {} 
Open the \sphinxstylestrong{meteorological file} in a text editor or in a spreadsheet
such as MS excel or LibreOffice (Open office).
\begin{itemize}
\item {} 
Data file is formatted for the UMEP plugin (in general) and the
SEBE plugin (in particular).

\item {} 
First four columns are \sphinxstyleemphasis{time related}.

\item {} 
Columns of interest are \sphinxstylestrong{kdown, kdiff and kdir}. These are
related to shortwave radiation and give global, diffuse and direct
radiation, respectively.

\item {} 
The meteorological file should be at least a year long, but
preferably multi-year.

\item {} 
One option is to use a \sphinxhref{https://en.wikipedia.org/wiki/Typical\_meteorological\_year}{**typical meteorological
year**}
as you will do in this tutorial

\end{itemize}

\end{enumerate}

Variables included in the \sphinxstylestrong{meteorological data file}. No. indicates
the column the file is in. Use indicates if it is \sphinxstylestrong{R \textendash{} required} or
\sphinxstyleemphasis{O- optional} (in this application) or \sphinxstylestrong{N- Not used in this
application}. All columns must be present but can be filled with
numbers to indicate they are not in use (e.g. -999).


\begin{savenotes}\sphinxattablestart
\centering
\begin{tabulary}{\linewidth}[t]{|T|T|T|T|}
\hline
\sphinxstyletheadfamily 
No.
&\sphinxstyletheadfamily 
USE
&\sphinxstyletheadfamily 
Column name
&\sphinxstyletheadfamily 
Description
\\
\hline
1
&
R
&
iy
&
Year {[}YYYY{]}
\\
\hline
2
&
R
&
id
&
Day of year
{[}DOY{]}
\\
\hline
3
&
R
&
it
&
Hour {[}H{]}
\\
\hline
4
&
R
&
imin
&
Minute {[}M{]}
\\
\hline
5
&
N
&
qn
&
Net all-wave
radiation {[}W
m$^{\text{-2}}${]}
\\
\hline
6
&
N
&
qh
&
Sensible heat
flux {[}W
m$^{\text{-2}}${]}
\\
\hline
7
&
N
&
qe
&
Latent heat
flux {[}W
m$^{\text{-2}}${]}
\\
\hline
8
&
N
&
qs
&
Storage heat
flux {[}W
m$^{\text{-2}}${]}
\\
\hline
9
&
N
&
qf
&
Anthropogenic
heat flux {[}W
m$^{\text{-2}}${]}
\\
\hline
10
&
N
&
U
&
Wind speed {[}m
s$^{\text{-1}}${]}
\\
\hline
11
&
O
&
RH
&
Relative
Humidity {[}\%{]}
\\
\hline
12
&
O
&
Tair
&
Air temperature
{[}°C{]}
\\
\hline
13
&
N
&
pres
&
Barometric
pressure {[}kPa{]}
\\
\hline
14
&
N
&
rain
&
Rainfall {[}mm{]}
\\
\hline
15
&
R
&
kdown
&
Incoming
shortwave
radiation {[}W
m$^{\text{-2}}${]}
Must be \textgreater{}= 0 W
m$^{\text{-2}}$.
\\
\hline
16
&
N
&
snow
&
Snow {[}mm{]}
\\
\hline
17
&
N
&
ldown
&
Incoming
longwave
radiation {[}W
m$^{\text{-2}}${]}
\\
\hline
18
&
N
&
fcld
&
Cloud fraction
{[}tenths{]}
\\
\hline
19
&
N
&
Wuh
&
External water
use {[}msup:\sphinxcode{\sphinxupquote{3}}{]}
\\
\hline
20
&
N
&
xsmd
&
Observed soil
moisture {[}m3
m$^{\text{-3}}$ or
kg
kg$^{\text{-1}}${]}
\\
\hline
21
&
N
&
lai
&
Observed leaf
area index {[}m2
m$^{\text{-2}}${]}
\\
\hline
22
&
O
&
kdiff
&
Diffuse
radiation {[}W
m$^{\text{-2}}${]}
\\
\hline
23
&
O
&
kdir
&
Direct
radiation {[}W
m$^{\text{-2}}${]}
\\
\hline
24
&
N
&
wdir
&
Wind direction
{[}°{]}
\\
\hline
\end{tabulary}
\par
\sphinxattableend\end{savenotes}


\subsection{Preparing data for SEBE}
\label{\detokenize{Tutorials/SEBE:preparing-data-for-sebe}}
SEBE plugin: located at \sphinxstyleemphasis{UMEP -\textgreater{} Processor -\textgreater{} Solar Energy -\textgreater{} Solar
Energy on Building Envelopes (SEBE)} in the menu bar.

\begin{figure}[htbp]
\centering
\capstart

\noindent\sphinxincludegraphics[width=514\sphinxpxdimen]{{SEBE_SEBE1}.png}
\caption{The interface for SEBE in UMEP}\label{\detokenize{Tutorials/SEBE:id4}}\end{figure}
\begin{enumerate}
\item {} 
\sphinxstyleemphasis{Top frame}: for input data for the SEBE calculations.
\begin{itemize}
\item {} 
Critical is the \sphinxstylestrong{building and ground}
{\hyperref[\detokenize{Abbreviations:abbreviations}]{\sphinxcrossref{\DUrole{std,std-ref,std,std-ref}{DSM}}}}
for the calculations in SEBE.

\item {} 
Optionally \sphinxstylestrong{vegetation} (trees and bushes) can be included as
they can shadow buildings, walls and roofs reducing the potential
solar energy production

\item {} 
Two vegetation DSMs are required when the Use vegetation DSMs is
ticked:
\begin{itemize}
\item {} 
One to describe the top of the vegetation (Vegetation Canopy DSM).

\item {} 
One to describe the bottom, underneath the canopies (Vegetation Trunk Zone DSM).

\end{itemize}

As Trunk Zone DSMs are very rare, an option to create this from the
canopy DSM is available. You can set the amount of light (shortwave radiation) that is
transmitted through the vegetation.

\end{itemize}

\item {} 
Two raster datasets, height and wall aspect, are needed to calculate
irradiance on building walls.
\begin{itemize}
\item {} 
The average albedo (one value is used for all surfaces) can be
changed.

\end{itemize}

\item {} 
The
\sphinxhref{https://en.wikipedia.org/wiki/Coordinated\_Universal\_Time}{UTC}
offset is needed to accurately estimate the sun position, positive
numbers for easterly position and negative for westerly. For example,
Gothenburg is located in CET which is UTC +1.

\item {} 
Meteorological file needs to be specified.

\item {} 
Wall data are created with the {\hyperref[\detokenize{pre-processor/Urban Geometry Wall Height and Aspect:wallheightandaspect}]{\sphinxcrossref{\DUrole{std,std-ref,std,std-ref}{UMEP plugin - **Wall Height and
Aspect**}}}}:
\begin{itemize}
\item {} 
This uses a 3 by 3 pixels kernel minimum filter where the four
cardinal points (N, W, S,E) are investigated. The pixels just
‘inside’ the buildings are identified and give values to indicate
they are a building edge. The aspect algorithm originates from a
linear filtering technique (Goodwin et al. 2009). It identifies
the linear features plus (a new addition) the aspect of the
identified line. Other more accurate techniques include using a
vector building layer and spatially relating this to the wall
pixels.

\end{itemize}

\item {} 
UMEP -\textgreater{} Pre-Processor -\textgreater{} Urban Geometry -\textgreater{} Wall Height and Aspect.

\item {} 
Close the SEBE plugin and open the Wall and Height and Aspect plugin

\item {} 
Use your ground and building DSM as input

\item {} 
Tick the option to Calculate wall aspect.

\item {} 
Create a folder in your Documents folder called e.g. SEBETutorial

\item {} 
Use this to save the result.

\item {} 
Name your new raster datasets aspect and height, respectively.

\item {} 
Tick: Add result to project and click OK.

\end{enumerate}


\subsection{Running the model}
\label{\detokenize{Tutorials/SEBE:running-the-model}}
Now you have all data ready to run the model.

\begin{figure}[htbp]
\centering
\capstart

\noindent\sphinxincludegraphics{{SEBE_SEBEnoVeg}.png}
\caption{Example of settings for running SEBE without vegetation.}\label{\detokenize{Tutorials/SEBE:id5}}\end{figure}
\begin{enumerate}
\item {} 
First run the model \sphinxstyleemphasis{without} including vegetation.
\begin{itemize}
\item {} 
Open the SEBE-plugin again

\item {} 
Make the setting according to the figure to the LHS

\item {} 
Save your results in a subfolder (\sphinxstylestrong{NoVeg}) of \sphinxstylestrong{SEBETutorial}.

\item {} 
The model takes some time to calculate irradiance on all the
surfaces.

\item {} 
The result added to your map canvas is the horizontal radiation,
i.e. irradiance on the ground and roofs.

\end{itemize}

\item {} 
Run the model again but this time also use the vegetation DSM.
\begin{itemize}
\item {} 
Save your result in a subfolder called \sphinxstylestrong{Veg}.

\end{itemize}

\end{enumerate}


\subsection{Irradiance on building envelopes (alternatively see the tips below \textendash{} currrently better)}
\label{\detokenize{Tutorials/SEBE:irradiance-on-building-envelopes-alternatively-see-the-tips-below-currrently-better}}
To determine the irradiance on building walls:
\begin{enumerate}
\item {} 
Open the SunAnalyser located at \sphinxstyleemphasis{UMEP \textgreater{} Post-Processor \textgreater{} Solar
Radiation \textgreater{} SEBE (Visualisation)}.
\begin{itemize}
\item {} 
This can be used to visualize the irradiance on both roofs and
walls.

\end{itemize}

\item {} 
Choose the input folder where you saved your result for one of the
runs.

\item {} 
Mark an area with the tool (Area of Visualisation) on the map canvas
by click first once

\item {} 
Drag to produce an area

\item {} 
Click again to finish.

\item {} 
Click Visualise. Now you should be able to see the results in 3D.

\end{enumerate}

\sphinxstylestrong{3D Visualisation for Mac currently not working properly}

Use the \sphinxstylestrong{Profile tool}, which is a plugin for QGIS, to see the range of values along a transect.
\begin{enumerate}
\item {} 
Plugins \textgreater{} Profile tool \textgreater{} Terrain profile.
\begin{itemize}
\item {} 
Draw a line across the screen on the area of interest. Double
click and you will see the profile drawn. Make certain you use the
correct layer (see Tips).

\end{itemize}

\item {} 
If this is not installed you will need to install it from official
QGIS-plugin reporistory (Plugins \textgreater{} Manage and Install Plugins).

\end{enumerate}


\subsection{Solar Energy Potential}
\label{\detokenize{Tutorials/SEBE:solar-energy-potential}}
In order to obtain the solar energy potential for a specific building:
\begin{enumerate}
\item {} 
The actual area of the roof needs to be considered.

\item {} 
Determine the area of each pixel (A$_{\text{P}}$): e.g. 1 m$^{\text{2}}$

\item {} 
As some roofs are tilting the area may be larger for some pixels. The
actual area (\sphinxstyleemphasis{A}$_{\text{A}}$) can be computed from:
\begin{quote}

\sphinxstyleemphasis{A}$_{\text{A}}$ = \sphinxstyleemphasis{A}$_{\text{P}}$ / \sphinxstyleemphasis{cos(S}$_{\text{i}}$)

where the slope (\sphinxstyleemphasis{S}$_{\text{i}}$) of the raster pixel should be in radians (1 deg = pi/180 rad).
\end{quote}

\end{enumerate}

\sphinxstylestrong{To make a slope raster:}
\begin{enumerate}
\item {} 
\sphinxstyleemphasis{Raster \textgreater{} Terrain analysis \textgreater{} Slope}. If the tool is missing, Go to
\sphinxstyleemphasis{Manage and Install Plugins} and activate (\sphinxstyleemphasis{Raster Terrain Analysis
Plugin})

\item {} 
Use the DSM for elevation layer

\item {} 
Create the slope z factor =1 - area

\end{enumerate}

\begin{figure}[htbp]
\centering
\capstart

\noindent\sphinxincludegraphics{{SEBE_Slope}.png}
\caption{The Slope tool in QGIS}\label{\detokenize{Tutorials/SEBE:id6}}\end{figure}

Use the raster menu: \sphinxstyleemphasis{Raster\textgreater{} Raster Calculator}.
\begin{enumerate}
\item {} 
To determine the area after you have removed the wall area from the
buildings.

\item {} 
Enter the equation indicated.

\item {} 
To visualize where to place solar panels the amount of energy
received needs to be cost effective. As irradiance below 900 kWh is
considered to be too low for solar energy production (\sphinxstyleemphasis{Per Jonsson
personal communication Tyréns Consultancy}), pixel cells lower than
900 can be filtered out (Figure LHS). Transparency \textendash{} allows you to
make visible above a threshold of interest.
\begin{itemize}
\item {} 
Right-click on the Energyyearroof-layer and go to \sphinxstylestrong{Properties}
and then \sphinxstylestrong{Transparency}.

\item {} 
Add a custom transparency (green cross) where values between 0 and
900 are set to 100\% transparency.

\end{itemize}

\end{enumerate}

\begin{figure}[htbp]
\centering
\capstart

\noindent\sphinxincludegraphics{{SEBE_RasterCalculator}.png}
\caption{The RasterCalculator in QGIS}\label{\detokenize{Tutorials/SEBE:id7}}\end{figure}


\subsubsection{Irradiance map with values less than 900 kWh filtered out}
\label{\detokenize{Tutorials/SEBE:irradiance-map-with-values-less-than-900-kwh-filtered-out}}
To estimate solar potential on building roofs we can use the Zonal
statistics tool:
\begin{enumerate}
\item {} 
Raster \textgreater{} Zonal statistics.
\begin{itemize}
\item {} 
Use the roof area raster layer (\sphinxstylestrong{energyPerm2\_slope65\_RoofArea})
created before and use \sphinxstylestrong{kr\_building.shp} as the polygon layer to
calculate as your zone layer. Make sure that you calculate sum
statistics.

\end{itemize}

\item {} 
On your building layer \textendash{} Right click Open Attribute Table

\item {} 
Or use the identifier to click a building (polygon) of interest to
see the statistics you have just calculated

Note that we will not consider the performance of the solar panels.

\begin{figure}[htbp]
\centering
\capstart

\noindent\sphinxincludegraphics{{SEBE_GOT_Irradiance}.png}
\caption{Irradiance map on building roofs in Gothenburg}\label{\detokenize{Tutorials/SEBE:id8}}\end{figure}

\end{enumerate}


\subsection{Covent Garden data set}
\label{\detokenize{Tutorials/SEBE:covent-garden-data-set}}
A second GIS data set is available for the Covent Garden area in London
\begin{enumerate}
\item {} 
Close the Gothenburg data (it may be easiest to completely close QGIS
and re-open).

\item {} 
Download from
\sphinxhref{https://drive.google.com/open?id=0B7D8dqiua0uzWWhwWmU4c1lnTG8}{1}

\item {} 
Add the Covent Garden data

\item {} 
Extract the data to a directory

\item {} 
Load the Raster data (DEM, DSM) files (as you did before)

\item {} 
Shadows
\begin{itemize}
\item {} 
{\hyperref[\detokenize{processor/Solar Radiation Daily Shadow Pattern:dailyshadowpattern}]{\sphinxcrossref{\DUrole{std,std-ref,std,std-ref}{UMEP -\textgreater{} Processor -\textgreater{} Solar Radiation -\textgreater{} Daily Shadow
pattern}}}}

\item {} 
Allows you to calculate the shadows for a particular time of day
and \sphinxhref{http://disc.sci.gsfc.nasa.gov/julian\_calendar.html}{Day of
Year}.

\end{itemize}

\end{enumerate}


\subsection{Questions for you to explore with UMEP:SEBE}
\label{\detokenize{Tutorials/SEBE:questions-for-you-to-explore-with-umep-sebe}}\begin{enumerate}
\item {} 
Use the Gothenburg dataset consider the impact of vegetation.
\begin{itemize}
\item {} 
What are the main differences between the two model runs with
respect to ground and roof surfaces?

\item {} 
To what extent are the building roofs affected by vegetation?

\end{itemize}

\item {} 
Consider the differences between London and Gothenburg. You can run
the model for different times of the year by modifying the
meteorological data so the file only has the period of interest.

\item {} 
For Covent Garden, determine the solar energy potential for a
specific building within the model domain. Work in groups to consider
different areas. What would be the impact of having a smaller/larger
area domain modelled for this building? Identify the possibilities of
solar energy production for that building.

\end{enumerate}


\subsection{References}
\label{\detokenize{Tutorials/SEBE:references}}\begin{itemize}
\item {} 
Goodwin NR, Coops NC, Tooke TR, Christen A, Voogt JA 2009:
Characterizing urban surface cover and structure with airborne lidar
technology. \sphinxhref{http://pubs.casi.ca/doi/abs/10.5589/m09-015?journalCode=cjrs}{Can J Remote Sens
35:297\textendash{}309}

\item {} 
Lindberg F, Jonsson P, Honjo T, Wästberg D 2015: Solar energy on
building envelopes - 3D modelling in a 2D environment. \sphinxhref{http://www.sciencedirect.com/science/article/pii/S0038092X15001164}{Solar Energy.
115,
369\textendash{}378}

\item {} 
Ratti CF, Richens P 1999: Urban texture analysis with image
processing techniques Proc CAADFutures99, Atlanta, GA

\end{itemize}

\sphinxstylestrong{Authors of this document}: Lindberg and Grimmond (2015, 2016)

\sphinxstyleemphasis{Contributors to the material covered}
\begin{itemize}
\item {} 
University of Gothenburg: Fredrik Lindberg

\item {} 
University of Reading: Sue Grimmond

\item {} 
Background work also comes from: UK (Ratti \& Richens 1999), Sweden
(Lindberg et al. 2015), Canada (Goodwin et al. 2009)

\end{itemize}

In the \sphinxhref{https://bitbucket.org/fredrik\_ucg/umep/}{repository} of UMEP you can find the code and report bugs and other suggestions on future improvments.


\subsection{Tips}
\label{\detokenize{Tutorials/SEBE:tips}}
\sphinxstylestrong{Meteorological} file in UMEP has a special format. If you have data
in another format there is a {\hyperref[\detokenize{pre-processor/Meteorological Data MetPreprocessor:metpreprocessor}]{\sphinxcrossref{\DUrole{std,std-ref,std,std-ref}{UMEP plugin that can convert your
meteorological data into the UMEP
format}}}}.
\begin{itemize}
\item {} 
Plugin is found at \sphinxstyleemphasis{UMEP \textgreater{} Pre-Processor \textgreater{} Meteorological data \textgreater{}Prepare Existing data}.

\end{itemize}

Plugin to \sphinxstylestrong{visualize data} in 3D: called
\sphinxhref{https://media.readthedocs.org/pdf/qgis2threejs/docs-release/qgis2threejs.pdf}{Qgis2Threejs}.
\begin{itemize}
\item {} 
Available for download from the official repository Plugins -\textgreater{} Manage
and Install Plugins.

\end{itemize}

\begin{figure}[htbp]
\centering
\capstart

\noindent\sphinxincludegraphics{{SEBE_CoventGarden}.png}
\caption{3D visualisation with Qgis2Threejs over Convent Garden}\label{\detokenize{Tutorials/SEBE:id9}}\end{figure}

TIFF (TIF) and ASC are \sphinxstylestrong{raster data file formats} In the left Hand
Side there is a list of layers.
\begin{itemize}
\item {} 
The layer that is checked at the top of the list is the layer that is
seen, If you want to see another layer you can either:

\end{itemize}

\#*Un-tick the layers above the one you are interested in and/or

\#*Move the layer you are interested in to the top of the list by
dragging it.

You can save all of you work for different areas as a project \textendash{} so you
can return to it as whole.
\begin{itemize}
\item {} 
Project \textgreater{} Save as

\end{itemize}

You can change the \sphinxstyleemphasis{shading etc}. on different layers.
\begin{itemize}
\item {} 
Right Click on the Layer name Properties \textgreater{} Style \textgreater{} Singlebandpseudo
color

\item {} 
Choose the color band you would like.

\item {} 
Classify

\item {} 
Numerous things can be modified from this point.

\end{itemize}

\sphinxhref{https://bitbucket.org/fredrik\_ucg/umep/}{UMEP repository}.


\section{Anthropogenic heat - LQF}
\label{\detokenize{Tutorials/LQF:anthropogenic-heat-lqf}}\label{\detokenize{Tutorials/LQF:lqf}}\label{\detokenize{Tutorials/LQF::doc}}
This tutorial demonstrates how the LQF software is used to simulate
anthropogenic heat fluxes for London, UK, in the year 2015, using a
mixture of administrative and meteorological data.


\subsection{Initial Practical steps}
\label{\detokenize{Tutorials/LQF:initial-practical-steps}}
UMEP is a python plugin used in conjunction with
\sphinxhref{http://www.qgis.org}{QGIS}. To install the software and the UMEP
plugin see the {\hyperref[\detokenize{Getting_Started:getting-started}]{\sphinxcrossref{\DUrole{std,std-ref,std,std-ref}{getting
started}}}}
section in the UMEP manual.

For this tutorial you will need certain python libraries: \sphinxstylestrong{Pandas} and
\sphinxstylestrong{NetCDF}. Instructions how to install python libraries are located
{\hyperref[\detokenize{Getting_Started:python-libraries}]{\sphinxcrossref{\DUrole{std,std-ref,std,std-ref}{here}}}}.

As UMEP is under constant development, some documentation may be missing
and/or there may be instability. Please report any issues or suggestions
to our \sphinxhref{https://bitbucket.org/fredrik\_ucg/umep/}{repository}.


\subsubsection{Data for this exercise}
\label{\detokenize{Tutorials/LQF:data-for-this-exercise}}
In order to proceed, you will need the zip file named LQF\_Inputs\_1.zip
and a copy of the LQF database. These are available at the following
locations:
\begin{enumerate}
\item {} 
\sphinxhref{https://urban-meteorology-reading.github.io/other\%20files/LQF\_Database.zip}{LQF database} - v1.2 was used to
produce this tutorial

\item {} 
\sphinxhref{https://urban-meteorology-reading.github.io/other\%20files/LQF\_Inputs\_1.zip}{LQF input
files}
from the UMEP tutorials data repository.

\end{enumerate}

You may also wish to consult the \sphinxhref{LQF\_Manual}{LQF user guide}

The LQF\_Inputs\_1.zip file contains several datasets to cover all of
the tutorials on this page:
\begin{itemize}
\item {} 
LondonBuildingProfiles.csv: Seasonal diurnal profiles of building
energy consumption (1 hour resolution; 7 days; 6 times of year)

\item {} 
weeklyTraffic.csv: Diurnal profile of road transport volume (1 hour
resolution; 7 days)

\item {} 
DataSources.nml: LQF data sources

\item {} 
Parameters.nml: LQF Calculation parameters

\item {} 
griddedResidentialPopulation.*: Spatial data containing residential
population counts

\item {} 
dailyTemperature\_2015.csv: Daily mean air temperature in London

\end{itemize}


\subsection{LQF Tutorial 1: Simple QF modelling}
\label{\detokenize{Tutorials/LQF:lqf-tutorial-1-simple-qf-modelling}}

\subsubsection{Preparing data}
\label{\detokenize{Tutorials/LQF:preparing-data}}

\paragraph{Manage input data files}
\label{\detokenize{Tutorials/LQF:manage-input-data-files}}\begin{itemize}
\item {} 
Extract the contents of LQF\_Inputs\_1.zip into a folder on your
local machine and note the path to each file (e.g.
C:\textbackslash{}LQFData\textbackslash{}LondonBuildingProfiles.csv)

\item {} 
Save the LQF database to the same folder as the other data.

\end{itemize}


\paragraph{Gather information about output areas shapefile}
\label{\detokenize{Tutorials/LQF:gather-information-about-output-areas-shapefile}}
The input data comes with one shapefile
(griddedResidentialPopulation.shp) (a collection of files with the same
name, one of which ends in .shp). For convenience, the same shapefile is
used to define both the model output areas and the residential
population density:
\begin{itemize}
\item {} 
\sphinxstylestrong{Model output areas}: The spatial units where each value of QF is
calculated.

\item {} 
\sphinxstylestrong{Residential population}: The number of people within each spatial
unit.

\end{itemize}

LQF needs several pieces of information about the input shapefiles:
\begin{enumerate}
\item {} 
\sphinxstylestrong{File path:} (e.g. c:\textbackslash{}LQFData\textbackslash{}PopulationData.shp) - the location
of the \sphinxstylestrong{.shp} file on your computer

\item {} 
\sphinxstylestrong{EPSG code:} A number that defines the coordinate reference system
(CRS) of the shapefile

\item {} 
\sphinxstylestrong{Feature ID field:} An attribute within the output areas file that
contains a unique identifier for each output area.

\item {} 
\sphinxstylestrong{Start date} (population data only): The earliest modelled date for
which this dataset should be used

\end{enumerate}

Points (2) and (3) need information from the shapefile itself, as they
change depending on the file used. \sphinxhref{LQF\_Manual\#Appendix\_B:\_Gathering\_information\_about\_shapefiles\_for\_QF\_modelling}{Click
here}
for a guide to finding the EPSG code and feature ID field using QGIS.


\subparagraph{Verify the population attribute name}
\label{\detokenize{Tutorials/LQF:verify-the-population-attribute-name}}
The shapefile used for population data in LQF must always contain an
attribute “Pop” that holds the total number of people in that spatial
unit. Check this is the case (a similar process to finding the Feature
ID field name). The population data was gridded using an algorithm, so
the number of people in each area may not be a whole number.


\subsubsection{Set up the DataSources.nml file}
\label{\detokenize{Tutorials/LQF:set-up-the-datasources-nml-file}}

\paragraph{Add shapefile information}
\label{\detokenize{Tutorials/LQF:add-shapefile-information}}
The data sources file needs to be updated so that it can find the
various data files, and understands what to do with them. A full
description of the Data sources file contents is available
\sphinxhref{LQF\_Manual\#Data\_sources\_file}{here}, but this section shows how to
build up the entries.
\begin{enumerate}
\item {} 
Open DataSources.nml using a text editor (we recommend Notepad++) and
update the entries according to the information gathered above.

\item {} 
There are several sections in the data sources file. Each is bounded
by \sphinxstylestrong{\&section\_name} and ends with \sphinxstylestrong{“/”} and deals with a
different part of the input data

\item {} 
The “\sphinxstylestrong{shapefile}” entry is the path to the file

\item {} 
The \sphinxstylestrong{epsgCode} and \sphinxstylestrong{featureIds} entries are as described earlier

\item {} 
The \sphinxstylestrong{start date} for the residential population data is January 1
2011, which needs to be entered as YYYY-mm-dd (2011-01-01).

\item {} 
The population shapefile is used for both the output areas and the
residential population, so the same filename, EPSG code and featureID
are used to describe both of them

\end{enumerate}

The first two sections of the DataSources.nml file should now look like
this:

\fvset{hllines={, ,}}%
\begin{sphinxVerbatim}[commandchars=\\\{\}]
\PYG{o}{\PYGZam{}}\PYG{n}{outputAreas}
   \PYG{n}{shapefile} \PYG{o}{=} \PYG{l+s+s1}{\PYGZsq{}}\PYG{l+s+s1}{C:/Some/Path/To/Files/griddedResidentialPopulation.shp}\PYG{l+s+s1}{\PYGZsq{}}
   \PYG{n}{epsgCode} \PYG{o}{=} \PYG{l+m+mi}{32631}
   \PYG{n}{featureIds} \PYG{o}{=} \PYG{l+s+s1}{\PYGZsq{}}\PYG{l+s+s1}{ID}\PYG{l+s+s1}{\PYGZsq{}}
\PYG{o}{/}
\PYG{o}{\PYGZam{}}\PYG{n}{residentialPop}
   \PYG{n}{shapefiles} \PYG{o}{=} \PYG{l+s+s1}{\PYGZsq{}}\PYG{l+s+s1}{C:/Some/Path/To/Files/griddedResidentialPopulation.shp}\PYG{l+s+s1}{\PYGZsq{}}
   \PYG{n}{startDates} \PYG{o}{=} \PYG{l+s+s1}{\PYGZsq{}}\PYG{l+s+s1}{2011\PYGZhy{}01\PYGZhy{}01}\PYG{l+s+s1}{\PYGZsq{}}
   \PYG{n}{epsgCodes} \PYG{o}{=} \PYG{l+m+mi}{32631}
   \PYG{n}{featureIds} \PYG{o}{=} \PYG{l+s+s1}{\PYGZsq{}}\PYG{l+s+s1}{ID}\PYG{l+s+s1}{\PYGZsq{}}
\end{sphinxVerbatim}


\paragraph{Add the LQF database and mean daily temperature files}
\label{\detokenize{Tutorials/LQF:add-the-lqf-database-and-mean-daily-temperature-files}}
LQF needs to know the location of its database of national parameters.

\fvset{hllines={, ,}}%
\begin{sphinxVerbatim}[commandchars=\\\{\}]
\PYG{o}{\PYGZam{}}\PYG{n}{database}
   \PYG{n}{path} \PYG{o}{=} \PYG{l+s+s1}{\PYGZsq{}}\PYG{l+s+s1}{C:/Some/Path/To/Files/LQFDatabase\PYGZus{}V1\PYGZhy{}2.sqlite}\PYG{l+s+s1}{\PYGZsq{}}
\PYG{o}{/}
\end{sphinxVerbatim}
\begin{itemize}
\item {} 
The daily temperature file (dailyTemperature\_2015.csv in the zip
file) must be formatted appropriately for LQF. \sphinxhref{LQF\_Manual\#Daily\_temperature}{See the
manual} for a detailed description
of the file format

\end{itemize}

\fvset{hllines={, ,}}%
\begin{sphinxVerbatim}[commandchars=\\\{\}]
\PYGZam{}temporal
    ! Air temperature each day for a year
    dailyTemperature = \PYGZsq{}C:\PYGZbs{}Some\PYGZbs{}Path\PYGZbs{}To\PYGZbs{}Files \PYGZbs{}dailyTemperature\PYGZus{}2015.csv\PYGZsq{}
/
\end{sphinxVerbatim}

The data sources file should now look similar to the example shown in
\sphinxhref{LQF\_Manual\#Data\_sources\_file}{the LQF manual}. In this tutorial, the
default diurnal profiles of traffic and building energy use stored in
the database will be used, but {\hyperref[\detokenize{Tutorials/LQF:Tutorial_2a:_Custom_diurnal_profiles}]{\emph{they can be
overridden}}} by adding options
to the data sources file.


\subsubsection{Run LQF}
\label{\detokenize{Tutorials/LQF:run-lqf}}
Under UMEP \textgreater{} Processor \textgreater{} Urban Energy Balance, choose LQf (LUCY)

\begin{figure}[htbp]
\centering

\noindent\sphinxincludegraphics[width=300\sphinxpxdimen]{{LQF_Tutorial_RunUMEP}.png}
\end{figure}

This loads the model interface dialog box:

\begin{figure}[htbp]
\centering
\capstart

\noindent\sphinxincludegraphics{{LQF_Tutorial_RunLQF}.png}
\caption{\sphinxcode{\sphinxupquote{{}`to do{}`}}}\label{\detokenize{Tutorials/LQF:id1}}\end{figure}


\paragraph{Choose configuration files and output folder}
\label{\detokenize{Tutorials/LQF:choose-configuration-files-and-output-folder}}
Working from the top of the dialog box to the bottom…
\begin{enumerate}
\item {} 
Click the … buttons in the “Configuration and raw input data” panel
to browse to the parameters.nml and DataSources.nml files. A pop-up
error message will warn of any problems inside the files.

\item {} 
\sphinxstylestrong{Output path:} A folder in which the model outputs will be stored.
It is \sphinxstylestrong{strongly recommended} that a new folder is used each time.

\item {} 
\sphinxstylestrong{Extra spatial disaggregation} step is not used here

\item {} 
Click \sphinxstylestrong{Prepare input data using Data Sources} button. This may be a
time-consuming step: It matches each output area with a population
and national parameters from the database, which contains different
values for each country. If the output areas and population areas are
not identical, it also splits the population across output areas
based on their overlapping fractions.

\item {} 
Once this step is complete, the \sphinxstylestrong{“available at}:” box will become
populated. This folder contains the disaggregated data needed to run
the model.

\end{enumerate}

\sphinxstylestrong{Tip:} Save time in future: If the exact same input data files are
used in a later study, then the “prepare” step can be skipped: click the
“…” button and navigate to a folder that contains the relevant
disaggregated data. It will then be copied to the new output folder and
used as normal.


\paragraph{Run the model for 1 week}
\label{\detokenize{Tutorials/LQF:run-the-model-for-1-week}}
Choose a start date of 11 May 2015, using the start and end date boxes,
then select “Run”.

\begin{figure}[htbp]
\centering
\capstart

\noindent\sphinxincludegraphics{{LQF_Tutorial_DateRange}.png}
\caption{\sphinxcode{\sphinxupquote{{}`to do{}`}}}\label{\detokenize{Tutorials/LQF:id2}}\end{figure}


\subsubsection{Visualise results}
\label{\detokenize{Tutorials/LQF:visualise-results}}
Once the model run this is finished, press “visualise outputs” to view
some of the model results to open the visualisation tool.

\begin{figure}[htbp]
\centering
\capstart

\noindent\sphinxincludegraphics[width=300\sphinxpxdimen]{{LQF_Tutorial_Visualise}.png}
\caption{\sphinxcode{\sphinxupquote{{}`to do{}`}}}\label{\detokenize{Tutorials/LQF:id3}}\end{figure}


\paragraph{Create a map of total QF at noon}
\label{\detokenize{Tutorials/LQF:create-a-map-of-total-qf-at-noon}}
Use the visualisation tool to create a map of all the QF components at
noon (11:00-12:00 UTC) on May 11 by selecting that time and pressing
“Add to canvas”. This may take a moment to process. Close the
visualisation tool and return to the main canvas to inspect the four new
layers that have appeared.

\begin{figure}[htbp]
\centering
\capstart

\noindent\sphinxincludegraphics{{525px-LQF_Tutorial_QfMap_1}.png}
\caption{\sphinxcode{\sphinxupquote{{}`to do{}`}}}\label{\detokenize{Tutorials/LQF:id4}}\end{figure}

Each layer corresponds to a different QF component, Qm (metabolism) and
is plotted on the top layer. De-selecting a layer in the Layers panel
removes it from view.

Leaving just Qf (total QF) visible, there isn’t much structure in the
colours. Add some contrast to it by choosing a different colour scale.

Right-click the Qf layer, go to Properties \textgreater{} Style, change the colour
ramp to “Reds” and choose Mode: Natural Breaks (Jenks). This shows much
more structure, although the grid borders are distracting. These can be
removed by double-clicking the colour levels and choosing a border
colour the same as the fill colour.

\begin{figure}[htbp]
\centering
\capstart

\noindent\sphinxincludegraphics{{525px-LQF_Tutorial_QfMap_2}.png}
\caption{\sphinxcode{\sphinxupquote{{}`to do{}`}}}\label{\detokenize{Tutorials/LQF:id5}}\end{figure}


\paragraph{Plot a time series of QF in the centre of the city}
\label{\detokenize{Tutorials/LQF:plot-a-time-series-of-qf-in-the-centre-of-the-city}}
A time series can be shown for any of the output areas. To identify one
of interest, zoom into the city centre, choose the selection tool

\begin{figure}[htbp]
\centering
\capstart

\noindent\sphinxincludegraphics{{LQF_Tutorial_SelectFeatureIcon}.png}
\caption{\sphinxcode{\sphinxupquote{{}`to do{}`}}}\label{\detokenize{Tutorials/LQF:id6}}\end{figure}

and click an output area of
interest.

This turns yellow. Right-click it and select the option that comes up.
\begin{quote}

\begin{figure}[htbp]
\centering
\capstart

\noindent\sphinxincludegraphics{{LQF_Tutorial_SelectFeature}.png}
\caption{Information about the output area}\label{\detokenize{Tutorials/LQF:id7}}\end{figure}
\end{quote}

then appears on the left, with the ID shown. Make a note of this.
\begin{quote}

\begin{figure}[htbp]
\centering
\capstart

\noindent\sphinxincludegraphics{{LQF_Tutorial_FeatureInfo}.png}
\caption{\sphinxcode{\sphinxupquote{{}`to do{}`}}}\label{\detokenize{Tutorials/LQF:id8}}\end{figure}
\end{quote}

Return to the visualisation tool, choose output area 5448 and click
“show plot”. Time series of each QF component then appear for the week.
Note the lower traffic activity on Saturday and Sunday, and the higher
building emissions on Thursday 15th when the weather is colder.
\begin{quote}

\begin{figure}[htbp]
\centering
\capstart

\noindent\sphinxincludegraphics{{525px-LQF_Tutorial_Temporal_standardcase}.png}
\caption{\sphinxcode{\sphinxupquote{{}`to do{}`}}}\label{\detokenize{Tutorials/LQF:id9}}\end{figure}
\end{quote}


\paragraph{Recycling of input data}
\label{\detokenize{Tutorials/LQF:recycling-of-input-data}}
Ideally the model is run only for dates covered by the daily temperature
data, but the data is recycled if the model runs beyond the end of the
available temperature data. In this tutorial, only 2015 temperatures are
provided. If the model ran into 2016, a suitable date from the 2015
temperature data would be selected based on the time of year. Except at
the very start or end of the year, the date from 2015 used will be
within a few days of the same date in 2016.


\subsection{Tutorials II: Refining LQF results}
\label{\detokenize{Tutorials/LQF:tutorials-ii-refining-lqf-results}}
Once a basic QF estimate has been made (as in the previous section),
there are several options to refining this using additional data that
may be available.

The following mini-tutorials show how each of these refinements are
applied, and the output of the model is compared to that of the standard
case.


\subsubsection{Tutorial 2a: Custom diurnal profiles}
\label{\detokenize{Tutorials/LQF:tutorial-2a-custom-diurnal-profiles}}
In this scenario, new diurnal profiles for building energy consumption
and road vehicle traffic are available for London. These profiles are
assumed to better represent the city than the default profiles in the
LQF database. In this example, we will re-run LQF using the new
profiles.

Each country in the LQF database is associated with two diurnal profiles
for transport (a weekend and a weekday version), and the same for
building emissions. LQF takes in a week-long profile, starting on
Monday, for transport and buildings (shown below), and different
profiles can be used for different times of year (\sphinxhref{LQF\_Manual\#Diurnal\_variations}{click here for full
information about diurnal profile
files}).
\begin{quote}

\begin{figure}[htbp]
\centering
\capstart

\noindent\sphinxincludegraphics{{525px-LQF_Tutorial_WeeklyTraffic}.png}
\caption{Custom traffic profile}\label{\detokenize{Tutorials/LQF:id10}}\end{figure}
\end{quote}

\begin{figure}[htbp]
\centering
\capstart

\noindent\sphinxincludegraphics{{525px-LQF_Tutorial_BuildingProfiles}.png}
\caption{\sphinxcode{\sphinxupquote{{}`to do{}`}}}\label{\detokenize{Tutorials/LQF:id11}}\end{figure}

\sphinxstylestrong{Step 1:}
Create a duplicate of the DataSources.nml file used earlier

\sphinxstylestrong{Step 2:} create a new folder for the model outputs.

\sphinxstylestrong{Step 3:} Note the paths to the weeklyTraffic.csv and
LondonBuildingProfiles.csv files. These contain new profile data

\sphinxstylestrong{Step 4:} Add these to the \&temporal section of the new DataSources
file using the optional diurnTraffic and diurnEnergy entries. The
section should now resemble this:

\fvset{hllines={, ,}}%
\begin{sphinxVerbatim}[commandchars=\\\{\}]
\PYG{o}{\PYGZam{}}\PYG{n}{temporal}
    \PYG{n}{dailyTemperature} \PYG{o}{=} \PYG{l+s+s1}{\PYGZsq{}}\PYG{l+s+s1}{C:/Some/Path/To/Files/dailyTemperature\PYGZus{}2015.csv}\PYG{l+s+s1}{\PYGZsq{}}
    \PYG{n}{diurnTraffic} \PYG{o}{=} \PYG{l+s+s1}{\PYGZsq{}}\PYG{l+s+s1}{C:/Some/Path/To/Files/weeklyTraffic.csv}\PYG{l+s+s1}{\PYGZsq{}}
    \PYG{n}{diurnEnergy} \PYG{o}{=} \PYG{l+s+s1}{\PYGZsq{}}\PYG{l+s+s1}{C:/Some/Path/To/Files/LondonBuildingProfiles.csv}\PYG{l+s+s1}{\PYGZsq{}}
\PYG{o}{/}
\end{sphinxVerbatim}

\sphinxstylestrong{Step 5:} Re-run LQF for 7 days week, starting 11 May 2015, specifying
the new DataSources file at run time. Visualising the time series for
output area 5448 again:

\begin{figure}[htbp]
\centering

\noindent\sphinxincludegraphics[width=350\sphinxpxdimen]{{525px-LQF_Tutorial_Temporal_customDiurnals}.png}
\end{figure}

Note how the building and transport emission patterns now change on
different days of the week. This is especially noticeable in transport
emissions on the final 3 days of the week.


\subsubsection{Tutorial 2b: Updating national parameters in the LQF database}
\label{\detokenize{Tutorials/LQF:tutorial-2b-updating-national-parameters-in-the-lqf-database}}
LQF takes the latest national attributes (population, vehicle count and
energy consumption) up to and including the year(s) modelled. The copy
of the LQF database used in this tutorial contains national UK
population values in 2010 and 2016. This means the 2010 population is
used when 2015 is simulated. This can be updated or added to if data
becomes available. The database can be edited using software such as
SQLite Browser.

Fictional scenario: The UK population in 2015 was approximate twice that
in 2010, but energy consumption remained the same.

\sphinxstylestrong{Step 1:} Make a copy of the LQF database as a backup

\sphinxstylestrong{Step 2:} Open the LQF database in SQLite browser or other suitable
software

\sphinxstylestrong{Step 3:} Browse the “attributes” table, which contains national
attributes for all countries

\sphinxstylestrong{Step 4:} Locate the population in the UK 2010 row (the value is
62036000.0)

\sphinxstylestrong{Step 5:} Create a new row for the UK in 2015 with the following
entries:


\begin{savenotes}\sphinxattablestart
\centering
\begin{tabular}[t]{|\X{50}{100}|\X{50}{100}|}
\hline
\sphinxstyletheadfamily 
Database column
&\sphinxstyletheadfamily 
Value
\\
\hline
id
&
United Kingdom
\\
\hline
as\_of\_year
&
2015
\\
\hline
population
&
120000000
\\
\hline
population\_datasource
&
Fake value for test
\\
\hline
\end{tabular}
\par
\sphinxattableend\end{savenotes}

\sphinxstylestrong{Step 6:} Run the model as in the first example.

\sphinxstylestrong{Step 7:} Visualise the data for output area 5448. Note how the
building emissions are approximately half of those in the first example,
because the national energy consumption per-capita is now half as much.
The vehicle emissions are the same because they are specified per 10,000
people in the LQF database.

\begin{figure}[htbp]
\centering

\noindent\sphinxincludegraphics[width=350\sphinxpxdimen]{{525px-LQF_Tutorial_Temporal_customdatabase}.png}
\end{figure}

\sphinxstylestrong{Step 8: Restore the original LQF database so that the test values do
not corrupt future modelling studies}

\sphinxstylestrong{Tip:} To check which national values were used at a given time, check
the log folder of the model output directory: NationalParameters.txt
contains a list of the values used for each modelled year and country.
The following example shows the 2014 value of energy consumption
(kwh\_year) being looked up for model runs in 2015.

“DB value for United Kingdom kwh\_year in modelled year 2015: 966862000000.0 (2014 value)”

See \sphinxhref{LQF\_Manual\#Wide-area\_data}{the manual} for a list of the
parameters stored in the LQF database.


\subsubsection{Tutorial 2c: Custom temperature response function}
\label{\detokenize{Tutorials/LQF:tutorial-2c-custom-temperature-response-function}}
Building emissions are governed by a function that relates the daily
mean air temperature to energy consumption. This is a simple treatment
that may not capture the full relationship, so a custom function with
more parameters can also be used in LQF (\sphinxhref{LQF\_Manual\#Temperature\_response\_functions}{full
details}).

This example shows how to activate the custom temperature response
function, and how it affects the results.

The parameters of the custom function are specified using optional
entries in the \sphinxhref{LQF\_Manual\#Parameters\_file}{Parameters file}. In this
example, we will assume that:
\begin{itemize}
\item {} 
The use of energy stops increasing when the temperature exceeds 20C
(weighting is 0.5)

\item {} 
Energy use increases steeply by 0.5 per degree below 20C

\end{itemize}

\sphinxstylestrong{Step 1:} Open the parameters.nml file

\sphinxstylestrong{Step 2:} Copy the optional entries needed for the custom temperature
response from the manual and ensure the values are consistent with those
below

\fvset{hllines={, ,}}%
\begin{sphinxVerbatim}[commandchars=\\\{\}]
\PYG{o}{\PYGZam{}}\PYG{n}{CustomTemperatureResponse}
   \PYG{n}{Th} \PYG{o}{=} \PYG{l+m+mi}{20}
   \PYG{n}{Tc} \PYG{o}{=} \PYG{l+m+mi}{20}
   \PYG{n}{Ah} \PYG{o}{=} \PYG{l+m+mf}{0.5}
   \PYG{n}{Ac} \PYG{o}{=} \PYG{l+m+mf}{0.5}
   \PYG{n}{c} \PYG{o}{=} \PYG{l+m+mf}{0.5}
   \PYG{n}{Tmax} \PYG{o}{=} \PYG{l+m+mi}{50}
   \PYG{n}{Tmin} \PYG{o}{=} \PYG{o}{\PYGZhy{}}\PYG{l+m+mi}{10}
\PYG{o}{/}
\end{sphinxVerbatim}

\sphinxstylestrong{Step 3} : Save the parameters file and run the model as in the
original tutorial. Note how the day-to-day variations in the building
emissions are much greater than in {\hyperref[\detokenize{Tutorials/LQF:LQF_Tutorial_1:_Simple_QF_modelling}]{\emph{Tutorial
1}}}, but the transport and
metabolism emissions remains the same as before.

\begin{figure}[htbp]
\centering
\capstart

\noindent\sphinxincludegraphics[width=350\sphinxpxdimen]{{525px-LQF_Tutorial_Temporal_customResponse}.png}
\caption{\sphinxcode{\sphinxupquote{{}`to do{}`}}}\label{\detokenize{Tutorials/LQF:id12}}\end{figure}


\section{Anthropogenic heat - GQF}
\label{\detokenize{Tutorials/GQF:anthropogenic-heat-gqf}}\label{\detokenize{Tutorials/GQF:gqf}}\label{\detokenize{Tutorials/GQF::doc}}
This tutorial demonstrates how the GQF software is used to simulate
anthropogenic heat fluxes for London, UK, in the year 2015, using a
mixture of administrative and meteorological data.


\subsection{Initial Practical steps}
\label{\detokenize{Tutorials/GQF:initial-practical-steps}}
UMEP is a python plugin used in conjunction with
\sphinxhref{http://www.qgis.org}{QGIS}. To install the software and the UMEP
plugin see the {\hyperref[\detokenize{Getting_Started:getting-started}]{\sphinxcrossref{\DUrole{std,std-ref,std,std-ref}{getting
started}}}}
section in the UMEP manual.

As UMEP is under constant development, some documentation may be missing
and/or there may be instability. Please report any issues or suggestions
to our \sphinxhref{https://bitbucket.org/fredrik\_ucg/umep/}{repository}.


\subsubsection{Data for this exercise}
\label{\detokenize{Tutorials/GQF:data-for-this-exercise}}
In order to proceed, you will need the zip file named
\sphinxhref{https://urban-meteorology-reading.github.io/other\%20files/GQF\_Inputs\_1.zip}{GQF\_Inputs\_1.zip}
from the UMEP tutorials data reopository.

You may also wish to consult the \sphinxhref{GQF\_Manual}{GQF user guide}

The GQF\_Inputs\_1.zip file contains several datasets to cover the
comprehensive requirements of GQF:
\begin{quote}


\begin{savenotes}\sphinxattablestart
\centering
\begin{tabular}[t]{|\X{33}{99}|\X{33}{99}|\X{33}{99}|}
\hline
\sphinxstyletheadfamily 
Filename
&\sphinxstyletheadfamily 
Description
&\sphinxstyletheadfamily 
Notes
\\
\hline
500m\_Residential\_from\_100m.shp
&
Residential population
&
Attribute: “Pop”, Feature ID: “ID”
\\
\hline
500m\_Workday\_from\_100m.shp
&
Workday population
&
Attribute: “Pop”
\\
\hline
MSOA\_elec\_gas\_2014.shp
&
Industrial gas use (annual)
&
Attribute:“GasInd”
\\
\hline
LA\_energy\_2014.shp
&
Industrial electricity use (annual)
&
Attribute:“ElInd\_kWh”
\\
\hline
LSOA\_elec\_gas\_2014.shp
&
Domestic gas and electricity use (annual)
&
Attributes:“GasDom”, “TelDom”
\\
\hline
2015GasElecDD.csv
&
Day-to-day variations in gas and electricity use
&
Year 2015
\\
\hline
BuildingLoadings\_DomUnre.csv
&
Weekend and weekday diurnal profiles of domestic energy use
&
6 seasonal variants
\\
\hline
BuildingLoadings\_Industrial.csv
&
Weekend and weekday diurnal profiles of industrial energy use
&
6 seasonal variants
\\
\hline
BuildingLoadings\_EC7.csv
&
Weekend and weekday diurnal profiles of domestic economy 7 energy use
&
File required for model execution but data not actually used
\\
\hline
LAEI2013\_AADTVKm\_2013\_link.shp
&
Road link map with vehicle flows broken down by fuel and vehicle type
&
Feature ID: “OBJECTID”
\\
\hline
fuelConsumption.csv
&
Fuel efficiency (g/km) of different vehicle classes
&\\
\hline
Transport.csv
&
Weekend and weekday diurnal profiles for each vehicle class
&\\
\hline
Metabolism.csv
&
Weekend and weekday diurnal profiles of metabolic activity
&\\
\hline
Parameters.nml
&
Configuration file containing model run parameters
&\\
\hline
DataSources.nml
&
Configuration file specifying different input data sources for model
&\\
\hline
\end{tabular}
\par
\sphinxattableend\end{savenotes}
\end{quote}


\subsection{GQF Tutorial 1: Comprehensive QF modelling for Greater London}
\label{\detokenize{Tutorials/GQF:gqf-tutorial-1-comprehensive-qf-modelling-for-greater-london}}

\subsubsection{Preparing data}
\label{\detokenize{Tutorials/GQF:preparing-data}}

\paragraph{Manage input data files}
\label{\detokenize{Tutorials/GQF:manage-input-data-files}}\begin{itemize}
\item {} 
Extract the contents of GQF\_Inputs\_1.zip into a folder on your
local machine and note the path to each file (e.g.
C:\textbackslash{}GQFData\textbackslash{}BuildingLoadings\_Industrial.csv)

\end{itemize}


\paragraph{Gather information about shapefile inputs}
\label{\detokenize{Tutorials/GQF:gather-information-about-shapefile-inputs}}
GQF uses multiple shapefiles (ending .shp) to build up a picture of
energy use, population and road transport across the city. Five pieces
of information are needed for each of these:
\begin{enumerate}
\item {} 
Filename: The full path to the .shp file (e.g.
c:\textbackslash{}path\textbackslash{}to\textbackslash{}file.shp)

\item {} 
Start date: The modelled date from which this data should be used

\item {} 
EPSG code: A numeric code that determines which co-ordinate reference
system (CRS) to use for the shapefile

\item {} 
Attribute to use. A shapefile attaches one or more attributes (e.g.
population or energy consumption) to each spatial unit. The name of
the relevant attribute must be specified here.

\item {} 
Feature IDs: The name of an attribute that contains a unique
identifier for each spatial unit

\end{enumerate}

\sphinxhref{LQF\_Manual\#Appendix\_B:\_Gathering\_information\_about\_shapefiles\_for\_QF\_modelling}{Click
here}
for a guide explaining how to identify the feature ID, attribute to use
and EPSG code of a shapefile using QGIS.

A shapefile also defines the so-called output areas, which are the
spatial units (sometimes pixels) of model output (one QF estimate per
area). These are needed because the spatial units of the various input
files may not all match up. The output areas can either be one of the
input files, or a totally different set of areas. In this tutorial, one
of the population datasets is used to keep things simple.


\subsubsection{Set up the DataSources.nml file}
\label{\detokenize{Tutorials/GQF:set-up-the-datasources-nml-file}}
The data sources file needs to be updated so that it can find the
various data files, and understands what to do with them. A full
description of the Data sources file contents is available
\sphinxhref{LQF\_Manual\#Data\_sources\_file}{here}, but this section shows how to
build up the entries.

There are several sections in the data sources file. Each is bounded by
\sphinxstylestrong{\&section\_name} and ends with \sphinxstylestrong{“/”} and deals with a different part
of the input data.
\begin{enumerate}
\item {} 
Open DataSources.nml using a text editor (we recommend Notepad++).

\item {} 
The following steps show how to update the entries according to the
information gathered above.

\end{enumerate}


\paragraph{Add shapefile information}
\label{\detokenize{Tutorials/GQF:add-shapefile-information}}\begin{enumerate}
\item {} 
The “\sphinxstylestrong{shapefile}” entry is the path to the file

\item {} 
The \sphinxstylestrong{epsgCode} and \sphinxstylestrong{featureIds} entries are found by inspecting
each file using QGIS. Note that each file has different values for
these

\item {} 
The attribToUse entry for each file is covered in the table above

\item {} 
An arbitrary start date of 2011-01-01 (1st january) can be used for
the data shown.

\end{enumerate}

For brevity, just the first two sections of the DataSources.nml file are
shown here: Using the workday population spatial units as model output
areas. This section does not need to use an attribute or know about a
start date:

\fvset{hllines={, ,}}%
\begin{sphinxVerbatim}[commandchars=\\\{\}]
\PYG{o}{\PYGZam{}}\PYG{n}{outputAreas}
     \PYG{n}{shapefile} \PYG{o}{=} \PYG{l+s+s1}{\PYGZsq{}}\PYG{l+s+s1}{C:}\PYG{l+s+s1}{\PYGZbs{}}\PYG{l+s+s1}{path}\PYG{l+s+se}{\PYGZbs{}t}\PYG{l+s+s1}{o}\PYG{l+s+s1}{\PYGZbs{}}\PYG{l+s+s1}{data}\PYG{l+s+se}{\PYGZbs{}500}\PYG{l+s+s1}{m\PYGZus{}Workday\PYGZus{}from\PYGZus{}100m.shp}\PYG{l+s+s1}{\PYGZsq{}}
     \PYG{n}{epsgCode} \PYG{o}{=} \PYG{l+m+mi}{32631}
     \PYG{n}{featureIds} \PYG{o}{=} \PYG{l+s+s1}{\PYGZsq{}}\PYG{l+s+s1}{ID}\PYG{l+s+s1}{\PYGZsq{}}
\PYG{o}{/}
\end{sphinxVerbatim}

Specifying the residential population data.

\fvset{hllines={, ,}}%
\begin{sphinxVerbatim}[commandchars=\\\{\}]
! \PYGZsh{}\PYGZsh{}\PYGZsh{} Population data
\PYGZam{}residentialPop
   shapefiles = \PYGZsq{}C:\PYGZbs{}path\PYGZbs{}to\PYGZbs{}data\PYGZbs{}500m\PYGZus{}Residential\PYGZus{}from\PYGZus{}100m.shp\PYGZsq{}
   startDates = \PYGZsq{}2011\PYGZhy{}01\PYGZhy{}01\PYGZsq{}
   attribToUse = \PYGZsq{}Pop\PYGZsq{}
   featureIds = \PYGZsq{}ID\PYGZsq{}
/
\end{sphinxVerbatim}

The same pattern is used for the other spatial input datasets:
\begin{itemize}
\item {} 
\sphinxstylestrong{workplacePop}: Workplace/workday population dataset

\item {} 
\sphinxstylestrong{annualIndGas}: Industrial gas use

\item {} 
\sphinxstylestrong{annualIndElec}: Industrial electricity use

\item {} 
\sphinxstylestrong{annualDomGas}: Domestic gas use

\item {} 
\sphinxstylestrong{annualDomElec}: Domestic electricity use (same file as domestic
gas, but different attribute)

\end{itemize}

For the \sphinxstylestrong{annualEco7} section, we shall assume zero consumption. This
doesn’t need a shapefile - a single number indicating the whole-city
consumption should be used instead, along with dummy EPSG code,
attribToUse and featureIds:

\fvset{hllines={, ,}}%
\begin{sphinxVerbatim}[commandchars=\\\{\}]
\PYGZam{}annualEco7
   ! Spatial variations of economy 7 electricity use
   shapefiles = 0.0
   startDates = \PYGZsq{}2014\PYGZhy{}01\PYGZhy{}01\PYGZsq{}
   epsgCodes = 1
   attribToUse = \PYGZsq{}IndGas\PYGZsq{} !A dummy name
   featureIds = \PYGZsq{}\PYGZsq{}
/
\end{sphinxVerbatim}


\paragraph{Add temporal data files}
\label{\detokenize{Tutorials/GQF:add-temporal-data-files}}

\subparagraph{Day-to-day energy demand changes}
\label{\detokenize{Tutorials/GQF:day-to-day-energy-demand-changes}}
GQF uses annual total energy consumption shapefiles, and needs to know
how to vary energy consumption on different dates (e.g. winter is likely
to have more fuel use than summer). This is captured using real data
from the energy grid. The 2015GasElecDD.csv file contains each day’s
total gas and electricity consumption. GQF then scales the annual
consumption based on this each day.

\fvset{hllines={, ,}}%
\begin{sphinxVerbatim}[commandchars=\\\{\}]
\PYG{o}{\PYGZam{}}\PYG{n}{dailyEnergyUse}
   \PYG{n}{Daily} \PYG{n}{variations} \PYG{o+ow}{in} \PYG{n}{total} \PYG{n}{power} \PYG{n}{use}
   \PYG{n}{profileFiles} \PYG{o}{=} \PYG{l+s+s1}{\PYGZsq{}}\PYG{l+s+s1}{C:}\PYG{l+s+s1}{\PYGZbs{}}\PYG{l+s+s1}{Path}\PYG{l+s+s1}{\PYGZbs{}}\PYG{l+s+s1}{To}\PYG{l+s+se}{\PYGZbs{}201}\PYG{l+s+s1}{5GasElecDD.csv}\PYG{l+s+s1}{\PYGZsq{}}
\PYG{o}{/}
\end{sphinxVerbatim}

Only the year(s) represented by the data should be modelled, but if only
past years are available GQF will recycle it for later years, offering
the closest sensible match to time of week and time of year.


\subparagraph{Metabolism file}
\label{\detokenize{Tutorials/GQF:metabolism-file}}
The metabolism file controls:
\begin{itemize}
\item {} 
How much energy each the average person emits at each time of day

\item {} 
The fraction of an area’s workday population actually at work (and by
extension the fraction of the residential population at home)

\end{itemize}

The ‘metabolism.csv’ file contains a weekday, saturday and sunday
variant of this information, and copies for each daylight savings regime
in the UK to account for changes in the summer.

\fvset{hllines={, ,}}%
\begin{sphinxVerbatim}[commandchars=\\\{\}]
! Temporal metabolism data
\PYGZam{}diurnalMetabolism
    profileFiles = \PYGZsq{}N:\PYGZbs{}QF\PYGZus{}London\PYGZbs{}GreaterQF\PYGZus{}input\PYGZbs{}London\PYGZbs{}Profiles\PYGZbs{}\PYGZbs{}Metabolism.csv\PYGZsq{}
/
\end{sphinxVerbatim}


\subparagraph{Building diurnal profiles}
\label{\detokenize{Tutorials/GQF:building-diurnal-profiles}}
As shown above, the different kinds of building energy consumption are
separated in GQF. Their diurnal profiles are also different so that the
different behaviours of households and businesses are represented
accurately. This means that each of the building energy inputs also
requires a diurnal profile data file:

\fvset{hllines={, ,}}%
\begin{sphinxVerbatim}[commandchars=\\\{\}]
\PYGZam{}diurnalDomElec
   ! Diurnal variations in total domestic electricity use (metadata provided in file; files can contain multiple seasons)
   profileFiles =
\PYGZsq{}C:\PYGZbs{}Path\PYGZbs{}To\PYGZbs{}BuildingLoadings\PYGZus{}DomUnre.csv\PYGZsq{}
/
\PYGZam{}diurnalDomGas
   ! Diurnal variations in total domestic gas use (metadata provided in file; files can contain multiple seasons)
   profileFiles = \PYGZsq{}C:\PYGZbs{}Path\PYGZbs{}To\PYGZbs{}BuildingLoadings\PYGZus{}DomUnre.csv\PYGZsq{}
/
\PYGZam{}diurnalIndElec
   ! Diurnal variations in total industrial electricity use (metadata provided in file; files can contain multiple seasons)
   profileFiles = \PYGZsq{}C:\PYGZbs{}Path\PYGZbs{}To\PYGZbs{}BuildingLoadings\PYGZus{}Industrial.csv\PYGZsq{}
/
\PYGZam{}diurnalIndGas
   ! Diurnal variations in total industrial gas use (metadata provided in file; files can contain multiple seasons)
   profileFiles = \PYGZsq{}C:\PYGZbs{}Path\PYGZbs{}To\PYGZbs{}BuildingLoadings\PYGZus{}Industrial.csv\PYGZsq{}
/
\PYGZam{}diurnalEco7
   ! Diurnal variations in total economy 7 electricity use (metadata provided in file; files can contain multiple seasons)
   profileFiles = \PYGZsq{}C:\PYGZbs{}Path\PYGZbs{}To\PYGZbs{}BuildingLoadings\PYGZus{}EC7.csv\PYGZsq{}
/
\end{sphinxVerbatim}


\paragraph{Add information about transport}
\label{\detokenize{Tutorials/GQF:add-information-about-transport}}
The transport input data files are very detailed and therefore needs a
lot of descriptive information in the \&transportData section of
DataSouces.nml


\subparagraph{Shapefile}
\label{\detokenize{Tutorials/GQF:shapefile}}
To save time, the DataSources file is mostly completed in advance with
entries that reflect the transport shapefile, but some of the key
entries still need completing as part of the tutorial:
\begin{itemize}
\item {} 
The location, EPSG code, feature ID and start date of the road
transport shapefile

\item {} 
Information about what is available in the shapefile

\end{itemize}

It should be possible to complete and/or verify the first four entries
using the table and information above.

The next three entries should be all be set to 1 to signify that they
are provided by the shapefile
\begin{itemize}
\item {} 
speed\_available: vehicle speed provided for each road link

\item {} 
total\_AADT\_available: annual average daily traffic (traffic flow)
provided for each road link

\item {} 
vehicle\_AADT available: AADT is broken down by vehicle type for each
road link

\end{itemize}

\fvset{hllines={, ,}}%
\begin{sphinxVerbatim}[commandchars=\\\{\}]
\PYGZam{}transportData
 ! Vector data containing all road segments for study area
 shapefiles = \PYGZsq{}C:\PYGZbs{}path\PYGZbs{}to\PYGZbs{}data\PYGZbs{}LAEI2013\PYGZus{}AADTVKm\PYGZus{}2013\PYGZus{}link.shp\PYGZsq{}
 startDates = \PYGZsq{}2008\PYGZhy{}01\PYGZhy{}01\PYGZsq{}
 epsgCodes = 27700
 featureIds = \PYGZsq{}OBJECTID\PYGZsq{}
 ! What data is available for each road segment in this shapefile? 1 = Yes; 0 = No
 speed\PYGZus{}available = 1                 ! Speed data. If not available then default values from parameters file are used
 total\PYGZus{}AADT\PYGZus{}available = 1            ! Total annual average daily total (AADT: total vehicles passing over each segment each day)
 vehicle\PYGZus{}AADT\PYGZus{}available = 1          ! AADT available for specific vehicle types
\end{sphinxVerbatim}

The rest of the section tells GQF which attributes to use for various
aspects of the traffic data, and what different kinds of roads are
called:

\fvset{hllines={, ,}}%
\begin{sphinxVerbatim}[commandchars=\\\{\}]
    ! Road classification information. This is used with assumed values for AADT
    class\PYGZus{}field = \PYGZsq{}DESC\PYGZus{}TERM\PYGZsq{}           ! The shapefile attribute that contains road classification
    ! Strings that identify each class of road
    motorway\PYGZus{}class = \PYGZsq{}Motorway\PYGZsq{}
    primary\PYGZus{}class = \PYGZsq{}A Road\PYGZsq{}
    secondary\PYGZus{}class = \PYGZsq{}B Road\PYGZsq{}
    ! All other road types will be considered as \PYGZbs{} “other”
    ! Average speed for each road segment
    speed\PYGZus{}field = \PYGZsq{}Speed\PYGZus{}kph\PYGZsq{}           ! Field name
    speed\PYGZus{}multiplier = 1.0              ! Factor that converts data to km/h (1.0 if data is already in km/h)
    ! Annual average daily total (mean number of vehicles per day) passing over each road segment in the shapefile
    ! Specify attribute names if data is present in the shapefile.
    AADT\PYGZus{}total = \PYGZsq{}AADTTOTAL\PYGZsq{}            ! Total AADT for all vehicles. Leave blank (\PYGZsq{}\PYGZsq{}) if not available
    ! AADT for cars of different fuels (leave as \PYGZsq{}\PYGZsq{} if not available)
    AADT\PYGZus{}diesel\PYGZus{}car = \PYGZsq{}AADTDcar\PYGZsq{}        ! Petrol cars
    AADT\PYGZus{}petrol\PYGZus{}car = \PYGZsq{}AADTPcar\PYGZsq{}        ! Diesel cars
    ! Secondary option: Use total AADT for cars and break down using assumed fuel fractions from model parameters file
    AADT\PYGZus{}total\PYGZus{}car = \PYGZsq{}\PYGZsq{}               ! Total AADT for all cars (required if the other car fields are \PYGZsq{}\PYGZsq{}; ignored if they are specified)
    ! AADT for LGVs of different fuels leave as \PYGZsq{}\PYGZsq{} if not available)
    AADT\PYGZus{}diesel\PYGZus{}LGV = \PYGZsq{}AADTDLgv\PYGZsq{}        ! Petrol LGVs
    AADT\PYGZus{}petrol\PYGZus{}LGV = \PYGZsq{}AADTPLgv\PYGZsq{}        ! Diesel LGVs
    ! Secondary option: Use total LGV AADT and assumed fuel fractions from parameters file
    AADT\PYGZus{}total\PYGZus{}LGV = \PYGZsq{}\PYGZsq{}               ! Total AADT for all LGVs (required if the other LGV fields are \PYGZsq{}\PYGZsq{}; ignored if they are specified)
    ! AADT for other vehicles. These are broken down into diesel/petrol based on fuel fractions (see model parameters file)
    ! Specify shapefile attribute name or leave as \PYGZsq{}\PYGZsq{} if not available
    AADT\PYGZus{}motorcycle = \PYGZsq{}AADTMotorc\PYGZsq{}      ! Motorcycles
    AADT\PYGZus{}taxi = \PYGZsq{}AADTTaxi\PYGZsq{}              ! Taxis
    AADT\PYGZus{}bus = \PYGZsq{}AADTLtBus\PYGZsq{}                  ! Buses
    AADT\PYGZus{}coach = \PYGZsq{}AADTCoach\PYGZsq{}                ! Coaches
    AADT\PYGZus{}rigid = \PYGZsq{}AADTRigid\PYGZsq{}                ! Rigid goods vehicles
    AADT\PYGZus{}artic = \PYGZsq{}AADTArtic\PYGZsq{}                ! Articulated trucks
/
\end{sphinxVerbatim}


\subparagraph{Vehicle fuel efficiency data}
\label{\detokenize{Tutorials/GQF:vehicle-fuel-efficiency-data}}
The fuelConsumption.csv file contains a list of vehicle fuel efficiency
by fuel, vehicle type and era. This is used to calculate each road
link’s fuel consumption:

\fvset{hllines={, ,}}%
\begin{sphinxVerbatim}[commandchars=\\\{\}]
\PYGZam{}fuelConsumption
   ! File containing fuel consumption performance data for each vehicle type as standards change over the years
   profileFiles = \PYGZsq{}C:\PYGZbs{}Path\PYGZbs{}To\PYGZbs{}fuelConsumption.csv\PYGZsq{}
/
\end{sphinxVerbatim}


\subparagraph{Diurnal profiles by vehicle type}
\label{\detokenize{Tutorials/GQF:diurnal-profiles-by-vehicle-type}}
Each vehicle type has a different activity profile. For example, freight
and taxi vehicle may operate later at night than passenger cars. The
Transport.csv file contains a profile for each of these:

\fvset{hllines={, ,}}%
\begin{sphinxVerbatim}[commandchars=\\\{\}]
\PYGZam{}diurnalTraffic
   ! Diurnal cycles of transport flow for different vehicle types
   profileFiles = \PYGZsq{}C:\PYGZbs{}Path\PYGZbs{}To\PYGZbs{}Transport.csv\PYGZsq{}
/
\end{sphinxVerbatim}

Each profile is a week long, and these profiles control changes to the
total volume of traffic each day.


\subsubsection{Run GQF}
\label{\detokenize{Tutorials/GQF:run-gqf}}
Under UMEP \textgreater{} Processor \textgreater{} Urban Energy Balance, choose GQf (GreateRQF)
\begin{description}
\item[{This loads the model interface dialog box:}] \leavevmode
\begin{figure}[htbp]
\centering
\capstart

\noindent\sphinxincludegraphics{{Gqf_dialog}.png}
\caption{\sphinxcode{\sphinxupquote{{}`to do{}`}}}\label{\detokenize{Tutorials/GQF:id1}}\end{figure}

\end{description}


\paragraph{Choose configuration files and output folder}
\label{\detokenize{Tutorials/GQF:choose-configuration-files-and-output-folder}}
Working from the top of the dialog box to the bottom…
\begin{enumerate}
\item {} 
Click the … buttons in the “Configuration and raw input data” panel
to browse to the parameters.nml and DataSources.nml files. A pop-up
error message will warn of any problems inside the files.

\item {} 
\sphinxstylestrong{Output path:} A folder in which the model outputs will be stored.
It is \sphinxstylestrong{strongly recommended} that a new folder is used each time.

\item {} 
Click \sphinxstylestrong{Prepare input data using Data Sources} button. This may be a
time-consuming step: It matches the various inputs to each output
area. Where output areas and input shapes are not identical, it also
splits population or energy use across output areas based on their
overlapping fractions.

\item {} 
Once this step is complete, the \sphinxstylestrong{“available at}:” box will become
populated. This folder contains the disaggregated data needed to run
the model.

\end{enumerate}

\sphinxstylestrong{Tip:} Save time in future: If the exact same input data files are
used in a later study, then the “prepare” step can be skipped: click the
“…” button and navigate to a folder that contains the relevant
disaggregated data. It will then be copied to the new output folder and
used as normal.


\paragraph{Run the model for 1 week}
\label{\detokenize{Tutorials/GQF:run-the-model-for-1-week}}
Choose a start date of 11 May 2015, using the start and end date boxes,
then select “Run”.
\begin{quote}

\begin{figure}[htbp]
\centering
\capstart

\noindent\sphinxincludegraphics{{Gqf_timerange}.png}
\caption{\sphinxcode{\sphinxupquote{{}`to do{}`}}}\label{\detokenize{Tutorials/GQF:id2}}\end{figure}
\end{quote}


\subsubsection{Visualise results}
\label{\detokenize{Tutorials/GQF:visualise-results}}
Once the model run this is finished, press “visualise outputs” to view
some of the model results to open the visualisation tool.


\paragraph{Create emissions maps at noon}
\label{\detokenize{Tutorials/GQF:create-emissions-maps-at-noon}}
Use the visualisation tool to create a map of all the QF components at
noon (11:00-12:00 UTC) on May 11 by selecting that time and pressing
“Add to canvas”. This may take a moment to process. Close the
visualisation took and return to the main canvas to inspect the four new
layers that have appeared.

Each layer corresponds to a different QF component:
\begin{itemize}
\item {} 
\sphinxstylestrong{Metab}: Metabolism

\item {} 
\sphinxstylestrong{TransTot}: Total from all road transport sources

\item {} 
\sphinxstylestrong{AllTot}: Total QF from all emissions

\item {} 
\sphinxstylestrong{BldTot}: Total building emissions

\end{itemize}

De-selecting a layer in the Layers panel removes it from view.

Leaving just AllTot(total QF) visible, there isn’t much structure in the
colours.
\begin{quote}

\begin{figure}[htbp]
\centering
\capstart

\noindent\sphinxincludegraphics{{525px-Gqf_totalqf_map}.png}
\caption{Total QF at Noon on May 11}\label{\detokenize{Tutorials/GQF:id3}}\end{figure}
\end{quote}

Add some contrast to it by choosing a different colour scale:

Right-click the Qf layer, go to Properties \textgreater{} Style, change the colour
ramp to “Reds” and choose Mode: Natural Breaks (Jenks). This shows much
more structure, although the grid borders are distracting. These can be
removed by double-clicking the colour levels and choosing a border
colour the same as the fill colour.
\begin{quote}

\begin{figure}[htbp]
\centering
\capstart

\noindent\sphinxincludegraphics{{525px-Gqf_totalqf_map_recoloured}.png}
\caption{Total QF at Noon on May 11}\label{\detokenize{Tutorials/GQF:id4}}\end{figure}
\end{quote}

The roads have a very different spatial pattern to buildings, so these
can also be visualised by selecting the TransTot layer and re-colouring
accordingly:
\begin{quote}

\begin{figure}[htbp]
\centering
\capstart

\noindent\sphinxincludegraphics{{525px-Gqf_transportqf_map_recoloured}.png}
\caption{GQF Transport QF at 1200 UTC}\label{\detokenize{Tutorials/GQF:id5}}\end{figure}
\end{quote}


\paragraph{Plot a time series of QF in the centre of the city}
\label{\detokenize{Tutorials/GQF:plot-a-time-series-of-qf-in-the-centre-of-the-city}}
A time series can be shown for any of the output areas. To identify one
of interest, zoom into the city centre, choose the selection tool
\begin{quote}

\begin{figure}[htbp]
\centering
\capstart

\noindent\sphinxincludegraphics{{LQF_Tutorial_SelectFeatureIcon}.png}
\caption{\sphinxcode{\sphinxupquote{{}`to do{}`}}}\label{\detokenize{Tutorials/GQF:id6}}\end{figure}
\end{quote}

and click an output area of
interest.
\begin{description}
\item[{This turns yellow. Right-click it and select the option that comes up.}] \leavevmode
\begin{figure}[htbp]
\centering
\capstart

\noindent\sphinxincludegraphics{{LQF_Tutorial_SelectFeature}.png}
\caption{\sphinxcode{\sphinxupquote{{}`to do{}`}}}\label{\detokenize{Tutorials/GQF:id7}}\end{figure}

\end{description}

Information about the output area
then appears on the left, with the ID shown. Make a note of this.
\begin{quote}

\begin{figure}[htbp]
\centering
\capstart

\noindent\sphinxincludegraphics{{LQF_Tutorial_FeatureInfo}.png}
\caption{\sphinxcode{\sphinxupquote{{}`to do{}`}}}\label{\detokenize{Tutorials/GQF:id8}}\end{figure}
\end{quote}

Return to the visualisation tool, choose output area 5448 and click
“show plot”. Time series of each QF component then appear for the week.
Note the lower traffic activity and different behaviours on Saturday and
Sunday, when people are expected to not be at work.
\begin{quote}

\begin{figure}[htbp]
\centering

\noindent\sphinxincludegraphics{{600px-Gqf_timeseries_default}.png}
\end{figure}
\end{quote}


\subsection{Tutorials 2: Refining GQF results}
\label{\detokenize{Tutorials/GQF:tutorials-2-refining-gqf-results}}
There are several extra options in GQF. The following mini-tutorials
show how they are used:


\subsubsection{Tutorial 2a: Add a public holiday}
\label{\detokenize{Tutorials/GQF:tutorial-2a-add-a-public-holiday}}
The parameters.nml file contains three entries related to public
holidays, which are treated as the second day of the weekend by GQF:
\begin{itemize}
\item {} 
Use\_UK\_holidays: Religious and recurrent public holidays from the
UK are calculated automatically

\item {} 
Use\_custom\_holidays: Set to 1 in order to have GQF read in a list
of user-provided holidays

\item {} 
custom\_holidays: A comma-separated list of dates that should be
treated as holidays in format YYYY-mm-dd (e.g. “2015-05-07”,
“2015-07-30”)

\end{itemize}

In this example, a fictional public holiday of 2015-05-13 is entered
into the parameters.nml file. The model is then run as in Tutorial 1,
and the resulting time series in output area 5448 is shown below:
\begin{quote}

\begin{figure}[htbp]
\centering
\capstart

\noindent\sphinxincludegraphics{{600px-Gqf_timeseries_default}.png}
\caption{Time series with extra public holiday on May 13}\label{\detokenize{Tutorials/GQF:id9}}\end{figure}
\end{quote}

Compared against the results from Tutorial 1, the curve on May 13 in
each sub-plot now resembles May 17 (a Sunday) rather than the weekdays
around it.


\subsubsection{Tutorial 2b: Remove the sensible heat component}
\label{\detokenize{Tutorials/GQF:tutorial-2b-remove-the-sensible-heat-component}}
Anthropogenic heat is made up of three parts:
\begin{itemize}
\item {} 
Sensible: Transported by convection (usually the largest share)

\item {} 
Latent: Transported by the vaporisation of water

\item {} 
Wastewater: Heat in water ejected by buildings

\end{itemize}

GQF includes all of these in the calculated fluxes by default, but one
or more of them can be removed at model run-time using the checkboxes:

In this example, the week of 11 to 18 May 2015 is again modelled but the
“Sensible” and “Wastewater” checkboxes are un-ticked. This means the
modelled QF will contain only latent heat. The resulting time series in
area 5448 is shown below:
\begin{quote}

\begin{figure}[htbp]
\centering
\capstart

\noindent\sphinxincludegraphics{{525px-Gqf_timeseries_holiday_nosensible}.png}
\caption{Time series with only latent and wastewater contributions included, and extra public holiday on May 13}\label{\detokenize{Tutorials/GQF:id10}}\end{figure}
\end{quote}

The emissions are far lower than those in Tutorial 2a, showing how
latent heat is a relatively small contribution. Consuming electricity
emits no latent heat, unlike gas, while metabolism now represents a
larger fraction of the total.


\chapter{Other Manuals}
\label{\detokenize{OtherManuals/OtherManuals:other-manuals}}\label{\detokenize{OtherManuals/OtherManuals:othermanuals}}\label{\detokenize{OtherManuals/OtherManuals::doc}}
This section include individual and more detailed manuals for some of the models used UMEP.


\section{GQF Manual}
\label{\detokenize{OtherManuals/GQF_Manual:gqf-manual}}\label{\detokenize{OtherManuals/GQF_Manual::doc}}

\subsection{Overview}
\label{\detokenize{OtherManuals/GQF_Manual:overview}}
GEF provides a method to calculate the anthropogenic heat flux. It uses
energy consumption, traffic and population data recorded within a city
to produce estimates of the anthropogenic heat flux from buildings,
transport and human metabolism at 30 minute intervals, using the highest
possible spatial scale.
\begin{itemize}
\item {} 
Spatial resolution is maximised by attributing annual industrial and
domestic energy consumption (available at coarse spatial scales) to
finer scales based on working and residential populations

\item {} 
The best available traffic and road network maps are used to
attribute traffic, and therefore the heat released by burning fuel,
to each spatial unit. This can be further increased based on
high-resolution land cover fraction data.

\item {} 
Temporal resolution is maximised by applying empirically measured
diurnal, day-of-week and seasonal variations to the data.

\item {} 
Latent, sensible and/or wastewater components of Q$_{\text{F}}$ can be
calculated.

\end{itemize}


\subsubsection{Workflow to model Q$_{\text{F}}$}
\label{\detokenize{OtherManuals/GQF_Manual:workflow-to-model-qf}}\begin{enumerate}
\item {} 
Select parameters and data sources files

\item {} 
Select output path: This contains model outputs, logs and any
pre-processed data that is produced

\item {} 
Perform pre-processing of the data or select existing pre-processed
data: This is a time-consuming step but need only to be performed
once for a set of input data.

\item {} 
Optionally: Specify land cover fractions at high spatial resolution:
Allows the spatial resolution of the modelled outputs to be enhanced

\item {} 
Run the model: Executes the model for the chosen date range and
Q$_{\text{F}}$ components.

\item {} 
Visualise outputs: A simple tool is provided to generate maps and
time series from the model outputs.

\end{enumerate}


\subsection{Main user interface}
\label{\detokenize{OtherManuals/GQF_Manual:main-user-interface}}
The main user interface allows the user to select the temporal extent
and configuration files for the model. Since the model contains many
configuration options and parameters, these are stored in two files that
must be managed by the user and are chosen at run time.

\begin{figure}[htbp]
\centering
\capstart

\noindent\sphinxincludegraphics{{300px-GQF_main}.png}
\caption{GQF main dialogue box}\label{\detokenize{OtherManuals/GQF_Manual:id4}}\end{figure}
\begin{itemize}
\item {} 
\sphinxstylestrong{Setting up and running GQF}
\begin{quote}
\begin{itemize}
\item {} 
GQF requires spatial and temporal information describing the population, energy consumption and transport in the study area. Before Q$_{\text{F}}$ can be calculated for each part of the study area, the energy use and road network data must be disaggregated to match the chosen output areas. They are then temporally disaggregated to 30 minute resolution based on template diurnal cycles and scaling data that reflects the time of year. At the end of this process, the data are ready for use in Q$_{\text{F}}$ calculations.    GQF main dialogue box

\end{itemize}
\begin{enumerate}
\item {} 
Specify model configuration files and output path:
\begin{itemize}
\item {} 
GQF needs configuration files that specify the spatial and temporal information to model Q$_{\text{F}}$:

\item {} 
Model {\hyperref[\detokenize{OtherManuals/GQF_Manual:id3}]{\sphinxcrossref{Parameters\_file}}}: Fortran-90 namelist file containing numerical parameters required in model calculations

\item {} 
{\hyperref[\detokenize{OtherManuals/GQF_Manual:data-sources-file}]{\sphinxcrossref{Data sources file}}}: Fortran-90 namelist file that contains the locations of spatial and temporal input files used by the model

\item {} 
\sphinxstylestrong{Output Path}: Directory into which {\hyperref[\detokenize{OtherManuals/GQF_Manual:model-outputs}]{\sphinxcrossref{Model outputs}}} and associated data will be stored. \sphinxstyleemphasis{Any existing files will be overwritten.}

\end{itemize}

\item {} 
Process input data
\begin{itemize}
\item {} 
This step disaggregates the input data specified by the Data Sources file so that they all use the same spatial units.

\item {} 
The disaggregated data files are saved in the /DownscaledData/ subfolder of the chosen model output directory and can be inspected if required. This step can take up to several hours for large grids (thousands of cells), and QGIS will not respond to input while this process is going on.

\item {} 
If processed input data already exists elsewhere it can be used instead by specifying the path using the \sphinxstylestrong{Available at:} box. The processed files are copied to the /DownscaledData/ subfolder of the chosen model output directory. This removes the need for repeated disaggregation of the same data.

\end{itemize}

\item {} 
Choose temporal domain:
\begin{itemize}
\item {} 
\sphinxstylestrong{Dates to model} (outputs are produced at 60-minute intervals). Either:

\item {} 
\sphinxstylestrong{Date range}: The first and final dates are specified and the whole period is simulated.

\item {} 
\sphinxstylestrong{Date list}: A comma-separated list of dates in YYYY-mm-dd format (e.g. 2015-01-02, 2016-03-05, 2014-05-03) is provided. These dates are simulated in their entirety.

\end{itemize}

\item {} 
Run model and visualise results:
\begin{itemize}
\item {} 
The \sphinxstylestrong{Run Model} button executes the model, which applies the temporal disaggregations and calculates Q$_{\text{F}}$ components in each output area. This takes up to several hours for high resolution or long study periods. During this time QGIS will not respond to input.

\item {} 
Results are visualised using the \sphinxstylestrong{Visualise…} button

\item {} 
Previous model results are retrieved using the \sphinxstylestrong{Load Results} button, which allows a previous model output folder to be selected.

\end{itemize}

\end{enumerate}
\end{quote}

\item {} 
\sphinxstylestrong{Visualising output} A simple visualisation tool accompanies the model, which produces maps and time series plots of the most recent run by default.
\begin{quote}
\begin{itemize}
\item {} 
The results of previous runs can also be visualised without re-running the model: Select the relevant output directory and Data Sources file are chosen in the GQF UI before pressing the “Visualise” button.    GQF results visualisation dialogue box

\end{itemize}

\begin{figure}[htbp]
\centering
\capstart

\noindent\sphinxincludegraphics{{300px-Visualise}.png}
\caption{GQF results visualisation dialogue box}\label{\detokenize{OtherManuals/GQF_Manual:id5}}\end{figure}
\end{quote}

\item {} 
\sphinxstylestrong{Time series plots}
\begin{itemize}
\item {} 
One plot per output area is produced for all of the time steps present in the model output directory, showing the three Q$_{\text{F}}$ components on separate axes. To plot a time series, select the output area of interest and click the “Show” button.

\end{itemize}

\item {} 
\sphinxstylestrong{Maps}
\begin{itemize}
\item {} 
One map per Q$_{\text{F}}$ component and time step is produced, coloured on a logarithmic scale according to the Q$_{\text{F}}$ value in each output area. The map is updated in the main QGIS window each time a different Q$_{\text{F}}$ component or time step is selected.

\end{itemize}

\end{itemize}


\subsection{Model outputs}
\label{\detokenize{OtherManuals/GQF_Manual:model-outputs}}
Model outputs are stored in the /ModelOutput/ subdirectory of the
selected model output directory. A separate data file is produced for
each time step of the model run. Each file contains a column per heat
flux component and a row for each spatial feature.
\begin{itemize}
\item {} 
Output files are timestamped with the
pattern\sphinxstylestrong{GQFYYYYmmdd\_HH-MM}.csv, with times stated in UTC.
\begin{itemize}
\item {} 
YYYY: 4-digit year

\item {} 
mm: 2-digit month

\item {} 
dd: 2-digit day of month

\item {} 
HH: 2-digit hour (00 to 23)

\item {} 
MM: 2-digit minute

\end{itemize}

\item {} 
The first model output is labelled 00:30 UTC and covers the period
00:00-00:30 UTC.

\item {} 
Each data file is in comma-separated value (CSV) format

\end{itemize}


\subsection{Synthesised shapefiles}
\label{\detokenize{OtherManuals/GQF_Manual:synthesised-shapefiles}}
If pre-processing of the input data has taken place, the Disaggregated
energy, transport and population shapefiles are stored in the
\sphinxstylestrong{/DownscaledData/} subdirectory of the model outputs, with filenames
that reflect the time period they represent. This folder can be used as
the source of processed input data in the future to save time, provided
that the data sources file have not changed.

If previously processed input data are being used, the
\sphinxstylestrong{/DownscaledData/} subdirectory remains empty.


\subsection{Logs}
\label{\detokenize{OtherManuals/GQF_Manual:logs}}
Several log files are saved in the \sphinxstylestrong{/Logs/} subdirectory. The logs are
intended to help interpretation of model outputs by providing a
traceable history of why a particular spatial or temporal disaggregation
value was looked up.
\begin{enumerate}
\item {} 
The steps taken to disaggregate spatial data, including which
attributes were involved

\item {} 
The day of week and the time of day that was returned from each
diurnal and annual profile data source when it was queried with a
particular model time step.

\end{enumerate}


\subsection{Configuration files}
\label{\detokenize{OtherManuals/GQF_Manual:configuration-files}}
The Parameters and Data Sources file are copied to the \sphinxstylestrong{/ConfigFiles/}
subdirectory of the model output directory for future reference.


\subsection{Input data}
\label{\detokenize{OtherManuals/GQF_Manual:input-data}}
Input data consists of spatial and temporal information, a lookup table
for vehicle fuel efficiency and (optionally) land use cover data to
further enhance the spatial resolution of the model output.
\begin{itemize}
\item {} 
Spatial information:
\begin{itemize}
\item {} 
\sphinxstylestrong{Residential} (evening) and \sphinxstylestrong{work day} (daytime) absolute
population

\item {} 
District-scale domestic and industrial \sphinxstylestrong{energy consumption}
{[}kWh/year{]}

\item {} 
\sphinxstylestrong{Road network} topography and associated traffic flows

\end{itemize}

\item {} 
Temporal information (provided via CSV files):
\begin{itemize}
\item {} 
Template diurnal cycles for \sphinxstylestrong{energy consumption, traffic flow}
and \sphinxstylestrong{human activity}

\item {} 
Variations of these cycles for different \sphinxstylestrong{days of week}

\item {} 
Variations of the above at different \sphinxstylestrong{times of year}.

\end{itemize}

\end{itemize}


\subsubsection{Spatial data}
\label{\detokenize{OtherManuals/GQF_Manual:spatial-data}}
This section lists the spatial data (provided via ESRI shapefiles)
required by the model. Each shapefile must contain:
\begin{itemize}
\item {} 
Polygons representing each spatial area (except for Transport)

\item {} 
An attribute that contains a unique identifier for each polygon. This
is needed for objective cross-referencing of data within the model.

\end{itemize}


\paragraph{Population data}
\label{\detokenize{OtherManuals/GQF_Manual:population-data}}
Population data {[}number of people per spatial unit{]} is used by the model
in two ways:
\begin{enumerate}
\item {} 
Calculating metabolic emissions in different areas

\item {} 
Attributing domestic and industrial energy use at a finer spatial
scale.

\end{enumerate}

Two types of population are needed:
\begin{itemize}
\item {} 
\sphinxstylestrong{Residential/evening population}: The population residing in each
area

\item {} 
\sphinxstylestrong{Workday/daytime population}: The population at work or home during
the daytime in each area

\end{itemize}

Since population data are key to the model method, it is important to
use the finest available spatial scale.

The model must output results for a consistent set of spatial units, so
the populations are assigned to the model output areas based on how much
each spatial unit of population is intersected each output area. It is
\sphinxstylestrong{recommended} that a population shapefile is chosen as the output
areas.


\paragraph{Energy consumption data}
\label{\detokenize{OtherManuals/GQF_Manual:energy-consumption-data}}
The total annual energy consumption {[}kWh/year{]} must be provided five
sub-sectors
\begin{enumerate}
\item {} 
Industrial electricity

\item {} 
Industrial gas

\item {} 
Domestic electricity

\item {} 
Domestic gas

\item {} 
Domestic “Economy 7”: an electrical supply with a distinct diurnal
pattern (may be set to zero in the data sources file if not
available)

\end{enumerate}

This data is used to calculate heat emissions from residential and
industrial buildings, and is generally available in coarse spatial
units. Residential and workday population data are therefore used to
spatially disaggregate it into the model output areas.


\paragraph{Transportation data}
\label{\detokenize{OtherManuals/GQF_Manual:transportation-data}}
A comprehensive road network shapefile is required.
\begin{itemize}
\item {} 
Minimum: vector line for each segment of the road network, together
with the type of road each segment represents.

\end{itemize}

Four road classes are assumed in the model:


\begin{savenotes}\sphinxattablestart
\centering
\begin{tabular}[t]{|\X{50}{100}|\X{50}{100}|}
\hline

\sphinxstylestrong{Motorway}
&
Purpose-built highways
\\
\hline
\sphinxstylestrong{Primary road}
&
Major thoroughfares
\\
\hline
\sphinxstylestrong{Secondary road}
&
Thoroughfares with less traffic
\\
\hline
\sphinxstylestrong{Other}
&
Any other road segments: Assumed to have minor traffic flow
\\
\hline
\end{tabular}
\par
\sphinxattableend\end{savenotes}

The naming convention used in the shapefile must be defined in the
transport section of the {\hyperref[\detokenize{OtherManuals/GQF_Manual:data-sources-file}]{\sphinxcrossref{Data sources file}}}
for the first three.

Diesel and petrol consumption are calculated for seven vehicle types
indicated using any segment-specific traffic flow and speed data
available. This is combined with fuel consumption data. The vehicle
types are:


\begin{savenotes}\sphinxattablestart
\centering
\begin{tabular}[t]{|\X{50}{100}|\X{50}{100}|}
\hline
\sphinxstyletheadfamily 
Name in model
&\sphinxstyletheadfamily 
Description
\\
\hline
Motorcycle
&
Motorcycles
\\
\hline
Taxi
&
Taxis
\\
\hline
Bus
&
Buses and coaches
\\
\hline
Artic
&
Articulated trucks
\\
\hline
Rigid
&
Rigid body trucks
\\
\hline
LGV
&
Light Goods Vehicle
\\
\hline
Car
&
Ordinary cars
\\
\hline
\end{tabular}
\par
\sphinxattableend\end{savenotes}

Fuel consumption for a given vehicle type on a particular road segment
{[}g/day{]} is estimated by multiplying:
\begin{enumerate}
\item {} 
Speed, fuel and vehicle-dependent consumption rates {[}g/km{]} from the
COPERT-II database, which lists consumption for different vehicle
types under different Euro-class regimes that apply to vehicles
manufactured after a particular date.

\item {} 
Length of the road segment {[}km{]}

\item {} 
Vehicle type and fuel-dependent average daily total (AADT) number of
vehicles passing over the road segment.

\end{enumerate}

Each road segment in the shapefile would ideally be accompanied by a
speed for the segment and an AADT for each vehicle type that is further
broken down into diesel and petrol components for cars and LGVs. It is
not always possible to obtain some or even any of these, so default
representative values must also be specified in the {\color{red}\bfseries{}{}`}model parameters
file {\hyperref[\detokenize{OtherManuals/GQF_Manual:id3}]{\sphinxcrossref{Parameters\_file}}}:


\begin{savenotes}\sphinxattablestart
\centering
\begin{tabular}[t]{|\X{50}{100}|\X{50}{100}|}
\hline

AADT
&
A representative AADT associated with each road class
\\
\hline
Road fleet fraction
&
Contribution of different vehicle types to the total traffic on each road classification.
\\
\hline
Fuel fraction
&
Fraction of each vehicle type powered by diesel and petrol
\\
\hline
Speed
&
Typical speed of traffic on each road classification
\\
\hline
\end{tabular}
\par
\sphinxattableend\end{savenotes}

The use of the default parameters depends upon the available information
in the shapefile. This relations are summarised below: when parameters
are used if certain information are (green) or are not (red) available.
\begin{itemize}
\item {} 
Available in shapefile

\end{itemize}


\begin{savenotes}\sphinxattablestart
\centering
\begin{tabular}[t]{|\X{25}{100}|\X{25}{100}|\X{25}{100}|\X{25}{100}|}
\hline
\sphinxstyletheadfamily 
Total AADT
&\sphinxstyletheadfamily 
AADT by vehicle
&\sphinxstyletheadfamily 
AADT by vehicle \& fuel
&\sphinxstyletheadfamily 
Speed
\\
\hline
X
&
X
&
X
&
X
\\
\hline
/
&
X
&
X
&
X
\\
\hline
/
&
/
&
X
&
X
\\
\hline
/
&
/
&
/
&
X
\\
\hline
X
&
X
&
X
&
/
\\
\hline
X
&
X
&
X
&
/
\\
\hline
/
&
X
&
X
&
/
\\
\hline
/
&
/
&
X
&
/
\\
\hline
/
&
/
&
/
&
/
\\
\hline
\end{tabular}
\par
\sphinxattableend\end{savenotes}
\begin{itemize}
\item {} 
Looked up from parameters

\end{itemize}


\begin{savenotes}\sphinxattablestart
\centering
\begin{tabular}[t]{|\X{25}{100}|\X{25}{100}|\X{25}{100}|\X{25}{100}|}
\hline
\sphinxstyletheadfamily 
AADT
&\sphinxstyletheadfamily 
Fuel fraction
&\sphinxstyletheadfamily 
Fleet fraction
&\sphinxstyletheadfamily 
Speed
\\
\hline
/
&
/
&
/
&
/
\\
\hline
X
&
/
&
/
&
/
\\
\hline
X
&
/
&
X
&
/
\\
\hline
X
&
X
&
X
&
/
\\
\hline
/
&
/
&
/
&
X
\\
\hline
/
&
/
&
/
&
X
\\
\hline
X
&
/
&
/
&
X
\\
\hline
X
&
/
&
X
&
X
\\
\hline
X
&
X
&
X
&
X
\\
\hline
\end{tabular}
\par
\sphinxattableend\end{savenotes}

The fuel consumption that a segment contributes to a model output area
(OA) is calculated by determining the proportion of the segment that
intersects the OA and multiplying the total segment consumption by this.
Total fuel consumption inside an output area is calculated by summing
over all the segments that intersect it. This yields a new shapefile in
which each output area is associated with a daily petrol and diesel
consumption.

Daily fuel consumption in an OA is converted to mean heat flux {[}W
m$^{\text{-2}}${]} using the heat of combustion {[}J kg$^{\text{-1}}${]}, number of
seconds in a day and the area of the OA {[}m$^{\text{-2}}${]}. This is
disaggregated to half-hour time steps using empirical diurnal cycle data
for each day of the week.


\paragraph{Time indexing of spatial data}
\label{\detokenize{OtherManuals/GQF_Manual:time-indexing-of-spatial-data}}
A series of shapefiles, each associated with a different start date, can
be loaded into the model to capture the time evolution of energy use,
transport or population. The following example describes how such a
series is treated by the model implementation:

Two shapefiles are provided for population. The first is correct as of
2015-01-01, and the second is correct as of 2016-01-01. The model is set
to calculate Q$_{\text{F}}$ from 2014 to 2017 continuously:
\begin{enumerate}
\item {} 
Model time steps representing dates before 2015-01-01 use the
earliest available shapefile (2015-01-01).

\item {} 
Model time steps on/after 2015-01-01 but before 2016-01-01 use the
2015-01-01 shapefile

\item {} 
Model time steps on/after 2016-01-01 use the 2016-01-01 shapefile. No
transition is assumed between the shapefiles.

\end{enumerate}

Since energy consumption data is disaggregated to finer spatial units
based on population, the energy consumption on/before 2015-12-31 is
disaggregated using the 2015-01-01 population data, while energy
consumption associated with 2016-01-01 or later is disaggregated using
the 2016-01-01 population data.


\subsubsection{Temporal files required by GQF}
\label{\detokenize{OtherManuals/GQF_Manual:temporal-files-required-by-gqf}}
\sphinxstylestrong{Overview}
\begin{itemize}
\item {} 
Four temporal profile files (summarised below) contain information
about half-hourly, daily and seasonal variations in traffic,
metabolic activity and energy use. These allow the annualised data
provided by the shapefiles to be temporally disaggreated into time
series.

\item {} 
Each file must contain:
\begin{enumerate}
\item {} 
A time series of values at 30 minute intervals, binned to the
right hand side. The first entry of every file represents the
period 00:00-00:30 and is labelled 00:30.

\item {} 
Values for every part of every year mentioned in the file. \sphinxstylestrong{Gaps
are not allowed}.

\item {} 
The time zone represented by the file
(“\sphinxhref{https://en.wikipedia.org/wiki/Coordinated\_Universal\_Time}{UTC}”
or of the style “Europe/London”). If “UTC” is specified, then
values must be explicitly provided for each daylight savings
regime to capture shifts in human behaviour. Note that the model
outputs are always UTC, with the necessary conversion taking place
in the software.

\item {} 
The start and end dates of the period represented by the data.
This allows seasonality to be captured.

\end{enumerate}

\end{itemize}


\begin{savenotes}\sphinxattablestart
\centering
\begin{tabular}[t]{|\X{33}{99}|\X{33}{99}|\X{33}{99}|}
\hline
\sphinxstyletheadfamily 
Q$_{\text{F}}$ component
&\sphinxstyletheadfamily 
File description(s)
&\sphinxstyletheadfamily 
Size of file
\\
\hline
Metabolism
&
Diurnal cycles of metabolic activity for each day of week and each season
&
48 half-hours * 7 days * N seasons
\\
\hline
Transport
&
Traffic flows for each vehicle type during each day of the week
&
336 half-hours (=48 * 7) * 7 vehicle types
\\
\hline
Building energy
&
Seasonal variations: Daily total gas and electricity consumption variation (one file for electricity and gas)
&
365 (or 366) days * 2 fuel types
\\
\hline

&
Diurnal variations: Template cycles for weekdays, Saturdays and Sundays for each season (separate file for each fuel)
&
48 half-hours * 3 day types * N seasons
\\
\hline
\end{tabular}
\par
\sphinxattableend\end{savenotes}

Ideally these files contain data taken from the period being modelled,
but this is not always practical. In this case, temporal profile data
from the most recent available year is looked up for the same day of
week (taking into account public holidays), season and daylight savings
regime if applicable. Different variants are used for traffic, energy
and metabolism, and each of these is described below.


\paragraph{Details of temporal files}
\label{\detokenize{OtherManuals/GQF_Manual:details-of-temporal-files}}

\subparagraph{Traffic flow profiles}
\label{\detokenize{OtherManuals/GQF_Manual:traffic-flow-profiles}}
A template week of traffic variations at 30 min intervals (336 entries,
48 * 7) beginning on Monday must be specified for each vehicle type, so
that day of week effects are captured.

An example is shown below. The first header line must be \sphinxstylestrong{exactly} as
shown because it specifies the vehicle types used in the model. Each
file may contain only one set of values. Subsequent periods or years
must be stored in separate files.


\begin{savenotes}\sphinxatlongtablestart\begin{longtable}{|\X{12}{96}|\X{12}{96}|\X{12}{96}|\X{12}{96}|\X{12}{96}|\X{12}{96}|\X{12}{96}|\X{12}{96}|}
\hline
\sphinxstyletheadfamily 
TransportType
&\sphinxstyletheadfamily 
motorcycles
&\sphinxstyletheadfamily 
taxis
&\sphinxstyletheadfamily 
cars
&\sphinxstyletheadfamily 
Buses
&\sphinxstyletheadfamily 
LGVs
&\sphinxstyletheadfamily 
rigids
&\sphinxstyletheadfamily 
artics
\\
\hline
\endfirsthead

\multicolumn{8}{c}%
{\makebox[0pt]{\sphinxtablecontinued{\tablename\ \thetable{} -- continued from previous page}}}\\
\hline
\sphinxstyletheadfamily 
TransportType
&\sphinxstyletheadfamily 
motorcycles
&\sphinxstyletheadfamily 
taxis
&\sphinxstyletheadfamily 
cars
&\sphinxstyletheadfamily 
Buses
&\sphinxstyletheadfamily 
LGVs
&\sphinxstyletheadfamily 
rigids
&\sphinxstyletheadfamily 
artics
\\
\hline
\endhead

\hline
\multicolumn{8}{r}{\makebox[0pt][r]{\sphinxtablecontinued{Continued on next page}}}\\
\endfoot

\endlastfoot

StartDate
&
2016-01-01
&&&&&&\\
\hline
EndDate
&
2016-12-31
&&&&&&\\
\hline
Timezone
&
Europe/London
&&&&&&\\
\hline
00:30
&
0.237
&
1.125
&
0.398
&
0.594
&
0.198
&
0.435
&
0.436
\\
\hline
01:00
&
0.178
&
1.003
&
0.312
&
0.433
&
0.172
&
0.393
&
0.4
\\
\hline
01:30
&
0.12
&
0.881
&
0.226
&
0.272
&
0.146
&
0.352
&
0.365
\\
\hline
02:00
&
0.093
&
0.647
&
0.192
&
0.234
&
0.138
&
0.378
&
0.378
\\
\hline
02:30
&
0.066
&
0.412
&
0.159
&
0.197
&
0.13
&
0.404
&
0.39
\\
\hline
03:00
&
0.065
&
0.349
&
0.147
&
0.189
&
0.148
&
0.355
&
0.366
\\
\hline
03:30
&
0.063
&
0.286
&
0.135
&
0.18
&
0.167
&
0.306
&
0.342
\\
\hline
04:00
&
0.086
&
0.276
&
0.149
&
0.204
&
0.215
&
0.413
&
0.427
\\
\hline
04:30
&
0.109
&
0.267
&
0.163
&
0.229
&
0.262
&
0.52
&
0.511
\\
\hline
05:00
&
0.199
&
0.343
&
0.226
&
0.367
&
0.341
&
0.7
&
0.664
\\
\hline
05:30
&
0.288
&
0.419
&
0.288
&
0.505
&
0.42
&
0.88
&
0.817
\\
\hline
06:00
&
0.699
&
0.565
&
0.54
&
0.721
&
0.934
&
1.195
&
1.161
\\
\hline
06:30
&
1.11
&
0.71
&
0.791
&
0.937
&
1.448
&
1.511
&
1.504
\\
\hline
07:00
&
1.62
&
0.786
&
1.086
&
1.184
&
1.771
&
1.5
&
1.646
\\
\hline
07:30
&
2.129
&
0.861
&
1.381
&
1.431
&
2.094
&
1.49
&
1.788
\\
\hline
08:00
&
2.375
&
0.873
&
1.461
&
1.435
&
1.875
&
1.498
&
1.739
\\
\hline
08:30
&
2.62
&
0.885
&
1.54
&
1.438
&
1.656
&
1.507
&
1.689
\\
\hline
09:00
&
2.166
&
0.897
&
1.424
&
1.487
&
1.672
&
1.693
&
1.791
\\
\hline
09:30
&
1.712
&
0.909
&
1.308
&
1.537
&
1.689
&
1.88
&
1.892
\\
\hline
10:00
&
1.452
&
0.983
&
1.23
&
1.499
&
1.724
&
1.96
&
1.956
\\
\hline
10:30
&
1.192
&
1.057
&
1.152
&
1.462
&
1.76
&
2.041
&
2.02
\\
\hline
11:00
&
1.165
&
1.095
&
1.144
&
1.404
&
1.765
&
2.077
&
2.025
\\
\hline
11:30
&
1.138
&
1.133
&
1.136
&
1.347
&
1.77
&
2.112
&
2.031
\\
\hline
12:00
&
1.167
&
1.125
&
1.168
&
1.335
&
1.76
&
2.118
&
2.034
\\
\hline
12:30
&
1.196
&
1.117
&
1.2
&
1.324
&
1.75
&
2.124
&
2.037
\\
\hline
13:00
&
1.239
&
1.143
&
1.209
&
1.339
&
1.748
&
2.072
&
1.988
\\
\hline
13:30
&
1.282
&
1.169
&
1.219
&
1.354
&
1.746
&
2.021
&
1.94
\\
\hline
14:00
&
1.292
&
1.281
&
1.231
&
1.392
&
1.775
&
1.97
&
1.862
\\
\hline
14:30
&
1.302
&
1.393
&
1.244
&
1.43
&
1.804
&
1.919
&
1.784
\\
\hline
15:00
&
1.375
&
1.321
&
1.31
&
1.454
&
1.838
&
1.853
&
1.678
\\
\hline
15:30
&
1.447
&
1.248
&
1.376
&
1.477
&
1.872
&
1.788
&
1.572
\\
\hline
16:00
&
1.671
&
1.337
&
1.448
&
1.504
&
1.887
&
1.665
&
1.468
\\
\hline
16:30
&
1.894
&
1.425
&
1.521
&
1.531
&
1.902
&
1.542
&
1.363
\\
\hline
17:00
&
2.237
&
1.447
&
1.606
&
1.47
&
1.714
&
1.419
&
1.241
\\
\hline
17:30
&
2.579
&
1.469
&
1.691
&
1.41
&
1.525
&
1.296
&
1.119
\\
\hline
18:00
&
2.518
&
1.414
&
1.647
&
1.377
&
1.314
&
1.214
&
1.038
\\
\hline
18:30
&
2.458
&
1.36
&
1.604
&
1.343
&
1.103
&
1.132
&
0.956
\\
\hline
19:00
&
2.086
&
1.394
&
1.54
&
1.33
&
0.973
&
0.799
&
0.733
\\
\hline
19:30
&
1.715
&
1.429
&
1.476
&
1.318
&
0.843
&
0.466
&
0.511
\\
\hline
20:00
&
1.417
&
1.445
&
1.314
&
1.195
&
0.724
&
0.462
&
0.498
\\
\hline
20:30
&
1.119
&
1.461
&
1.153
&
1.071
&
0.604
&
0.459
&
0.485
\\
\hline
21:00
&
0.963
&
1.396
&
1.054
&
0.971
&
0.52
&
0.384
&
0.427
\\
\hline
21:30
&
0.807
&
1.331
&
0.954
&
0.871
&
0.437
&
0.31
&
0.37
\\
\hline
22:00
&
0.705
&
1.301
&
0.893
&
0.807
&
0.384
&
0.338
&
0.381
\\
\hline
22:30
&
0.602
&
1.271
&
0.832
&
0.744
&
0.331
&
0.365
&
0.393
\\
\hline
23:00
&
0.525
&
1.287
&
0.748
&
0.745
&
0.3
&
0.409
&
0.424
\\
\hline
23:30
&
0.447
&
1.304
&
0.665
&
0.747
&
0.269
&
0.453
&
0.455
\\
\hline
00:00
&
0.346
&
1.235
&
0.539
&
0.681
&
0.237
&
0.452
&
0.453
\\
\hline
00:30
&
0.246
&
1.167
&
0.412
&
0.616
&
0.206
&
0.451
&
0.451
\\
\hline
01:00
&
0.185
&
1.04
&
0.323
&
0.449
&
0.178
&
0.408
&
0.415
\\
\hline
01:30
&
0.125
&
0.914
&
0.234
&
0.282
&
0.151
&
0.365
&
0.378
\\
\hline
02:00
&
0.097
&
0.671
&
0.2
&
0.243
&
0.143
&
0.392
&
0.391
\\
\hline
02:30
&
0.069
&
0.428
&
0.165
&
0.205
&
0.134
&
0.419
&
0.404
\\
\hline
03:00
&
0.067
&
0.362
&
0.153
&
0.195
&
0.154
&
0.368
&
0.379
\\
\hline
03:30
&
0.066
&
0.297
&
0.14
&
0.186
&
0.173
&
0.317
&
0.354
\\
\hline
04:00
&
0.089
&
0.287
&
0.155
&
0.212
&
0.222
&
0.428
&
0.442
\\
\hline
04:30
&
0.113
&
0.277
&
0.17
&
0.238
&
0.272
&
0.539
&
0.53
\\
\hline
05:00
&
0.206
&
0.355
&
0.234
&
0.381
&
0.354
&
0.725
&
0.688
\\
\hline
05:30
&
0.299
&
0.434
&
0.299
&
0.524
&
0.436
&
0.911
&
0.847
\\
\hline
06:00
&
0.726
&
0.586
&
0.56
&
0.748
&
0.968
&
1.239
&
1.203
\\
\hline
06:30
&
1.153
&
0.737
&
0.821
&
0.972
&
1.5
&
1.566
&
1.559
\\
\hline
07:00
&
1.676
&
0.813
&
1.124
&
1.225
&
1.832
&
1.552
&
1.703
\\
\hline
07:30
&
2.199
&
0.89
&
1.427
&
1.478
&
2.163
&
1.539
&
1.847
\\
\hline
08:00
&
2.47
&
0.908
&
1.519
&
1.491
&
1.947
&
1.557
&
1.807
\\
\hline
08:30
&
2.74
&
0.925
&
1.611
&
1.504
&
1.732
&
1.576
&
1.767
\\
\hline
09:00
&
2.264
&
0.937
&
1.488
&
1.554
&
1.748
&
1.769
&
1.871
\\
\hline
09:30
&
1.787
&
0.949
&
1.366
&
1.605
&
1.763
&
1.963
&
1.976
\\
\hline
10:00
&
1.515
&
1.025
&
1.283
&
1.564
&
1.799
&
2.045
&
2.04
\\
\hline
10:30
&
1.243
&
1.101
&
1.201
&
1.523
&
1.834
&
2.127
&
2.105
\\
\hline
11:00
&
1.211
&
1.138
&
1.189
&
1.459
&
1.834
&
2.158
&
2.105
\\
\hline
11:30
&
1.18
&
1.174
&
1.177
&
1.396
&
1.835
&
2.189
&
2.105
\\
\hline
12:00
&
1.207
&
1.163
&
1.207
&
1.381
&
1.82
&
2.19
&
2.103
\\
\hline
12:30
&
1.233
&
1.152
&
1.237
&
1.366
&
1.805
&
2.191
&
2.101
\\
\hline
13:00
&
1.275
&
1.176
&
1.245
&
1.378
&
1.799
&
2.133
&
2.047
\\
\hline
13:30
&
1.317
&
1.201
&
1.252
&
1.391
&
1.794
&
2.076
&
1.993
\\
\hline
14:00
&
1.329
&
1.317
&
1.266
&
1.432
&
1.825
&
2.025
&
1.914
\\
\hline
14:30
&
1.34
&
1.434
&
1.28
&
1.472
&
1.856
&
1.974
&
1.836
\\
\hline
15:00
&
1.416
&
1.36
&
1.349
&
1.497
&
1.892
&
1.908
&
1.728
\\
\hline
15:30
&
1.491
&
1.286
&
1.418
&
1.522
&
1.929
&
1.843
&
1.62
\\
\hline
16:00
&
1.721
&
1.377
&
1.492
&
1.549
&
1.944
&
1.715
&
1.512
\\
\hline
16:30
&
1.95
&
1.468
&
1.566
&
1.576
&
1.959
&
1.588
&
1.404
\\
\hline
17:00
&
2.318
&
1.499
&
1.663
&
1.522
&
1.774
&
1.469
&
1.285
\\
\hline
17:30
&
2.686
&
1.53
&
1.761
&
1.469
&
1.589
&
1.35
&
1.166
\\
\hline
18:00
&
2.635
&
1.48
&
1.723
&
1.44
&
1.374
&
1.27
&
1.086
\\
\hline
18:30
&
2.583
&
1.43
&
1.686
&
1.412
&
1.16
&
1.189
&
1.005
\\
\hline
19:00
&
2.182
&
1.456
&
1.608
&
1.389
&
1.017
&
0.836
&
0.767
\\
\hline
19:30
&
1.78
&
1.482
&
1.531
&
1.366
&
0.874
&
0.483
&
0.529
\\
\hline
20:00
&
1.471
&
1.498
&
1.363
&
1.239
&
0.75
&
0.479
&
0.516
\\
\hline
20:30
&
1.162
&
1.515
&
1.196
&
1.111
&
0.626
&
0.475
&
0.503
\\
\hline
21:00
&
1
&
1.448
&
1.093
&
1.007
&
0.539
&
0.398
&
0.443
\\
\hline
21:30
&
0.838
&
1.381
&
0.989
&
0.903
&
0.452
&
0.322
&
0.383
\\
\hline
22:00
&
0.732
&
1.349
&
0.926
&
0.837
&
0.398
&
0.35
&
0.395
\\
\hline
22:30
&
0.625
&
1.318
&
0.863
&
0.772
&
0.343
&
0.378
&
0.407
\\
\hline
23:00
&
0.545
&
1.335
&
0.776
&
0.773
&
0.311
&
0.424
&
0.439
\\
\hline
23:30
&
0.464
&
1.352
&
0.69
&
0.774
&
0.279
&
0.47
&
0.471
\\
\hline
00:00
&
0.355
&
1.261
&
0.552
&
0.696
&
0.243
&
0.461
&
0.462
\\
\hline
00:30
&
0.247
&
1.171
&
0.414
&
0.618
&
0.207
&
0.452
&
0.453
\\
\hline
01:00
&
0.186
&
1.044
&
0.324
&
0.45
&
0.179
&
0.409
&
0.416
\\
\hline
01:30
&
0.125
&
0.917
&
0.235
&
0.283
&
0.152
&
0.366
&
0.38
\\
\hline
02:00
&
0.097
&
0.673
&
0.2
&
0.244
&
0.143
&
0.393
&
0.393
\\
\hline
02:30
&
0.069
&
0.429
&
0.166
&
0.205
&
0.135
&
0.421
&
0.406
\\
\hline
03:00
&
0.067
&
0.363
&
0.153
&
0.196
&
0.154
&
0.369
&
0.381
\\
\hline
03:30
&
0.066
&
0.298
&
0.141
&
0.187
&
0.174
&
0.318
&
0.356
\\
\hline
04:00
&
0.09
&
0.288
&
0.155
&
0.213
&
0.223
&
0.43
&
0.444
\\
\hline
04:30
&
0.114
&
0.278
&
0.17
&
0.238
&
0.273
&
0.541
&
0.532
\\
\hline
05:00
&
0.207
&
0.357
&
0.235
&
0.382
&
0.355
&
0.728
&
0.691
\\
\hline
05:30
&
0.3
&
0.436
&
0.3
&
0.526
&
0.437
&
0.915
&
0.851
\\
\hline
06:00
&
0.729
&
0.588
&
0.562
&
0.751
&
0.972
&
1.243
&
1.208
\\
\hline
06:30
&
1.157
&
0.739
&
0.823
&
0.976
&
1.507
&
1.572
&
1.566
\\
\hline
07:00
&
1.695
&
0.822
&
1.136
&
1.238
&
1.851
&
1.567
&
1.721
\\
\hline
07:30
&
2.233
&
0.904
&
1.449
&
1.501
&
2.196
&
1.562
&
1.876
\\
\hline
08:00
&
2.496
&
0.918
&
1.535
&
1.507
&
1.97
&
1.574
&
1.827
\\
\hline
08:30
&
2.759
&
0.932
&
1.621
&
1.514
&
1.743
&
1.586
&
1.779
\\
\hline
09:00
&
2.273
&
0.94
&
1.493
&
1.559
&
1.753
&
1.774
&
1.877
\\
\hline
09:30
&
1.786
&
0.949
&
1.365
&
1.604
&
1.762
&
1.962
&
1.975
\\
\hline
10:00
&
1.518
&
1.028
&
1.286
&
1.568
&
1.803
&
2.05
&
2.046
\\
\hline
10:30
&
1.249
&
1.107
&
1.207
&
1.531
&
1.844
&
2.139
&
2.116
\\
\hline
11:00
&
1.219
&
1.145
&
1.197
&
1.469
&
1.847
&
2.172
&
2.119
\\
\hline
11:30
&
1.189
&
1.183
&
1.186
&
1.407
&
1.849
&
2.206
&
2.121
\\
\hline
12:00
&
1.212
&
1.168
&
1.213
&
1.387
&
1.828
&
2.2
&
2.112
\\
\hline
12:30
&
1.235
&
1.153
&
1.239
&
1.367
&
1.807
&
2.193
&
2.103
\\
\hline
13:00
&
1.278
&
1.179
&
1.247
&
1.381
&
1.802
&
2.137
&
2.051
\\
\hline
13:30
&
1.321
&
1.204
&
1.255
&
1.395
&
1.798
&
2.081
&
1.998
\\
\hline
14:00
&
1.333
&
1.321
&
1.27
&
1.436
&
1.83
&
2.031
&
1.92
\\
\hline
14:30
&
1.344
&
1.439
&
1.284
&
1.477
&
1.862
&
1.981
&
1.842
\\
\hline
15:00
&
1.421
&
1.364
&
1.353
&
1.502
&
1.899
&
1.915
&
1.734
\\
\hline
15:30
&
1.497
&
1.29
&
1.423
&
1.527
&
1.936
&
1.849
&
1.626
\\
\hline
16:00
&
1.733
&
1.386
&
1.502
&
1.559
&
1.956
&
1.726
&
1.521
\\
\hline
16:30
&
1.968
&
1.481
&
1.58
&
1.591
&
1.977
&
1.603
&
1.417
\\
\hline
17:00
&
2.317
&
1.5
&
1.664
&
1.524
&
1.777
&
1.471
&
1.287
\\
\hline
17:30
&
2.666
&
1.519
&
1.748
&
1.458
&
1.577
&
1.34
&
1.157
\\
\hline
18:00
&
2.623
&
1.473
&
1.716
&
1.434
&
1.368
&
1.264
&
1.081
\\
\hline
18:30
&
2.58
&
1.428
&
1.684
&
1.41
&
1.158
&
1.188
&
1.004
\\
\hline
19:00
&
2.183
&
1.457
&
1.61
&
1.391
&
1.018
&
0.837
&
0.768
\\
\hline
19:30
&
1.786
&
1.487
&
1.536
&
1.372
&
0.877
&
0.485
&
0.531
\\
\hline
20:00
&
1.476
&
1.504
&
1.368
&
1.243
&
0.753
&
0.481
&
0.518
\\
\hline
20:30
&
1.166
&
1.52
&
1.201
&
1.115
&
0.629
&
0.477
&
0.505
\\
\hline
21:00
&
1.004
&
1.453
&
1.097
&
1.011
&
0.542
&
0.4
&
0.445
\\
\hline
21:30
&
0.841
&
1.385
&
0.993
&
0.906
&
0.454
&
0.323
&
0.385
\\
\hline
22:00
&
0.734
&
1.354
&
0.929
&
0.84
&
0.399
&
0.351
&
0.397
\\
\hline
22:30
&
0.627
&
1.322
&
0.866
&
0.775
&
0.344
&
0.38
&
0.409
\\
\hline
23:00
&
0.546
&
1.34
&
0.779
&
0.776
&
0.312
&
0.426
&
0.441
\\
\hline
23:30
&
0.466
&
1.357
&
0.692
&
0.777
&
0.28
&
0.472
&
0.473
\\
\hline
00:00
&
0.357
&
1.269
&
0.555
&
0.7
&
0.244
&
0.464
&
0.465
\\
\hline
00:30
&
0.249
&
1.181
&
0.418
&
0.623
&
0.208
&
0.456
&
0.457
\\
\hline
01:00
&
0.188
&
1.053
&
0.327
&
0.454
&
0.181
&
0.413
&
0.42
\\
\hline
01:30
&
0.126
&
0.925
&
0.237
&
0.285
&
0.153
&
0.369
&
0.383
\\
\hline
02:00
&
0.098
&
0.679
&
0.202
&
0.246
&
0.145
&
0.397
&
0.396
\\
\hline
02:30
&
0.069
&
0.433
&
0.167
&
0.207
&
0.136
&
0.424
&
0.409
\\
\hline
03:00
&
0.068
&
0.367
&
0.155
&
0.198
&
0.156
&
0.372
&
0.384
\\
\hline
03:30
&
0.066
&
0.3
&
0.142
&
0.189
&
0.175
&
0.321
&
0.359
\\
\hline
04:00
&
0.091
&
0.29
&
0.157
&
0.215
&
0.225
&
0.433
&
0.448
\\
\hline
04:30
&
0.115
&
0.28
&
0.172
&
0.24
&
0.275
&
0.546
&
0.536
\\
\hline
05:00
&
0.209
&
0.36
&
0.237
&
0.385
&
0.358
&
0.734
&
0.697
\\
\hline
05:30
&
0.303
&
0.44
&
0.303
&
0.53
&
0.441
&
0.923
&
0.858
\\
\hline
06:00
&
0.735
&
0.593
&
0.567
&
0.757
&
0.98
&
1.254
&
1.218
\\
\hline
06:30
&
1.167
&
0.746
&
0.831
&
0.984
&
1.519
&
1.585
&
1.578
\\
\hline
07:00
&
1.707
&
0.828
&
1.144
&
1.247
&
1.864
&
1.578
&
1.733
\\
\hline
07:30
&
2.247
&
0.909
&
1.458
&
1.51
&
2.209
&
1.572
&
1.887
\\
\hline
08:00
&
2.506
&
0.921
&
1.541
&
1.514
&
1.978
&
1.581
&
1.835
\\
\hline
08:30
&
2.764
&
0.933
&
1.625
&
1.517
&
1.747
&
1.59
&
1.782
\\
\hline
09:00
&
2.282
&
0.945
&
1.5
&
1.567
&
1.762
&
1.783
&
1.886
\\
\hline
09:30
&
1.8
&
0.956
&
1.375
&
1.617
&
1.776
&
1.977
&
1.99
\\
\hline
10:00
&
1.524
&
1.031
&
1.291
&
1.573
&
1.809
&
2.057
&
2.053
\\
\hline
10:30
&
1.249
&
1.107
&
1.206
&
1.53
&
1.843
&
2.137
&
2.115
\\
\hline
11:00
&
1.217
&
1.143
&
1.194
&
1.466
&
1.842
&
2.168
&
2.114
\\
\hline
11:30
&
1.185
&
1.179
&
1.182
&
1.401
&
1.842
&
2.198
&
2.113
\\
\hline
12:00
&
1.214
&
1.17
&
1.215
&
1.389
&
1.831
&
2.204
&
2.116
\\
\hline
12:30
&
1.244
&
1.162
&
1.248
&
1.377
&
1.82
&
2.209
&
2.119
\\
\hline
13:00
&
1.296
&
1.195
&
1.264
&
1.4
&
1.827
&
2.166
&
2.079
\\
\hline
13:30
&
1.347
&
1.228
&
1.281
&
1.423
&
1.834
&
2.123
&
2.038
\\
\hline
14:00
&
1.359
&
1.348
&
1.295
&
1.465
&
1.867
&
2.072
&
1.958
\\
\hline
14:30
&
1.371
&
1.467
&
1.31
&
1.506
&
1.9
&
2.021
&
1.879
\\
\hline
15:00
&
1.443
&
1.387
&
1.375
&
1.526
&
1.93
&
1.946
&
1.762
\\
\hline
15:30
&
1.515
&
1.306
&
1.44
&
1.546
&
1.96
&
1.872
&
1.646
\\
\hline
16:00
&
1.746
&
1.397
&
1.514
&
1.572
&
1.973
&
1.741
&
1.535
\\
\hline
16:30
&
1.977
&
1.488
&
1.588
&
1.598
&
1.986
&
1.61
&
1.423
\\
\hline
17:00
&
2.339
&
1.513
&
1.679
&
1.537
&
1.791
&
1.483
&
1.298
\\
\hline
17:30
&
2.701
&
1.538
&
1.77
&
1.477
&
1.597
&
1.357
&
1.172
\\
\hline
18:00
&
2.657
&
1.492
&
1.738
&
1.452
&
1.385
&
1.28
&
1.094
\\
\hline
18:30
&
2.613
&
1.446
&
1.705
&
1.428
&
1.173
&
1.203
&
1.017
\\
\hline
19:00
&
2.207
&
1.473
&
1.627
&
1.405
&
1.029
&
0.846
&
0.776
\\
\hline
19:30
&
1.801
&
1.5
&
1.549
&
1.383
&
0.885
&
0.489
&
0.536
\\
\hline
20:00
&
1.489
&
1.517
&
1.38
&
1.254
&
0.759
&
0.485
&
0.522
\\
\hline
20:30
&
1.176
&
1.534
&
1.211
&
1.125
&
0.634
&
0.481
&
0.509
\\
\hline
21:00
&
1.012
&
1.465
&
1.106
&
1.019
&
0.546
&
0.403
&
0.448
\\
\hline
21:30
&
0.848
&
1.397
&
1.001
&
0.914
&
0.458
&
0.326
&
0.388
\\
\hline
22:00
&
0.741
&
1.366
&
0.937
&
0.848
&
0.403
&
0.354
&
0.4
\\
\hline
22:30
&
0.633
&
1.334
&
0.873
&
0.781
&
0.347
&
0.383
&
0.412
\\
\hline
23:00
&
0.551
&
1.351
&
0.786
&
0.782
&
0.315
&
0.429
&
0.445
\\
\hline
23:30
&
0.47
&
1.369
&
0.698
&
0.784
&
0.283
&
0.476
&
0.477
\\
\hline
00:00
&
0.358
&
1.271
&
0.557
&
0.702
&
0.245
&
0.465
&
0.466
\\
\hline
00:30
&
0.247
&
1.174
&
0.415
&
0.619
&
0.207
&
0.453
&
0.454
\\
\hline
01:00
&
0.186
&
1.047
&
0.325
&
0.451
&
0.179
&
0.41
&
0.417
\\
\hline
01:30
&
0.126
&
0.92
&
0.235
&
0.283
&
0.152
&
0.367
&
0.38
\\
\hline
02:00
&
0.097
&
0.675
&
0.201
&
0.245
&
0.144
&
0.394
&
0.393
\\
\hline
02:30
&
0.069
&
0.43
&
0.166
&
0.206
&
0.135
&
0.422
&
0.407
\\
\hline
03:00
&
0.068
&
0.364
&
0.154
&
0.197
&
0.155
&
0.37
&
0.382
\\
\hline
03:30
&
0.066
&
0.299
&
0.141
&
0.187
&
0.174
&
0.319
&
0.356
\\
\hline
04:00
&
0.09
&
0.288
&
0.156
&
0.213
&
0.224
&
0.431
&
0.445
\\
\hline
04:30
&
0.114
&
0.278
&
0.171
&
0.239
&
0.273
&
0.542
&
0.533
\\
\hline
05:00
&
0.207
&
0.358
&
0.236
&
0.383
&
0.356
&
0.73
&
0.692
\\
\hline
05:30
&
0.301
&
0.437
&
0.301
&
0.527
&
0.438
&
0.917
&
0.852
\\
\hline
06:00
&
0.73
&
0.589
&
0.563
&
0.752
&
0.974
&
1.246
&
1.21
\\
\hline
06:30
&
1.159
&
0.741
&
0.825
&
0.978
&
1.509
&
1.575
&
1.568
\\
\hline
07:00
&
1.677
&
0.815
&
1.125
&
1.226
&
1.834
&
1.555
&
1.706
\\
\hline
07:30
&
2.195
&
0.888
&
1.424
&
1.475
&
2.158
&
1.535
&
1.843
\\
\hline
08:00
&
2.456
&
0.903
&
1.511
&
1.483
&
1.938
&
1.549
&
1.798
\\
\hline
08:30
&
2.718
&
0.918
&
1.598
&
1.492
&
1.718
&
1.563
&
1.753
\\
\hline
09:00
&
2.25
&
0.932
&
1.479
&
1.546
&
1.738
&
1.76
&
1.861
\\
\hline
09:30
&
1.781
&
0.946
&
1.361
&
1.6
&
1.757
&
1.956
&
1.969
\\
\hline
10:00
&
1.51
&
1.022
&
1.279
&
1.559
&
1.793
&
2.039
&
2.034
\\
\hline
10:30
&
1.239
&
1.098
&
1.197
&
1.519
&
1.828
&
2.121
&
2.099
\\
\hline
11:00
&
1.216
&
1.143
&
1.194
&
1.465
&
1.842
&
2.167
&
2.114
\\
\hline
11:30
&
1.193
&
1.188
&
1.19
&
1.412
&
1.856
&
2.214
&
2.129
\\
\hline
12:00
&
1.22
&
1.176
&
1.221
&
1.396
&
1.84
&
2.214
&
2.126
\\
\hline
12:30
&
1.247
&
1.164
&
1.251
&
1.38
&
1.824
&
2.214
&
2.124
\\
\hline
13:00
&
1.29
&
1.19
&
1.259
&
1.394
&
1.82
&
2.158
&
2.071
\\
\hline
13:30
&
1.334
&
1.216
&
1.268
&
1.408
&
1.816
&
2.102
&
2.017
\\
\hline
14:00
&
1.346
&
1.334
&
1.282
&
1.45
&
1.848
&
2.052
&
1.939
\\
\hline
14:30
&
1.358
&
1.453
&
1.297
&
1.492
&
1.881
&
2.001
&
1.861
\\
\hline
15:00
&
1.431
&
1.375
&
1.364
&
1.514
&
1.914
&
1.93
&
1.748
\\
\hline
15:30
&
1.505
&
1.297
&
1.43
&
1.535
&
1.946
&
1.859
&
1.634
\\
\hline
16:00
&
1.741
&
1.392
&
1.509
&
1.567
&
1.966
&
1.734
&
1.529
\\
\hline
16:30
&
1.977
&
1.488
&
1.587
&
1.598
&
1.986
&
1.61
&
1.423
\\
\hline
17:00
&
2.335
&
1.511
&
1.677
&
1.535
&
1.789
&
1.482
&
1.296
\\
\hline
17:30
&
2.693
&
1.534
&
1.766
&
1.473
&
1.593
&
1.353
&
1.169
\\
\hline
18:00
&
2.659
&
1.493
&
1.739
&
1.453
&
1.386
&
1.281
&
1.095
\\
\hline
18:30
&
2.625
&
1.452
&
1.713
&
1.434
&
1.178
&
1.208
&
1.021
\\
\hline
19:00
&
2.207
&
1.472
&
1.626
&
1.404
&
1.029
&
0.847
&
0.777
\\
\hline
19:30
&
1.789
&
1.491
&
1.54
&
1.374
&
0.879
&
0.486
&
0.532
\\
\hline
20:00
&
1.479
&
1.508
&
1.372
&
1.246
&
0.754
&
0.482
&
0.519
\\
\hline
20:30
&
1.168
&
1.524
&
1.203
&
1.117
&
0.63
&
0.478
&
0.505
\\
\hline
21:00
&
1.005
&
1.457
&
1.099
&
1.013
&
0.542
&
0.401
&
0.445
\\
\hline
21:30
&
0.843
&
1.389
&
0.995
&
0.908
&
0.455
&
0.324
&
0.385
\\
\hline
22:00
&
0.736
&
1.358
&
0.931
&
0.842
&
0.4
&
0.352
&
0.397
\\
\hline
22:30
&
0.628
&
1.326
&
0.868
&
0.776
&
0.345
&
0.38
&
0.409
\\
\hline
23:00
&
0.547
&
1.343
&
0.781
&
0.777
&
0.313
&
0.427
&
0.442
\\
\hline
23:30
&
0.466
&
1.361
&
0.694
&
0.779
&
0.281
&
0.473
&
0.474
\\
\hline
00:00
&
0.337
&
1.188
&
0.526
&
0.66
&
0.23
&
0.439
&
0.439
\\
\hline
00:30
&
0.207
&
1.015
&
0.358
&
0.54
&
0.18
&
0.405
&
0.403
\\
\hline
01:00
&
0.156
&
0.905
&
0.281
&
0.394
&
0.156
&
0.367
&
0.37
\\
\hline
01:30
&
0.105
&
0.795
&
0.203
&
0.247
&
0.132
&
0.328
&
0.337
\\
\hline
02:00
&
0.082
&
0.583
&
0.173
&
0.213
&
0.125
&
0.352
&
0.349
\\
\hline
02:30
&
0.058
&
0.372
&
0.144
&
0.179
&
0.118
&
0.377
&
0.361
\\
\hline
03:00
&
0.057
&
0.315
&
0.133
&
0.171
&
0.134
&
0.331
&
0.339
\\
\hline
03:30
&
0.055
&
0.258
&
0.122
&
0.163
&
0.151
&
0.285
&
0.317
\\
\hline
04:00
&
0.075
&
0.249
&
0.134
&
0.186
&
0.194
&
0.385
&
0.395
\\
\hline
04:30
&
0.095
&
0.241
&
0.147
&
0.208
&
0.238
&
0.485
&
0.473
\\
\hline
05:00
&
0.174
&
0.309
&
0.204
&
0.334
&
0.309
&
0.653
&
0.615
\\
\hline
05:30
&
0.252
&
0.378
&
0.26
&
0.46
&
0.381
&
0.821
&
0.757
\\
\hline
06:00
&
0.612
&
0.509
&
0.486
&
0.656
&
0.847
&
1.115
&
1.075
\\
\hline
06:30
&
0.972
&
0.641
&
0.712
&
0.853
&
1.313
&
1.409
&
1.393
\\
\hline
07:00
&
1.155
&
0.599
&
0.783
&
0.905
&
1.309
&
1.144
&
1.223
\\
\hline
07:30
&
1.338
&
0.556
&
0.854
&
0.957
&
1.305
&
0.88
&
1.053
\\
\hline
08:00
&
1.567
&
0.584
&
0.944
&
0.979
&
1.217
&
0.967
&
1.115
\\
\hline
08:30
&
1.796
&
0.612
&
1.035
&
1
&
1.129
&
1.054
&
1.177
\\
\hline
09:00
&
1.606
&
0.683
&
1.07
&
1.145
&
1.279
&
1.342
&
1.411
\\
\hline
09:30
&
1.416
&
0.753
&
1.105
&
1.289
&
1.429
&
1.63
&
1.644
\\
\hline
10:00
&
1.286
&
0.884
&
1.13
&
1.359
&
1.592
&
1.83
&
1.84
\\
\hline
10:30
&
1.155
&
1.015
&
1.155
&
1.428
&
1.756
&
2.03
&
2.035
\\
\hline
11:00
&
1.167
&
1.09
&
1.185
&
1.411
&
1.818
&
2.15
&
2.118
\\
\hline
11:30
&
1.178
&
1.165
&
1.215
&
1.394
&
1.88
&
2.27
&
2.201
\\
\hline
12:00
&
1.213
&
1.161
&
1.247
&
1.385
&
1.87
&
2.288
&
2.2
\\
\hline
12:30
&
1.247
&
1.157
&
1.279
&
1.376
&
1.859
&
2.306
&
2.199
\\
\hline
13:00
&
1.299
&
1.19
&
1.289
&
1.395
&
1.858
&
2.235
&
2.138
\\
\hline
13:30
&
1.351
&
1.224
&
1.299
&
1.413
&
1.857
&
2.164
&
2.076
\\
\hline
14:00
&
1.328
&
1.307
&
1.28
&
1.423
&
1.84
&
2.048
&
1.938
\\
\hline
14:30
&
1.305
&
1.39
&
1.261
&
1.433
&
1.823
&
1.932
&
1.8
\\
\hline
15:00
&
1.335
&
1.28
&
1.27
&
1.407
&
1.784
&
1.791
&
1.629
\\
\hline
15:30
&
1.365
&
1.17
&
1.278
&
1.382
&
1.745
&
1.651
&
1.458
\\
\hline
16:00
&
1.517
&
1.212
&
1.289
&
1.367
&
1.688
&
1.476
&
1.298
\\
\hline
16:30
&
1.669
&
1.254
&
1.3
&
1.351
&
1.631
&
1.301
&
1.138
\\
\hline
17:00
&
1.951
&
1.268
&
1.358
&
1.293
&
1.457
&
1.202
&
1.045
\\
\hline
17:30
&
2.232
&
1.281
&
1.416
&
1.234
&
1.283
&
1.103
&
0.953
\\
\hline
18:00
&
2.262
&
1.284
&
1.472
&
1.26
&
1.17
&
1.114
&
0.955
\\
\hline
18:30
&
2.293
&
1.286
&
1.529
&
1.285
&
1.057
&
1.126
&
0.958
\\
\hline
19:00
&
1.897
&
1.288
&
1.429
&
1.242
&
0.911
&
0.78
&
0.715
\\
\hline
19:30
&
1.501
&
1.289
&
1.329
&
1.199
&
0.765
&
0.435
&
0.473
\\
\hline
20:00
&
1.24
&
1.303
&
1.184
&
1.087
&
0.657
&
0.431
&
0.461
\\
\hline
20:30
&
0.98
&
1.318
&
1.038
&
0.975
&
0.548
&
0.427
&
0.449
\\
\hline
21:00
&
0.843
&
1.259
&
0.949
&
0.883
&
0.472
&
0.358
&
0.396
\\
\hline
21:30
&
0.707
&
1.201
&
0.859
&
0.792
&
0.396
&
0.289
&
0.342
\\
\hline
22:00
&
0.617
&
1.174
&
0.804
&
0.734
&
0.348
&
0.315
&
0.353
\\
\hline
22:30
&
0.527
&
1.147
&
0.749
&
0.677
&
0.3
&
0.34
&
0.363
\\
\hline
23:00
&
0.459
&
1.162
&
0.674
&
0.678
&
0.272
&
0.381
&
0.392
\\
\hline
23:30
&
0.391
&
1.176
&
0.599
&
0.679
&
0.244
&
0.423
&
0.421
\\
\hline
00:00
&
0.203
&
0.899
&
0.53
&
0.55
&
0.179
&
0.221
&
0.245
\\
\hline
00:30
&
0.015
&
0.622
&
0.46
&
0.421
&
0.114
&
0.019
&
0.07
\\
\hline
01:00
&
0.012
&
0.523
&
0.367
&
0.315
&
0.094
&
0.017
&
0.061
\\
\hline
01:30
&
0.009
&
0.425
&
0.275
&
0.209
&
0.075
&
0.014
&
0.052
\\
\hline
02:00
&
0.007
&
0.357
&
0.231
&
0.168
&
0.065
&
0.014
&
0.052
\\
\hline
02:30
&
0.005
&
0.288
&
0.188
&
0.128
&
0.055
&
0.014
&
0.052
\\
\hline
03:00
&
0.006
&
0.262
&
0.181
&
0.136
&
0.054
&
0.017
&
0.062
\\
\hline
03:30
&
0.007
&
0.237
&
0.174
&
0.145
&
0.053
&
0.02
&
0.073
\\
\hline
04:00
&
0.007
&
0.231
&
0.187
&
0.2
&
0.067
&
0.023
&
0.083
\\
\hline
04:30
&
0.007
&
0.226
&
0.2
&
0.255
&
0.081
&
0.026
&
0.093
\\
\hline
05:00
&
0.008
&
0.257
&
0.254
&
0.399
&
0.138
&
0.044
&
0.156
\\
\hline
05:30
&
0.01
&
0.287
&
0.308
&
0.542
&
0.194
&
0.062
&
0.219
\\
\hline
06:00
&
0.014
&
0.304
&
0.404
&
0.691
&
0.304
&
0.082
&
0.288
\\
\hline
06:30
&
0.018
&
0.32
&
0.501
&
0.839
&
0.413
&
0.102
&
0.357
\\
\hline
07:00
&
0.024
&
0.365
&
0.6
&
0.932
&
0.533
&
0.118
&
0.413
\\
\hline
07:30
&
0.029
&
0.409
&
0.7
&
1.025
&
0.653
&
0.134
&
0.468
\\
\hline
08:00
&
0.032
&
0.481
&
0.761
&
1.033
&
0.66
&
0.124
&
0.433
\\
\hline
08:30
&
0.035
&
0.553
&
0.823
&
1.041
&
0.667
&
0.114
&
0.398
\\
\hline
09:00
&
0.038
&
0.646
&
0.976
&
1.057
&
0.666
&
0.099
&
0.347
\\
\hline
09:30
&
0.041
&
0.738
&
1.129
&
1.073
&
0.665
&
0.085
&
0.297
\\
\hline
10:00
&
0.045
&
0.795
&
1.281
&
1.021
&
0.682
&
0.082
&
0.288
\\
\hline
10:30
&
0.049
&
0.852
&
1.433
&
0.969
&
0.698
&
0.08
&
0.28
\\
\hline
11:00
&
0.047
&
0.889
&
1.511
&
0.915
&
0.712
&
0.074
&
0.259
\\
\hline
11:30
&
0.046
&
0.926
&
1.589
&
0.861
&
0.726
&
0.068
&
0.238
\\
\hline
12:00
&
0.049
&
0.912
&
1.639
&
0.844
&
0.714
&
0.062
&
0.219
\\
\hline
12:30
&
0.052
&
0.897
&
1.689
&
0.827
&
0.703
&
0.057
&
0.2
\\
\hline
13:00
&
0.052
&
0.908
&
1.705
&
0.828
&
0.695
&
0.054
&
0.19
\\
\hline
13:30
&
0.052
&
0.919
&
1.721
&
0.83
&
0.688
&
0.051
&
0.18
\\
\hline
14:00
&
0.055
&
0.925
&
1.725
&
0.847
&
0.676
&
0.051
&
0.178
\\
\hline
14:30
&
0.058
&
0.931
&
1.729
&
0.863
&
0.665
&
0.05
&
0.177
\\
\hline
15:00
&
0.053
&
0.954
&
1.717
&
0.878
&
0.65
&
0.048
&
0.17
\\
\hline
15:30
&
0.049
&
0.978
&
1.704
&
0.892
&
0.634
&
0.046
&
0.163
\\
\hline
16:00
&
0.052
&
0.982
&
1.694
&
0.915
&
0.619
&
0.045
&
0.159
\\
\hline
16:30
&
0.056
&
0.986
&
1.684
&
0.938
&
0.605
&
0.044
&
0.155
\\
\hline
17:00
&
0.054
&
1.005
&
1.693
&
0.931
&
0.597
&
0.044
&
0.154
\\
\hline
17:30
&
0.052
&
1.025
&
1.702
&
0.924
&
0.59
&
0.043
&
0.153
\\
\hline
18:00
&
0.054
&
1.027
&
1.717
&
0.935
&
0.586
&
0.045
&
0.16
\\
\hline
18:30
&
0.056
&
1.03
&
1.733
&
0.946
&
0.582
&
0.047
&
0.167
\\
\hline
19:00
&
0.05
&
0.982
&
1.558
&
0.884
&
0.523
&
0.047
&
0.167
\\
\hline
19:30
&
0.045
&
0.934
&
1.383
&
0.821
&
0.465
&
0.047
&
0.166
\\
\hline
20:00
&
0.04
&
0.871
&
1.239
&
0.78
&
0.413
&
0.055
&
0.194
\\
\hline
20:30
&
0.035
&
0.807
&
1.095
&
0.739
&
0.362
&
0.063
&
0.221
\\
\hline
21:00
&
0.032
&
0.754
&
1
&
0.719
&
0.328
&
0.065
&
0.227
\\
\hline
21:30
&
0.03
&
0.701
&
0.905
&
0.699
&
0.294
&
0.066
&
0.233
\\
\hline
22:00
&
0.029
&
0.699
&
0.874
&
0.723
&
0.281
&
0.063
&
0.222
\\
\hline
22:30
&
0.029
&
0.697
&
0.842
&
0.747
&
0.269
&
0.06
&
0.21
\\
\hline
23:00
&
0.027
&
0.679
&
0.761
&
0.77
&
0.255
&
0.065
&
0.227
\\
\hline
23:30
&
0.025
&
0.661
&
0.68
&
0.793
&
0.241
&
0.069
&
0.244
\\
\hline
00:00
&
0.131
&
0.893
&
0.539
&
0.693
&
0.22
&
0.252
&
0.34
\\
\hline
\end{longtable}\sphinxatlongtableend\end{savenotes}


\subparagraph{Building energy profiles}
\label{\detokenize{OtherManuals/GQF_Manual:building-energy-profiles}}

\subparagraph{Seasonal variations}
\label{\detokenize{OtherManuals/GQF_Manual:seasonal-variations}}
This file records daily variations in total gas and electricity
consumption over a wide area, so that seasonal variations are
reconstructed by the model. The values in the files are converted to
scaling factors when the file is read by the model software, so the unit
of measurement is not important.

The file consists of three columns. The first is the day of year; the
second and third must be headed “Elec” and “Gas” for electricity and gas
consumption, respectively. Based on the start and end date chosen, the
file must contain 365 or 366 entries. A truncated example of the file
covering the first 7 days of the year is shown below to demonstrate the
format:


\begin{savenotes}\sphinxattablestart
\centering
\begin{tabular}[t]{|\X{33}{99}|\X{33}{99}|\X{33}{99}|}
\hline
\sphinxstyletheadfamily 
Fuel
&\sphinxstyletheadfamily 
Elec
&\sphinxstyletheadfamily 
Gas
\\
\hline
StartDate
&
2008-01-01
&\\
\hline
EndDate
&
2008-12-31
&\\
\hline
Timezone
&
Europe/London
&\\
\hline
1
&
0.942515348
&
1.097280899
\\
\hline
2
&
1.133871156
&
1.309574671
\\
\hline
3
&
1.237227268
&
1.461329099
\\
\hline
4
&
1.214487757
&
1.346215615
\\
\hline
5
&
1.063433309
&
1.251089375
\\
\hline
6
&
1.046604939
&
1.258738219
\\
\hline
7
&
1.195052511
&
1.347154599
\\
\hline
\end{tabular}
\par
\sphinxattableend\end{savenotes}


\subparagraph{Diurnal variations}
\label{\detokenize{OtherManuals/GQF_Manual:diurnal-variations}}
Each file contains triplets of 24-hour cycles at 30 minute resolution
showing the relative variation of energy use during (i) a weekday, (ii)
a Saturday and (iii) a Sunday.

Note that five separate input files must be provided for domestic
electricity, domestic gas, industrial electricity, industrial gas and
Economy 7 diurnal cycles. The link between file and energy type is made
in the {\hyperref[\detokenize{OtherManuals/GQF_Manual:data-sources-file}]{\sphinxcrossref{Data sources file}}}.

Aside from the standard headers, this file contains headers for:
\begin{itemize}
\item {} 
\sphinxstylestrong{Season}: A name for the period represented by each triplet of
columns. Must be consistent within each triplet.

\item {} 
\sphinxstylestrong{Day of week} represented by the cycle: “Wd”: Weekday, “Sat”:
Saturday or “Sun”: Sunday

\item {} 
\sphinxstylestrong{Tariff}: A brief description of tariff (for user information only)

\end{itemize}

The values for each day are normalised inside the model software so that
they average to 1.

An example is shown below for a diurnal variations file that contains
entries for 2014: Autumn (Aut), High Summer (HSr), Summer (Smr), Spring
(Spr) and Winter (Wtr), which appears at the start and end of the year
so that 2014 is fully covered. Any number of seasons/periods of year can
be added to a single file.

The actual file contains 48 rows of data, but the version shown here is
shortened.


\begin{savenotes}\sphinxattablestart
\centering
\begin{tabular}[t]{|\X{5}{95}|\X{5}{95}|\X{5}{95}|\X{5}{95}|\X{5}{95}|\X{5}{95}|\X{5}{95}|\X{5}{95}|\X{5}{95}|\X{5}{95}|\X{5}{95}|\X{5}{95}|\X{5}{95}|\X{5}{95}|\X{5}{95}|\X{5}{95}|\X{5}{95}|\X{5}{95}|\X{5}{95}|}
\hline
\sphinxstyletheadfamily 
Season
&\sphinxstyletheadfamily 
Aut
&\sphinxstyletheadfamily 
Aut
&\sphinxstyletheadfamily 
Aut
&\sphinxstyletheadfamily 
HSr
&\sphinxstyletheadfamily 
HSr
&\sphinxstyletheadfamily 
HSr
&\sphinxstyletheadfamily 
Smr
&\sphinxstyletheadfamily 
Smr
&\sphinxstyletheadfamily 
Smr
&\sphinxstyletheadfamily 
Spr
&\sphinxstyletheadfamily 
Spr
&\sphinxstyletheadfamily 
Spr
&\sphinxstyletheadfamily 
Wtr\_1
&\sphinxstyletheadfamily 
Wtr\_1
&\sphinxstyletheadfamily 
Wtr\_1
&\sphinxstyletheadfamily 
Wtr\_2
&\sphinxstyletheadfamily 
Wtr\_2
&\sphinxstyletheadfamily 
Wtr\_2
\\
\hline
Day
&
Wd
&
Sat
&
Sun
&
Wd
&
Sat
&
Sun
&
Wd
&
Sat
&
Sun
&
Wd
&
Sat
&
Sun
&
Wd
&
Sat
&
Sun
&
Wd
&
Sat
&
Sun
\\
\hline
Tariff
&
DomUnr
&
DomUnr
&
DomUnr
&
DomUnr
&
DomUnr
&
DomUnr
&
DomUnr
&
DomUnr
&
DomUnr
&
DomUnr
&
DomUnr
&
DomUnr
&
DomUnr
&
DomUnr
&
DomUnr
&
DomUnr
&
DomUnr
&
DomUnr
\\
\hline
StartDate
&
2014-09-01
&
2014-09-01
&
2014-09-01
&
2014-07-20
&
2014-07-20
&
2014-07-20
&
2014-05-11
&
2014-05-11
&
2014-05-11
&
2014-03-30
&
2014-03-30
&
2014-03-30
&
2014-01-01
&
2014-01-01
&
2014-01-01
&
2014-10-26
&
2014-10-26
&
2014-10-26
\\
\hline
EndDate
&
2014-10-25
&
2014-10-25
&
2014-10-25
&
2014-08-31
&
2014-08-31
&
2014-08-31
&
2014-07-19
&
2014-07-19
&
2014-07-19
&
2014-05-10
&
2014-05-10
&
2014-05-10
&
2014-03-29
&
2014-03-29
&
2014-03-29
&
2014-12-31
&
2014-12-31
&
2014-12-31
\\
\hline
Timezone
&
Europe/London
&&&&&&&&&&&&&&&&&\\
\hline
00:30
&
0.31
&
0.33
&
0.339
&
0.315
&
0.325
&
0.324
&
0.314
&
0.333
&
0.344
&
0.338
&
0.351
&
0.366
&
0.352
&
0.387
&
0.391
&
0.352
&
0.387
&
0.391
\\
\hline
01:00
&
0.273
&
0.294
&
0.306
&
0.287
&
0.291
&
0.296
&
0.276
&
0.301
&
0.306
&
0.304
&
0.312
&
0.312
&
0.313
&
0.344
&
0.348
&
0.313
&
0.344
&
0.348
\\
\hline
01:30
&
0.252
&
0.268
&
0.277
&
0.26
&
0.269
&
0.276
&
0.256
&
0.271
&
0.28
&
0.279
&
0.304
&
0.286
&
0.294
&
0.322
&
0.32
&
0.294
&
0.322
&
0.32
\\
\hline
02:00
&
0.236
&
0.248
&
0.259
&
0.242
&
0.249
&
0.255
&
0.247
&
0.249
&
0.259
&
0.258
&
0.262
&
0.271
&
0.278
&
0.3
&
0.299
&
0.278
&
0.3
&
0.299
\\
\hline
02:30
&
0.23
&
0.24
&
0.249
&
0.234
&
0.238
&
0.243
&
0.229
&
0.236
&
0.241
&
0.25
&
0.251
&
0.26
&
0.266
&
0.284
&
0.283
&
0.266
&
0.284
&
0.283
\\
\hline
…
&
…
&
…
&
…
&
…
&
…
&
…
&
…
&
…
&
…
&
…
&
…
&
…
&
…
&
…
&
…
&
…
&
…
&
…
\\
\hline
23:00
&
0.496
&
0.488
&
0.497
&
0.474
&
0.469
&
0.467
&
0.481
&
0.481
&
0.485
&
0.532
&
0.503
&
0.513
&
0.566
&
0.576
&
0.57
&
0.566
&
0.576
&
0.57
\\
\hline
23:30
&
0.423
&
0.443
&
0.423
&
0.415
&
0.424
&
0.404
&
0.438
&
0.43
&
0.425
&
0.461
&
0.469
&
0.396
&
0.487
&
0.518
&
0.485
&
0.487
&
0.518
&
0.485
\\
\hline
00:00
&
0.36
&
0.393
&
0.358
&
0.359
&
0.374
&
0.353
&
0.377
&
0.396
&
0.366
&
0.39
&
0.367
&
0.335
&
0.414
&
0.452
&
0.415
&
0.414
&
0.452
&
0.415
\\
\hline
\end{tabular}
\par
\sphinxattableend\end{savenotes}


\subparagraph{Metabolic activity}
\label{\detokenize{OtherManuals/GQF_Manual:metabolic-activity}}
Metabolism profiles contain multiple seasons per file and describe the
variation in metabolic activity of the whole population on the average
weekday, Saturday and Sunday at 30-minute intervals. Each weekday,
Saturday and Sunday has 2 columns: \sphinxstylestrong{Energy} emitted per person, and
\sphinxstylestrong{Fraction} of residents who are at work at each point in the day. Both
workers and residents are assumed to emit the same amount of heat per
person at each time of day.

Headers specific to this file:
\begin{itemize}
\item {} 
\sphinxstylestrong{Season}: A name for the season being described. Must be consistent
within all six columns describing a season

\item {} 
\sphinxstylestrong{Day}: “Weekday”, “Saturday” or “Sunday”, exactly as shown below

\item {} 
\sphinxstylestrong{Type}: “Energy” and “Fraction” as described above.

\end{itemize}


\begin{savenotes}\sphinxatlongtablestart\begin{longtable}{|\X{5}{95}|\X{5}{95}|\X{5}{95}|\X{5}{95}|\X{5}{95}|\X{5}{95}|\X{5}{95}|\X{5}{95}|\X{5}{95}|\X{5}{95}|\X{5}{95}|\X{5}{95}|\X{5}{95}|\X{5}{95}|\X{5}{95}|\X{5}{95}|\X{5}{95}|\X{5}{95}|\X{5}{95}|}
\hline
\sphinxstyletheadfamily 
Season
&\sphinxstyletheadfamily 
GMT
&\sphinxstyletheadfamily 
GMT
&\sphinxstyletheadfamily 
GMT
&\sphinxstyletheadfamily 
GMT
&\sphinxstyletheadfamily 
GMT
&\sphinxstyletheadfamily 
GMT
&\sphinxstyletheadfamily 
BST
&\sphinxstyletheadfamily 
BST
&\sphinxstyletheadfamily 
BST
&\sphinxstyletheadfamily 
BST
&\sphinxstyletheadfamily 
BST
&\sphinxstyletheadfamily 
BST
&\sphinxstyletheadfamily 
GMT2
&\sphinxstyletheadfamily 
GMT2
&\sphinxstyletheadfamily 
GMT2
&\sphinxstyletheadfamily 
GMT2
&\sphinxstyletheadfamily 
GMT2
&\sphinxstyletheadfamily 
GMT2
\\
\hline
\endfirsthead

\multicolumn{19}{c}%
{\makebox[0pt]{\sphinxtablecontinued{\tablename\ \thetable{} -- continued from previous page}}}\\
\hline
\sphinxstyletheadfamily 
Season
&\sphinxstyletheadfamily 
GMT
&\sphinxstyletheadfamily 
GMT
&\sphinxstyletheadfamily 
GMT
&\sphinxstyletheadfamily 
GMT
&\sphinxstyletheadfamily 
GMT
&\sphinxstyletheadfamily 
GMT
&\sphinxstyletheadfamily 
BST
&\sphinxstyletheadfamily 
BST
&\sphinxstyletheadfamily 
BST
&\sphinxstyletheadfamily 
BST
&\sphinxstyletheadfamily 
BST
&\sphinxstyletheadfamily 
BST
&\sphinxstyletheadfamily 
GMT2
&\sphinxstyletheadfamily 
GMT2
&\sphinxstyletheadfamily 
GMT2
&\sphinxstyletheadfamily 
GMT2
&\sphinxstyletheadfamily 
GMT2
&\sphinxstyletheadfamily 
GMT2
\\
\hline
\endhead

\hline
\multicolumn{19}{r}{\makebox[0pt][r]{\sphinxtablecontinued{Continued on next page}}}\\
\endfoot

\endlastfoot

Day
&
Weekday
&
Weekday
&
Saturday
&
Saturday
&
Sunday
&
Sunday
&
Weekday
&
Weekday
&
Saturday
&
Saturday
&
Sunday
&
Sunday
&
Weekday
&
Weekday
&
Saturday
&
Saturday
&
Sunday
&
Sunday
\\
\hline
Type
&
Energy
&
Fraction
&
Energy
&
Fraction
&
Energy
&
Fraction
&
Energy
&
Fraction
&
Energy
&
Fraction
&
Energy
&
Fraction
&
Energy
&
Fraction
&
Energy
&
Fraction
&
Energy
&
Fraction
\\
\hline
StartDate
&
2008-01-01
&
2008-01-01
&
2008-01-01
&
2008-01-01
&
2008-01-01
&
2008-01-01
&
2008-03-30
&
2008-03-30
&
2008-03-30
&
2008-03-30
&
2008-03-30
&
2008-03-30
&
2008-10-26
&
2008-10-26
&
2008-10-26
&
2008-10-26
&
2008-10-26
&
2008-10-26
\\
\hline
EndDate
&
2008-03-29
&
2008-03-29
&
2008-03-29
&
2008-03-29
&
2008-03-29
&
2008-03-29
&
2008-10-25
&
2008-10-25
&
2008-10-25
&
2008-10-25
&
2008-10-25
&
2008-10-25
&
2008-12-31
&
2008-12-31
&
2008-12-31
&
2008-12-31
&
2008-12-31
&
2008-12-31
\\
\hline
Timezone
&
Europe/London
&&&&&&&&&&&&&&&&&\\
\hline
00:30
&
64.3
&
0
&
0
&
0
&
0
&
0
&
64.3
&
0
&
0
&
0
&
0
&
0
&
64.3
&
0
&
0
&
0
&
0
&
0
\\
\hline
01:00
&
64.3
&
0
&
0
&
0
&
0
&
0
&
64.3
&
0
&
0
&
0
&
0
&
0
&
64.3
&
0
&
0
&
0
&
0
&
0
\\
\hline
01:30
&
64.3
&
0
&
0
&
0
&
0
&
0
&
64.3
&
0
&
0
&
0
&
0
&
0
&
64.3
&
0
&
0
&
0
&
0
&
0
\\
\hline
02:00
&
64.3
&
0
&
0
&
0
&
0
&
0
&
64.3
&
0
&
0
&
0
&
0
&
0
&
64.3
&
0
&
0
&
0
&
0
&
0
\\
\hline
02:30
&
64.3
&
0
&
0
&
0
&
0
&
0
&
64.3
&
0
&
0
&
0
&
0
&
0
&
64.3
&
0
&
0
&
0
&
0
&
0
\\
\hline
03:00
&
64.3
&
0
&
0
&
0
&
0
&
0
&
64.3
&
0
&
0
&
0
&
0
&
0
&
64.3
&
0
&
0
&
0
&
0
&
0
\\
\hline
03:30
&
64.3
&
0
&
0
&
0
&
0
&
0
&
64.3
&
0
&
0
&
0
&
0
&
0
&
64.3
&
0
&
0
&
0
&
0
&
0
\\
\hline
04:00
&
64.3
&
0
&
0
&
0
&
0
&
0
&
64.3
&
0
&
0
&
0
&
0
&
0
&
64.3
&
0
&
0
&
0
&
0
&
0
\\
\hline
04:30
&
64.3
&
0
&
0
&
0
&
0
&
0
&
64.3
&
0
&
0
&
0
&
0
&
0
&
64.3
&
0
&
0
&
0
&
0
&
0
\\
\hline
05:00
&
64.3
&
0
&
0
&
0
&
0
&
0
&
64.3
&
0
&
0
&
0
&
0
&
0
&
64.3
&
0
&
0
&
0
&
0
&
0
\\
\hline
05:30
&
64.3
&
0
&
0
&
0
&
0
&
0
&
64.3
&
0
&
0
&
0
&
0
&
0
&
64.3
&
0
&
0
&
0
&
0
&
0
\\
\hline
06:00
&
64.3
&
0
&
0
&
0
&
0
&
0
&
64.3
&
0
&
0
&
0
&
0
&
0
&
64.3
&
0
&
0
&
0
&
0
&
0
\\
\hline
06:30
&
64.3
&
0
&
0
&
0
&
0
&
0
&
64.3
&
0
&
0
&
0
&
0
&
0
&
64.3
&
0
&
0
&
0
&
0
&
0
\\
\hline
07:00
&
68
&
0
&
0
&
0
&
0
&
0
&
68
&
0
&
0
&
0
&
0
&
0
&
68
&
0
&
0
&
0
&
0
&
0
\\
\hline
07:30
&
80
&
0.02
&
0
&
0
&
0
&
0
&
80
&
0.02
&
0
&
0
&
0
&
0
&
80
&
0.02
&
0
&
0
&
0
&
0
\\
\hline
08:00
&
110
&
0.08
&
0
&
0
&
0
&
0
&
110
&
0.08
&
0
&
0
&
0
&
0
&
110
&
0.08
&
0
&
0
&
0
&
0
\\
\hline
08:30
&
150
&
0.2
&
0
&
0
&
0
&
0
&
150
&
0.2
&
0
&
0
&
0
&
0
&
150
&
0.2
&
0
&
0
&
0
&
0
\\
\hline
09:00
&
166
&
0.4
&
0
&
0
&
0
&
0
&
166
&
0.4
&
0
&
0
&
0
&
0
&
166
&
0.4
&
0
&
0
&
0
&
0
\\
\hline
09:30
&
170.5
&
0.6
&
0
&
0
&
0
&
0
&
170.5
&
0.6
&
0
&
0
&
0
&
0
&
170.5
&
0.6
&
0
&
0
&
0
&
0
\\
\hline
10:00
&
170.5
&
0.9
&
0
&
0
&
0
&
0
&
170.5
&
0.9
&
0
&
0
&
0
&
0
&
170.5
&
0.9
&
0
&
0
&
0
&
0
\\
\hline
10:30
&
170.5
&
0.98
&
0
&
0
&
0
&
0
&
170.5
&
0.98
&
0
&
0
&
0
&
0
&
170.5
&
0.98
&
0
&
0
&
0
&
0
\\
\hline
11:00
&
170.5
&
1
&
0
&
0
&
0
&
0
&
170.5
&
1
&
0
&
0
&
0
&
0
&
170.5
&
1
&
0
&
0
&
0
&
0
\\
\hline
11:30
&
170.5
&
1
&
0
&
0
&
0
&
0
&
170.5
&
1
&
0
&
0
&
0
&
0
&
170.5
&
1
&
0
&
0
&
0
&
0
\\
\hline
12:00
&
170.5
&
1
&
0
&
0
&
0
&
0
&
170.5
&
1
&
0
&
0
&
0
&
0
&
170.5
&
1
&
0
&
0
&
0
&
0
\\
\hline
12:30
&
170.5
&
1
&
0
&
0
&
0
&
0
&
170.5
&
1
&
0
&
0
&
0
&
0
&
170.5
&
1
&
0
&
0
&
0
&
0
\\
\hline
13:00
&
170.5
&
1
&
0
&
0
&
0
&
0
&
170.5
&
1
&
0
&
0
&
0
&
0
&
170.5
&
1
&
0
&
0
&
0
&
0
\\
\hline
13:30
&
170.5
&
1
&
0
&
0
&
0
&
0
&
170.5
&
1
&
0
&
0
&
0
&
0
&
170.5
&
1
&
0
&
0
&
0
&
0
\\
\hline
14:00
&
170.5
&
1
&
0
&
0
&
0
&
0
&
170.5
&
1
&
0
&
0
&
0
&
0
&
170.5
&
1
&
0
&
0
&
0
&
0
\\
\hline
14:30
&
170.5
&
1
&
0
&
0
&
0
&
0
&
170.5
&
1
&
0
&
0
&
0
&
0
&
170.5
&
1
&
0
&
0
&
0
&
0
\\
\hline
15:00
&
170.5
&
1
&
0
&
0
&
0
&
0
&
170.5
&
1
&
0
&
0
&
0
&
0
&
170.5
&
1
&
0
&
0
&
0
&
0
\\
\hline
15:30
&
170.5
&
1
&
0
&
0
&
0
&
0
&
170.5
&
1
&
0
&
0
&
0
&
0
&
170.5
&
1
&
0
&
0
&
0
&
0
\\
\hline
16:00
&
170.5
&
1
&
0
&
0
&
0
&
0
&
170.5
&
1
&
0
&
0
&
0
&
0
&
170.5
&
1
&
0
&
0
&
0
&
0
\\
\hline
16:30
&
170.5
&
1
&
0
&
0
&
0
&
0
&
170.5
&
1
&
0
&
0
&
0
&
0
&
170.5
&
1
&
0
&
0
&
0
&
0
\\
\hline
17:00
&
170.5
&
0.98
&
0
&
0
&
0
&
0
&
170.5
&
0.98
&
0
&
0
&
0
&
0
&
170.5
&
0.98
&
0
&
0
&
0
&
0
\\
\hline
17:30
&
170.5
&
0.9
&
0
&
0
&
0
&
0
&
170.5
&
0.9
&
0
&
0
&
0
&
0
&
170.5
&
0.9
&
0
&
0
&
0
&
0
\\
\hline
18:00
&
170.5
&
0.6
&
0
&
0
&
0
&
0
&
170.5
&
0.6
&
0
&
0
&
0
&
0
&
170.5
&
0.6
&
0
&
0
&
0
&
0
\\
\hline
18:30
&
170.5
&
0.4
&
0
&
0
&
0
&
0
&
170.5
&
0.4
&
0
&
0
&
0
&
0
&
170.5
&
0.4
&
0
&
0
&
0
&
0
\\
\hline
19:00
&
170.5
&
0.2
&
0
&
0
&
0
&
0
&
170.5
&
0.2
&
0
&
0
&
0
&
0
&
170.5
&
0.2
&
0
&
0
&
0
&
0
\\
\hline
19:30
&
170.5
&
0.08
&
0
&
0
&
0
&
0
&
170.5
&
0.08
&
0
&
0
&
0
&
0
&
170.5
&
0.08
&
0
&
0
&
0
&
0
\\
\hline
20:00
&
170.5
&
0.02
&
0
&
0
&
0
&
0
&
170.5
&
0.02
&
0
&
0
&
0
&
0
&
170.5
&
0.02
&
0
&
0
&
0
&
0
\\
\hline
20:30
&
170.5
&
0
&
0
&
0
&
0
&
0
&
170.5
&
0
&
0
&
0
&
0
&
0
&
170.5
&
0
&
0
&
0
&
0
&
0
\\
\hline
21:00
&
170.5
&
0
&
0
&
0
&
0
&
0
&
170.5
&
0
&
0
&
0
&
0
&
0
&
170.5
&
0
&
0
&
0
&
0
&
0
\\
\hline
21:30
&
170.5
&
0
&
0
&
0
&
0
&
0
&
170.5
&
0
&
0
&
0
&
0
&
0
&
170.5
&
0
&
0
&
0
&
0
&
0
\\
\hline
22:00
&
166
&
0
&
0
&
0
&
0
&
0
&
166
&
0
&
0
&
0
&
0
&
0
&
166
&
0
&
0
&
0
&
0
&
0
\\
\hline
22:30
&
150
&
0
&
0
&
0
&
0
&
0
&
150
&
0
&
0
&
0
&
0
&
0
&
150
&
0
&
0
&
0
&
0
&
0
\\
\hline
23:00
&
110
&
0
&
0
&
0
&
0
&
0
&
110
&
0
&
0
&
0
&
0
&
0
&
110
&
0
&
0
&
0
&
0
&
0
\\
\hline
23:30
&
80
&
0
&
0
&
0
&
0
&
0
&
80
&
0
&
0
&
0
&
0
&
0
&
80
&
0
&
0
&
0
&
0
&
0
\\
\hline
00:00
&
68
&
0
&
0
&
0
&
0
&
0
&
68
&
0
&
0
&
0
&
0
&
0
&
68
&
0
&
0
&
0
&
0
&
0
\\
\hline
\end{longtable}\sphinxatlongtableend\end{savenotes}


\paragraph{Recycling of temporal data}
\label{\detokenize{OtherManuals/GQF_Manual:recycling-of-temporal-data}}
The model calculates fluxes for any date provided there is temporal data
for the corresponding time of year. If daily energy loadings and/or
diurnal cycles are not available for the date being modelled, a series
of lookups is performed on the available temporal data to find a
suitable match. This process accounts for changes in public holidays,
leap years and changing DST switch dates.

For diurnal cycle data, the lookup operates by building and then
reducing a shortlist of cycles that may be suitable:
\begin{enumerate}
\item {} 
Based on the modelled time step, cycles from a suitable year are
added to the shortlist. A year is deemed suitable if it contains data
covering the time of year being modelled
\begin{itemize}
\item {} 
If the modelled year is later than available data, the latest
suitable year is used

\item {} 
If the modelled year is earlier than the available data, the
earliest suitable year is used

\end{itemize}

\item {} 
The modelled day of week is established (set to Sunday if a public
holiday)

\item {} 
The lookup date is set as the same day of week, month and time of
month as the modelled date, but in the year identified as suitable.
\begin{itemize}
\item {} 
This operation sometimes causes late December dates to become
early January. Such dates are moved into the final week of
December.

\end{itemize}

\item {} 
The daylight savings time (DST) state is identified for the lookup
date, based on the time shift at noon.

\item {} 
Down-select the available cycles based on the DST state:
\begin{itemize}
\item {} 
If the cycles are not provided in the local time of the city being
modelled, the search is narrowed to those cycles for
periods/seasons matching this DST state

\item {} 
If the cycles are provided in the local time of the city being
modelled, all periods/seasons are available

\end{itemize}

\item {} 
Remove any cycles that do not contain the necessary day of week from
the shortlist

\item {} 
The most recent cycle with respect to the lookup date is used, and
the modelled time and day of week is chosen from the cycle

\end{enumerate}

The same process is used to identify a relevant daily energy loading,
except in this case a single value is looked up instead of a cycle, and
each day of the year is its own season to improve resolution.


\subsubsection{Fuel consumption file}
\label{\detokenize{OtherManuals/GQF_Manual:fuel-consumption-file}}
This file provides the fuel consumption of each Euro-class on urban
roads and motorways, broken down by vehicle type and euro-class. Each
euro-class corresponds to vehicles manufacturers on/after a certain
date. This information is used with assumed vehicle age to capture the
time evolution of fuel efficiency.

The layout of this file is distinct from the other temporal files shown
here, but the column headings, vehicle names and fuel types must be
exactly as shown here. Since this is a CSV file, the reference text must
also contain no commas (,).


\begin{savenotes}\sphinxattablestart
\centering
\begin{tabular}[t]{|\X{12}{96}|\X{12}{96}|\X{12}{96}|\X{12}{96}|\X{12}{96}|\X{12}{96}|\X{12}{96}|\X{12}{96}|}
\hline
\sphinxstyletheadfamily 
Reference
&\sphinxstyletheadfamily &\sphinxstyletheadfamily &\sphinxstyletheadfamily &\sphinxstyletheadfamily &\sphinxstyletheadfamily &\sphinxstyletheadfamily &\sphinxstyletheadfamily \\
\hline
StartDate
&
Fuel
&
vehicle
&
Standard
&
urban
&
rural\_single
&
rural\_dual
&
motorway
\\
\hline
1996-01-01
&
Petrol
&
car
&
Euro II
&
57.6
&
46.8
&
72.3
&
69
\\
\hline
1996-01-01
&
Diesel
&
car
&
Euro II
&
42.4
&
30.1
&
36.2
&
35.1
\\
\hline
1996-01-01
&
Petrol
&
lgv
&
Euro II
&
76.6
&
60.4
&
90.7
&
86.6
\\
\hline
1996-01-01
&
Diesel
&
lgv
&
Euro II
&
88.3
&
75.8
&
101.6
&
98.2
\\
\hline
1996-01-01
&
Petrol
&
taxi
&
Euro II
&
57.6
&
46.8
&
72.3
&
69
\\
\hline
1996-01-01
&
Diesel
&
taxi
&
Euro II
&
42.4
&
30.1
&
36.2
&
35.1
\\
\hline
1996-01-01
&
Petrol
&
motorcycle
&
Euro II
&
30.1
&
33.1
&
38.7
&
38.2
\\
\hline
1996-01-01
&
Diesel
&
motorcycle
&
Euro II
&
0
&
0
&
0
&
0
\\
\hline
1996-01-01
&
Petrol
&
rigid
&
Euro II
&
0
&
0
&
0
&
0
\\
\hline
1996-01-01
&
Diesel
&
rigid
&
Euro II
&
168
&
155
&
175
&
181
\\
\hline
1996-01-01
&
Petrol
&
artic
&
Euro II
&
0
&
0
&
0
&
0
\\
\hline
1996-01-01
&
Diesel
&
artic
&
Euro II
&
364
&
299
&
311
&
319
\\
\hline
1996-01-01
&
Petrol
&
bus
&
Euro II
&
0
&
0
&
0
&
0
\\
\hline
1996-01-01
&
Diesel
&
bus
&
Euro II
&
415
&
203
&
202
&
206
\\
\hline
\end{tabular}
\par
\sphinxattableend\end{savenotes}


\subsubsection{Further spatial disaggregation}
\label{\detokenize{OtherManuals/GQF_Manual:further-spatial-disaggregation}}
This is optional. It assigns transport, building and metabolism heat
fluxes to only those regions of that map with compatible land covers.
Since land cover fraction data are often available at high spatial
resolution, this increases the resolution of the model outputs beyond
the output areas that were specified initially.

Each model output area is divided into a number of “refined output
areas” (ROAs). The land cover fraction lists the proportion of each ROA
occupied by:
\begin{itemize}
\item {} 
Water

\item {} 
Paved surfaces

\item {} 
Buildings

\item {} 
Soil

\item {} 
Low vegetation

\item {} 
High vegetation

\item {} 
Grass

\end{itemize}

The GQF user interface requires two input files for this process.
\begin{itemize}
\item {} 
\sphinxstylestrong{Land cover fractions}: Land cover fractions calculated using the
{\hyperref[\detokenize{pre-processor/Urban Land Cover Land Cover Reclassifier:landcoverreclassifier}]{\sphinxcrossref{\DUrole{std,std-ref,std,std-ref}{Urban Land Cover: Land Cover Reclassifier}}}} in the pre-processing toolbox.

\item {} 
\sphinxstylestrong{Corresponding polygon grid}: The ESRI shapefile grid of polygons
represented by the land cover fractions. This is a required input for
the UMEP land cover classifier.

\end{itemize}

‘’Note that this feature may be very slow and memory limitations may
cause it to fail or produce very large output files. ‘’

The overall building, transport and metabolic Q$_{\text{F}}$ components in
an MOA are attributed to each ROA based on a set of weightings that
associate land cover classes with Q$_{\text{F}}$ components.

A fixed set of weightings determines the behaviour of this routine and
ensure the following principles are satisfied:
\begin{enumerate}
\item {} 
Transport heat flux only occurs on paved areas (roads)

\item {} 
Building heat flux only occurs where there are buildings

\item {} 
Metabolic energy reflects the distribution of people between indoor
and outdoor environments

\end{enumerate}


\begin{savenotes}\sphinxattablestart
\centering
\begin{tabular}[t]{|\X{25}{100}|\X{25}{100}|\X{25}{100}|\X{25}{100}|}
\hline
\sphinxstyletheadfamily 
Land cover class
&\sphinxstyletheadfamily &\sphinxstyletheadfamily 
Weightings (columns must sum to 1)
&\sphinxstyletheadfamily \\
\hline&
Q$_{\text{F,B}}$
&
Q$_{\text{F,M}}$
&
Q$_{\text{F,T}}$
\\
\hline
Building
&
1
&
0.8
&
0
\\
\hline
Paved
&
0
&
0.05
&
1
\\
\hline
Water
&
0
&
0.0
&
0
\\
\hline
Soil
&
0
&
0.05
&
0
\\
\hline
Grass
&
0
&
0.05
&
0
\\
\hline
High vegetation
&
0
&
0.0
&
0
\\
\hline
Low vegetation
&
0
&
0.05
&
0
\\
\hline
\end{tabular}
\par
\sphinxattableend\end{savenotes}

Current limitations:
\begin{itemize}
\item {} 
Building height not accounted for: same fraction of Q$_{\text{F}}$ would
be assigned to a very tall building and short building if they
occupied the same footprint, despite the former being likely to emit
more heat per square metre of the surface it occupies

\item {} 
Land cover data: assumed to be consistent with the original input
data. If non-zero building energy is calculated in an MOA that has a
building land cover of zero, then this energy is lost.

\end{itemize}


\subsection{Configuration data}
\label{\detokenize{OtherManuals/GQF_Manual:configuration-data}}
The GQF software has two input files:
\begin{itemize}
\item {} 
{\hyperref[\detokenize{OtherManuals/GQF_Manual:data-sources-file}]{\sphinxcrossref{Data sources file}}}: Manages the various input
data files and their associated metadata

\item {} 
{\hyperref[\detokenize{OtherManuals/GQF_Manual:id3}]{\sphinxcrossref{Parameters\_file}}}: Contains numerical values and
assumptions used in model calculations.

\end{itemize}


\subsubsection{Parameters file}
\label{\detokenize{OtherManuals/GQF_Manual:parameters-file}}\phantomsection\label{\detokenize{OtherManuals/GQF_Manual:id3}}
The GQF parameters file contains public holidays and numeric values used
in calculations. The table below describes the entries in each
parameters file


\begin{savenotes}\sphinxatlongtablestart\begin{longtable}{|\X{50}{100}|\X{50}{100}|}
\hline
\sphinxstyletheadfamily 
Parameter name
&\sphinxstyletheadfamily 
Description
\\
\hline
\endfirsthead

\multicolumn{2}{c}%
{\makebox[0pt]{\sphinxtablecontinued{\tablename\ \thetable{} -- continued from previous page}}}\\
\hline
\sphinxstyletheadfamily 
Parameter name
&\sphinxstyletheadfamily 
Description
\\
\hline
\endhead

\hline
\multicolumn{2}{r}{\makebox[0pt][r]{\sphinxtablecontinued{Continued on next page}}}\\
\endfoot

\endlastfoot

\sphinxstylestrong{params: Model run parameters}
&\\
\hline
city
&
Area model is being run for. Expressed in Continent/City format (e.g. Europe/London)
\\
\hline
use\_uk\_holidays
&
Set to 1 to use UK public holidays (calculated automatically) or 0 otherwise
\\
\hline
use\_custom\_holidays
&
Set to 1 to use a list of public holidays (specified separately) or 0 otherwise
\\
\hline
custom\_holidays
&
A list of custom public holidays in YYYY-mm-dd format.
\\
\hline
heaterEffic\_elec
&
Electrical heating efficiency (values from 0 to 1)
\\
\hline
heaterEffic\_gas
&
Gas heating efficiency (values from 0 to 1)
\\
\hline
metabolicLatentHeatFract
&
Fraction of metabolic partitioned into latent heat (values from 0 to 1)
\\
\hline
metabolicSensibleHeatFract
&
Fraction of metabolic heat partitioned into sensible heat (values from 0 to 1)
\\
\hline
vehicleAge
&
Assumed vehicle age (years) relative to the current model time step
\\
\hline
\sphinxstylestrong{waterHeatingFractions: Fraction of building energy spent on heating water} \sphinxstyleemphasis{Values from 0 to 1}
&\\
\hline
domestic\_elec
&
Domestic electricity
\\
\hline
domestic\_gas
&
Domestic gas
\\
\hline
industrial\_elec
&
Industrial electricity
\\
\hline
industrial\_gas
&
Industrial gas
\\
\hline
industrial\_other
&
Other industrial energy sources
\\
\hline
\sphinxstylestrong{heatOfCombustion: Heat of combustion for different fuels} \sphinxstyleemphasis{Two values per entry: net and gross (respectively) {[}MJ/kg{]}}
&\\
\hline
natural\_gas
&
Natural gas
\\
\hline
Petrol\_Fuel
&
Petrol
\\
\hline
Diesel\_Fuel
&
Diesel
\\
\hline
Crude\_Oil
&
Crude Oil
\\
\hline
\sphinxstylestrong{petrolDieselFractions: Vehicle fuel fractions} \sphinxstyleemphasis{Two values per entry: petrol and diesel (respectively). Must be between 0 and 1}
&\\
\hline
motorcycle
&
Motorcycles
\\
\hline
taxi
&
Taxis
\\
\hline
car
&
Cars
\\
\hline
bus
&
Buses (and long-distance coaches)
\\
\hline
lgv
&
LGVs
\\
\hline
rigid
&
Rigid HGVs
\\
\hline
artic
&
Articulated HGVs
\\
\hline
\sphinxstylestrong{vehicleFractions: Breakdown of vehicle types by road classification} \sphinxstyleemphasis{Each entry contains 7 values respectively for car, LGV, motorcycle, taxi, bus, rigid, artic Values in each entry must sum to 1. Used when transport shapefile does not include vehicle-specific AADT}
&\\
\hline
motorway
&
Motorways
\\
\hline
primary\_road
&
Primary roads
\\
\hline
secondary\_road
&
Secondary roads
\\
\hline
other
&
Other roads
\\
\hline
\sphinxstylestrong{roadSpeeds: Default speeds {[}km/h{]} traffic speeds for each road classification} \sphinxstyleemphasis{Used if transport shapefile does not provide speeds for each road segment}
&\\
\hline
motorway
&
Motorway speed
\\
\hline
primary\_road
&
Primary road speed
\\
\hline
secondary\_road
&
Secondary road speed
\\
\hline
other
&
Other road speed
\\
\hline
\sphinxstylestrong{roadAADTs: Default AADTs (annual average daily total) for each road classification} \sphinxstyleemphasis{Used if transport shapefile does not provide AADTs for each road segment}
&\\
\hline
motorway
&
Motorway AADT
\\
\hline
primary\_road
&
Primary AADT
\\
\hline
secondary\_road
&
Secondary AADT
\\
\hline
other
&
Other AADT
\\
\hline
\end{longtable}\sphinxatlongtableend\end{savenotes}


\paragraph{Example parameters file}
\label{\detokenize{OtherManuals/GQF_Manual:example-parameters-file}}
A model configuration for the UK, with two more public holidays than are
ordinarily present. Cars make up the majority of the transport fleet,
and the majority of cars are found on motorways and primary roads. All
other vehicles are found exclusively on primary roads.

\fvset{hllines={, ,}}%
\begin{sphinxVerbatim}[commandchars=\\\{\}]
\PYGZam{}params
   use\PYGZus{}uk\PYGZus{}holidays = 1
   use\PYGZus{}custom\PYGZus{}holidays = 0
   custom\PYGZus{}holidays = \PYGZsq{}2000\PYGZhy{}10\PYGZhy{}30\PYGZsq{}, \PYGZsq{}2000\PYGZhy{}11\PYGZhy{}14
   heaterEffic\PYGZus{}elec = 0.98
   heaterEffic\PYGZus{}gas = 0.85
   metabolicLatentHeatFract = 0.3
   metabolicSensibleHeatFract = 0.7
/
\PYGZam{}waterHeatingFractions
   domestic\PYGZus{}elec = 0.139
   domestic\PYGZus{}gas = 0.27
   industrial\PYGZus{}elec = 0.036
   industrial\PYGZus{}gas = 0.146
   industrial\PYGZus{}other = 0.084
/
\PYGZam{}heatOfCombustion
   natural\PYGZus{}gas = 35.5, 39.4
   Petrol\PYGZus{}Fuel = 44.7, 47.1
   Diesel\PYGZus{}Fuel = 43.3, 45.5
   Crude\PYGZus{}Oil = 43.4, 45.7
/
\PYGZam{}petrolDieselFractions
   motorcycle = 1,0
   taxi = 0,1
   car = 0.84, 0.16
   bus = 0,1
   lgv = 0.1, 0.9
   rigid = 0,1
   artic = 0, 1
/
\PYGZam{}vehicleFractions
   ! Overall fractions of the fleet
   fractions =      0.4,  0.1, 0.1, 0.1, 0.1, 0.1, 0.1
   ! Proportions of each vehicle found on different types of road
   motorway =       0.4,  0,   0,    0,   0,   0,0
   primary\PYGZus{}road =   0.4,  1,   1,    1,   1,   1,1
   secondary\PYGZus{}road = 0.15, 0,   0,    0,   0,   0,0
   other =          0.05, 0,   0,    0,   0,   0,0
/

\PYGZam{}roadSpeeds
   motorway = 80
   primary\PYGZus{}road = 60
   secondary\PYGZus{}road = 40
   other = 20
/

\PYGZam{}roadAADTs
   motorway = 8000
   primary\PYGZus{}road = 4000
   secondary\PYGZus{}road = 2000
   other = 10
/
\end{sphinxVerbatim}


\subsubsection{Data sources file}
\label{\detokenize{OtherManuals/GQF_Manual:data-sources-file}}
The data sources file manages the library of shapefiles and temporal
profile files used by the model. It is divided into a number of sections
(described below).

Everything in the data sources file is \sphinxstylestrong{case-sensitive}.


\paragraph{Output areas}
\label{\detokenize{OtherManuals/GQF_Manual:output-areas}}
The shapefile that defines the model output areas to be used: all input
data are disaggregated into these spatial units, and the model results
are shown using them. There are three entries:


\begin{savenotes}\sphinxattablestart
\centering
\begin{tabular}[t]{|\X{50}{100}|\X{50}{100}|}
\hline
\sphinxstyletheadfamily 
Parameter
&\sphinxstyletheadfamily 
Description
\\
\hline
Shapefile
&
Location of the shapefile on the local machine
\\
\hline
epsgCode
&
EPSG code (numeric) of the shapefile coordinate reference system
\\
\hline
featureIds
&
Column that contains a unique identifier for each output area (optional: order of the output areas in the file is used if empty). This is used for cross-referencing and is shown in the model outputs.
\\
\hline
\end{tabular}
\par
\sphinxattableend\end{savenotes}

An example:

\fvset{hllines={, ,}}%
\begin{sphinxVerbatim}[commandchars=\\\{\}]
\PYG{o}{\PYGZam{}}\PYG{n}{outputAreas}
   \PYG{n}{shapefile} \PYG{o}{=} \PYG{l+s+s1}{\PYGZsq{}}\PYG{l+s+s1}{C:}\PYG{l+s+s1}{\PYGZbs{}}\PYG{l+s+s1}{GreaterQF}\PYG{l+s+s1}{\PYGZbs{}}\PYG{l+s+s1}{PopDens\PYGZus{}2014.shp}\PYG{l+s+s1}{\PYGZsq{}}
   \PYG{n}{epsgCode} \PYG{o}{=} \PYG{l+m+mi}{27700}
   \PYG{n}{featureIds} \PYG{o}{=} \PYG{l+s+s1}{\PYGZsq{}}\PYG{l+s+s1}{LSOA11CD}\PYG{l+s+s1}{\PYGZsq{}}
\PYG{o}{/}
\end{sphinxVerbatim}


\paragraph{Spatial data: Population and energy use shapefiles}
\label{\detokenize{OtherManuals/GQF_Manual:spatial-data-population-and-energy-use-shapefiles}}
The population and energy use shapefiles are specified using a
standardised pattern, each of which consists of four entries:


\begin{savenotes}\sphinxattablestart
\centering
\begin{tabular}[t]{|\X{50}{100}|\X{50}{100}|}
\hline
\sphinxstyletheadfamily 
Parameter
&\sphinxstyletheadfamily 
Description
\\
\hline
shapefiles
&
Location of the shapefile(s) on the local machine
\\
\hline
startDates
&
Start of the time period(s) represented by the shapefile(s) (YYYY-mm-dd format)
\\
\hline
epsgCodes
&
EPSG code (numeric) of the shapefile(s) coordinate reference system
\\
\hline
attribToUse
&
Attribute(s) of the input shapefile(s) that contains the data of interest
\\
\hline
featureIds
&
Name of field that contains unique identifier (integer or string) for each polygon in each shapefile
\\
\hline
\end{tabular}
\par
\sphinxattableend\end{savenotes}

Entries for the \sphinxstyleemphasis{residentialPop} and \sphinxstyleemphasis{workplacePop} sections of the data
sources file (residential and workplace population data) example:

\fvset{hllines={, ,}}%
\begin{sphinxVerbatim}[commandchars=\\\{\}]
\PYG{o}{\PYGZam{}}\PYG{n}{residentialPop}
   \PYG{n}{shapefiles} \PYG{o}{=} \PYG{l+s+s1}{\PYGZsq{}}\PYG{l+s+s1}{C:}\PYG{l+s+s1}{\PYGZbs{}}\PYG{l+s+s1}{GreaterQF}\PYG{l+s+s1}{\PYGZbs{}}\PYG{l+s+s1}{popOA2014.shp}\PYG{l+s+s1}{\PYGZsq{}}
   \PYG{n}{startDates} \PYG{o}{=} \PYG{l+s+s1}{\PYGZsq{}}\PYG{l+s+s1}{2014\PYGZhy{}01\PYGZhy{}01}\PYG{l+s+s1}{\PYGZsq{}}
   \PYG{n}{epsgCodes} \PYG{o}{=} \PYG{l+m+mi}{27700}
   \PYG{n}{attribToUse} \PYG{o}{=} \PYG{l+s+s1}{\PYGZsq{}}\PYG{l+s+s1}{Pop}\PYG{l+s+s1}{\PYGZsq{}}
   \PYG{n}{featureIds} \PYG{o}{=} \PYG{l+s+s1}{\PYGZsq{}}\PYG{l+s+s1}{ID\PYGZus{}CODE}\PYG{l+s+s1}{\PYGZsq{}}
\PYG{o}{/}
\PYG{o}{\PYGZam{}}\PYG{n}{workplacePop}
   \PYG{n}{shapefiles} \PYG{o}{=}\PYG{l+s+s1}{\PYGZsq{}}\PYG{l+s+s1}{C:/GreaterQF/2011OAworkdaypop.shp}\PYG{l+s+s1}{\PYGZsq{}}
   \PYG{n}{startDates} \PYG{o}{=} \PYG{l+s+s1}{\PYGZsq{}}\PYG{l+s+s1}{2014\PYGZhy{}01\PYGZhy{}01}\PYG{l+s+s1}{\PYGZsq{}}
   \PYG{n}{epsgCodes} \PYG{o}{=} \PYG{l+m+mi}{27700}
   \PYG{n}{attribToUse} \PYG{o}{=} \PYG{l+s+s1}{\PYGZsq{}}\PYG{l+s+s1}{WorkPop}\PYG{l+s+s1}{\PYGZsq{}}
   \PYG{n}{featureIds} \PYG{o}{=} \PYG{l+s+s1}{\PYGZsq{}}\PYG{l+s+s1}{FEATURE\PYGZus{}ID}\PYG{l+s+s1}{\PYGZsq{}}
\PYG{o}{/}
\end{sphinxVerbatim}

Same patterns are used to specify energy consumption data. The full list
of input shapefile section headings are:


\begin{savenotes}\sphinxattablestart
\centering
\begin{tabular}[t]{|\X{50}{100}|\X{50}{100}|}
\hline
\sphinxstyletheadfamily 
Parameter
&\sphinxstyletheadfamily 
Description
\\
\hline
residentialPop
&
Residential population
\\
\hline
workplacePop
&
Workday (daytime) population
\\
\hline
annualIndGas
&
Annual industrial gas consumption
\\
\hline
annualIndElec
&
Annual industrial electricity consumption
\\
\hline
annualDomGas
&
Annual domestic gas consumption
\\
\hline
annualDomElec
&
Annual domestic electricity consumption
\\
\hline
annualEco7
&
Annual domestic economy 7 electricity consumption
\\
\hline
\end{tabular}
\par
\sphinxattableend\end{savenotes}


\paragraph{Specifying multiple shapefiles per section}
\label{\detokenize{OtherManuals/GQF_Manual:specifying-multiple-shapefiles-per-section}}
The examples above show the use of a single shapefile for each energy
and population data, but multiple shapefiles can also be used in order
to capture variations over time. This is achieved by using a list of
values. An example is shown below for residential population, in which
populations for 2014 and 2015 are added and different CRS, attributes
and ID fields are used for each file:

\fvset{hllines={, ,}}%
\begin{sphinxVerbatim}[commandchars=\\\{\}]
\PYG{o}{\PYGZam{}}\PYG{n}{residentialPop}
   \PYG{n}{shapefiles} \PYG{o}{=} \PYG{l+s+s1}{\PYGZsq{}}\PYG{l+s+s1}{C:}\PYG{l+s+s1}{\PYGZbs{}}\PYG{l+s+s1}{GreaterQF}\PYG{l+s+s1}{\PYGZbs{}}\PYG{l+s+s1}{popOA2014.shp}\PYG{l+s+s1}{\PYGZsq{}}\PYG{p}{,} \PYG{l+s+s1}{\PYGZsq{}}\PYG{l+s+s1}{C:}\PYG{l+s+s1}{\PYGZbs{}}\PYG{l+s+s1}{GreaterQF}\PYG{l+s+s1}{\PYGZbs{}}\PYG{l+s+s1}{popOA2015.shp}\PYG{l+s+s1}{\PYGZsq{}}\PYG{p}{,}
   \PYG{n}{startDates} \PYG{o}{=} \PYG{l+s+s1}{\PYGZsq{}}\PYG{l+s+s1}{2014\PYGZhy{}01\PYGZhy{}01}\PYG{l+s+s1}{\PYGZsq{}}\PYG{p}{,} \PYG{l+s+s1}{\PYGZsq{}}\PYG{l+s+s1}{2015\PYGZhy{}01\PYGZhy{}01}\PYG{l+s+s1}{\PYGZsq{}}
   \PYG{n}{epsgCodes} \PYG{o}{=} \PYG{l+m+mi}{27700}\PYG{p}{,} \PYG{l+m+mi}{32631}
   \PYG{n}{attribToUse} \PYG{o}{=} \PYG{l+s+s1}{\PYGZsq{}}\PYG{l+s+s1}{Pop}\PYG{l+s+s1}{\PYGZsq{}}\PYG{p}{,} \PYG{l+s+s1}{\PYGZsq{}}\PYG{l+s+s1}{Pop2015}\PYG{l+s+s1}{\PYGZsq{}}
   \PYG{n}{featureIds} \PYG{o}{=} \PYG{l+s+s1}{\PYGZsq{}}\PYG{l+s+s1}{2014\PYGZus{}code}\PYG{l+s+s1}{\PYGZsq{}}\PYG{p}{,} \PYG{l+s+s1}{\PYGZsq{}}\PYG{l+s+s1}{2015\PYGZus{}code}\PYG{l+s+s1}{\PYGZsq{}}
\PYG{o}{/}
\end{sphinxVerbatim}

Note that a “startDate”, “epsgCode” and “attribToUse” must be specified
for every shapefile.


\subparagraph{Temporal data: Metabolism, energy use and transportation temporal profiles}
\label{\detokenize{OtherManuals/GQF_Manual:temporal-data-metabolism-energy-use-and-transportation-temporal-profiles}}
Temporal profile files are each added using the same pattern, with a
list of “profileFiles” added for each category. A complete list is shown
below as an example:

\fvset{hllines={, ,}}%
\begin{sphinxVerbatim}[commandchars=\\\{\}]
\PYGZam{}dailyEnergyUse
   ! Daily variations in total power use
   profileFiles = \PYGZsq{}C:\PYGZbs{}\PYGZbs{}GreaterQF\PYGZbs{}\PYGZbs{}testDailyEnergy.csv\PYGZsq{}
/
\PYGZam{}diurnalDomElec
   ! Diurnal variations in total domestic electricity use (metadata provided in file; files can contain multiple seasons)
   profileFiles = \PYGZsq{}\PYGZsq{}C:\PYGZbs{}\PYGZbs{}GreaterQF\PYGZbs{}\PYGZbs{}BuildingLoadings\PYGZus{}DomUnre.csv\PYGZsq{}
/
\PYGZam{}diurnalDomGas
   ! Diurnal variations in total domestic gas use (metadata provided in file; files can contain multiple seasons)
   profileFiles = \PYGZsq{}\PYGZsq{}C:\PYGZbs{}\PYGZbs{}GreaterQF\PYGZbs{}\PYGZbs{}BuildingLoadings\PYGZus{}DomUnre.csv\PYGZsq{}
/
\PYGZam{}diurnalIndElec
   ! Diurnal variations in total industrial electricity use (metadata provided in file; files can contain multiple seasons)
   profileFiles = \PYGZsq{}\PYGZsq{}C:\PYGZbs{}\PYGZbs{}GreaterQF\PYGZbs{}\PYGZbs{}BuildingLoadings\PYGZus{}Industrial.csv\PYGZsq{}
/
\PYGZam{}diurnalIndGas
   ! Diurnal variations in total industrial gas use (metadata provided in file; files can contain multiple seasons)
   profileFiles = \PYGZsq{}C:\PYGZbs{}\PYGZbs{}GreaterQF\PYGZbs{}\PYGZbs{}BuildingLoadings\PYGZus{}Industrial.csv\PYGZsq{}
/
\PYGZam{}diurnalEco7
   ! Diurnal variations in total economy 7 electricity use (metadata provided in file; files can contain multiple seasons)
   profileFiles = \PYGZsq{}C:\PYGZbs{}\PYGZbs{}GreaterQF\PYGZbs{}\PYGZbs{}BuildingLoadings\PYGZus{}EC7.csv\PYGZsq{}
/
! Temporal transport data
\PYGZam{}diurnalTraffic
   ! Diurnal cycles of transport flow for different vehicle types
   profileFiles = \PYGZsq{}C:\PYGZbs{}\PYGZbs{}GreaterQF\PYGZbs{}\PYGZbs{}testTransport.csv\PYGZsq{}
/
! Temporal metabolism data
\PYGZam{}diurnalMetabolism
   profileFiles = \PYGZsq{}C:\PYGZbs{}\PYGZbs{}GreaterQF\PYGZbs{}\PYGZbs{}testMetabolism.csv\PYGZsq{}
/
\PYGZam{}fuelConsumption
   ! File containing fuel consumption performance data for each vehicle type as standards change over the years
   profileFiles = \PYGZsq{}C:\PYGZbs{}\PYGZbs{}GreaterQF\PYGZbs{}\PYGZbs{}fuelConsumption.csv\PYGZsq{}
/
\end{sphinxVerbatim}

The section headings in the data sources file must exactly match those
shown. The complete list of required section headings is:


\begin{savenotes}\sphinxattablestart
\centering
\begin{tabular}[t]{|\X{50}{100}|\X{50}{100}|}
\hline
\sphinxstyletheadfamily 
Section header
&\sphinxstyletheadfamily 
Model input
\\
\hline
dailyEnergyUse
&
Daily variations in energy consumption
\\
\hline
diurnalDomElec
&
Seasonal diurnal cycles: Domestic electricity
\\
\hline
diurnalDomGas
&
Seasonal diurnal cycles: Domestic gas
\\
\hline
diurnalIndElec
&
Seasonal diurnal cycles: Industrial electricity
\\
\hline
diurnalIndGas
&
Seasonal diurnal cycles: Industrial gas
\\
\hline
diurnalEco7
&
Annual domestic electricity consumption
\\
\hline
annualEco7
&
Seasonal diurnal cycles: Economy 7 electricity
\\
\hline
diurnalTraffic
&
Traffic weekly cycles
\\
\hline
diurnalMetabolism
&
Seasonal diurnal cycles: Metabolism
\\
\hline
fuelConsumption
&
Fuel consumption file
\\
\hline
\end{tabular}
\par
\sphinxattableend\end{savenotes}


\paragraph{Using multiple temporal profile files}
\label{\detokenize{OtherManuals/GQF_Manual:using-multiple-temporal-profile-files}}
As with shapefiles, multiple temporal profile files can be loaded into
the model to capture different periods of time. All of the data is
combined into a single file inside the model, provided that none of the
periods described within the files clash.


\paragraph{Transport data}
\label{\detokenize{OtherManuals/GQF_Manual:transport-data}}
The transport data shapefile section is longer than the others because
the model must deal with different levels of data describing traffic
flow and speed for each road segment. Traffic flow data are read in as
the Annual Average Daily Traffic (AADT; equivalent to vehicle kilometres
divided by road segment length), defined as the mean number of vehicles
passing over the road segment daily.

The data availability scenarios covered by the software are as listed
below. These correspond directly to the scenarios shown in (TABLE IN
PREVIOUS SECTION).


\begin{savenotes}\sphinxattablestart
\centering
\begin{tabular}[t]{|\X{50}{100}|\X{50}{100}|}
\hline
\sphinxstyletheadfamily 
Scenario
&\sphinxstyletheadfamily 
Available data
\\
\hline
Minimum data
&
Classification for every road segment that allows a default AADT and speed to be applied. Default values are specified in the parameters file.
\\
\hline
diurnalDomElec
&
Seasonal diurnal cycles: Domestic electricity
\\
\hline
Scenario 1a
&
Minimum data + Total AADT available for each road segment
\\
\hline
Scenario 1b
&
Minimum data + Total AADT + Mean speed available for each road segment
\\
\hline
Scenario 2a
&
Minimum data + Vehicle-specific AADT available for each road segment
\\
\hline
Scenario 2b
&
Minimum data + Vehicle-specific AADT + Mean speed available for each road segment
\\
\hline
Scenario 3
&
Minimum data + Vehicle-specific AADT, with car and LGV AADTs further broken down into diesel and petrol vehicles, and bus AADT broken down into local buses and long-distance coaches.
\\
\hline
Best case
&
Scenario 3 + Mean speed for each road segment
\\
\hline
\end{tabular}
\par
\sphinxattableend\end{savenotes}

\sphinxstylestrong{Gap-filling for incomplete data}

If AADT or speed is generally available in the shapefile but found to be
missing for a particular road segment, the software will attempt to
gap-fill using mean value from the nearest ten road segments with the
same classification. Default or calculated values for speed and/or
vehicle-specific AADT in each road segment are required in all but the
best-case scenario. These are based upon values stored in the GQF
parameters file.


\subparagraph{Transport parameters}
\label{\detokenize{OtherManuals/GQF_Manual:transport-parameters}}
The table below shows the parameters used in the transport section of
the data sources file.

The shapefile(s) to load are specified using the same format as in the
energy and population sections. Additionally, there are flags to signal
whether certain types of data are available in the shapefile(s), and
mappings to shapefile attributes so that the software refers to the
correct input data.


\begin{savenotes}\sphinxatlongtablestart\begin{longtable}{|\X{33}{99}|\X{33}{99}|\X{33}{99}|}
\hline
\sphinxstyletheadfamily 
Name
&\sphinxstyletheadfamily 
Description
&\sphinxstyletheadfamily 
When used
\\
\hline
\endfirsthead

\multicolumn{3}{c}%
{\makebox[0pt]{\sphinxtablecontinued{\tablename\ \thetable{} -- continued from previous page}}}\\
\hline
\sphinxstyletheadfamily 
Name
&\sphinxstyletheadfamily 
Description
&\sphinxstyletheadfamily 
When used
\\
\hline
\endhead

\hline
\multicolumn{3}{r}{\makebox[0pt][r]{\sphinxtablecontinued{Continued on next page}}}\\
\endfoot

\endlastfoot

\sphinxstylestrong{Standard shapefile descriptors}
&&\\
\hline
shapefiles
&
One or more shapefiles containing road segments, classifications and (optionally) traffic counts and speeds for each road segment
&
Always
\\
\hline
startDates
&
Start date(s) for shapefile(s)
&
Always
\\
\hline
epsgCodes
&
Numeric EPSG code(s) for shapefile(s)
&
Always
\\
\hline
\sphinxstylestrong{Data availability flags} (1=True, 0=False) \textendash{} applies to all transport shapefiles
&&\\
\hline
speed\_available
&
Speed data is available for each road segment
&
Always
\\
\hline
total\_AADT\_available
&
Total AADT is available for each road segment
&
Always
\\
\hline
vehicle\_AADT\_available
&
AADT for each vehicle type is available for each road segment
&
Always
\\
\hline
\sphinxstylestrong{Information on road segment classifications}
&
Note: Any other classifications in the shapefile are treated as “other”: small local roads.
&\\
\hline
class\_field
&
Attribute name: road segment classification field
&
Always
\\
\hline
motorway\_class
&
The name used for motorways
&
Always
\\
\hline
primary\_class
&
The name used for primary roads (UK “A” roads)
&
Always
\\
\hline
secondary\_class
&
The name used for secondary roads (UK “B” roads)
&
Always
\\
\hline
\sphinxstylestrong{Shapefile attribute names}
&&\\
\hline
speed\_field
&
Speed of each road segment
&
speed\_available = 1
\\
\hline
speed\_multiplier
&
A multiplicative conversion factor to convert the speed data to km h-1
&
speed\_available = 1
\\
\hline
AADT\_total
&
Total AADT for all vehicle types
&
total\_AADT\_available=1 and vehicle\_AADT\_available=0
\\
\hline
\sphinxstylestrong{Attribute names for vehicle-specific AADT}
&
\sphinxstylestrong{vehicle\_AADT\_available = 1}
&\\
\hline
AADT\_diesel\_car
&
AADT of diesel cars
&
AADT\_petrol\_car and AADT\_diesel\_car filled in
\\
\hline
AADT\_petrol\_car
&
AADT of petrol cars
&
AADT\_petrol\_car and AADT\_diesel\_car filled in
\\
\hline
AADT\_total\_car
&
AADT for all cars (used when )
&
AADT\_petrol\_car =’’and/or AADT\_petrol\_car =’’
\\
\hline
AADT\_diesel\_LGV
&
AADT of diesel LGVs
&
AADT\_petrol\_LGV and AADT\_diesel\_LGV filled in
\\
\hline
AADT\_petrol\_LGV
&
AADT of petrol LGVs
&
AADT\_petrol\_LGV and AADT\_diesel\_LGV filled in
\\
\hline
AADT\_total\_LGV
&
AADT for all cars
&
AADT\_petrol\_car =’’ and/or AADT\_petrol\_car =’’
\\
\hline
AADT\_motorcycle
&
AADT of all motorcycles
&\\
\hline
AADT\_taxi
&
AADT of all motorcycles
&\\
\hline
AADT\_bus
&
AADT ofbuses
&\\
\hline
AADT\_coach
&
AADT of long-distance coaches
&
If specified
\\
\hline
AADT\_rigid
&
AADT of all rigid HGVs
&\\
\hline
AADT\_artic
&
AADT of all articulated HGVs
&\\
\hline
\end{longtable}\sphinxatlongtableend\end{savenotes}


\paragraph{Example data sources files}
\label{\detokenize{OtherManuals/GQF_Manual:example-data-sources-files}}
Examples of the transport section of the data sources file that deal
with different levels of available data. All parameters must be
specified in every case, but must be left blank if not available (as
shown)

\sphinxstylestrong{Example 1: Best-case scenario \textendash{} all data available in the shapefile}

\fvset{hllines={, ,}}%
\begin{sphinxVerbatim}[commandchars=\\\{\}]
\PYGZam{}transportData
    ! Vector data containing all road segments for study area
    shapefiles = \PYGZsq{}C:\PYGZbs{}GreaterQF\PYGZbs{}RoadSegments.shp\PYGZsq{}
    startDates = \PYGZsq{}2008\PYGZhy{}01\PYGZhy{}01\PYGZsq{}
    epsgCodes = 27700
    speed\PYGZus{}available = 1
    total\PYGZus{}AADT\PYGZus{}available = 1
    vehicle\PYGZus{}AADT\PYGZus{}available = 1
    class\PYGZus{}field = \PYGZsq{}DESC\PYGZus{}’
    motorway\PYGZus{}class = \PYGZsq{}Motorway\PYGZsq{}
    primary\PYGZus{}class = \PYGZsq{}A Road\PYGZsq{}
    secondary\PYGZus{}class = \PYGZsq{}B Road\PYGZsq{}
    speed\PYGZus{}field = \PYGZsq{}Speed\PYGZus{}kph\PYGZsq{}
    speed\PYGZus{}multiplier = 1.0
    AADT\PYGZus{}total = \PYGZsq{}\PYGZsq{}                     ! Left blank because vehicle\PYGZhy{}specific AADTs available
    AADT\PYGZus{}diesel\PYGZus{}car = \PYGZsq{}AADTPcar\PYGZsq{}
    AADT\PYGZus{}petrol\PYGZus{}car = \PYGZsq{}AADTDcar\PYGZsq{}
    AADT\PYGZus{}total\PYGZus{}car = \PYGZsq{}\PYGZsq{}                 ! Left blank because petrol + diesel cars specified
    AADT\PYGZus{}diesel\PYGZus{}LGV = \PYGZsq{}AADTPcar\PYGZsq{}
    AADT\PYGZus{}petrol\PYGZus{}LGV = \PYGZsq{}AADTDcar\PYGZsq{}
    AADT\PYGZus{}total\PYGZus{}LGV = \PYGZsq{}\PYGZsq{}               ! Left blank because petrol + diesel LGVs specified
    AADT\PYGZus{}motorcycle = \PYGZsq{}AADTMotorc\PYGZsq{}
    AADT\PYGZus{}taxi = \PYGZsq{}AADTTaxi\PYGZsq{}
    AADT\PYGZus{}bus = \PYGZsq{}AADTLtBus\PYGZsq{}
    AADT\PYGZus{}coach = \PYGZsq{}AADTCoach\PYGZsq{}
    AADT\PYGZus{}rigid = \PYGZsq{}AADTRigid\PYGZsq{}
    AADT\PYGZus{}artic = \PYGZsq{}AADTArtic\PYGZsq{}
/
\end{sphinxVerbatim}

\sphinxstylestrong{Example 2: Scenario 1b \textendash{} total AADT and speed data available for each
road segment}

\fvset{hllines={, ,}}%
\begin{sphinxVerbatim}[commandchars=\\\{\}]
  \PYGZam{}transportData
    ! Vector data containing all road segments for study area
    shapefiles = \PYGZsq{}C:\PYGZbs{}GreaterQF\PYGZbs{}RoadSegments.shp\PYGZsq{}
    startDates = \PYGZsq{}2008\PYGZhy{}01\PYGZhy{}01\PYGZsq{}
    epsgCodes = 27700
    ! Data available for each road segment
    speed\PYGZus{}available = 1
    total\PYGZus{}AADT\PYGZus{}available = 1
    vehicle\PYGZus{}AADT\PYGZus{}available = 0
    ! Road classification information.
    class\PYGZus{}field = \PYGZsq{}DESC\PYGZus{}’
    motorway\PYGZus{}class = \PYGZsq{}Motorway\PYGZsq{}
    primary\PYGZus{}class = \PYGZsq{}A Road\PYGZsq{}
    secondary\PYGZus{}class = \PYGZsq{}B Road\PYGZsq{}
    speed\PYGZus{}field = \PYGZsq{}Speed\PYGZus{}kph\PYGZsq{}
    speed\PYGZus{}multiplier = 1.0
    AADT\PYGZus{}total = \PYGZsq{}AADTTOTAL\PYGZsq{}
    AADT\PYGZus{}diesel\PYGZus{}car = \PYGZsq{}\PYGZsq{}
    AADT\PYGZus{}petrol\PYGZus{}car = \PYGZsq{}\PYGZsq{}
    AADT\PYGZus{}total\PYGZus{}car = \PYGZsq{}\PYGZsq{}
    AADT\PYGZus{}diesel\PYGZus{}LGV = \PYGZsq{}\PYGZsq{}
    AADT\PYGZus{}petrol\PYGZus{}LGV = \PYGZsq{}\PYGZsq{}
    AADT\PYGZus{}total\PYGZus{}LGV = \PYGZsq{}\PYGZsq{}
    AADT\PYGZus{}motorcycle = \PYGZsq{}\PYGZsq{}
    AADT\PYGZus{}taxi = \PYGZsq{}\PYGZsq{}
    AADT\PYGZus{}bus = \PYGZsq{}\PYGZsq{}
    AADT\PYGZus{}coach = \PYGZsq{}\PYGZsq{}
    AADT\PYGZus{}rigid = \PYGZsq{}\PYGZsq{}
    AADT\PYGZus{}artic = \PYGZsq{}\PYGZsq{}
/
\end{sphinxVerbatim}

\sphinxstylestrong{Example 3: Minimum required data available in the shapefile}

\fvset{hllines={, ,}}%
\begin{sphinxVerbatim}[commandchars=\\\{\}]
\PYGZam{}transportData
   ! Minimum data available from the shapefile
   ! Vector data containing all road segments for study area
   shapefiles = \PYGZsq{}C:\PYGZbs{}GreaterQF\PYGZbs{}RoadSegments.shp\PYGZsq{}
   startDates = \PYGZsq{}2008\PYGZhy{}01\PYGZhy{}01\PYGZsq{}
   epsgCodes = 27700

   ! Data available for each road segment
   speed\PYGZus{}available = 0
   total\PYGZus{}AADT\PYGZus{}available = 0
   vehicle\PYGZus{}AADT\PYGZus{}available = 0

   ! Road classification information.
   class\PYGZus{}field = \PYGZsq{}DESC\PYGZus{}’
   motorway\PYGZus{}class = \PYGZsq{}Motorway\PYGZsq{}
   primary\PYGZus{}class = \PYGZsq{}A Road\PYGZsq{}
   secondary\PYGZus{}class = \PYGZsq{}B Road\PYGZsq{}

   speed\PYGZus{}field =
   speed\PYGZus{}multiplier = 1.0

   AADT\PYGZus{}total =

   AADT\PYGZus{}diesel\PYGZus{}car =
   AADT\PYGZus{}petrol\PYGZus{}car =
   AADT\PYGZus{}total\PYGZus{}car =
   AADT\PYGZus{}diesel\PYGZus{}LGV =
   AADT\PYGZus{}petrol\PYGZus{}LGV =
   AADT\PYGZus{}total\PYGZus{}LGV =

   AADT\PYGZus{}motorcycle =
   AADT\PYGZus{}taxi =
   AADT\PYGZus{}bus =
   AADT\PYGZus{}coach =
   AADT\PYGZus{}rigid =
   AADT\PYGZus{}artic =
/
\end{sphinxVerbatim}


\section{LQF Manual}
\label{\detokenize{OtherManuals/LQF_Manual:lqf-manual}}\label{\detokenize{OtherManuals/LQF_Manual::doc}}

\subsection{Overview}
\label{\detokenize{OtherManuals/LQF_Manual:overview}}
LQF provides a method to calculate the anthropogenic heat flux. It uses
wide-area energy consumption and vehicle ownership values, and uses and
higher-resolution residential population data estimate the heat flux
from buildings, transport and human metabolism at 60 minute intervals at
the spatial resolution of the population data.
\begin{itemize}
\item {} 
Energy and vehicle use are assumed to be correlated to residential
(night-time) populations

\item {} 
Temporal resolution is maximised by applying empirically measured
diurnal, day-of-week and seasonal variations to the data.

\end{itemize}


\subsubsection{Workflow to model Q$_{\text{F}}$}
\label{\detokenize{OtherManuals/LQF_Manual:workflow-to-model-qf}}\begin{enumerate}
\item {} 
Select parameters and data sources files

\item {} 
Select output path: This contains model outputs, logs and any
pre-processed data that is produced

\item {} 
Perform pre-processing of the data or select existing pre-processed
data: A single set of processed input data can be re-used in
subsequent runs

\item {} 
Optionally: Specify land cover fractions at high spatial resolution:
Allows the spatial resolution of the modelled outputs to be enhanced

\item {} 
Run the model: Executes the model for the chosen date range
components.

\item {} 
Visualise outputs: A simple tool is provided to generate maps and
time series from the model outputs.

\end{enumerate}


\subsection{Main user interface}
\label{\detokenize{OtherManuals/LQF_Manual:main-user-interface}}
The main user interface allows the user to select the temporal extent of
the model run, and the configuration files. The configuration files
describe model assumptions and the library of available data files.
\begin{quote}

\sphinxstylestrong{Running the model}
\begin{quote}

\begin{figure}[htbp]
\centering
\capstart

\noindent\sphinxincludegraphics{{300px-LUCY_main}.png}
\caption{LQF main dialogue box}\label{\detokenize{OtherManuals/LQF_Manual:id1}}\end{figure}
\end{quote}

1: Specify model configuration files and output path:
\begin{itemize}
\item {} 
LQF needs spatial and temporal information about the population, energy consumption and transport in order to model Q$_{\text{F}}$ at high temporal and spatial resolution:

\item {} 
Model parameters file {\color{red}\bfseries{}{}`Parameters\_file{}`\_}: Fortran-90 namelist file containing numerical parameters required in model calculations

\item {} 
{\hyperref[\detokenize{OtherManuals/LQF_Manual:data-sources-file}]{\sphinxcrossref{Data sources file}}}: Fortran-90 namelist file that contains the locations of spatial and temporal input files used by the model

\item {} 
\sphinxstylestrong{Output Path}: Directory into which {\hyperref[\detokenize{OtherManuals/LQF_Manual:model-outputs}]{\sphinxcrossref{Model outputs}}} and associated data will be stored. \sphinxstyleemphasis{Any existing files will be overwritten.}

\end{itemize}

2: Input data is pre-processed:
\begin{itemize}
\item {} 
Before it can be used in the model, the wide-area energy use, vehicle ownership and population data in the data sources file must be (dis)aggregated using local population data to match the chosen output areas.

\item {} 
Data is processed using the \sphinxstylestrong{Prepare input data using data sources} button. This performs disaggregation and saves the output files to the /DownscaledData/ subfolder of the chosen model output directory. This step can take up to several hours, during which QGIS will not respond to input.

\item {} 
If processed input data already exists elsewhere it can be used instead by specifying the path using the \sphinxstylestrong{Available at:} box. The processed files are copied to the /DownscaledData/ subfolder of the chosen model output directory.

\item {} 
Optional: \sphinxstylestrong{Extra disaggregation} uses an additional set of inputs so the data can be disaggregated to a higher spatial resolution:
\begin{itemize}
\item {} 
\sphinxstylestrong{Land cover fractions}: Land cover fractions calculated using the UMEP land cover classifier in the pre-processing toolbox.

\item {} 
\sphinxstylestrong{Corresponding polygon grid}: The ESRI shapefile grid of polygons represented by the land cover fractions.

\item {} 
\sphinxstylestrong{Grid cell ID field}: The field of the polygon grid shapefile that contains a unique identifier for each cell. This is used to cross-reference model outputs.

\end{itemize}

\end{itemize}

3: Choose temporal domain:
\begin{itemize}
\item {} 
\sphinxstylestrong{Dates to model} (outputs are produced at 60-minute intervals). Either:
\begin{itemize}
\item {} 
\sphinxstylestrong{Date range}: The first and final dates are specified and the whole period is simulated.

\item {} 
\sphinxstylestrong{Date list}: A comma-separated list of dates in YYYY-mm-dd format (e.g. 2015-01-02, 2016-03-05, 2014-05-03) is provided. These dates are simulated in their entirety.

\end{itemize}

\end{itemize}

4: Run model and visualise results:
\begin{itemize}
\item {} 
The \sphinxstylestrong{Run Model} button executes the model, which applies the temporal disaggregations and calculates Q$_{\text{F}}$ components in each output area. This takes up to several hours for high resolution or long study periods. During this time QGIS will not respond to input.

\item {} 
Results are visualised using the \sphinxstylestrong{Visualise…} button

\item {} 
Previous model results are retrieved using the \sphinxstylestrong{Load Results} button, which allows a previous model output folder to be selected.

\end{itemize}

Data are ready for use in Q$_{\text{F}}$ calculations after this point

\sphinxstylestrong{Visualising output}
\begin{quote}

\begin{figure}[htbp]
\centering
\capstart

\noindent\sphinxincludegraphics{{300px-Visualise}.png}
\caption{LQF results visualisation dialogue box}\label{\detokenize{OtherManuals/LQF_Manual:id2}}\end{figure}

\begin{figure}[htbp]
\centering
\capstart

\noindent\sphinxincludegraphics{{300px-Timeseries}.png}
\caption{Time series output example}\label{\detokenize{OtherManuals/LQF_Manual:id3}}\end{figure}
\end{quote}

A simple visualisation tool accompanies the model which produces maps and time series plots of the available data.

\sphinxstylestrong{Time series plots}
\begin{itemize}
\item {} 
One plot per output area is produced for all of the time steps present in the model output directory, showing the three Q$_{\text{F}}$ components on separate axes. To plot a time series, select the output area of interest and click the “Show plot” button. The plot areas can be manipulated and graphs exported using the tools in the plot window.

\end{itemize}

\sphinxstylestrong{Maps}
\begin{itemize}
\item {} 
One map per Q$_{\text{F}}$ component and time step is produced, coloured on a logarithmic scale according to the Q$_{\text{F}}$ value in each output area. One or more LQF time steps is selected in the list, and every Q$_{\text{F}}$ component is displayed for each date in the QGIS window by pressing “Add to canvas”.

\end{itemize}

Note: Rendering maps may take several minutes for high-resolution model results.
\end{quote}


\subsection{Model outputs}
\label{\detokenize{OtherManuals/LQF_Manual:model-outputs}}
Model outputs are stored in the /ModelOutput/ subdirectory of the
selected model output directory. A separate data file is produced for
each time step of the model run. Each file contains four columns (one
each for total, building, transport and metabolism) and a row for each
output area.
\begin{itemize}
\item {} 
Output files are timestamped with the pattern
\sphinxstylestrong{LQFYYYYmmdd\_HH-MM}.csv, with times stated in UTC.
\begin{itemize}
\item {} 
YYYY: 4-digit year

\item {} 
mm: 2-digit month

\item {} 
dd: 2-digit day of month

\item {} 
HH: 2-digit hour (00 to 23)

\item {} 
MM: 2-digit minute

\end{itemize}

\item {} 
The first model output is labelled 01:00UTC and covers the period
00:00-01:00 UTC.

\item {} 
Each data file is in comma-separated value (CSV) format

\end{itemize}


\subsection{Synthesised shapefiles}
\label{\detokenize{OtherManuals/LQF_Manual:synthesised-shapefiles}}
If pre-processing of the input data has taken place, the Disaggregated
energy, transport and population shapefiles are stored in the
\sphinxstylestrong{/DownscaledData/} subdirectory of the model outputs, with filenames
that reflect the time period they represent. This folder can be used as
the source of processed input data in future runs to save time, provided
that the data sources file has not changed.

If previously processed input data are being used, these are copied to
the \sphinxstylestrong{/DownscaledData/} subdirectory of the current model run


\subsection{Logs}
\label{\detokenize{OtherManuals/LQF_Manual:logs}}
Several log files are saved in the \sphinxstylestrong{/Logs/} subdirectory. The logs are
intended to help interpretation of model outputs by providing a
traceable history of why a particular spatial or temporal disaggregation
value was looked up.
\begin{enumerate}
\item {} 
The steps taken to disaggregate spatial data, including which
attributes were involved

\item {} 
The day of week and the time of day that was returned from each
diurnal and annual profile data source when it was queried with a
particular model time step.

\end{enumerate}


\subsection{Configuration files}
\label{\detokenize{OtherManuals/LQF_Manual:configuration-files}}
The Parameters and Data Sources file are copied to the \sphinxstylestrong{/ConfigFiles/}
subdirectory of the model output directory for future reference.


\subsection{Input data}
\label{\detokenize{OtherManuals/LQF_Manual:input-data}}
Input data consists of spatial and temporal information, a lookup table
for vehicle fuel efficiency and (optionally) land use cover data to
further enhance the spatial resolution of the model output.


\subsubsection{Spatial information}
\label{\detokenize{OtherManuals/LQF_Manual:spatial-information}}

\paragraph{Wide-area data}
\label{\detokenize{OtherManuals/LQF_Manual:wide-area-data}}
An internal database contains nation-level parameters. These are
disaggregated and downscaled based on residential population data. Any
output areas spatially outside a territory will be labelled as belonging
to no nation, and therefore receive zero vehicles, energy consumption or
metabolism.

The database contains the following data for each country. Some of these
are time varying, which values stored for each year that data is
available (1950 onwards). The data can be added to using standard SQL
tools such as SQLite browser, the pandas package in Python or
open-source programming tools. Data can be added for any or all
time-varying quantities, and non-consecutive years are permitted. The
entries are as follows:


\begin{savenotes}\sphinxattablestart
\centering
\begin{tabular}[t]{|\X{33}{99}|\X{33}{99}|\X{33}{99}|}
\hline
\sphinxstyletheadfamily 
Attribute
&\sphinxstyletheadfamily 
Description
&\sphinxstyletheadfamily 
Units
\\
\hline
kwh\_year
&
Total annual primary energy consumption (time-varying)
&
kWh per year
\\
\hline
motorcycles
&
Total motorcycle ownership (time varying)
&
Per 1,000 people
\\
\hline
cars
&
Total passenger car ownership (time varying)
&
Per 1,000 people
\\
\hline
freight
&
Total freight vehicles (time varying)
&
Per 1,000 people
\\
\hline
ecostatus
&
World Bank national income classification (1 to 4, 1 being highest)
&\begin{itemize}
\item {} 
\end{itemize}
\\
\hline
summer\_cooling
&
Whether summer cooling is a significant impact on energy consumption (1=Yes, 0=No)
&\begin{itemize}
\item {} 
\end{itemize}
\\
\hline
wake\_hour
&
Time when 50\% of the population has woken up in the morning
&
Hour of day (local time)
\\
\hline
sleep\_hour
&
Time when 50\% of the population has gone to sleep at night
&
Hour of day (local time)
\\
\hline
transition\_time
&
Timescale over which waking and sleeping occurs
&
Hours
\\
\hline
population
&
Total population (time-varying)
&\begin{itemize}
\item {} 
\end{itemize}
\\
\hline
fixedHolidays
&
Days of the year that contain fixed public holidays for each country (e.g. December 25 in the UK)
&
DOY (non-leap year. Adjusted values used when leap year modelled)
\\
\hline
weekendDays
&
The days of the week that are assumed as weekends in each country
&
1 (weekend) or 0 (weekday)
\\
\hline
weekendCycles
&
Country-specific diurnal variation for weekend building energy consumption and traffic flow
&
Local time
\\
\hline
weekdayCycles
&
Country-specific diurnal variation for weekday building energy consumption and traffic flow
&
Local time
\\
\hline
\end{tabular}
\par
\sphinxattableend\end{savenotes}


\paragraph{Time indexing of wide-area data}
\label{\detokenize{OtherManuals/LQF_Manual:time-indexing-of-wide-area-data}}
The model selects an appropriate time-varying value (e.g. population)
from the database as follows:
\begin{enumerate}
\item {} 
If the model time step is before the first available year, the model
will report an error.

\item {} 
If the model time step is after the final available year, the latest
value is used.

\item {} 
If the model time step is in between two available years, the earlier
year is used.

\end{enumerate}


\paragraph{Local data}
\label{\detokenize{OtherManuals/LQF_Manual:local-data}}
An ESRI shapefile containing spatially resolved population data. This is
used to disaggregate the wide-area totals and estimate metabolism across
the study area.
\begin{itemize}
\item {} 
Since population data are key to the model method, it is important to
use the finest available spatial scale.

\item {} 
The model must output results for a consistent set of spatial units,
so the populations are assigned to the model output areas based on
how much each spatial unit of population is intersected each output
area. It is \sphinxstylestrong{recommended} that a population shapefile is chosen as
the output areas.

\item {} 
The field containing the population must be labelled “Pop” in the
shapefile attributes

\end{itemize}


\subsubsection{Temporal information}
\label{\detokenize{OtherManuals/LQF_Manual:temporal-information}}

\paragraph{Information needed by LQF}
\label{\detokenize{OtherManuals/LQF_Manual:information-needed-by-lqf}}
Temporal data allows the annualised data provided by the shapefiles to
be temporally disaggreated into time series. LQF requires daily and
hourly information:
\begin{enumerate}
\item {} 
\sphinxstylestrong{Daily information}: The mean daily temperature (degrees Celsius)
for the region being studied, covering the period of study. The model
estimates day-to-day changes in building energy consumption based on
the daily mean temperature. The temperature input file for each year
is provided by a file with 365 (or 366) entries.

\item {} 
\sphinxstylestrong{Hourly information}: Template diurnal cycles at 60-minute
intervals for total energy consumption, total traffic flow, metabolic
heat emitted per person and the proportion of the population emitting
this heat.
\begin{itemize}
\item {} 
Variations of these cycles for different \sphinxstylestrong{days of week}

\item {} 
Variations of the above at different \sphinxstylestrong{times of year} (if
available)

\end{itemize}

\item {} 
\sphinxstyleemphasis{‘ Time zone information}’: Temporal files must contain the time zone
represented by the data file. Time zones are specified using the list
of \sphinxhref{https://en.wikipedia.org/wiki/List\_of\_tz\_database\_time\_zones\_standard\_time\_zone\_names.}{https://en.wikipedia.org/wiki/List\_of\_tz\_database\_time\_zones
standard time zone
names.}.

\end{enumerate}

Metabolism is based purely on data in the LQF database and can’t be
overridden. The LQF database contains one default diurnal profile for
traffic flow and building energy consumption, but these should be
overridden with local data files whenever possible:


\begin{savenotes}\sphinxattablestart
\centering
\begin{tabular}[t]{|\X{33}{99}|\X{33}{99}|\X{33}{99}|}
\hline
\sphinxstyletheadfamily 
Q$_{\text{F}}$ component
&\sphinxstyletheadfamily 
File description(s)
&\sphinxstyletheadfamily 
Size of file
\\
\hline
Transport
&
Traffic flows for each vehicle type during each day of the week
&
7 days * 24 hours * N seasons
\\
\hline
Building
&
Building energy consumption during each day of the week
&
7 days * 24 hours * N seasons
\\
\hline
\end{tabular}
\par
\sphinxattableend\end{savenotes}

Each temporal file contains headers that store metadata used by the
model to interpret the data:
\begin{enumerate}
\item {} 
The time zone represented by the file
(“\sphinxhref{https://en.wikipedia.org/wiki/Coordinated\_Universal\_Time}{UTC}”
or of the style “Europe/London”). If “UTC” is specified, then values
must be explicitly provided for each daylight savings regime to
capture shifts in human behaviour. Note that the model outputs are
always UTC, with the necessary conversion taking place in the
software.

\item {} 
The start and end dates of the period represented by the data. This
allows seasonality to be captured.

\end{enumerate}

Ideally these files contain data taken from the period being modelled,
but this is not always practical. In this case, temporal profile data
from the most recent available year is looked up for the same day of
week (taking into account public holidays), season and daylight savings
regime if applicable. Different variants are used for traffic, energy
and metabolism, and each of these is described below.


\paragraph{Details of temporal input files}
\label{\detokenize{OtherManuals/LQF_Manual:details-of-temporal-input-files}}

\subparagraph{Daily temperature}
\label{\detokenize{OtherManuals/LQF_Manual:daily-temperature}}
This file records daily air temperature, from which the model estimates
the response in building energy consumption. These are expressed in
degrees Celsius.

The file consists of Two columns. The first is the day of year; the
second is the temperature. The file must contain values for the days
from StartDate to EndDate (inclusive), and the column and row headers
must be identical to those shown.


\begin{savenotes}\sphinxattablestart
\centering
\begin{tabular}[t]{|\X{50}{100}|\X{50}{100}|}
\hline
\sphinxstyletheadfamily 
Data
&\sphinxstyletheadfamily 
T\_Celsius
\\
\hline
StartDate
&
2015-01-01
\\
\hline
EndDate
&
2015-12-31
\\
\hline
Timezone
&
Europe/London
\\
\hline
1
&
9.161881378
\\
\hline
2
&
9.582277749
\\
\hline
3
&
5.615161127
\\
\hline
4
&
3.62641677
\\
\hline
5
&
8.310810996
\\
\hline
6
&
8.237201333
\\
\hline
7
&
7.586860408
\\
\hline
\end{tabular}
\par
\sphinxattableend\end{savenotes}


\subparagraph{Diurnal variations}
\label{\detokenize{OtherManuals/LQF_Manual:diurnal-variations}}
The same file format is used for both traffic flow and building energy
consumption. Each file contains 7 days of data at 1 hour resolution (168
rows). The first row represents the period 00:00-01:00 on Monday
morning, and the final row represents 23:00-00:00 on Sunday Evening
(into Monday).

The following header lines must be present:
\begin{itemize}
\item {} 
\sphinxstylestrong{Season}: A name for the period represented by each column.

\item {} 
\sphinxstylestrong{Start Date}: The first day of the period (e.g. season) represented
by the data

\item {} 
\sphinxstylestrong{End Date}: The final day of this period

\end{itemize}

Notes:
\begin{itemize}
\item {} 
Periods are not allowed to overlap

\item {} 
The units of measurement are not important: The values within a given
day are normalised after they are loaded into the model software

\end{itemize}

The example below shows the first 24 rows of a file that contains
entries for the 4 quarters of 2014. Any number of seasons/periods of
year can be added to a single file, and multiple files can be added.


\begin{savenotes}\sphinxattablestart
\centering
\begin{tabular}[t]{|\X{20}{100}|\X{20}{100}|\X{20}{100}|\X{20}{100}|\X{20}{100}|}
\hline
\sphinxstyletheadfamily 
Season
&\sphinxstyletheadfamily 
Q1
&\sphinxstyletheadfamily 
Q2
&\sphinxstyletheadfamily 
Q3
&\sphinxstyletheadfamily 
Q4
\\
\hline
StartDate
&
2014-01-01
&
2014-04-01
&
2014-07-01
&
2014-10-20
\\
\hline
EndDate
&
2014-03-31
&
2014-06-30
&
2014-09-30
&
2014-12-31
\\
\hline
Timezone
&
Europe/London
&&&\\
\hline
01:00
&
0.273
&
0.294
&
0.306
&
0.287
\\
\hline
02:00
&
0.236
&
0.248
&
0.259
&
0.242
\\
\hline
03:00
&
0.228
&
0.238
&
0.24
&
0.228
\\
\hline
04:00
&
0.219
&
0.228
&
0.227
&
0.222
\\
\hline
05:00
&
0.226
&
0.226
&
0.227
&
0.222
\\
\hline
06:00
&
0.254
&
0.245
&
0.238
&
0.238
\\
\hline
07:00
&
0.355
&
0.297
&
0.275
&
0.304
\\
\hline
08:00
&
0.477
&
0.395
&
0.349
&
0.387
\\
\hline
09:00
&
0.487
&
0.509
&
0.48
&
0.448
\\
\hline
10:00
&
0.473
&
0.542
&
0.532
&
0.456
\\
\hline
11:00
&
0.45
&
0.51
&
0.567
&
0.442
\\
\hline
12:00
&
0.448
&
0.502
&
0.576
&
0.44
\\
\hline
13:00
&
0.458
&
0.507
&
0.591
&
0.439
\\
\hline
14:00
&
0.436
&
0.487
&
0.552
&
0.421
\\
\hline
15:00
&
0.431
&
0.478
&
0.539
&
0.402
\\
\hline
16:00
&
0.468
&
0.478
&
0.563
&
0.417
\\
\hline
17:00
&
0.554
&
0.533
&
0.629
&
0.482
\\
\hline
18:00
&
0.65
&
0.649
&
0.698
&
0.547
\\
\hline
19:00
&
0.723
&
0.691
&
0.763
&
0.569
\\
\hline
20:00
&
0.709
&
0.665
&
0.757
&
0.545
\\
\hline
21:00
&
0.661
&
0.622
&
0.685
&
0.555
\\
\hline
22:00
&
0.593
&
0.572
&
0.606
&
0.548
\\
\hline
23:00
&
0.496
&
0.488
&
0.497
&
0.474
\\
\hline
00:00
&
0.36
&
0.393
&
0.358
&
0.359
\\
\hline
\end{tabular}
\par
\sphinxattableend\end{savenotes}


\subparagraph{Metabolic activity}
\label{\detokenize{OtherManuals/LQF_Manual:metabolic-activity}}
Metabolic activity is calculated based on the parameters in the
database, which do not change over time (unlike energy consumption,
population and vehicle ownership).

The populace is assumed to emit more metabolic energy during waking
hours than during sleep, with a linear transition between these two
states based on the time people generally wake and sleep in each
country. A study area spanning national boundaries therefore shows
spatial variation in metabolic activity in the morning and evening if
the countries have different waking and sleeping hours in the LQF
database.


\paragraph{Recycling of temporal data}
\label{\detokenize{OtherManuals/LQF_Manual:recycling-of-temporal-data}}
The model calculates fluxes for any date provided there is temporal data
for the corresponding time of year. If daily temperatures and/or diurnal
cycles are not available for the date being modelled, a series of
lookups is performed on the available temporal data to find a suitable
match. This process accounts for changes in public holidays, leap years
and changing DST switch dates.

For diurnal cycle data, the lookup operates by building and then
reducing a shortlist of cycles that may be suitable:
\begin{enumerate}
\item {} 
Based on the modelled time step, cycles from a suitable year are
added to the shortlist. A year is deemed suitable if it contains data
covering the time of year being modelled
\begin{itemize}
\item {} 
If the modelled year is later than available data, the latest
suitable year is used

\item {} 
If the modelled year is earlier than the available data, the
earliest suitable year is used

\end{itemize}

\item {} 
The modelled day of week is established (set to Sunday if a public
holiday)

\item {} 
The lookup date is set as the same day of week, month and time of
month as the modelled date, but in the year identified as suitable.
\begin{itemize}
\item {} 
This operation sometimes causes late December dates to become
early January. Such dates are moved into the final week of
December.

\end{itemize}

\item {} 
The daylight savings time (DST) state is identified for the lookup
date, based on the time shift at noon.

\item {} 
Down-select the available cycles based on the DST state
\sphinxstyleemphasis{(user-provided diurnal profile files only, when timezone of the
modelled city is not the same as that in the profile file)}:
\begin{itemize}
\item {} 
If the cycles are not provided in the local time of the city being
modelled, the search is narrowed to those cycles for
periods/seasons matching this DST state

\item {} 
If the cycles are provided in the local time of the city being
modelled, all periods/seasons are available

\end{itemize}

\item {} 
Remove any cycles that do not contain the necessary day of week from
the shortlist

\item {} 
The most recent cycle with respect to the lookup date is used

\end{enumerate}

The same process is used to identify a relevant daily temperature,
except in this case a single value is looked up instead of a cycle and
each day of the year is its own season to improve resolution.


\subsubsection{Further spatial disaggregation}
\label{\detokenize{OtherManuals/LQF_Manual:further-spatial-disaggregation}}
This is optional. It assigns transport, building and metabolism heat
fluxes to only those regions of that map with compatible land covers.
Since land cover fraction data are often available at high spatial
resolution, this increases the resolution of the model outputs beyond
the output areas that were specified initially.

Each model output area is divided into a number of “refined output
areas” (ROAs). The land cover fraction lists the proportion of each ROA
occupied by:
\begin{itemize}
\item {} 
Water

\item {} 
Paved surfaces

\item {} 
Buildings

\item {} 
Soil

\item {} 
Deciduous Trees

\item {} 
Coniferous Trees

\item {} 
Grass

\end{itemize}

The GQF user interface requires two input files for this process.
\begin{itemize}
\item {} 
\sphinxstylestrong{Land cover fractions}: Land cover fractions calculated using the
{\hyperref[\detokenize{pre-processor/Urban Land Cover Land Cover Reclassifier:landcoverreclassifier}]{\sphinxcrossref{\DUrole{std,std-ref,std,std-ref}{Urban Land Cover: Land Cover Reclassifier}}}}
\begin{quote}

in the pre-processing toolbox.
\end{quote}

\item {} 
\sphinxstylestrong{Corresponding polygon grid}: The ESRI shapefile grid of polygons
represented by the land cover fractions. This is a required input for
the UMEP land cover classifier.

\end{itemize}

‘’Note that this feature may be very slow and memory limitations may
cause it to fail or produce very large output files. ‘’

The overall building, transport and metabolic Q$_{\text{F}}$ components in
an MOA are attributed to each ROA based on a set of weightings that
associate land cover classes with Q$_{\text{F}}$ components.

A fixed set of weightings determines the behaviour of this routine and
ensure the following principles are satisfied:
\begin{enumerate}
\item {} 
Transport heat flux only occurs on paved areas (roads)

\item {} 
Building heat flux only occurs where there are buildings

\item {} 
Metabolic energy reflects the distribution of people between indoor
and outdoor environments

\end{enumerate}


\begin{savenotes}\sphinxattablestart
\centering
\begin{tabular}[t]{|\X{25}{100}|\X{25}{100}|\X{25}{100}|\X{25}{100}|}
\hline
\sphinxstyletheadfamily 
Land cover class
&\sphinxstyletheadfamily &\sphinxstyletheadfamily 
Weightings (columns must sum to 1)
&\sphinxstyletheadfamily \\
\hline&
Q$_{\text{F,B}}$
&
Q$_{\text{F,M}}$
&
Q$_{\text{F,T}}$
\\
\hline
Building
&
1
&
0.8
&
0
\\
\hline
Paved
&
0
&
0.05
&
1
\\
\hline
Water
&
0
&
0.0
&
0
\\
\hline
Soil
&
0
&
0.05
&
0
\\
\hline
Grass
&
0
&
0.05
&
0
\\
\hline
Deciduous Trees
&
0
&
0.0
&
0
\\
\hline
Coniferous Trees
&
0
&
0.05
&
0
\\
\hline
\end{tabular}
\par
\sphinxattableend\end{savenotes}

Current limitations:
\begin{itemize}
\item {} 
Building height not accounted for: same fraction of Q$_{\text{F}}$ would
be assigned to a very tall building and short building if they
occupied the same footprint, despite the former being likely to emit
more heat per square metre of the surface it occupies

\item {} 
Land cover data: assumed to be consistent with the original input
data. If non-zero building energy is calculated in an MOA that has a
building land cover of zero, then this energy is lost.

\end{itemize}


\subsubsection{Temperature response functions}
\label{\detokenize{OtherManuals/LQF_Manual:temperature-response-functions}}

\paragraph{Built-in response}
\label{\detokenize{OtherManuals/LQF_Manual:built-in-response}}
LQF contains a database of country-specific parameters that link
temperature to building energy consumption via heating degree days (and
cooling degree days if air conditioning is assumed to be significant in
that country). This forms a temperature response function.

In the model, mean daily building energy consumption is estimated by
dividing the annual consumption by the number of days in a year. For
each modelled day, this figure is multiplied by the temperature response
function for that day. This allows the model to estimate seasonal and
day-to-day variations in energy consumption and therefore QF. \sphinxhref{http://www.sciencedirect.com/science/article/pii/S2212095513000059}{Lindberg
et al.
(2013)}
details the response function and how it varies from country to country.


\paragraph{User-defined response}
\label{\detokenize{OtherManuals/LQF_Manual:user-defined-response}}
An alternative temperature response function can be used to override the
built-in values. This uses 7 parameters, illustrated below:

\begin{figure}[htbp]
\centering
\capstart

\noindent\sphinxincludegraphics{{T_response}.png}
\caption{\sphinxcode{\sphinxupquote{{}`to do{}`}}}\label{\detokenize{OtherManuals/LQF_Manual:id4}}\end{figure}
\begin{enumerate}
\item {} 
Tc: Temperature above which air conditioning is used {[}°C{]}

\item {} 
Th: Temperature below which heating is used {[}°C{]}

\item {} 
Ac: Coefficient relating temperature above Tc to energy consumption

\item {} 
Ah: Coefficient relating temperature below Th to energy consumption

\item {} 
c: Constant that sets minimum value

\item {} 
Tmin: Temperature below which energy use from heating stops varying
{[}°C{]}

\item {} 
Tmax: Temperature above which energy use from cooling stops varying
{[}°C{]}

\end{enumerate}

Despite the direction of the slopes, Ah and Ac are both positive
coefficients that act on the absolute difference between T and Th or Tc
(respectively).

To activate the custom response function, the parameters must be
specified in the parameters file.


\subsection{Configuration data}
\label{\detokenize{OtherManuals/LQF_Manual:configuration-data}}
The LQFsoftware has two configuration files:
\begin{itemize}
\item {} 
{\hyperref[\detokenize{OtherManuals/LQF_Manual:data-sources-file}]{\sphinxcrossref{Data sources file}}}: Manages the various input
data files and their associated metadata

\item {} 
{\color{red}\bfseries{}{}`Parameters\_file{}`\_}: Contains numerical values and
assumptions used in model calculations.

\end{itemize}


\subsubsection{Parameters file}
\label{\detokenize{OtherManuals/LQF_Manual:parameters-file}}
The LQF parameters file contains public holidays and numeric values used
in calculations. The table below describes the entries in each
parameters file.


\begin{savenotes}\sphinxattablestart
\centering
\begin{tabular}[t]{|\X{50}{100}|\X{50}{100}|}
\hline
\sphinxstyletheadfamily 
Parameter name
&\sphinxstyletheadfamily 
Description
\\
\hline
\sphinxstylestrong{params: Model run parameters}
&\\
\hline\begin{itemize}
\item {} 
timezone

\end{itemize}
&
The time zone of the modelled area. Expressed in Continent/City format (e.g. Europe/London). \sphinxhref{https://en.wikipedia.org/wiki/List\_of\_tz\_database\_time\_zones\_List\_of\_valid\_time\_zones.}{https://en.wikipedia.org/wiki/List\_of\_tz\_database\_time\_zones List of valid time zones.}.
\\
\hline\begin{itemize}
\item {} 
use\_uk\_holidays

\end{itemize}
&
Set to 1 to use UK public holidays (calculated automatically) or 0 otherwise
\\
\hline\begin{itemize}
\item {} 
use\_custom\_holidays

\end{itemize}
&
Set to 1 to use a list of public holidays (specified separately) or 0 otherwise
\\
\hline\begin{itemize}
\item {} 
custom\_holidays

\end{itemize}
&
A list of custom public holidays in YYYY-mm-dd format.
\\
\hline\begin{itemize}
\item {} 
avgspeed

\end{itemize}
&
Mean speed (metres per hour) of traffic
\\
\hline\begin{itemize}
\item {} 
emissionfactors

\end{itemize}
&
Emissions factors in {[}W.m-2{]} for cars, motorcycles and freight vehicles
\\
\hline\begin{itemize}
\item {} 
balance\_point\_temperature

\end{itemize}
&
Outdoor air temperature below/above which the building energy is assumed to change as a result of active heading/cooling.
\\
\hline\begin{itemize}
\item {} 
balance\_point\_multfactor

\end{itemize}
&
Factor applied to the difference between air temperature and balance point temperature to estimate the building energy response
\\
\hline\begin{itemize}
\item {} 
QV\_multfactor

\end{itemize}
&
Assumed proportion of vehicle fleet in use per day
\\
\hline\begin{itemize}
\item {} 
sleep\_metab

\end{itemize}
&
Assumed metabolic heat emission per person {[}W{]} while resting (sleep)
\\
\hline\begin{itemize}
\item {} 
work\_metab

\end{itemize}
&
Assumed metabolic heat emission per person {[}W{]} while active (awake)
\\
\hline
\sphinxstylestrong{CustomTemperatureResponse}:
&
\sphinxstylestrong{Optional parameters for a custom {}`temperature response \textless{}\#Temperature\_response\_functions\textgreater{}{}`\_\_ function}
\\
\hline\begin{itemize}
\item {} 
Th

\end{itemize}
&
Daily mean Temperature below which heating is used (celsius)
\\
\hline\begin{itemize}
\item {} 
Tc

\end{itemize}
&
Daily mean Temperature above which artificial cooling is used (celsius)
\\
\hline\begin{itemize}
\item {} 
Ah

\end{itemize}
&
Coefficient relating temperature below Th to energy consumption
\\
\hline\begin{itemize}
\item {} 
Ac

\end{itemize}
&
Coefficient relating temperature above Tc to energy consumption
\\
\hline\begin{itemize}
\item {} 
c

\end{itemize}
&
Constant that sets minimum value of response function
\\
\hline\begin{itemize}
\item {} 
Tmax

\end{itemize}
&
Temperature above which energy use is constant with temperature
\\
\hline\begin{itemize}
\item {} 
Tmin

\end{itemize}
&
Temperature below which energy use is constant with temperature
\\
\hline
\end{tabular}
\par
\sphinxattableend\end{savenotes}

Values for the land cover weightings discussed above are also included
in the parameters file.


\paragraph{Example parameters file (without user-defined temperature response)}
\label{\detokenize{OtherManuals/LQF_Manual:example-parameters-file-without-user-defined-temperature-response}}
A model configuration for the UK, with two more public holidays than are
ordinarily present.

\fvset{hllines={, ,}}%
\begin{sphinxVerbatim}[commandchars=\\\{\}]
\PYGZam{}params
   timezone = \PYGZbs{} “Europe/London”
   use\PYGZus{}uk\PYGZus{}holidays = 1
   use\PYGZus{}custom\PYGZus{}holidays = 1
   custom\PYGZus{}holidays = \PYGZsq{}2016\PYGZhy{}06\PYGZhy{}21\PYGZsq{}, \PYGZsq{}2016\PYGZhy{}06\PYGZhy{}22\PYGZsq{}
   avgspeed = 48000.
   emissionfactors = 25.92, 13.16, 108.42
   balance\PYGZus{}point\PYGZus{}temperature = 12.
   balance\PYGZus{}point\PYGZus{}multfactor = 0.7
   QV\PYGZus{}multfactor = 0.8
   sleep\PYGZus{}metab = 75
   work\PYGZus{}metab = 175
/
\PYGZam{}landCoverWeights
   ! For optional additional spatial disaggregation, triplets of weightings for land cover classes
   ! Values for [Building, Transport, Metabolism] respectively
   grass           = 0, 0, 0.025
   baresoil        = 0, 0, 0
   paved           = 0, 1, 0.10
   buildings       = 1, 0, 0.85
   water           = 0, 0, 0
   decidioustrees  = 0, 0, 0.025
   evergreentrees  = 0, 0, 0
/
\end{sphinxVerbatim}


\paragraph{User-defined temperature response section}
\label{\detokenize{OtherManuals/LQF_Manual:user-defined-temperature-response-section}}
To override the built-in temperature response function, the following
section must be added to the parameters file (arbitrary values are used
here as examples)

\fvset{hllines={, ,}}%
\begin{sphinxVerbatim}[commandchars=\\\{\}]
\PYG{o}{\PYGZam{}}\PYG{n}{CustomTemperatureResponse}
   \PYG{n}{Th} \PYG{o}{=} \PYG{l+m+mi}{10}
   \PYG{n}{Tc} \PYG{o}{=} \PYG{l+m+mi}{20}
   \PYG{n}{Ah} \PYG{o}{=} \PYG{l+m+mf}{0.1}
   \PYG{n}{Ac} \PYG{o}{=} \PYG{l+m+mf}{0.2}
   \PYG{n}{c} \PYG{o}{=} \PYG{l+m+mf}{0.5}
   \PYG{n}{Tmax} \PYG{o}{=} \PYG{l+m+mi}{50}
   \PYG{n}{Tmin} \PYG{o}{=} \PYG{o}{\PYGZhy{}}\PYG{l+m+mi}{10}
\PYG{o}{/}
\end{sphinxVerbatim}


\subsubsection{Data sources file}
\label{\detokenize{OtherManuals/LQF_Manual:data-sources-file}}
The data sources file manages the library of shapefiles and temporal
profile files used by the model. It is divided into a number of sections
(described below).


\paragraph{Output areas}
\label{\detokenize{OtherManuals/LQF_Manual:output-areas}}
The shapefile that defines the model output areas to be used: all input
data are disaggregated into these spatial units, and the model results
are shown using them. In the simplest case, the same shapefile is used
for both outputAreas and Residential population (see below).

There are three entries:


\begin{savenotes}\sphinxattablestart
\centering
\begin{tabular}[t]{|\X{50}{100}|\X{50}{100}|}
\hline
\sphinxstyletheadfamily 
Parameter
&\sphinxstyletheadfamily 
Description
\\
\hline
Shapefile
&
Location of the shapefile on the local machine
\\
\hline
epsgCode
&
EPSG code (numeric) of the shapefile coordinate reference system
\\
\hline
featureIds
&
Column that contains a unique identifier for each output area (optional: order of the output areas in the file is used if empty). This is used for cross-referencing and is shown in the model outputs.
\\
\hline
\end{tabular}
\par
\sphinxattableend\end{savenotes}

An example:

\fvset{hllines={, ,}}%
\begin{sphinxVerbatim}[commandchars=\\\{\}]
\PYG{o}{\PYGZam{}}\PYG{n}{outputAreas}
  \PYG{n}{shapefile} \PYG{o}{=} \PYG{l+s+s1}{\PYGZsq{}}\PYG{l+s+s1}{C:}\PYG{l+s+s1}{\PYGZbs{}}\PYG{l+s+s1}{LQF}\PYG{l+s+s1}{\PYGZbs{}}\PYG{l+s+s1}{PopDens\PYGZus{}2014.shp}\PYG{l+s+s1}{\PYGZsq{}}
  \PYG{n}{epsgCode} \PYG{o}{=} \PYG{l+m+mi}{27700}
  \PYG{n}{featureIds} \PYG{o}{=} \PYG{l+s+s1}{\PYGZsq{}}\PYG{l+s+s1}{LSOA11CD}\PYG{l+s+s1}{\PYGZsq{}}
  \PYG{o}{/}
\end{sphinxVerbatim}


\paragraph{International database}
\label{\detokenize{OtherManuals/LQF_Manual:international-database}}
Nation-level population, vehicle registrations, energy consumption and
socio-economic data for multiple years are stored in a Spatialite
database file. The location of this file is specified in the data
sources file as follows:

\fvset{hllines={, ,}}%
\begin{sphinxVerbatim}[commandchars=\\\{\}]
\PYG{o}{\PYGZam{}}\PYG{n}{database}
   \PYG{n}{path} \PYG{o}{=} \PYG{l+s+s1}{\PYGZsq{}}\PYG{l+s+s1}{C:}\PYG{l+s+s1}{\PYGZbs{}}\PYG{l+s+s1}{LQF}\PYG{l+s+s1}{\PYGZbs{}}\PYG{l+s+s1}{InternationalDatabase.sqlite}\PYG{l+s+s1}{\PYGZsq{}}
\PYG{o}{/}
\end{sphinxVerbatim}


\paragraph{Residential population shapefile}
\label{\detokenize{OtherManuals/LQF_Manual:residential-population-shapefile}}
Entries for the ‘residentialPop’ section of the data sources file
(residential population data) example:

\fvset{hllines={, ,}}%
\begin{sphinxVerbatim}[commandchars=\\\{\}]
\PYG{o}{\PYGZam{}}\PYG{n}{residentialPop}
   \PYG{n}{shapefiles} \PYG{o}{=} \PYG{l+s+s1}{\PYGZsq{}}\PYG{l+s+s1}{C:}\PYG{l+s+s1}{\PYGZbs{}}\PYG{l+s+s1}{LQF}\PYG{l+s+s1}{\PYGZbs{}}\PYG{l+s+s1}{popOA2014.shp}\PYG{l+s+s1}{\PYGZsq{}}
   \PYG{n}{startDates} \PYG{o}{=} \PYG{l+s+s1}{\PYGZsq{}}\PYG{l+s+s1}{2014\PYGZhy{}01\PYGZhy{}01}\PYG{l+s+s1}{\PYGZsq{}}
   \PYG{n}{epsgCodes} \PYG{o}{=} \PYG{l+m+mi}{27700}
\PYG{o}{/}
\end{sphinxVerbatim}

\sphinxstylestrong{Note:} The population \sphinxstylestrong{must} appear under the attribute “Pop” in
the residential shapefile.

Note that a “startDate” and “epsgCode” must be specified for each
shapefile. Providing the incorrect EPSG code will result in incorrect or
zero heat fluxes being modelled because the mis-projected model areas
never overlap.


\paragraph{Temporal data: Metabolism, energy use and transportation temporal profiles}
\label{\detokenize{OtherManuals/LQF_Manual:temporal-data-metabolism-energy-use-and-transportation-temporal-profiles}}

\subparagraph{Air temperature (required)}
\label{\detokenize{OtherManuals/LQF_Manual:air-temperature-required}}
Daily mean temperature (in the local time zone of the location being
studied) is a required input. Data can be provided for multiple years
using a comma-separated list of files.


\subparagraph{Energy consumption and traffic flow profiles (optional)}
\label{\detokenize{OtherManuals/LQF_Manual:energy-consumption-and-traffic-flow-profiles-optional}}
The LQF database contains default diurnal profiles for traffic and
building energy consumption, and this varies if the study area overlaps
countries with different profiles. These profiles are overridden if
user-specified data are supplied instead, and the user-specified values
are applied to the entire study area.

An example that provides all three temporal data sources is shown below,
and two years of data are provided for air temperature.

\fvset{hllines={, ,}}%
\begin{sphinxVerbatim}[commandchars=\\\{\}]
\PYGZam{}temporal
   ! Mean daily air temperature data
   dailyTemperature = \PYGZsq{}C:\PYGZbs{}LQF\PYGZbs{}dailyT\PYGZus{}2013.csv\PYGZsq{}, \PYGZsq{}C:\PYGZbs{}LQF\PYGZbs{}dailyT\PYGZus{}2014.csv\PYGZsq{}
   ! Diurnal profiles
   ! Omit entries to use default LQF database values
   diurnEnergy = \PYGZsq{}C:\PYGZbs{}LQF\PYGZbs{}buildingProfiles.csv\PYGZsq{}
   diurnTraffic = \PYGZsq{}C:\PYGZbs{}LQF\PYGZbs{}transportProfiles.csv\PYGZsq{}
/
\end{sphinxVerbatim}


\subparagraph{Using multiple temporal profile files}
\label{\detokenize{OtherManuals/LQF_Manual:using-multiple-temporal-profile-files}}
As with shapefiles, multiple temporal profile files can be loaded into
the model to capture different periods of time. All of the data is
combined into a single file inside the model, provided that none of the
periods described within the files clash.


\paragraph{Example data sources file}
\label{\detokenize{OtherManuals/LQF_Manual:example-data-sources-file}}
A complete data sources file appears as follows. Note that two data
files are specified for the daily temperature data so that a longer time
series can be modelled.

\fvset{hllines={, ,}}%
\begin{sphinxVerbatim}[commandchars=\\\{\}]
! \PYGZsh{}\PYGZsh{}\PYGZsh{} Model output polygons
\PYGZam{}outputAreas
   shapefile = \PYGZsq{}C:\PYGZbs{}LQF\PYGZbs{}population.shp\PYGZsq{}
   epsgCode = 32631
   featureIds = \PYGZsq{}ID\PYGZsq{} ! The attribute to use as a unique ID for each areas (optional; for cross\PYGZhy{}referencing)
/
! \PYGZsh{}\PYGZsh{}\PYGZsh{} Residential population data for the city being studied
! Must contain total population in each area under the attribute \PYGZbs{} “Pop”
\PYGZam{}residentialPop
   shapefiles = \PYGZsq{}C:\PYGZbs{}LQF\PYGZbs{}population.shp\PYGZsq{}
   startDates = \PYGZsq{}2014\PYGZhy{}01\PYGZhy{}01\PYGZsq{}
   epsgCodes = 32631
   featureIds = \PYGZsq{}ID\PYGZsq{}
/
\PYGZam{}database
   path = \PYGZsq{}C:\PYGZbs{}LQF\PYGZbs{}InternationalDatabase.sqlite\PYGZsq{}
/
\PYGZam{}temporal
   ! Air temperature each day for a year
   dailyTemperature = \PYGZsq{}C:\PYGZbs{}LQF\PYGZbs{}temp\PYGZus{}2013.csv\PYGZsq{}, \PYGZsq{}C:\PYGZbs{}LQF\PYGZbs{}temp\PYGZus{}2014.csv\PYGZsq{}
   ! Provide file(s) for building energy consumption and/or traffic flow diurnal cycles
   ! Omit entries to use default LQF database values
   diurnEnergy = \PYGZsq{}C:\PYGZbs{}LQF\PYGZbs{}buildingProfiles.csv\PYGZsq{}
   diurnTraffic = \PYGZsq{}C:\PYGZbs{}LQF\PYGZbs{}transportProfiles.csv\PYGZsq{}
/
\end{sphinxVerbatim}


\subsection{Troubleshooting}
\label{\detokenize{OtherManuals/LQF_Manual:troubleshooting}}

\subsubsection{Known issues}
\label{\detokenize{OtherManuals/LQF_Manual:known-issues}}

\paragraph{Time zone problem}
\label{\detokenize{OtherManuals/LQF_Manual:time-zone-problem}}
Sometimes, a valid time zone in the Parameters or temporal input files
will be rejected by the model, resulting in a “Time zone problem” error
message.

This is usually fixed by upgrading the Python time zone library. In
Windows:
\begin{enumerate}
\item {} 
Find Osgeo4w shell in Start \textgreater{} Programs

\item {} 
Right-click it and select “run as administrator”

\item {} 
Enter the following command:

\end{enumerate}

pip install pytz \textendash{}upgrade

Restart QGIS and try again.


\paragraph{QGIS crashes and quits}
\label{\detokenize{OtherManuals/LQF_Manual:qgis-crashes-and-quits}}
An unresolved bug causes QGIS 2.18.x to crash and quit immediately after
the “preparing input data using data sources” has finished. After
restarting QGIS, the model run can be resumed by
\begin{itemize}
\item {} 
Using the same parameters and data sources files

\item {} 
Setting a new output folder

\item {} 
Rather than processing the input data again, selecting the prepared
input data from the old output folder.

\item {} 
Run the model as normal

\end{itemize}

This allows the preparation step to be skipped, making use of the
results from last time round.


\subsection{Appendix A: Converting a population raster to a vector shapefile using QGIS}
\label{\detokenize{OtherManuals/LQF_Manual:appendix-a-converting-a-population-raster-to-a-vector-shapefile-using-qgis}}
Global population datasets are generally available as raster files, but
LQF requires a set of population counts as vector polygons. This guide
explains how to convert a raster dataset to a set of polygons for use in
LQF. Examples are shown using a Greater London population count dataset
at 250m resolution.
\begin{enumerate}
\item {} 
Load the raster file into QGIS

\end{enumerate}

\begin{figure}[htbp]
\centering
\capstart

\noindent\sphinxincludegraphics{{RasterConvert1}.png}
\caption{\sphinxcode{\sphinxupquote{{}`to do{}`}}}\label{\detokenize{OtherManuals/LQF_Manual:id5}}\end{figure}
\begin{enumerate}
\item {} 
Rename the layer to “Pop” (this saves time later)

\item {} 
Make sure the project coordinate reference system (CRS) is the same
as for the raster. To change it, click the label and choose the
correct CRS from the list:

\end{enumerate}

\begin{figure}[htbp]
\centering
\capstart

\noindent\sphinxincludegraphics{{RasterConvert2}.png}
\caption{\sphinxcode{\sphinxupquote{{}`to do{}`}}}\label{\detokenize{OtherManuals/LQF_Manual:id6}}\end{figure}
\begin{enumerate}
\item {} 
Create a vector grid aligned to the raster:
\begin{itemize}
\item {} 
Vector -\textgreater{} Research Tools -\textgreater{} Vector Grid

\begin{figure}[htbp]
\centering
\capstart

\noindent\sphinxincludegraphics{{RasterConvert3}.png}
\caption{\sphinxcode{\sphinxupquote{{}`to do{}`}}}\label{\detokenize{OtherManuals/LQF_Manual:id7}}\end{figure}

\item {} 
This will show the Vector Grid dialog box:

\begin{figure}[htbp]
\centering
\capstart

\noindent\sphinxincludegraphics{{RasterConvert4}.png}
\caption{\sphinxcode{\sphinxupquote{{}`to do{}`}}}\label{\detokenize{OtherManuals/LQF_Manual:id8}}\end{figure}
\begin{itemize}
\item {} 
In \sphinxstylestrong{Grid Extent}:
\begin{itemize}
\item {} 
Choose “Pop”

\item {} 
Click \sphinxstyleemphasis{Align extends and resolution to the selected raster
layer} (unless you want to choose the grid parameters
manually to extract a subset of the raster)

\item {} 
Click \sphinxstyleemphasis{Update extents from layer} to fill in the text boxes
\begin{itemize}
\item {} 
If this option is not available, you will need to get the
resolution of the raster layer by inspecting its metadata
(right click the layer \textgreater{} Properties \textgreater{} Metadata \textgreater{} Pixel
size)

\end{itemize}

\end{itemize}

\item {} 
In \sphinxstylestrong{Parameters}:
\begin{itemize}
\item {} 
Check \sphinxstyleemphasis{Output grid as polygons}

\item {} 
Choose where to save the resulting shapefile containing the
grid

\item {} 
Check \sphinxstyleemphasis{Add results to canvas} so the grid can be used

\end{itemize}

\end{itemize}

\end{itemize}

\item {} 
The raster values must now be extracted from the raster layer into
the vector grid. Use the “Add raster values to features” tool from
\sphinxstylestrong{Processing} \textgreater{} \sphinxstylestrong{Toolbox} \textgreater{} \sphinxstylestrong{SAGA} \textgreater{} \sphinxstylestrong{Vector to raster}:

\begin{figure}[htbp]
\centering
\capstart

\noindent\sphinxincludegraphics{{Saga1}.png}
\caption{\sphinxcode{\sphinxupquote{{}`to do{}`}}}\label{\detokenize{OtherManuals/LQF_Manual:id9}}\end{figure}

\begin{figure}[htbp]
\centering
\capstart

\noindent\sphinxincludegraphics{{Saga2}.png}
\caption{\sphinxcode{\sphinxupquote{{}`to do{}`}}}\label{\detokenize{OtherManuals/LQF_Manual:id10}}\end{figure}
\begin{itemize}
\item {} 
In \sphinxstyleemphasis{Parameters}, choose:
\begin{itemize}
\item {} 
Shapes: The vector grid that you created

\item {} 
Grids: Press “…” and select the “Pop” raster layer

\item {} 
Interpolation: Nearest neighbour (selects the nearest raster
data point)

\item {} 
Result: The location of a new shapefile that contains the
vector grid and the population in each cell

\end{itemize}

\item {} 
Press “Run”. The resulting shapefile will be added to the layers.
It contains a “Pop” column for the population

\item {} 
Use this shapefile as the residential population in LQF (in the
{\hyperref[\detokenize{OtherManuals/LQF_Manual:data-sources-file}]{\sphinxcrossref{Data sources file}}})

\end{itemize}

\end{enumerate}


\subsection{Appendix B: Gathering information about shapefiles for QF modelling}
\label{\detokenize{OtherManuals/LQF_Manual:appendix-b-gathering-information-about-shapefiles-for-qf-modelling}}
LQF and GQF usually need two pieces of information from within a
shapefile. This section explains how to find that information:
\begin{enumerate}
\item {} 
The EPSG code, which defines the coordinate reference system. This is
needed so the model can convert between positions and units of
measurement.

\item {} 
Feature ID field: An attribute within the output areas file that
contains a unique identifier for each output area. This allows the
model to cross-reference between areas.

\end{enumerate}

Firstly, open QGIS and load the griddedResidentialPopulation.shp file by
dragging it into the map area (canvas). An opaque grid should appear.


\subsubsection{Finding the shapefile EPSG code}
\label{\detokenize{OtherManuals/LQF_Manual:finding-the-shapefile-epsg-code}}
In the Layers panel, right-click “griddedResidentialPopulation” and
choose “Set project CRS from Layer”.

\begin{figure}[htbp]
\centering
\capstart

\noindent\sphinxincludegraphics{{300px-LQF_Tutorial_GetEPSG1}.png}
\caption{\sphinxcode{\sphinxupquote{{}`to do{}`}}}\label{\detokenize{OtherManuals/LQF_Manual:id11}}\end{figure}

The
project CRS code in the bottom right-hand corner of the QGIS window will
then change to match that of the output areas file. Use the numeric part
of this to fill in the EPSGcode: entry of the data sources file:

\begin{figure}[htbp]
\centering
\capstart

\noindent\sphinxincludegraphics{{GetEPSG2}.png}
\caption{\sphinxcode{\sphinxupquote{{}`to do{}`}}}\label{\detokenize{OtherManuals/LQF_Manual:id12}}\end{figure}


\subsubsection{Finding the unique feature identifier}
\label{\detokenize{OtherManuals/LQF_Manual:finding-the-unique-feature-identifier}}
Right-click the layer again, and choose “Open Attribute Table”. The
table that appears contains one row for every output area in the file,
and one attribute for each column.

\begin{figure}[htbp]
\centering
\capstart

\noindent\sphinxincludegraphics{{LQF_Tutorial_FindIdentifier}.png}
\caption{\sphinxcode{\sphinxupquote{{}`to do{}`}}}\label{\detokenize{OtherManuals/LQF_Manual:id13}}\end{figure}

In this case, the column with a unique value for every output area is
called “ID”. Use this name in the DataSources file.


\section{SOLWEIG Manual}
\label{\detokenize{OtherManuals/SOLWEIG:solweig-manual}}\label{\detokenize{OtherManuals/SOLWEIG:solweigmanual}}\label{\detokenize{OtherManuals/SOLWEIG::doc}}
The current version of SOLWEIG is v2016a (released 9 September 2016).

NEW in this version: see {\hyperref[\detokenize{OtherManuals/SOLWEIG:version-history}]{\sphinxcrossref{Version History}}}.

The manual for SOLWEIG should be referenced as follows:

\sphinxstyleemphasis{F Lindberg, CSB Grimmond 2016. SOLWEIG\_v2016a Department of Earth Sciences, University of Gothenburg, Sweden, University of Reading, UK.}


\subsection{Introduction}
\label{\detokenize{OtherManuals/SOLWEIG:introduction}}
SOLWEIG is a model which can be used to estimate spatial variations of
3D radiation fluxes and mean radiant temperature (T$_{\text{mrt}}$) in
complex urban settings. The SOLWEIG model follows the same approach
commonly adopted to observe T$_{\text{mrt}}$ (as used, for example, by
Höppe (1992)  %
\begin{footnote}[1]\sphinxAtStartFootnote
Höppe P (1992) A new procedure to determine the mean radiant
temperature outdoors. Wetter Leben 44:147\textendash{}151.
%
\end{footnote}, with shortwave and longwave radiation fluxes from
six directions being individually calculated to derive T$_{\text{mrt}}$.
The model requires a limited number of inputs, such as direct, diffuse
and global shortwave radiation, air temperature, relative humidity,
urban geometry and geographical information (latitude, longitude and
elevation). Additional vegetation and ground cover information can also
be used to imporove the estimation of T$_{\text{mrt}}$. Below is a
flowchart of the model.

\begin{figure}[htbp]
\centering
\capstart

\noindent\sphinxincludegraphics{{SOLWEIG_flowchart}.png}
\caption{Overview of SOLWEIG}\label{\detokenize{OtherManuals/SOLWEIG:id19}}\end{figure}


\subsubsection{Suggested reading}
\label{\detokenize{OtherManuals/SOLWEIG:suggested-reading}}
Read the manual and papers listed below to get a full explanation of the
model and previous evaluation:
\begin{itemize}
\item {} 
Lindberg, F., Onomura, S. and Grimmond, C.S.B (2016) Influence of
ground surface characteristics on the mean radiant temperature in
urban areas. International Journal of Biometeorology. 60(9),
1439-1452. (\sphinxhref{http://link.springer.com/article/10.1007/s00484-016-1135-x}{link to
paper})

\item {} 
Lindberg, F. and Grimmond, C. (2011) The influence of vegetation and
building morphology on shadow patterns and mean radiant temperature
in urban areas: model development and evaluation. Theoretical and
Applied Climatology 105:3, 311-323. (\sphinxhref{http://link.springer.com/article/10.1007/s00704-010-0382-8}{link to
paper})

\item {} 
Lindberg, F., Holmer, B. and Thorsson, S. (2008) SOLWEIG 1.0 \textendash{}
Modelling spatial variations of 3D radiant fluxes and mean radiant
temperature in complex urban settings. International Journal of
Biometeorology 52, 697\textendash{}713. (\sphinxhref{http://link.springer.com/article/10.1007/s00484-008-0162-7}{link to
paper})

\item {} 
Holmer, B., Lindberg, F., Rayner, D. and Thorsson, S. (2015) How to
transform the standing man from a box to a cylinder \textendash{} a modified
methodology to calculate mean radiant temperature in field studies
and models, ICUC9 \textendash{} 9 th International Conference on Urban Climate
jointly with 12th Symposium on the Urban Environment, BPH5: Human
perception and new indicators. Toulouse, July 2015. (\sphinxhref{http://www.meteo.fr/icuc9/LongAbstracts/bph5-2-3271344\_a.pdf}{link to
paper})

\end{itemize}


\subsubsection{Install the model}
\label{\detokenize{OtherManuals/SOLWEIG:install-the-model}}
As SOLWEIG is included in UMEP (as from version 0.2.1) follow the
{\hyperref[\detokenize{Getting_Started:getting-started}]{\sphinxcrossref{\DUrole{std,std-ref,std,std-ref}{Getting Started}}}}
on how to install QGIS and UMEP. When installed successfully, SOLWEIG is
found under \sphinxstyleemphasis{UMEP -\textgreater{} Processor -\textgreater{} Outdoor Thermal Comfort -\textgreater{} Mean
Radiant Temperature (SOLWEIG)}.


\subsection{Input data}
\label{\detokenize{OtherManuals/SOLWEIG:input-data}}
There are two categories of data needed to run SOLWEIG. The first
category is the spatial information in the some various raster grids
explained below. The other is meteorological data and other settings
such as environmental and human exposure parameters.


\subsubsection{Surface models}
\label{\detokenize{OtherManuals/SOLWEIG:surface-models}}
One essential prerequisite for performing successful model calculation
is that all surface models are of the same extent and pixel resolution.


\paragraph{Ground and building DSM}
\label{\detokenize{OtherManuals/SOLWEIG:ground-and-building-dsm}}
As the name suggest this DSM consist of both ground and building heights
(masl).


\paragraph{Vegetation DSMs}
\label{\detokenize{OtherManuals/SOLWEIG:vegetation-dsms}}
3D Vegetation is described by two different grids. First a canopy DSM
(CDSM) which represents the top of the vegetation and second, a trunk
zone DSM (TDSM) that describes the bottom of the vegetation (see
schematic figure). Pixel without any 3D vegetation has the value of zero
and vegetation pixels are given in magl. For a detailed description, see
Lindberg and Grimmond (2011)  %
\begin{footnote}[2]\sphinxAtStartFootnote
Lindberg F, Grimmond CSB, 2011: The influence of vegetation and
building morphology on shadow patterns and mean radiant temperature
in urban areas: model development and evaluation. Theoretical and
Applied Climatology. 105(3), s. 311-323.
%
\end{footnote}.

\begin{figure}[htbp]
\centering
\capstart

\noindent\sphinxincludegraphics{{Vegdems}.png}
\caption{Schematic cross section of the vegetation representation in SOLWEIG.
\sphinxstylestrong{a} Conifer tree (left) and bush (right), \sphinxstylestrong{b} the canopy DEM and
\sphinxstylestrong{c} trunk zone DEM based on (a). From Lindberg and Grimmond (2011)}\label{\detokenize{OtherManuals/SOLWEIG:id20}}\end{figure}

\begin{figure}[htbp]
\centering
\capstart

\noindent\sphinxincludegraphics{{Dsm_gbg}.png}
\caption{Example of a ground and building DSM}\label{\detokenize{OtherManuals/SOLWEIG:id21}}\end{figure}


\paragraph{Digital elevation model}
\label{\detokenize{OtherManuals/SOLWEIG:digital-elevation-model}}
One essential information needed is to know what pixels that are
occupied with buildings. In order to locate these pixels, there are two
possible ways:
\begin{enumerate}
\item {} 
By including a DEM which can bu used in conjunction with the ground
and building DSM to derive building pixels.

\item {} 
By using a land cover grid (see below) where buildings are
represented.

\end{enumerate}

If a DEM is used, is has to be the same spatial resolution and extent as
all the other grids used.


\paragraph{Ground cover grid}
\label{\detokenize{OtherManuals/SOLWEIG:ground-cover-grid}}
If the ground cover scheme as presented in Lindberg et al. (2016)  %
\begin{footnote}[3]\sphinxAtStartFootnote
Lindberg, F., Onomura, S. and Grimmond, C.S.B (2016) Influence of
ground surface characteristics on the mean radiant temperature in
urban areas. International Journal of Biometeorology. 60(9),
1439-1452.
%
\end{footnote}
should be used, a ground cover grid should be included. The Ground cover
grid should be in the UMEP standard format \sphinxstylestrong{except} for the two tree
classes (deciduous and conifer). In other words a ground cover grid in
SOLWEIG represents what is on the ground surface. A UMEP ground cover
grid can be prepared from other data using the Land Cover Reclassifier
(\sphinxstyleemphasis{UMEP -\textgreater{} Pre-Processor -\textgreater{}} {\hyperref[\detokenize{pre-processor/Urban Land Cover Land Cover Reclassifier:landcoverreclassifier}]{\sphinxcrossref{\DUrole{std,std-ref,std,std-ref}{Urban Land Cover: Land Cover Reclassifier}}}})


\subsubsection{Meteorological data}
\label{\detokenize{OtherManuals/SOLWEIG:meteorological-data}}
The format and variables used for meteorological data is the same as for
other parts of the UMEP-plugin. The time resolution is optional. To
prepare such dataset from existing data the {\hyperref[\detokenize{pre-processor/Meteorological Data MetPreprocessor:metpreprocessor}]{\sphinxcrossref{\DUrole{std,std-ref,std,std-ref}{Metdata
Preprocessor}}}}
could be used. If no data is available a single point in time can
modelled using the SOLWEIG interface. There is also a possibility to
download a dataset for any location on Earth using the {\hyperref[\detokenize{pre-processor/Meteorological Data Download data (WATCH):watch}]{\sphinxcrossref{\DUrole{std,std-ref,std,std-ref}{Download data
(WATCH)}}}}-plugin.
The variables required for SOLWIEG are:
\begin{itemize}
\item {} 
\sphinxstyleemphasis{Air temperature} {[}degC{]}

\item {} 
\sphinxstyleemphasis{Relative humidity} {[}\%{]}

\item {} 
\sphinxstyleemphasis{Incoming shortwave radiation} {[}W m$^{\text{-2}}${]}

\end{itemize}

Required are also the components of \sphinxstyleemphasis{diffuse} and \sphinxstyleemphasis{direct} shortwave
radiation. If these are unavailable, and submodel developed by Reindl et
al. (1990)  %
\begin{footnote}[4]\sphinxAtStartFootnote
Reindl D T, Beckman WA, Duffie JA, 1990: “Diffuse fraction
correlation.” Solar energy 45(1): 1-7.
%
\end{footnote} is included in SOLWEIG. Direct radiation perpendicular
to the solar beam should be used.


\subsubsection{Environmental parameters}
\label{\detokenize{OtherManuals/SOLWEIG:environmental-parameters}}
Four main environmental parameters are mandatory; albedo and emissivity
of ground and walls. For building walls, these are bulk albedo values
with a default of 0.20 (albedo) and 0.90 (emissivity). If the ground
cover scheme is not used the bulk ground values are 0.15 (albedo) and
0.95 (emissivity).

If the ground cover scheme is activated (specific tick box found in the
plugin-interface), the variables for albedo, emissivity and how surface
temperature is parameterised for different surfaces is found in
\sphinxstylestrong{landcoverclasses\_v2016a.txt}. For as detailed description of the
ground cover scheme, see Lindberg et al. (2016)  %
\begin{footnote}[5]\sphinxAtStartFootnote
%
\end{footnote}.
\sphinxstylestrong{landcoverclasses\_v2016a.txt} can be found in
\sphinxstyleemphasis{C:\textbackslash{}Users\textbackslash{}your\_username.qgis2\textbackslash{}python\textbackslash{}plugins\textbackslash{}UMEP\textbackslash{}SOLWEIG}.

It should be noted that it is only grass and impervious surfaces that
has been parameterisised and evaluated. Other surfaces such as bare soil
and water are only first order approximations at this point.


\subsubsection{Human exposure parameters}
\label{\detokenize{OtherManuals/SOLWEIG:human-exposure-parameters}}
There are three human exposure parameters available:
\begin{itemize}
\item {} 
\sphinxstyleemphasis{Absorption of shortwave radiation} (default value=0.70)

\item {} 
\sphinxstyleemphasis{Absorption of longwave radiation} (default value=0.95)

\item {} 
\sphinxstyleemphasis{Posture} (default value=Standing)

\end{itemize}


\subsubsection{Optional settings}
\label{\detokenize{OtherManuals/SOLWEIG:optional-settings}}\begin{itemize}
\item {} 
The original model as described in Lindberg et al. (2008)  %
\begin{footnote}[6]\sphinxAtStartFootnote
Lindberg F, Thorsson S, Holmer B, 2008: SOLWEIG 1.0 \textendash{} Modelling
spatial variations of 3D radiant fluxes and mean radiant temperature
in complex urban settings. International Journal of Biometeorology
(2008) 52:697\textendash{}713.
%
\end{footnote} used
an adjustment of sky emissivity (Jonsson et al. (2006)  %
\begin{footnote}[7]\sphinxAtStartFootnote
Jonsson P, Eliasson I, Holmer B, Grimmond CSB (2006) Longwave
incoming radiation in the Tropics: results from field work in three
African cities. Theor Appl Climatol 85:185\textendash{}201
%
\end{footnote}
calculated using the method presented in Prata (1996)  %
\begin{footnote}[8]\sphinxAtStartFootnote
Prata AJ (1996) A new long-wave formula for estimating downward
clearsky radiation at the surface. Q J R Meteorol Soc 122:1127\textendash{}1151
%
\end{footnote}. This is
now removed but can be added as an option.

\item {} 
As from version 2015a it is possible to consider the human as a
cyliner instead of a box. See Holmer et al. (2015)  %
\begin{footnote}[9]\sphinxAtStartFootnote
Holmer B, Lindberg F, Thorsson S, Rayner D, 2015: How to transform
the standing man from a box to a cylinder \textendash{} a modified methodology to
calculate mean radiant temperature in field studies and models. ICUC9
- 9th International Conference on Urban Climate jointly with 12th
Symposium on the Urban Environment.
%
\end{footnote} for more
details.

\end{itemize}


\subsection{Output data}
\label{\detokenize{OtherManuals/SOLWEIG:output-data}}
There are two forms of output available, calculated grids of various
parameters and full model outputs from certain point of interests (POIs)
within the model domain.


\subsubsection{Surface grids}
\label{\detokenize{OtherManuals/SOLWEIG:surface-grids}}
There are six different grids that can be saved from each model
iteration:
\begin{enumerate}
\item {} 
Mean radiation temperature

\item {} 
Incoming shortwave radiation

\item {} 
Outgoing shortwave radiation

\item {} 
Incoming longwave radiation

\item {} 
Outgoing longwave radiation

\item {} 
Shadow patterns

\end{enumerate}

A post-processing plugin (SOLWEIG Analyzer) for the output grids are
planned to be included in future versions of UMEP.


\subsubsection{POI.txt}
\label{\detokenize{OtherManuals/SOLWEIG:poi-txt}}
By ticking in the option to include POIs (Point of Interest), a vector
point layer can be added and full model output are written out to text
files for the specific POI. Multiple POIs can be used by including many
points in the vector file. In the table below is the output variables
specifiedː


\begin{savenotes}\sphinxatlongtablestart\begin{longtable}{|\X{5}{100}|\X{20}{100}|\X{75}{100}|}
\hline
\sphinxstyletheadfamily 
Column
&\sphinxstyletheadfamily 
Name
&\sphinxstyletheadfamily 
Description
\\
\hline
\endfirsthead

\multicolumn{3}{c}%
{\makebox[0pt]{\sphinxtablecontinued{\tablename\ \thetable{} -- continued from previous page}}}\\
\hline
\sphinxstyletheadfamily 
Column
&\sphinxstyletheadfamily 
Name
&\sphinxstyletheadfamily 
Description
\\
\hline
\endhead

\hline
\multicolumn{3}{r}{\makebox[0pt][r]{\sphinxtablecontinued{Continued on next page}}}\\
\endfoot

\endlastfoot

1
&
iy
&
Year {[}YYYY{]}
\\
\hline
2
&
id
&
Day of year {[}DOY{]}
\\
\hline
3
&
it
&
Hour {[}H{]}
\\
\hline
4
&
imin
&
Minute {[}M{]}
\\
\hline
5
&
dectime
&
Decimal time {[}-{]}
\\
\hline
6
&
altitude
&
altitude of the Sun {[}°{]}
\\
\hline
7
&
azimuth
&
azimuth of the Sun {[}°{]}
\\
\hline
8
&
kdir
&
direct beam solar radiation (from meteorological data) {[}W m$^{\text{-2}}${]}
\\
\hline
9
&
kdiff
&
diffuse component of radiation (from meteorological data) {[}W m$^{\text{-2}}${]}
\\
\hline
10
&
kglobal
&
global radiation (from meteorological data) {[}W m$^{\text{-2}}${]}
\\
\hline
11
&
kdown
&
Incoming shortwave radiation {[}W m$^{\text{-2}}${]}
\\
\hline
12
&
kup
&
Outgoing shortwave radiation {[}W m$^{\text{-2}}${]}
\\
\hline
13
&
keast
&
Incoming shortwave radiation {[}W m$^{\text{-2}}${]}
\\
\hline
14
&
ksouth
&
Outgoing shortwave radiation {[}W m$^{\text{-2}}${]}
\\
\hline
15
&
kwest
&
Incoming shortwave radiation {[}W m$^{\text{-2}}${]}
\\
\hline
16
&
knorth
&
Outgoing shortwave radiation {[}W m$^{\text{-2}}${]}
\\
\hline
17
&
ldown
&
Incoming longwave radiation {[}W m$^{\text{-2}}${]}
\\
\hline
18
&
lup
&
Outgoing longwave radiation {[}W m$^{\text{-2}}${]}
\\
\hline
19
&
least
&
Incoming longwave radiation {[}W m$^{\text{-2}}${]}
\\
\hline
20
&
lsouth
&
Outgoing longwave radiation {[}W m$^{\text{-2}}${]}
\\
\hline
21
&
lwest
&
Incoming longwave radiation {[}W m$^{\text{-2}}${]}
\\
\hline
22
&
lnorth
&
Outgoing longwave radiation {[}W m$^{\text{-2}}${]}
\\
\hline
23
&
Ta
&
air temperature from meteorological data {[}°C{]}
\\
\hline
24
&
Tg
&
calculated surface temperature {[}°C{]}
\\
\hline
25
&
RH
&
relative humidity from meteorological data {[}percent{]}
\\
\hline
26
&
Esky
&
sky emissivity
\\
\hline
27
&
Tmrt
&
mean radiant temperature {[}°C{]}
\\
\hline
28
&
I0
&
theoretical value of maximum incoming solar radiation {[}W m$^{\text{-2}}${]}
\\
\hline
29
&
CI
&
clearness index
\\
\hline
30
&
Shadow
&
Shadow value
\\
\hline
31
&
SVF\_b
&
Sky View Factor from ground and buildings
\\
\hline
32
&
SVF\_b+v
&
Sky View Factor from ground, buildings and vegetation
\\
\hline
33
&
KsideI
&
Direct shortwave radiation from side if cylinder option is used
\\
\hline
\end{longtable}\sphinxatlongtableend\end{savenotes}


\subsection{How to run the model}
\label{\detokenize{OtherManuals/SOLWEIG:how-to-run-the-model}}
The following section provides information on how to run the model and
what consideration that should be taken into account in order for the
model to perform at its best.


\subsubsection{Run the model for example data}
\label{\detokenize{OtherManuals/SOLWEIG:run-the-model-for-example-data}}
Before running the model for your own data it is good to make certain
that you can run the test data and get the same results as in the
example files provided. Test/example files are given for Göteborg,
Sweden or London, UK. Here, you will use the Göteborg data.
\begin{enumerate}
\item {} 
Download and extract the test dataset to your computer
(\sphinxhref{https://bitbucket.org/fredrik\_ucg/umep/downloads/testdata\_UMEP.zip}{testdata\_UMEP.zip}).

\item {} 
Add the raster layers (DSM, CDSM and land cover) from the Goteborg
folder into a new QGIS session. The coordinate system of the grids is
Sweref99 1200 (EPSG:3007).

\item {} 
In order to run SOLWEIG, some additional datasets must be created
based on the raster grids you just added. Open the \sphinxstylestrong{SkyViewFactor
Calculator} from the UMEP Pre-processor and calculate SVFs using
both your DSM and CDSM. Leave all settings as default. This
calculation produces a file called \sphinxstylestrong{svf.zip}’ which is used later
in the calculations.

\item {} 
Open the \sphinxstylestrong{Wall height and aspect} plugin from the UMEP
Pre-processor and calculate both wall height and aspect using the DSM
and your input raster. Make sure to add the result to your project.

\item {} 
Now you are ready to generate your first T$_{\text{mrt}}$ map. Open
SOLWEIG and use the settings as shown below but replacing the paths
to fit your computer environment. When you are finished, press Run.

\end{enumerate}

\begin{figure}[htbp]
\centering
\capstart

\noindent\sphinxincludegraphics{{SOLWEIGfirsttry}.png}
\caption{Dialog for the SOLWEIG model}\label{\detokenize{OtherManuals/SOLWEIG:id22}}\end{figure}


\subsection{Tips and Tricks}
\label{\detokenize{OtherManuals/SOLWEIG:tips-and-tricks}}\begin{itemize}
\item {} 
The model is very sensitive to the timing global radiation, i.e..
that the peak of solar radiation occurs at local noon. If using a
meteorological file included a longer dataset, this could be checked
by comparing the global solar radiation and the theoretical maximum
of solar radiation (I0) from a solar exposed point of interest.

\item {} 
If using the land cover grid to derive the building grid, it is
important that it coincides with the ground and building DSM.
Otherwise strange results will be produced.

\item {} 
SOLWEIG focus on pedestrian radiation fluxes and it is not
recommended to consider fluxes on building roofs.

\end{itemize}


\subsection{Acknowledgements}
\label{\detokenize{OtherManuals/SOLWEIG:acknowledgements}}\begin{itemize}
\item {} 
People who have contributed to the development of SOLWEIG (plus
co-authors of papers):

\item {} 
Current contributors:
\begin{itemize}
\item {} 
C.S.B. Grimmond (University of Reading; previously Indiana
University, King’s College London, UK),

\item {} 
Fredrik Lindberg (Göteborg University, Sweden)

\item {} 
Björn Holmer (Göteborg University, Sweden)

\end{itemize}

\item {} 
Past Contributors:
\begin{itemize}
\item {} 
Shiho Onomura (Göteborg University, Sweden)

\item {} 
Sofia Thorsson (Göteborg University, Sweden)

\item {} 
Ingegärd Eliasson (Göteborg University, Sweden)

\item {} 
Janina Konarska (Göteborg University, Sweden)

\item {} 
David Rayner (Göteborg University, Sweden)

\end{itemize}

\item {} 
Funding to support development:
\begin{itemize}
\item {} 
FORMAS, National Science Foundation (USA, BCS-0095284,
ATM-0710631), EU Framework 7 BRIDGE (211345); EU emBRACE; UK Met
Office; NERC ClearfLO, NERC TRUC.

\end{itemize}

\end{itemize}


\subsection{Abbreviations}
\label{\detokenize{OtherManuals/SOLWEIG:abbreviations}}

\begin{savenotes}\sphinxattablestart
\centering
\begin{tabular}[t]{|\X{10}{60}|\X{50}{60}|}
\hline
\sphinxstyletheadfamily 
DEM
&\sphinxstyletheadfamily 
Digital Elevation Model
\\
\hline
DSM
&
Digital surface model
\\
\hline
DTM
&
Digital Terrain Model
\\
\hline
L↓
&
Incoming longwave radiation
\\
\hline
LAI
&
Leaf area index
\\
\hline
SOLWEIG
&
The solar and longwave environmental irradiance geometry model
\\
\hline
SVF
&
Sky view factor
\\
\hline
UMEP
&
{\hyperref[\detokenize{index:index-page}]{\sphinxcrossref{\DUrole{std,std-ref,std,std-ref}{UMEP Manual}}}}
\\
\hline
GUI
&
Graphical User Interface
\\
\hline
POI
&
Point of Interest
\\
\hline
\end{tabular}
\par
\sphinxattableend\end{savenotes}


\subsection{Development}
\label{\detokenize{OtherManuals/SOLWEIG:development}}
SOLWEIG is an an open source model that we are keen to get others inputs
and contributions. There are two main ways to contribute:
\begin{enumerate}
\item {} 
Submit comments or issues to the
\sphinxhref{https://bitbucket.org/fredrik\_ucg/umep/issues}{repository}

\item {} 
Participate in Coding or adding new
features {\hyperref[\detokenize{DevelopmentGuidelines:developmentguidelines}]{\sphinxcrossref{\DUrole{std,std-ref,std,std-ref}{DevelopmentGuidelines}}}}.

\end{enumerate}


\subsection{Version History}
\label{\detokenize{OtherManuals/SOLWEIG:version-history}}

\begin{savenotes}\sphinxattablestart
\centering
\begin{tabular}[t]{|\X{15}{100}|\X{85}{100}|}
\hline
\sphinxstyletheadfamily 
Version
&\sphinxstyletheadfamily 
Changes from previous version
\\
\hline
v2016a
&
First version released within UMEP. Python version of model is now released as open source.
\\
\hline
v2015a
&\begin{itemize}
\item {} 
Now includes a simple land cover scheme according to Lindberg et al. (2015)   * -

\item {} 
Option to consider man as cylinder included (Holmer et al. 2015)   * -

\item {} 
More options regarding incoming longwave radiation is added to the GUI

\end{itemize}
\\
\hline
v2014a
&\begin{itemize}
\item {} 
The model is now able to run at any time interval   * -

\item {} 
A new format of the input met. data is introduced   * -

\item {} 
The time stamp is now ‘fixed’ i.e., 1400 in an hourly dataset represent the hour before.

\end{itemize}
\\
\hline
2013a
&
A new GUI is introduced as well as options to load gridded vegetation DSMs.
\\
\hline
2.3
&
A new scheme for reflection concerning the shortwave fluxes is included taking into account sunlit and shaded walls
\\
\hline
2.2
&
Some major (and minor) bugs have been fixed such as:   * -
-  A major bug regarding the scale of trees and bushes is resolved
\\
\hline
2.0
&
A new vegetation scheme is now included (Lindberg and Grimmond 2011). The interface also has a wizard for generating vegetation data to be included in the calculations. The new vegetation scheme is again slowing down the calculation but the computation time is still acceptable.
\\
\hline
1.1
&
Longwave and shortwave radiation fluxes from the four cardinal points is now separated based on anisotropical Sky View Factor (SVF) images. Ground View Factors is introduced which is a parameter that is estimated based on what an instrument measuring Lup actually is seeing based on its height above ground and shadow patterns. In order to make accurate estimations of GVF, locations of building walls need to be known. Walls can be found automatically be the SOLWEIG-model. However, if the User wants to have more control over what are buildings and not, the User should use the marking tool included in the ‘Create/Edit Vegetation DEM’. A very simple approach taken from Offerle et al. (2003) is used to estimate nocturnal Ldown. Therefore Tmrt could also be estimated during night in version 1.1.
\\
\hline
1.0
&
First version as from Lindberg et al. (2008)
\\
\hline
\end{tabular}
\par
\sphinxattableend\end{savenotes}


\subsection{References}
\label{\detokenize{OtherManuals/SOLWEIG:references}}

\section{SUEWS Manual}
\label{\detokenize{OtherManuals/SUEWS:suews-manual}}\label{\detokenize{OtherManuals/SUEWS::doc}}
The manual of SUEWS can be accessed at \sphinxhref{http://suews-docs.readthedocs.io}{its own documentation site}.



\renewcommand{\indexname}{Index}
\printindex
\end{document}